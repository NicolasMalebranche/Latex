\documentclass[11pt]{article}
\usepackage[T1]{fontenc}
\usepackage[latin1]{inputenc}
\usepackage[german]{babel}
\usepackage{fourier}  % Use the Adobe Utopia font for the document
\usepackage{amsmath,amsthm,amssymb,amscd,color,graphicx,hyperref}

% Geschwungene Kleinbuchstaben im Mathemodus (benutze \mathpzc)
\DeclareFontFamily{OT1}{pzc}{}
\DeclareFontShape{OT1}{pzc}{m}{it}{<-> s * [1.10] pzcmi7t}{}
\DeclareMathAlphabet{\mathpzc}{OT1}{pzc}{m}{it}


%Struktur
\newcommand{\eintrag}{\subparagraph*}
  
%Abk�rzungen
\newcommand{\N}{\mathbb{N}}
\newcommand{\Z}{\mathbb{Z}}
\newcommand{\Q}{\mathbb{Q}}
\newcommand{\C}{\mathbb{C}}
\newcommand{\R}{\mathbb{R}}
\newcommand{\Cinf}{C^\infty}
\newcommand{\id}{\text{id}}
\renewcommand{\O}{\mathcal{O}\!}
\renewcommand{\S}{\mathfrak{S}}
\newcommand{\T}{\mathcal{T}}
\newcommand{\E}{\mathbb{E}}
\newcommand{\A}{\mathcal{A}}
\renewcommand{\a}{\mathfrak{a}}
\newcommand{\m}{\mathfrak{m}}
\newcommand{\Hom}{\mathcal{H}\!\!\!\mathpzc{om}}
\newcommand{\End}{\mathcal{E}\!\!\mathpzc{nd}}
\renewcommand{\d}{d\!}
\newcommand{\del}{\partial}
\newcommand{\delbar}{\overline{\partial}}
\newcommand{\dzbar}{\d\overline{z}}
\newcommand{\diff}[1]{\frac{\partial}{\partial #1}}
\newcommand{\vac}{\left|0\right>}
\DeclareMathOperator{\Sym}{Sym}
\DeclareMathOperator{\Spur}{Spur}
\newcommand{\vectline}[1]{\underline{\smash{ #1 }}} 
\newcommand{\normord}[1]{\mathop{:}\nolimits #1 \mathop{:}\nolimits\ }
\newcommand{\lie}[1]{\left[ #1 \right]}

%\binoppenalty=7000
%\relpenalty=5000 

% Title Page
\title{Tagebuch}
\author{Simon Kapfer}

\begin{document}
\maketitle
\begin{abstract}
Was mir an mathematisch Interessantem einf�llt.
\end{abstract}
\eintrag{p-adische Approximation mit erzeugenden Funktionen} Eine Potenzreihe $F(x) = \sum c_n x^n$ konvergiert (wenn die $c_n$ nicht zu schlecht sind, also z.B. ganze Zahlen) im p-adischen Sinne in $\Z_p$ bzw. $\Q_p$ f�r alle $x$, die Vielfache von $p$ sind. F�r $x_0= a_1 p + a_2 p^2 +\ldots $ konvergieren die Koeffizienten der Reihe
$$\frac{F(a_1pt+a_2p^2t^2+\ldots)}{1-t} $$ 
gegen $F(x_0)$. Idee dazu kam von folgender Frage: Gegeben ein modulo $p^n\ \forall n$ surjektives Polynom $P$ (vgl. Hensel Lemma), wie kann man eine Folge $(r_n)$ finden, so da� $P(r_n)$ durch $p^n$ teilbar ist? Man mu� die Nullstellen von $P$ n�mlich p-adisch approximieren, z.B. mittels Newton-Verfahren, oder direkt eine Potenzreihe f�r die Wurzel nehmen.

\eintrag{Poincar�-Birkhoff-Witt-Theorem} Die universell einh�llende Algebra einer Liealgebra $\mathfrak{g}$ ist der Quotient der freien Tensoralgebra durch die Relationen $\left<a\otimes b - b\otimes a -[a,b] \right>$. Die Aussage des sog. Theorems ist, da� $U(\mathfrak{g})$ als Vektorraum von den (linear unabh�ngigen) geordneten Tensoren der $\mathfrak{g}$-Basiselemente $e_{i_1} \otimes\ldots\otimes e_{i_k},\ i_1\leq\ldots \leq i_k$ aufgespannt wird. Die lineare Unabh�ngigkeit ist der nichttriviale Teil. \\ Meine Beweisidee: $U(\mathfrak{g}) = \bigoplus U^k$ wobei $U^k = \left<e_{i_1} \otimes\ldots\otimes e_{i_k},\ i_1\leq \ldots \leq i_k \right>$ und die Erzeuger von $U^k$ sind linear unabh�ngig. Bleibt zu zeigen, da� $U^k\cap U^l = 0$ f�r verschiedene $k$ und $l$. Die Relationen der Einh�llenden vermischen nur zwei benachbarte Grade, d. h. es reicht zu zeigen, da� $U^k\cap U^{k-1} = 0$. Mit anderen Worten, die Vertauschung von zwei Basisvektoren in $U^k$ soll sich zu einer Gruppenwirkung von $\S_k$ auf $U^k\oplus U^{k-1}$ fortsetzen lassen. $S_k$ wird von Transpositionen benachbarter Elemente $\sigma_i$ erzeugt, die den Relationen $\sigma_i^2 =1 ,\ \sigma_i\sigma_j = \sigma_j\sigma_i$ falls $ |i-j|>1$ und $ \sigma_i\sigma_{i+1}\sigma_i = \sigma_{i+1}\sigma_i \sigma_{i+1}$ gehorchen. Die erste Gleichung gilt wegen Antikommutativit�t der Lieklammer, die zweite trivialerweise, die dritte braucht Jacobi. Gezeigt ist nun: egal, in welcher Reihenfolge man ein Element in $U^k$ in Normalform (durch Vertauschen) bringt, man wird stets dieselbe Korrektur in $U^{k-1}$ erhalten. Insbesondere wird Umformen eines Elements in Normalform in sich selbst keine Korrekturterme produzieren.

\eintrag{Dimension von metrischen Graphen} Ein metrischer Raum aus $n+1$ Punkten kann stets als Eckenmenge eines $n$-Simplex dargestellt werden. Dieses Simplex ist manchmal entartet, was eine Reduzierung der Dimension bedeutet. Wie weit kann man die Dimension reduzieren, wenn man f�r die Abst�nde einen Fehler von bestimmter Gr��e zul��t? M�glicherweise gehts mit einer Eigenraumzerlegung der darstellenden Matrix � la Carina.

\eintrag{Faktorisierungen in der Symmetrischen Gruppe} Bezeichne mit $f_{n,k}$ die Anzahl der M�glichkeiten, die Identit�t in $\S_n$ als $k$-faches Produkt $n$-Zykeln zu schreiben (unter Ber�cksichtigung der Reihenfolge). Dann gilt f�r die erzeugenden Funktionen $F_n(x):= \sum_{k\geq 0}f_{n,k}x^k$:
\begin{align*}
F_0(x)=& 1 \\
F_1(x)=& \frac{1}{1-x} \\
F_2(x)=& \frac{1}{1-x^2} \\
F_3(x)=& \frac{1-x}{1-x-2x^2}\\
F_4(x)=& \frac{1-34x^2+24x^4}{1-40x^2+144x^4}  \\
F_5(x)=& \frac{1-22x-72x^2+384x^3+384x^4}{1-22x-96x^2+432x^3+1536x^4}\\
F_6(x)=& \frac{1-19320x^2+43720704x^4-18345277440x^6+300589056000x^8}{1-19440x^2+42418944x^4-16334438400x^6+303906816000x^8}\\
F_7(x)=&\frac{\text{Polynom vom Grad 7}}{\text{Polynom vom Grad 7}},\quad
F_8(x)=\frac{\text{Polynom vom Grad 14}}{\text{Polynom vom Grad 14}}
\end{align*}
Wir haben immer rationale Funktionen, auch wenn die Koeffizienten mit wachsendem $n$ regelrecht explodieren. Die Frage, $f_{n,k}$ zu bestimmen, kam von Kai. Mit dem Paper von Goupil und Schaeffer kommt man an eine Berechnungsmethode daf�r �ber Matrixpotenzen. Damit bekommt man auch obere Schranken f�r den Nennergrad der $F_n$, n�mlich die Partitionszahlen $1, 1, 2, 3, 5, 7, 11, 15, 22, \ldots$. Die Tatsache, da� f�r gerade $n$ nur gerade Funktionen auftauchen, sieht man, sobald man das Signum der Permutationen anschaut. $\deg F_{2n} = 2\deg F_{2n-1}$ folgt m�glicherweise auch daraus.

\eintrag{Charakteristische Polynome von Matrixpotenzen} Die Koeffizienten $c_k$ des charakteristischen Polynoms einer $d\times d$-Matrix $A$ sind bis aufs Vorzeichen die elementarsymmetrischen Funktionen in den Eigenwerten: $\mathrm{e}_k = (-1)^{k} c_k$. Was sind die charakteristischen Polynome von $A^n$?
$$
S_A(x) := \sum_n\text{Spur}(A^n)x^n = \frac{\text{Rev}\left(\diff{x}\chi_A(x)\right)}{\text{Rev}\left(\chi_A(x)\right)}
$$
Dabei steht $\text{Rev}: P(x) \longmapsto x^{\deg P}P\left(\frac{1}{x}\right)$ f�r das r�ckw�rts gelesene Polynom. $S_A$ ist nat�rlich eine symmetrische Funktion in den Eigenwerten, daher kann man von Plethysmus sprechen und es gilt:
$$(-1)^{k} \sum_n c_k(A^n)x^n = S_A[\mathrm{e}_k](x) $$
Also braucht man, um $\chi_{A^ n}$ auszurechnen, nichts weiter als $\chi_A$ und die plethystischen Formeln f�r $\mathrm{e}_j[\mathrm{e}_k]$. Falls $\det A =1$, hat man au�erdem eine nette Dualit�t, die man durch Betrachtung der Eigenwerte beweist, n�mlich
$$ (-1)^{d-1} \sum_n c_{d-1}(A^n)x^n = \frac{\text{Rev}\left(\diff{x} \text{Rev}\left(\chi_A(x)\right)\right)}{\chi_A(x)}
$$
Diese Dualit�t wird von Plethysmen respektiert, und so haben dann die erzeugenden Funktionen von $c_k(A^n)$ und $c_{d-k}(A^n)$ jeweils r�ckw�rts gelesene Nenner.

\eintrag{Ordinalzahlen}
Betrachte alle abz�hlbaren, nicht endliche Mengen, welche total geordnet sind. F�hre eine Relation zwischen diesen ein: $M\unlhd N$, falls eine injektive, monotone (steigende oder fallende) Abbildung $: M \rightarrow N$ existiert. Diese Relation besitzt ein kleinstes Element, n�mlich $\N$, sowie ein gr��tes Element, n�mlich $\Q$. 

\eintrag{Eulercharakteristik von symmetrischen Potenzen} Wenn $V=V^{+}\oplus V^{-}$ ein Supervektorraum ist und $\chi(V):= \dim V^{+} - \dim V^{-}$, dann ist 
$$ \sum\chi(\text{Sym}^nV)x^n = (1-x)^{-\chi(V)}. $$
Das sollte auch mit beliebigen gewichteten Zerlegungen von $V$ funkionieren und aus der entsprechenden Formel f�r Charaktere von Darstellungen folgen.

\eintrag{Hilbertschema von Punkten auf K3} Sei $X$ eine K3. Dann ist die Torsion der Cup-Produkt-Bilder von
\begin{itemize}
\item $\text{Sym}^2(H^2(X^{[2]};\Z))$ in $H^4(X^{[2]};\Z)$ gleich $\left(\frac{\Z}{2}\right)^{23} \oplus \frac{\Z}{5}$.
\item $\text{Sym}^3(H^2(X^{[2]};\Z))$ in $H^6(X^{[2]};\Z)$ gleich $\frac{\Z}{2}$.
\item $\text{Sym}^2(H^2(X^{[3]};\Z))$ in $H^4(X^{[3]};\Z)$ gleich $3$. Dabei kann $\frac{1}{3}\a_3(1)\vac$ nicht getroffen werden, sondern nur das Dreifache.
\item $h^6(X^{[3]}) = 2554$.
$$ \frac{H^6(X^{[3]};\Z)}{H^4(X^{[3]};\Z)\cup H^2(X^{[3]};\Z)} = \left(\frac{\Z}{3}\right)^{23}
$$ 

\end{itemize}

\eintrag{Primfaktoren z�hlen} Sei $p\geq 3$ eine Primzahl. Bezeichne $o(n)$ die Anzahl von $p$-Faktoren in $n!$. Dann ist $o(n)\leq \frac{n}{p-1}$ linear beschr�nkt und es gibt keinen kleineren Faktor als $\frac{1}{p-1}$, der es auch tut.
\eintrag{p-adischer Arcustangens} Der Arcustangens, definiert �ber die �bliche Potenzreihe, konvergiert im p-adischen nur f�r Argumente vom Betrag kleiner 1. Mit der Formel 
$$ 2 \arctan x = \arctan\left(\frac{2x}{1-x^2}\right)$$
l��t er sich jedoch f�r Primzahlen $p\in \{2\}\cup\{2^n - 1\}$ auf ganz $\Q_p \backslash \{\pm 1\}$ fortsetzen. Dort ist er nicht mehr injektiv, es gilt z. B. in $\Q_2$: $\arctan\left(8n\pm 1 \right) = \arctan\left(\pm\frac{4n}{4n\pm 1} \right)$.
Die Machinsche Formel f�r $\frac{\pi}{4}$ entartet zu: $4\arctan\frac{1}{5} - \arctan\frac{1}{239} = 0$.
\\
Nachtrag: Man hat allgemein die Formel: $ \arctan(x)=\Im \log (1+ix)$ und daher:
$$ \arctan(x) = \frac{1}{m}\Im \log((1+ix)^m) =  \frac{1}{m}\Im\log \left(1+ i \frac{\sum_k (-1)^k \binom{m}{2k+1}x^{2k+1}}{\sum_k (-1)^k \binom{m}{2k}x^{2k}}\right)
$$
bis auf Addition von $\pi$-Vielfachen. So lassen sich f�r $p=2$ oder $p= 4n - 1$ alle Argumente in etwas transformieren, wo die Potenzreihe konvergiert, zum Beispiel, in dem man $m:=p+1$ nimmt. F�r $p= 4n +1$ und $m:=p-1$ geht es auch fast �berall gut, mit Ausnahme von $\pm i \in \Q_p$. Da kann man aber auch gleich 
$ \arctan(x) := \frac{1}{2i}\log\left(\frac{x-i}{x+i}\right) $
setzen, so da� die beiden Definitionsl�cken offensichtlich werden.

\eintrag{Monotonie und Stetigkeit} Seien $f,g : \R\rightarrow \R$ monotone Funktionen, so da� $g\circ f=\id$. Dann ist $g$ stetig. Es reicht f�r die Stetigkeit von $g$ sogar: $g$ ist monoton und surjektiv. \\
\emph{Beweis.} Seien $a,b\in\R$ beliebig. Wegen Monotonie von $g$ ist $g^{-1}([a,b])$ ein Intervall. Setze $I:=\inf(g^{-1}([a,b]))$. Falls $I> -\infty$ und $g(I)<a$ w�re, so w�re $g$ nicht surjektiv. Also $g(I)=a\in[a,b]$. Also ist $g^{-1}([a,b])$ abgeschlossen. Also ist $g$ stetig.

\eintrag{Idempotente Permutationen} Sei $m(r,n)$ die Anzahl aller Permutationen $\sigma \in \S_n$, f�r die gilt: $\sigma^k =\id$. Dann hat man
$$\sum_n\frac{m(r,n)}{n!}x^n = \prod_{i\mid r}\exp\left(\frac{x^i}{i}\right).$$

\eintrag{Signatur von Hilbertschemata von Punkten auf einer K3} Bezeichne mit $s_n$ die Signatur von $H^{2n}(S^{[n]}, \mathbb{Z})$. Dann gilt: 
$$ \sum_n s_nx^n = \left[(1+x)(1+x^3)(1+x^5)\ldots\right]^{-16}\left[(1-x^2)(1-x^4)(1-x^6)\ldots\right]^{-20}
$$ Zum Beweis kann man eine Formel von Ellingsrud/G�ttsche/Lehn f�r das $\chi_y$-Geschlecht benutzen, die \hyperlink{http://arxiv.org/abs/math/9904095}{hier} zu finden ist.

\eintrag{Partitionszahlen und 5, 7, 11} Unter den Partitionszahlen $\{p_0,\ldots, p_N\}$ kommen (f�r gro�e $n$) solche, welche durch 5, 7 oder 11 teilbar sind, deutlich h�ufiger vor als man erwarten w�rde. Bezeichne $D_{p,N}$ die Anzahl aller Partitionszahlen $p_0$ bis $p_N$, welche durch $p$ teilbar sind. 
Existiert $L_p:=\lim \frac{1}{N} D_{p,N}$ und wenn ja, wie gro� ist er? $D_{5,150k}\approx 0.36$, $D_{7,150k} \approx 0.27$, $D_{11,150k}\approx 0.174$. Siehe auch \url{ericaklarreich.com/pieces%20of%20numbers%20b&w%20compressed.PDF}.

\eintrag{Differenzieren mit Zufallsvariablen �ber $\Z$} Um eine $\R$-wertige Koordinatenfunktion auf $\Z$ zu beschr�nken, gleichzeitig aber das Gef�hl des Kontinuums nicht aufzugeben, ersetzen wir Koordinaten durch Zufallsvariablen mit Werten in $\Z$. Wir definieren f�r jede derartige Zufallsvariable $X$ und jeden Parameter $\alpha\in\R$ affine Translationen $T_\alpha$ derart, da� $T_\alpha T_\beta X = T_{\alpha+\beta} X$ und, falls $\alpha\in\Z$, $T_\alpha X = X+\alpha$ gelten sollen. Vielleicht ist es auch gut, $\E[T_\alpha X] = \E[X] + \alpha$ im Falle der Existenz zu fordern.
Das Differential $\frac{dY}{dX}$ soll dann was wie 
$ \lim\limits_{h\rightarrow 0}\frac{T_hY-Y}{T_hX-X} $ sein. Vielleicht sollte man besser $ \lim\limits_{h\rightarrow 0} \frac{1}{h} \E \left[\left\| \frac{dY}{dX}\cdot (T_hX-X) - (T_hY-Y) \right\|\right]=0$ verlangen.
F�r ein Polynom $f$ und $Y = f(X)$ hat man dann $\frac{dY}{dX} = f'(X)$.

\eintrag{Fl�chentreue lineare Abbildungen} Sei $A\in\R^{n\times n}$. Wir suchen die Bedingung daf�r, da� $A$ Fl�cheninhalte erh�lt. Der Fl�cheninhalt des Parallelogramms, das von $u$ und $v$ im $\R^n$ aufgespannt wird, ist gleich $\sqrt{\|u\|^2\|v\|^2 - \left<u,v\right>^2} = \|u \wedge v\|$ mit dem induzierten Skalarprodukt auf $\Lambda^2\R^n$. Also erh�lt $A$ genau dann Fl�chen, wenn die induzierte Abbildung auf $\Lambda^2\R^n$ orthogonal ist. F�r $n=1$ ist das immer der Fall, f�r $n=2$ genau dann, wenn $\det A =\pm 1$, f�r $n=3$ ist die induzierte Abbildung gleich der Adjunkten Matrix, also bis auf eine Determinante gleich $A^{-1}$, also mu� $A^{-1}$ bzw. $A$ orthogonal sein. St�rungstheoretisch kann man $A=1+tR$ setzen und beobachtet, da� f�r $n\geq 3$ Fl�chentreue in erster N�herung Antisymmetrie von $R$ bedeutet, d.~h. durch $A=\exp(R)$ bekommt man eine Zusammenhangskomponente der gesuchten Matrizengruppe, isomorph zu $\text{SO}(n)$. Also ist die Bedingung allgemein, da� $A$ orthogonal ist.

Elementarer kann man zeigen, da�, falls $n\geq 3$ und eine nichttriviale orthogonale Zerlegung $\R^n = V_1\oplus V_2$ existiert, die von $A$ respektiert wird, Fl�chentreue gleich Orthogonalit�t ist. W�hle dazu normierte Vektoren $v_1 \in V_1,\ v_2,v_3\in V_2$. Dann mu� wegen Fl�chentreue $\|Av_1\|\|Av_2\| = 1 = \|Av_1\|\|Av_3\| $ sein, also insbesondere $\|Av_2\| = \|Av_3\|$. Also ist $\left. A\right|_{V_2}$ ein Vielfaches einer orthogonalen Matrix, also orthogonal. Also ist $A$ orthogonal.

\eintrag{Multiplikative Sequenzen, Differentiale} Seien $p_n$ Potenzsummen und $(u_n)$ eine multiplikative Sequenz symmetrischer Polynome in Variablen $\{x_i\}$, d. h. $$U(t):=\sum_{n}u_n t^n = \exp\left(\sum_i f(x_it)\right)\quad\text{mit}\quad f(t) = \sum_{k\geq 1} a_kt^k.$$
Dann ist die Wirkung des Differentialoperators $\frac{\del}{\del p_m}u_n = a_mu_{n-m} $, bzw.
$$ \text{mit}\ D(q):= \sum_{m\geq 1}\frac{\del}{\del p_m}q^m \quad\text{ist}\quad D(q)U(t) = f(qt)U(t). $$
Sei nun $V(t):=\sum_{n}v_n t^n = \exp\left(\sum_i g(x_it)\right)$ mit $g(t) = \sum_{k\geq 1} b_kt^k $ eine weitere multiplikative Sequenz. Wenn nun $a_1\neq 0\neq b_1$, so sind $f$ und $g$ durch eine lineare Transformation $L$ miteinander verwandt, d. h. es gibt Koeffizienten $\gamma_i$, so da\ss\ 
$ L\left[ f\right] := \sum_k\gamma_k t^k\frac{\del^k}{\del t^k} f = g$. Damit haben wir offensichtlich eine Formel,
$$V(t) = \exp\left(L\left[\log\left(U(t)\right)\right]\right),$$
die uns die Basen ineinander umrechnet. Mit $\overline{U}(q):= \sum_{m\geq 1}\frac{\del}{\del u_m}q^m$ folgt:
$$ \overline{U}(q)V(t)\ =\ L\left[\frac{1}{U(t)(1-qt)}\right]V(t)\ =\ L\left[\frac{\exp\left(-L^{-1}\left[\log(V(t))\right]\right)}{1-qt}\right]V(t),
$$
so da\ss\ wir auch $v_n$ nach $u_m$ ableiten k\"onnen.
       


\eintrag{Quadratsummen} Sei $Q=\{n^2 - 1\; | \; n \geq 1 \}$ die Menge der Quadratzahlen minus $1$. Die Zahlen, welche sich nicht als Summe von Elementen aus $Q$ darstellen lassen, sind $ \{1,2,4,5,7,10,13\}$.
Allgemeiner kann man definieren $Q(a):= \{n^2 - a^2\; | \; n \geq a \}$ und die Zahlen betrachten, die sich nicht als Summen von Elementen aus $Q(a)$ schreiben lassen. Sie scheinen durch $(a+7)^2$ beschr�nkt zu sein und ihre Anzahl durch $(a+3)^2$. 

\eintrag{Aufgabe von Nikulin} Sei $A$ endliche abelsche Gruppe, $H\subset A$ eine Untergruppe und $b:A\times A\longrightarrow \frac{\Q}{\Z}$ eine symmetrische Bilinearform, so da\ss\ $\left.b\right|_H$ nichtdegeneriert ist. Dann ist $A=H\oplus H^\perp$.\\
\emph{Beweis: } Da\ss\ $H\cap H^\perp = \{0\}$, ist klar. Nach einem Theorem von Wall ist $A$ die Diskriminantengruppe eines Gitters $L$  mit Bilinearform $\tilde{b}$, also $A= \frac{L^*}{L}$. Sei $\tilde{H}$ das Urbild von $H$ unter der Quotientenabbildung. Dann ist auch $\left.\tilde{b}\right|_{\tilde{H}}$ nichtdegeneriert. 
Damit ist die von $\tilde{b}$ induzierte Einbettung $\tilde{H}\otimes\Q\rightarrow \left(\tilde{H}\otimes\Q\right)^*$ wegen der endlichen Dimensionen ein Isomorphismus und wir haben 
$L^*\otimes\Q=(\tilde{H}\otimes\Q)\oplus(\tilde{H}^\perp\otimes\Q$).
Andererseits ist $\tilde{H}\subset L^*$ eine primitive Einbettung, da sonst $\left.b\right|_H$ degeneriert w�re. Damit gilt $L^*=\tilde{H}\oplus\tilde{H}^\perp$ und entsprechend auch $A=H\oplus H^\perp$.
\eintrag{Maximale Ordnung} (Danke an Fran�ois Court�s.) Sei $X$ eine (nicht unbedingt abz\"ahlbare) Menge und $\leq\; \subset X \times X$ eine partielle Ordnung. Dann existiert eine totale Ordnung $\preceq\;\subset X\times X$ mit $\leq\;\subset\; \preceq$. \\
\emph{Beweis:} (Transfinite) Induktion. Sei $\leq\; \subset\; R$ schon eine Erweiterung der Relation und $(x,y),(y,x)\notin R$. Dann enth\"alt h\"ochstens eines von $R\cup\{(x,y)\}$ und $R\cup\{(y,x)\}$ einen gerichteten Kreis. Wir machen mit einer kreisfreien Relation weiter.
\eintrag{Ein alter Hut} Sei $\vectline{x} = \vectline{x}\left(\vectline{y}\right)$ eine glatte Koordinatentransformation. Dann:
$$
\del\vectline{x} = \mathbf{M}\; \del\vectline{y}, \qquad \frac{\del}{\del\vectline{x}} = \mathbf{M}^{-\mathbf{T}} \frac{\del}{\del\vectline{y}} 
$$

\eintrag{Skalarprodukt auf symmetrischen Potenzen} Sei auf einem Vektorraum $V$ der Dimension $n$ eine symmetrische Bilinearform mit Signatur $(p,q)$ erkl\"art. Definiere eine symmetrische Bilinearform auf $\Sym^k V$ durch 
$$ 
\left( a_1\ldots a_k\,,\, b_1\ldots b_k \right):=\sum_{\mathcal{P}}\prod_{(x,y)\in\mathcal{P}}\left(x,y\right)
$$
wobei $\mathcal{P}$ alle Partitionen der Menge $\{a_1,\ldots,a_k,b_1,\ldots,b_k\}$ in Paare durchl\"auft. F\"ur den Index $i_k$ gilt dann, wie es sich geh\"ort:
$$
\sum_{k\geq 0} i_k x^k = \frac{1}{(1-x)^p(1+x)^q}
$$
Sei $d_n$ die Diskriminante der Bilinearform auf $V$ und $d_{n,k}=\det( G_k)$ diejenige der induzierten Bilinearform, $G_k$ die zugeh�rige Grammatrix. Dann gilt $$d_{n,k}=d_n^{\binom{n+k-1}{n}}\,(-1)^{\binom{n+k-1}{k}}a_{n,k}.$$ 
\"Uber die Konstanten $a_{n,k}$ denke ich:
\begin{gather*}
a_{n,1} =1,\qquad a_{n,2} = 2^{n-1}(n+2), \\
a_{1,k} = (2k-1)!!, \qquad a_{2,k} = (k!)^{k+1},\\
a_{n,k} = \left\{
\begin{array}{*2{l}p{5cm}}
 \displaystyle\prod_{i=1}^k i^{(n-1)\binom{k-i+n-1}{n-1}}\prod_{\substack{i=1 \\ i\ \text{ungerade}\\\ }}^{2k-2+n}i^{\binom{k-i+n-1}{n-1}} &\text{f�r ungerades }n, \\
 \displaystyle\prod_{i=1}^k i^{(n-1)\binom{k-i+n-1}{n-1}}\prod_{i=1}^{k-1+\frac{n}{2}} i^{\binom{k-i+n-1}{n-1} - \binom{k-2i+n-1}{n-1}} &\text{f�r gerades }n.
\end{array}
\right.
\end{gather*}
Au�erdem gilt f�r die Spur mit $f(t) = 1+t+3t^2+15t^3+105t^4+\ldots$ der erzugenden Funktion der doppelten Faktoriellen und $x_i$ den Eigenwerten der Bilinearform:
$$
\sum_k \Spur(G_k)t^k =  \prod_{i=1}^n f(x_i t), 
$$
\eintrag{\"Au\ss eres Skalarprodukt auf symmetrischen Potenzen} Alternativ kann man auch definieren:
$$ 
\left< a_1\ldots a_k\,,\, b_1\ldots b_k \right> :=\sum_{\pi\in\S_k}\prod_{i\in\{1\ldots k\}}\left<a_i,b_{\pi(i)}\right>
$$
Dann hat man kurioserweise die selbe Formel f\"ur den Index: $\hat{i}_k = i_k$. F\"ur die Konstanten $\hat{a}_{n,k}$ jedoch:
$$
\hat{a}_{n,k} = \left( \prod_{i=0}^{k-1} (k-i)!^{\binom{i+n-2}{i}} \right)^n
$$

\eintrag{Orthogonale homogene Polynome auf der Sph\"are} Sei eine Familie von Polynomen $p^{m}_n$, mit $0\leq 2n\leq m+1$ definiert durch:
\begin{align*}
p^{m}_0(x) &= 1, \qquad p^{m}_1(x) = x, \\
p^{m}_{n+1}(x) &= xp^{m}_n(x)-\frac{n(m-n+1)}{(m-2n)(m-2n+2)}p^{m}_{n-1}(x)
\end{align*}
Sei $(x)_k := x(x-1)\ldots(x-k+1)$ und $(x)_{!k}:= x(x-2)\ldots(x-2k+2)$. Dann ist 
$$p_n^m(x) = \sum_{k=0}^{\lfloor \frac{n}{2} \rfloor}(-1)^k\frac{(n)_{2k}}{(m-2n+2k)_{!k}(2k)!!} x^{n-2k}
= \sum_{\substack{l=0\\ n-l\ \text{gerade}}}^n \frac{n!\,(m-2n)!!\,(-1)^{\frac{n-l}{2}}}{l!\,(m-n-l)!!\,(n-l)!!}x^l
$$
und $p^{m}_n$ erf\"ullt die folgende Differentialgleichung:
$$ (x^2+1)xf'(x) = (nx^2-m+n-1)f(x) +(m-n+1)(m-n+2)\int_0^1f\left(\tfrac{x}{s}\right)s^{m-n+1}ds,
$$
folgende tolle Identit\"at f\"ur $n\geq 1$:
$$
\frac{d}{d\omega} \Big[\cos^{m-1}(\omega)\,p_{n-1}^{m-2}\big(\tan\omega\big) \Big]= (n-m) \cos^{m-1}(\omega)\,p_{n}^{m}\big(\tan\omega\big)
$$
und folgende Orthogonalrelation:
\begin{align*}
\int_0^\infty \int_{-\infty}^\infty y^{m-1}&p_n^m\left(\frac{x}{y}\right)p_{n'}^m\left(\frac{x}{y}\right) e^{-\frac{x^2+y^2}{2}}dxdy \ = \\  &=\  \delta_{n\,n'}\;\frac{\Gamma\left(\frac{m}{2}-n\right)\Gamma\left(\frac{m}{2}-n+1\right)\Gamma\left(n+1\right)\Gamma\left(\frac{m+1}{2}\right)}{\Gamma\left(m-n+1\right)}2^{\frac{3m-1}{2}-2n} 
\end{align*}
Seien $\alpha = (\alpha_0,\ldots,\alpha_d)$ und $\alpha' = (\alpha_0,\ldots,\alpha_{d-1})$ Multiindizes. Setze $r := \sqrt{x_0^2 +\ldots+x_{d-1}^2}$. Definiere homogene Polynome $h_\alpha$ durch $h_{(\alpha_0)}=x_0^{\alpha_0}$ und
$$ h_\alpha := p_{\alpha_d}^{2|\alpha|+d}\left(\tfrac{x_d}{r}\right)r^{\alpha_d}h_{\alpha'}. $$
Dann bilden die $h_\alpha$, f\"ur festes $|\alpha|$, eine Orthogonalbasis bzgl. des Skalarprodukts
$$ \left<f,g\right> =\frac{1}{\sqrt{2\pi}^{d+1}}\int_{\R^{d+1}}f(x)g(x)e^{-\frac{\|x\|^2}{2}}dx \propto \int_{S^d} fg ,
$$
die durch einen Gram-Schmidt-Proze\ss , angewandt auf die Monomialbasis in lexikographischer Reihenfolge, entsteht. Es gilt: 
$$\left<h_\alpha,x^\alpha\right>=\left<h_\alpha,h_\alpha\right> = 
\frac{\Gamma\left(|\alpha '|+\frac{d}{2}+1\right)\Gamma\left(|\alpha |+\frac{d+1}{2}\right)\alpha_d!}{\sqrt{\pi}\,\left(|\alpha '|+|\alpha |+d\right)!}2^{|\alpha '|+|\alpha |+d}\left<h_{\alpha '},h_{\alpha '}\right>
.$$

\eintrag{Ein paar Formeln mit Binomialkoeffizienten}
\begin{align*}
\prod_{j=0}^k (j!)^{\binom{k-j+d}{d}} =& \prod_{i=1}^k i^{\binom{k-i+d+1}{d+1}} \\
\prod_{j=0}^k (j!)^{\binom{j+b}{d}} =& \prod_{i=1}^k i^{\binom{k+b+1}{d+1}-\binom{i+b}{d+1}} = \frac{(k!)^{\binom{k+b+1}{d+1}}}{\prod_{i=1}^k i^{\binom{i+b}{d+1}}} \\
\prod_{j=0}^k (j+m)!^{\binom{j+d}{d}} =& \prod_{i=1}^{k+m}i^{ \binom{k+d+1}{k}-\binom{i+d-m}{i-m-1}}
\end{align*}

\eintrag{Operator von Goulden} Auf $\Q[p_1,p_2,\ldots]$ definiere lineare Operatoren 
\begin{align*}
q_n &:= \left\{ \begin{array}{ll} \displaystyle n\diff{p_n}, & n \leq 0, \\ p_{-n}, & n<0, \end{array}\right.
& \Delta &:= \frac{1}{2}\sum_{i,j\geq 1} \left(q_{-i-j}q_iq_j + q_{-i}q_{-j}q_{i+j} \right).
\end{align*}
Dann gilt $\lie{q_n,q_m} = n \delta_{n,-m}$, sowie
$ \displaystyle \lie{q_n,\Delta} = \frac{n}{2} \sum_{i\in\Z} \normord{q_iq_{n-i}}$ (Normalordnung ordnet Indizes aufsteigend) und $\lie{q_m,\lie{q_n,\Delta}} = nm\,q_{n+m}$. Insbesondere l\"a\ss t sich jeder Operator $q_n$ durch Kommutatoren von $q_1,\ q_{-1}$ und $\Delta$ darstellen.

\eintrag{Aus meiner Masterarbeit} Definiere einen Operator, so \"ahnlich wie Goulden, $\Psi : \bigoplus \Q[\S_n] \rightarrow \Q[p_1,p_2,\ldots]$, der eine Klassensumme $K_{\lambda}$ auf $\frac{p^{\lambda}}{z_{\lambda}}$ schickt. Dann ist
\begin{gather*}
\Psi^{-1}\left(\exp\left(\sum_{k\geq 1} \frac{a_k}{k!} \frac{p_k}{k} t^k \right)\right)\cup\Psi^{-1}\left(\exp\left(\sum_{k\geq 1} \frac{b_k}{k!} \frac{p_k}{k} t^k \right)\right)=\Psi^{-1}\left(\exp\left(\sum_{k\geq 1} \frac{c_k}{k!} \frac{p_k}{k} t^k \right)\right) \\
\text{wobei}\quad A(t) = \sum_{k\geq 1} \frac{a_k}{k!} t^k, \quad B(t)=\sum_{k\geq 1} \frac{b_k}{k!} t^k, \quad C(t)=\sum_{k\geq 1} \frac{c_k}{k!} t^k\\
\text{und} \quad t\cdot C^{-1}(t) = A^{-1}(t)B^{-1}(t).
\end{gather*}


\eintrag{Zusammenhangskoeffizienten via Charaktere} Sei $f^\nu$ die Dimension der irreduziblen Darstellung der symmetrischen Gruppe zur Partition $\nu$ und die Zusammenhangskoeffizienten, wie �blich: $\prod_{i=1}^k K_{\lambda^i} = \sum_{\mu} c^\mu_{\lambda^1\cdots\lambda^k} K_\mu$.
Dann ist 
$$
z_{\lambda^1}\ldots z_{\lambda^k}c^\mu_{\lambda^1\cdots\lambda^k} = \sum_{\nu}  \left(\frac{n!}{f^\nu}\right)^{k-1} \chi_{\lambda^1}^\nu \ldots  \chi_{\lambda^k}^\nu  \chi_{\mu}^\nu
$$

\end{document}          
