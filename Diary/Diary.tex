\documentclass[11pt]{article}
\usepackage[T1]{fontenc}
\usepackage[latin1]{inputenc}
\usepackage[german]{babel}
\usepackage{fourier}  % Use the Adobe Utopia font for the document
\usepackage{amsmath,amsthm,amssymb,amscd,color,graphicx}

% Geschwungene Kleinbuchstaben im Mathemodus (benutze \mathpzc)
\DeclareFontFamily{OT1}{pzc}{}
\DeclareFontShape{OT1}{pzc}{m}{it}{<-> s * [1.10] pzcmi7t}{}
\DeclareMathAlphabet{\mathpzc}{OT1}{pzc}{m}{it}


%Struktur
\newcommand{\eintrag}{\subparagraph*}
  
%Abk�rzungen
\newcommand{\N}{\mathbb{N}}
\newcommand{\Z}{\mathbb{Z}}
\newcommand{\Q}{\mathbb{Q}}
\newcommand{\C}{\mathbb{C}}
\newcommand{\R}{\mathbb{R}}
\newcommand{\Cinf}{C^\infty}
\newcommand{\id}{\text{id}}
\renewcommand{\O}{\mathcal{O}\!}
\renewcommand{\S}{\mathfrak{S}}
\newcommand{\T}{\mathcal{T}}
\newcommand{\E}{\mathcal{E}}
\newcommand{\A}{\mathcal{A}}
\newcommand{\Hom}{\mathcal{H}\!\!\!\mathpzc{om}}
\newcommand{\End}{\mathcal{E}\!\!\mathpzc{nd}}
\renewcommand{\d}{d\!}
\newcommand{\del}{\partial}
\newcommand{\delbar}{\overline{\partial}}
\newcommand{\dzbar}{\d\overline{z}}
\newcommand{\diff}[1]{\frac{\partial}{\partial #1}}
\newcommand{\vac}{\left|0\right>}

%\binoppenalty=7000
%\relpenalty=5000 

% Title Page
\title{Tagebuch}
\author{Simon Kapfer}

\begin{document}
\maketitle
\begin{abstract}
Was mir an mathematisch Interessantem einf�llt.
\end{abstract}
\eintrag{p-adische Approximation mit erzeugenden Funktionen} Eine Potenzreihe $F(x) = \sum c_n x^n$ konvergiert (wenn die $c_n$ nicht zu schlecht sind, also z.B. ganze Zahlen) im p-adischen Sinne in $\Z_p$ bzw. $\Q_p$ f�r alle $x$, die Vielfache von $p$ sind. F�r $x_0= a_1 p + a_2 p^2 +\ldots $ konvergieren die Koeffizienten der Reihe
$$\frac{F(a_1pt+a_2p^2t^2+\ldots)}{1-t} $$ 
gegen $F(x_0)$. Idee dazu kam von folgender Frage: Gegeben ein modulo $p^n\ \forall n$ surjektives Polynom $P$ (vgl. Hensel Lemma), wie kann man eine Folge $(r_n)$ finden, so da� $P(r_n)$ durch $p^n$ teilbar ist? Man mu� die Nullstellen von $P$ n�mlich p-adisch approximieren, z.B. mittels Newton-Verfahren, oder direkt eine Potenzreihe f�r die Wurzel nehmen.

\eintrag{Poincar�-Birkhoff-Witt-Theorem} Die universell einh�llende Algebra einer Liealgebra $\mathfrak{g}$ ist der Quotient der freien Tensoralgebra durch die Relationen $\left<a\otimes b - b\otimes a -[a,b] \right>$. Die Aussage des sog. Theorems ist, da� $U(\mathfrak{g})$ als Vektorraum von den (linear unabh�ngigen) geordneten Tensoren der $\mathfrak{g}$-Basiselemente $e_{i_1} \otimes\ldots\otimes e_{i_k},\ i_1\leq\ldots \leq i_k$ aufgespannt wird. Die lineare Unabh�ngigkeit ist der nichttriviale Teil. \\ Meine Beweisidee: $U(\mathfrak{g}) = \bigoplus U^k$ wobei $U^k = \left<e_{i_1} \otimes\ldots\otimes e_{i_k},\ i_1\leq \ldots \leq i_k \right>$ und die Erzeuger von $U^k$ sind linear unabh�ngig. Bleibt zu zeigen, da� $U^k\cap U^l = 0$ f�r verschiedene $k$ und $l$. Die Relationen der Einh�llenden vermischen nur zwei benachbarte Grade, d. h. es reicht zu zeigen, da� $U^k\cap U^{k-1} = 0$. Mit anderen Worten, die Vertauschung von zwei Basisvektoren in $U^k$ soll sich zu einer Gruppenwirkung von $\S_k$ auf $U^k\oplus U^{k-1}$ fortsetzen lassen. $S_k$ wird von Transpositionen benachbarter Elemente $\sigma_i$ erzeugt, die den Relationen $\sigma_i^2 =1 ,\ \sigma_i\sigma_j = \sigma_j\sigma_i$ falls $ |i-j|>1$ und $ \sigma_i\sigma_{i+1}\sigma_i = \sigma_{i+1}\sigma_i \sigma_{i+1}$ gehorchen. Die erste Gleichung gilt wegen Antikommutativit�t der Lieklammer, die zweite trivialerweise, die dritte braucht Jacobi. Gezeigt ist nun: egal, in welcher Reihenfolge man ein Element in $U^k$ in Normalform (durch Vertauschen) bringt, man wird stets dieselbe Korrektur in $U^{k-1}$ erhalten. Insbesondere wird Umformen eines Elements in Normalform in sich selbst keine Korrekturterme produzieren.

\eintrag{Dimension von metrischen Graphen} Ein metrischer Raum aus $n+1$ Punkten kann stets als Eckenmenge eines $n$-Simplex dargestellt werden. Dieses Simplex ist manchmal entartet, was eine Reduzierung der Dimension bedeutet. Wie weit kann man die Dimension reduzieren, wenn man f�r die Abst�nde einen Fehler von bestimmter Gr��e zul��t? M�glicherweise gehts mit einer Eigenraumzerlegung der darstellenden Matrix � la Carina.

\bibliographystyle{alpha}
\bibliography{bibl}
\end{document}          
