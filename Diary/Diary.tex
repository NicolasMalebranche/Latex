\documentclass[11pt]{article}
\usepackage[T1]{fontenc}
\usepackage[latin1]{inputenc}
\usepackage[german]{babel}
\usepackage{fourier}  % Use the Adobe Utopia font for the document
\usepackage{amsmath,amsthm,amssymb,amscd,color,graphicx,hyperref}

% Geschwungene Kleinbuchstaben im Mathemodus (benutze \mathpzc)
\DeclareFontFamily{OT1}{pzc}{}
\DeclareFontShape{OT1}{pzc}{m}{it}{<-> s * [1.10] pzcmi7t}{}
\DeclareMathAlphabet{\mathpzc}{OT1}{pzc}{m}{it}


%Struktur
\newcommand{\eintrag}{\subparagraph*}
  
%Abk�rzungen
\newcommand{\N}{\mathbb{N}}
\newcommand{\Z}{\mathbb{Z}}
\newcommand{\Q}{\mathbb{Q}}
\newcommand{\C}{\mathbb{C}}
\newcommand{\R}{\mathbb{R}}
\newcommand{\Cinf}{C^\infty}
\newcommand{\id}{\text{id}}
\renewcommand{\O}{\mathcal{O}\!}
\renewcommand{\S}{\mathfrak{S}}
\newcommand{\T}{\mathcal{T}}
\newcommand{\E}{\mathbb{E}}
\newcommand{\A}{\mathcal{A}}
\renewcommand{\a}{\mathfrak{a}}
\newcommand{\m}{\mathfrak{m}}
\newcommand{\Hom}{\mathcal{H}\!\!\!\mathpzc{om}}
\newcommand{\End}{\mathcal{E}\!\!\mathpzc{nd}}
\renewcommand{\d}{d\!}
\newcommand{\del}{\partial}
\newcommand{\delbar}{\overline{\partial}}
\newcommand{\dzbar}{\d\overline{z}}
\newcommand{\diff}[1]{\frac{\partial}{\partial #1}}
\newcommand{\vac}{\left|0\right>}
\DeclareMathOperator{\Sym}{Sym}
\newcommand{\vectline}[1]{\underline{\smash{ #1 }}} 

%\binoppenalty=7000
%\relpenalty=5000 

% Title Page
\title{Tagebuch}
\author{Simon Kapfer}

\begin{document}
\maketitle
\begin{abstract}
Was mir an mathematisch Interessantem einf�llt.
\end{abstract}
\input{From2014.tex}

\eintrag{Quadratsummen} Sei $Q=\{n^2 - 1\; | \; n \geq 1 \}$ die Menge der Quadratzahlen minus $1$. Die Zahlen, welche sich nicht als Summe von Elementen aus $Q$ darstellen lassen, sind $ \{1,2,4,5,7,10,13\}$.
Allgemeiner kann man definieren $Q(a):= \{n^2 - a^2\; | \; n \geq a \}$ und die Zahlen betrachten, die sich nicht als Summen von Elementen aus $Q(a)$ schreiben lassen. Sie scheinen durch $(a+7)^2$ beschr�nkt zu sein und ihre Anzahl durch $(a+3)^2$. 

\eintrag{Aufgabe von Nikulin} Sei $A$ endliche abelsche Gruppe, $H\subset A$ eine Untergruppe und $b:A\times A\longrightarrow \frac{\Q}{\Z}$ eine symmetrische Bilinearform, so da\ss\ $\left.b\right|_H$ nichtdegeneriert ist. Dann ist $A=H\oplus H^\perp$.\\
\emph{Beweis: } Da\ss\ $H\cap H^\perp = \{0\}$, ist klar. Nach einem Theorem von Wall ist $A$ die Diskriminantengruppe eines Gitters $L$  mit Bilinearform $\tilde{b}$, also $A= \frac{L^*}{L}$. Sei $\tilde{H}$ das Urbild von $H$ unter der Quotientenabbildung. Dann ist auch $\left.\tilde{b}\right|_{\tilde{H}}$ nichtdegeneriert. 
Damit ist die von $\tilde{b}$ induzierte Einbettung $\tilde{H}\otimes\Q\rightarrow \left(\tilde{H}\otimes\Q\right)^*$ wegen der endlichen Dimensionen ein Isomorphismus und wir haben 
$L^*\otimes\Q=(\tilde{H}\otimes\Q)\oplus(\tilde{H}^\perp\otimes\Q$).
Andererseits ist $\tilde{H}\subset L^*$ eine primitive Einbettung, da sonst $\left.b\right|_H$ degeneriert w�re. Damit gilt $L^*=\tilde{H}\oplus\tilde{H}^\perp$ und entsprechend auch $A=H\oplus H^\perp$.
\eintrag{Maximale Ordnung} (Danke an Fran�ois Courtes.) Sei $X$ eine (nicht unbedingt abz\"ahlbare) Menge und $\leq\; \subset X \times X$ eine partielle Ordnung. Dann existiert eine totale Ordnung $\preceq\;\subset X\times X$ mit $\leq\;\subset\; \preceq$. \\
\emph{Beweis:} (Transfinite) Induktion. Sei $\leq\; \subset\; R$ schon eine Erweiterung der Relation und $(x,y),(y,x)\notin R$. Dann enth\"alt h\"ochstens eines von $R\cup\{(x,y)\}$ und $R\cup\{(y,x)\}$ einen gerichteten Kreis. Wir machen mit einer kreisfreien Relation weiter.
\eintrag{Ein alter Hut} Sei $\vectline{x} = \vectline{x}\left(\vectline{y}\right)$ eine glatte Koordinatentransformation. Dann:
$$
\del\vectline{x} = \mathbf{M}\; \del\vectline{y}, \qquad \frac{\del}{\del\vectline{x}} = \mathbf{M}^{-\mathbf{T}} \frac{\del}{\del\vectline{y}} 
$$

\eintrag{Skalarprodukt auf symmetrischen Potenzen} Sei auf einem Vektorraum $V$ der Dimension $n$ eine symmetrische Bilinearform mit Signatur $(p,q)$ erkl\"art. Definiere eine symmetrische Bilinearform auf $\Sym^k V$ durch 
$$ 
\left( a_1\ldots a_k\,,\, b_1\ldots b_k \right):=\sum_{\mathcal{P}}\prod_{(x,y)\in\mathcal{P}}\left(x,y\right)
$$
wobei $\mathcal{P}$ alle Partitionen der Menge $\{a_1,\ldots,a_k,b_1,\ldots,b_k\}$ in Paare durchl\"auft. F\"ur den Index $i_k$ gilt dann, wie es sich geh\"ort:
$$
\sum_{k\geq 0} i_k x^k = \frac{1}{(1-x)^p(1+x)^q}
$$
 Sei $d_n$ die Diskriminante der Bilinearform auf $V$ und $d_{n,k}$ diejenige der induzierten Bilinearform. Dann gilt $$d_{n,k}=(-d_n)^{\binom{n+k-1}{k}}\,a_{n,k}.$$ 
\"Uber die Konstanten $a_{n,k}$ denke ich:
\begin{gather*}
a_{n,1} =1,\qquad a_{n,2} = 2^{n-1}(n+2), \\
a_{n,3} = b_n^n,\quad \text{wobei }\ \sum_n b_{n}x^n = \frac{6-9x}{(1-2x)^2}, \\
a_{1,k} = (2k-1)!!, \qquad a_{2,k} = (k!)^{k+1}
\end{gather*}
\eintrag{\"Au\ss eres Skalarprodukt auf symmetrischen Potenzen} Alternativ kann man auch definieren:
$$ 
\left< a_1\ldots a_k\,,\, b_1\ldots b_k \right> :=\sum_{\pi\in\S_k}\prod_{i\in\{1\ldots k\}}\left<a_i,b_{\pi(i)}\right>
$$
Dann hat man kurioserweise die selbe Formel f\"ur den Index: $\hat{i}_k = i_k$. F\"ur die Konstanten $\hat{a}_{n,k}$ jedoch:
$$
\hat{a}_{n,k} = \left( \prod_{i=0}^{k-1} (k-i)!^{\binom{i+n-2}{i}} \right)^n
$$

\end{document}          
