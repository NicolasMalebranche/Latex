\documentclass[a4paper,10pt]{article}
\usepackage[utf8]{inputenc}
\usepackage{a4wide, amsmath, german}
\usepackage[arrow, matrix, curve]{xy}

\newcommand{\Tor}{\mathrm{Tor}}
%opening
\title{Künneth Formel}
\author{Simon Kapfer}

\begin{document}

\maketitle

\emph{Künneth Formel} Seien $C,\ C'$ Komplexe aus Projektiven / Freien übre einem Hauptidealbereich. Dann hat man eine exakte Sequenz:
$$
0 \longrightarrow HC \otimes HC' \longrightarrow H(C\otimes C') \longrightarrow \Tor (HC,HC')[-1] \longrightarrow 0 
$$
\emph{Beweis:} Bezeichne $ZC'$ und $BC'$ den geschlossenen, bzw.~exakten Subkomplex von $C'$ (Kern und Bild vom Differential). 
Dann hat man (per Definition von $H$) eine kurze exakte Sequenz
$$
0 \longrightarrow BC' \longrightarrow ZC' \longrightarrow HC' \longrightarrow 0
$$ 
Als Unterkomplexe sind $ZC'$ und $BC'$ auch frei und damit flach. Da Tensorieren rechtsexakt ist und $\Tor(\ ,BC') = \Tor(\ ,ZC') =0$, hat man:
$$ 
0\longrightarrow \Tor(HC,HC') \longrightarrow HC\otimes BC' \longrightarrow HC\otimes ZC' \longrightarrow HC\otimes HC' \longrightarrow 0
$$
Da $H$ ein Deltafunktor ist, hat man eine lange exakte Sequenz:
$$
\longrightarrow(C\otimes ZC') \longrightarrow H(C\otimes HC') \longrightarrow H(C \otimes BC)[-1] \longrightarrow H(C \otimes ZC')[-1]\longrightarrow 
$$
Da $C$, $ZC'$ und $BC'$ Komplexe aus frei sind und damit flach, sind isomorph:
$$
H(C\otimes C') \cong H(C\otimes HC'),\quad H(C\otimes ZC') \cong HC \otimes ZC' \quad \text {und} \quad H(C\otimes BC') \cong HC\otimes BC' 
$$
Kombiniert man nun beide Sequenzen, und benutzt, daß $H(C\otimes C') = H(C\otimes HC')$, erhält man Faktorisierungen
$$
\xymatrix@C=1em{
HC \otimes BC'\ar[r]\ar@{=}[d] &  HC \otimes ZC'  \ar[r] \ar@{=}[d]    &  HC \otimes HC'  \ar@^{(->}[d] \ar[r]  & 0  &\\
HC \otimes BC'  \ar[r]         &  H(C\otimes ZC') \ar[r] \ar@{->>}[ru] &  H(C\otimes HC') \ar[r]\ar@{->>}[d] & 
H(C\otimes BC')[-1] \ar[r]\ar@{=}[d] & H(C\otimes ZC')[-1] \ar@{=}[d] \\
                               &          0\ar[r]            & \Tor(HC,HC')[-1] \ar[r]\ar@^{(->}[ru] & HC\otimes BC'[-1] \ar[r] & HC\otimes ZC'[-1]
  }
$$
Betrachte nun die mittlere vertikale Sequenz.
\end{document}
