\documentclass[a4paper,french]{scrartcl}
\usepackage[T1]{fontenc}
\usepackage[latin1]{inputenc}
\usepackage[french]{babel}
\usepackage{fourier}  % Use the Adobe Utopia font for the document
\usepackage{amsmath,amsthm,amssymb,amscd,color,graphicx,hyperref}
\usepackage[protrusion=true,expansion=true]{microtype}

\usepackage{lmodern}
\usepackage{tabto}
%Struktur
\newcommand{\point}{\vspace{3mm}\par \noindent \refstepcounter{subsection}{\bf \thesubsection.} }
	\numberwithin{equation}{subsection}
\newcommand{\tpoint}[1]{\vspace{3mm}\par \noindent \refstepcounter{subsection}{\bf \thesubsection.} 
  \numberwithin{equation}{subsection} {\em #1. ---} }
\newcommand{\epoint}[1]{\vspace{3mm}\par \noindent \refstepcounter{subsection}{\bf \thesubsection.} 
  \numberwithin{equation}{subsection} {\em #1.} }
\newcommand{\bpoint}[1]{\vspace{3mm}\par \noindent \refstepcounter{subsection}{\bf \thesubsection.} 
  \numberwithin{equation}{subsection} {\bf\em #1.} }
  
%Abk�rzungen
\newcommand{\N}{\mathbb{N}}
\newcommand{\Z}{\mathbb{Z}}
\newcommand{\Q}{\mathbb{Q}}
\newcommand{\C}{\mathbb{C}}
\renewcommand{\O}{\mathcal{O}}
\renewcommand{\S}{\mathfrak{S}}
\newcommand{\e}{\mathbf{e}}
\newcommand{\p}{\mathbf{p}}
\newcommand{\h}{\mathbf{h}}
\newcommand{\m}{\mathbf{m}}
\newcommand{\s}{\mathbf{s}}
\newcommand{\diff}[1]{\frac{\partial}{\partial #1}}
\newcommand{\End}{\mathrm{End}}
\newcommand{\Sym}{\mathrm{Sym}}
\newcommand{\Hom}{\mathrm{Hom}}
\newcommand{\sgn}{\mathrm{sgn}}
\newcommand{\Schur}{\mathbb{S}}
\newcommand{\Aut}{\mathrm{Aut}}
\newcommand{\LR}{N}
\newcommand{\Ko}{K}
\renewcommand{\i}{\mathbf{i}}
\newcommand{\ch}{\mathrm{ch}}
\newcommand{\z}{\mathrm{z}}

\renewcommand*\theenumi{\alph{enumi}}
\renewcommand{\labelenumi}{\theenumi)}
%\pagestyle{empty}

\begin{document}

\vspace*{-4em}
\begin{flushright}Universit� de Poitiers \\ S�minaire des doctorants \\ Simon Kapfer \end{flushright}

\begin{center}\Large \textbf{Exercices aux repr�sentations du groupe sym�trique} \\
A faire comme pr�paration
\end{center}
\vspace{2em}

\newbox{\mybox}
\setbox\mybox=\hbox{\textbf{Exercise 1:}}

\begin{list}{}{\labelwidth0em \leftmargin0em \itemindent0.5em \itemsep 1.3em}
\item[\textbf{Exercice 1:}] Soit $\S_n$ le groupe des permutations sur $n$ �l�ments. Lis l'article sur la d�composition des permutations en cycles sur \url{http://fr.wikipedia.org/wiki/Permutation#D.C3.A9composition_en_produit_de_cycles_.C3.A0_supports_disjoints}.
\begin{enumerate} 
\item Liste les �l�ments de $\S_3$ et $\S_4$ en termes de d�compositions en cycles.
\item Soit $\pi\in S_n$. Nous d�finissons $\chi(\pi)$ comme trace de la matrice de permutation associ�e � $\pi$. Liste  $\chi(\pi)$ pour $\pi \in \S_3$ et $\pi\in\S_4$. Comment d�terminer $\chi(\pi)$, si la d�composition en cycles est connue? 
\item Comment �num�rer les classes de conjugaison de $\S_n$?
\end{enumerate}

\item[\textbf{Exercice 2:}] Soit $G$ un groupe et $V$ un espace vectoriel complexe. Une r�pr�sentation de $G$ est un homomorphisme: $G\rightarrow \Aut(V)$. Le caract�re d'une repr�sentation $\chi_V : G \rightarrow \C$ est d�fini comme trace de la matrice correspondante. 
\begin{enumerate} 
\item V�rifie que la valeur de $\chi_V(g)$ ne depend que de la classe de conjugaison de $g$.
\item Lis \url{http://fr.wikipedia.org/wiki/Signature_d'une_permutation}. Verifie que $\sgn : \S_n \rightarrow \C^* $ est une repr�sentation.
\end{enumerate}

\item[\textbf{Exercice 3:}] \emph{Foncteurs de Schur.} Soit $V$ un espace vectoriel complexe de dimension $n$. On a une repr�sentation de $\S_2$ sur $V\otimes V$ qui �change les deux facteurs. 
\begin{enumerate}
\item Consid�re $V\otimes V$ comme espace des matrices quadratiques. V�rifie que $V\otimes V$ est isomorphe � $\Sym^2 V \oplus \Lambda^2 V$. �a donne deux sous-repr�sentations de $\S_2$. Peux-tu les d�crire?
\item On a une repr�sentation de $\S_3$ sur $V\otimes V \otimes V$ qui �change les deux facteurs. D�montre qu'il existe une d�composition de $V\otimes V \otimes V$
comme $\Sym^3 V \oplus \Lambda^3 V \oplus S$. Quelle est la dimension de $S$?
\end{enumerate} 


\end{list}


\bibliographystyle{plain}
\bibliography{bibl}
\end{document}         