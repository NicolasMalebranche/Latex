\documentclass[11pt]{article}
\usepackage[T1]{fontenc}
\usepackage[latin1]{inputenc}
\usepackage[german]{babel}
\usepackage{fourier}  % Use the Adobe Utopia font for the document
\usepackage{amsmath,amsthm,amssymb,amscd,color,graphicx}

%Struktur
\newcommand{\point}{\vspace{3mm}\par \noindent \refstepcounter{subsection}{\bf \thesubsection.} }
	\numberwithin{equation}{subsection}
\newcommand{\tpoint}[1]{\vspace{3mm}\par \noindent \refstepcounter{subsection}{\bf \thesubsection.} 
  \numberwithin{equation}{subsection} {\em #1. ---} }
\newcommand{\epoint}[1]{\vspace{3mm}\par \noindent \refstepcounter{subsection}{\bf \thesubsection.} 
  \numberwithin{equation}{subsection} {\em #1.} }
\newcommand{\bpoint}[1]{\vspace{3mm}\par \noindent \refstepcounter{subsection}{\bf \thesubsection.} 
  \numberwithin{equation}{subsection} {\bf\em #1.} }
  
%Abk�rzungen
\newcommand{\N}{\mathbb{N}}
\newcommand{\Z}{\mathbb{Z}}
\newcommand{\Q}{\mathbb{Q}}
\newcommand{\C}{\mathbb{C}}
\renewcommand{\O}{\mathcal{O}}
\renewcommand{\S}{\mathfrak{S}}
\newcommand{\e}{\mathbf{e}}
\newcommand{\p}{\mathbf{p}}
\newcommand{\h}{\mathbf{h}}
\newcommand{\m}{\mathbf{m}}
\newcommand{\s}{\mathbf{s}}
\newcommand{\diff}[1]{\frac{\partial}{\partial #1}}
\newcommand{\vac}{\left|0\right>}


% Title Page
\title{Memo: Eigenschaften symmetrischer Funktionen}
\author{Simon Kapfer}

\begin{document}
\maketitle
\begin{abstract}
Was im Stanley \cite{stanley2} dazu steht.
\end{abstract}
\section{Definitionen}
Wir arbeiten �ber $\Q[x_1,x_2,\ldots, x_n]$. Dabei ist $n$ beliebig, aber gro�. Es sollen $\lambda, \nu, \mu$ Partitionen oder auch Multi-Indizes sein. Die Darstellungen werden munter gemixt. Permutationen werden mit $\sigma, \tau \in \S_n$ bezeichnet.
\bpoint{Sonstige Bezeichnungen}
\begin{gather*}
\delta := (n-1,n-2,\ldots, 0) \\
z_\lambda := \prod i^{\lambda_i} \lambda_i!\\
K_{\lambda\mu}\quad\text{Kostka-Zahlen}
\end{gather*}

\bpoint{Monomial symmetrische Funktionen} Werden als einzige �ber $\m_\lambda = (x^\lambda)^\text{Sym}$ definiert. Die anderen alle �ber Produkte.

\bpoint{Schur-Funktionen und Determinanten}. 
\begin{gather*}a_\lambda := \sum_{\sigma\in\S_n} (-1)^{\sigma} \sigma(x^\lambda) = \det\left(x_i^{\lambda_j}\right)\quad\text{(antisymmetrisch)}\\
\s_\lambda :=\frac{a_{\lambda+\delta}}{a_\delta}\quad\text{(symmetrisch)}
\end{gather*}
Insbesondere ist $a_\delta$ die Vandermonde-Determinante.

\bpoint{Erzeugende Funktionen} 
\begin{gather*}
\sum_k{\e_kt^k} = \prod_i(1+x_it) \\
\sum_k{\h_kt^k}= \prod_i\frac{1}{1-x_it} =\exp\left(
\sum_k\tfrac{1}{k}\p_kt^k \right)
\end{gather*}

\section{Skalarprodukt und Involution}
\bpoint{Skalarprodukt} Das Skalarprodukt wird so definiert, da� gilt:
\begin{gather*}\left<\m_\lambda,\h_\mu\right> = \delta_{\lambda\mu} = \left<\s_\lambda,\s_\mu\right> \\
\left<\p_\lambda,\p_\mu\right> = \delta_{\lambda\mu} z_\lambda 
\end{gather*}
\bpoint{Adjungierte Multiplikationsperatoren} (7.15.2, und \cite{macdonald}, S. 44.)
\[\left<\s_\nu f,\s_\lambda\right>=\left<f,\s_{\lambda/\nu}\right> \]
Bezeichne mit $D(\_)$ den adjungierten Operator zur Multiplikation. Dann: 
\[ D(\p_n)\  =\ \sum_{r\geq 0}\h_r \diff{\h_{n+r}}\ =\ (-1)^{n-1} \sum_{r\geq 0}\e_r \diff{\e_{n+r}}\ =\ n\diff{\p_n}
\]
\bpoint{Involution} Definiere eine Involution $\omega$ durch 
\[\omega\, \e_\lambda = \h_\lambda. \] Dann hat $\omega$ folgende Eigenschaften:
\begin{align*}
\omega^2 &= \text{id}\\
\left<\omega f,\omega g\right> &=\left<f, g\right>  \\
\omega\,\p_\lambda &= \deg(\lambda) \p_\lambda & (7.7.5)\\
\omega\,\s_{\lambda/\nu} &= \s_{\lambda'/\nu'} & (7.15.6)
\end{align*}
\bpoint{Duale Basen} $\{\mathbf{u}_\lambda\},\ \{\mathbf{v}_\lambda\}$ seien zwei duale Basen f�r symmetrische Funktionen, d. h. 
$ \left<\mathbf{u}_\lambda ,\mathbf{v}_\nu\right> = \delta_{\lambda\nu} $.
Dann gilt:
\begin{gather*}  \sum_\lambda \mathbf{u}_\lambda(x)\mathbf{v}_\lambda(y) = \prod_{i,j}\frac{1}{1-x_iy_j} \\
\sum_\lambda \mathbf{u}_\lambda(x)\,\omega_y\mathbf{v}_\lambda(y) = \prod_{i,j}{1+x_iy_j}
\end{gather*}

\section{Beziehungen zwischen den Basen}
\bpoint{Darstellung durch $\m_\lambda$} Siehe 7.4.1, 7.5.1., 7.7.1.
\bpoint{Durch Potenzsummen} Siehe 7.7.6.
\bpoint{Durch Schur} (7.12.4, 7.15.3, 7.17.3)
\begin{gather*}
\s_\nu\h_\mu = \sum_{\lambda}K_{\lambda/\nu\,\mu}\;\s_\lambda \\
\s_\nu\e_\mu = \sum_{\lambda}K_{\lambda'/\nu'\,\mu}\;\s_\lambda \\
\s_\nu\p_\mu = \sum_{\lambda}\chi^{\lambda/\nu}(\mu)\;\s_\lambda
\end{gather*}
Die letzte Gleichung hei�t Murnagham-Nakayama Regel. $\chi$ wird in 7.17.3 definiert. Dort auch Border-Strip-Tableaus.
\bpoint{Durch Matrizen} Siehe \cite{macdonald} S. 56.
\bpoint{Jacobi-Trudy} (Stanley 7.16.1)
\[\s_{\lambda/\mu} = \det\left(\h_{\lambda_i-\mu_j+i-j}\right)
\]


\bibliographystyle{alpha}
\bibliography{bibl}
\end{document}          
