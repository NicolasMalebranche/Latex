\part{Symmetrische Funktionen}
Was im Stanley \cite{stanley2} dr�ber steht.
\section{Definitionen}
Wir arbeiten �ber $\Q[x_1,x_2,\ldots, x_n]$. Dabei ist $n$ beliebig, aber hinreichend gro�. Es sollen $\lambda, \nu, \mu$ Partitionen sein. Partitionen in Multiindex--Schreibweise werden fett $\i,\mathbf{j}$ notiert. Permutationen werden mit $\sigma, \tau \in \S_n$ bezeichnet. 
\bpoint{Sonstige Bezeichnungen}\label{notation}
\begin{gather*}
\delta := (n-1,n-2,\ldots, 0), \quad \text{$n$ ist Anzahl der Variablen} \\
\z_\lambda \; :=\; \z_\i \; :=\; \prod_k k^{i_k} i_k! \\
\Delta_\lambda(x) := \det(x_i^{\lambda_j})_{ij} \quad\text{(schiefsymmetrisch in $x_i$)}\\
\Delta(x):=\Delta_\delta(x)=\prod_{i<j}(x_i-x_j) \quad\text{Vandermonde--Determinante}\\
\Ko_{\lambda\mu}\quad\text{Kostka-Zahlen} \\
\LR_{\lambda\mu}^\nu \quad\text{Littlewood-Richardson Zahlen}
\end{gather*}

\bpoint{Standardbasen} 
\begin{itemize}
\item \textbf{Monomial} symmetrische Funktionen werden �ber $\m_\lambda = (x^\lambda)^\text{Sym}$ definiert. 
\item \textbf{Schurpolynome} sind �ber Determinanten definiert, \cite[7.15]{stanley2}:
$$\s_\lambda :=\frac{\Delta_{\lambda+\delta}}{\Delta_\delta}$$
\end{itemize}
Die anderen Basen werden �ber Produkte definiert:
$$ \e_\lambda = \prod_i \e_{\lambda_i}, \quad
 \h_\lambda = \prod_i \h_{\lambda_i}, \quad
  \p_\lambda = \prod_i \p_{\lambda_i}.
$$
\begin{itemize}
\item \textbf{Elementar--} und \textbf{vollst�ndige} symmetrische Funktionen $\e_k$ und $\h_k$:
$$\e_k = \sum_{i_1<i_2<\ldots <i_k} x_{i_1}x_{i_2}\ldots x_{i_k}, \qquad \h_k = \sum_{i_1\leq i_2\leq\ldots \leq i_k} x_{i_1}x_{i_2}\ldots x_{i_k}$$
\item \textbf{Potenzsummen} $\p_k := x_1^k + x_2^k+\ldots$
\end{itemize}
\bpoint{Erzeugende Funktionen} 
\begin{gather*}
\E(t)=\sum_{k\geq 0}{\e_kt^k} = \prod_i(1+x_it) =\exp\left(
\sum_{k\geq 1}\tfrac{-(-1)^k}{k}\p_kt^k \right)\\
\H(t)=\sum_{k\geq 0}{\h_kt^k}= \prod_i\frac{1}{1-x_it} =\exp\left(
\sum_{k\geq 1}\tfrac{1}{k}\p_kt^k \right)
\end{gather*}

\section{Skalarprodukt und Involution}
\bpoint{Skalarprodukt} Das Skalarprodukt wird so definiert, da� gilt:
\begin{gather*}\left<\m_\lambda,\h_\mu\right> = \delta_{\lambda\mu} = \left<\s_\lambda,\s_\mu\right> \\
\left<\p_\lambda,\p_\mu\right> = \delta_{\lambda\mu} \z_\lambda 
\end{gather*}
\bpoint{Adjungierte Multiplikationsperatoren} (7.15.2, und \cite{macdonald}, S. 44.)
\[\left<\s_\nu f,\s_\lambda\right>=\left<f,\s_{\lambda/\nu}\right> \]
Bezeichne mit $D(\_)$ den adjungierten Operator zur Multiplikation. Dann: 
\[ D(\p_n)\  =\ \sum_{r\geq 0}\h_r \diff{\h_{n+r}}\ =\ (-1)^{n-1} \sum_{r\geq 0}\e_r \diff{\e_{n+r}}\ =\ n\diff{\p_n}
\]
\bpoint{Differentialoperatoren} Man kann alternativ auch schreiben:
$$ 
\diff{\p_n} \H(t) = \frac{1}{k}t^n \H(t)\quad \text{und} \quad \diff{\p_n} \E(t) = (-1)^{n-1}t^n \E(t)
$$
\bpoint{Involution} Definiere eine Involution $\omega$ durch 
\[\omega\, \e_\lambda = \h_\lambda. \] Dann hat $\omega$ folgende Eigenschaften:
\begin{align*}
\omega^2 &= \text{id}\\
\left<\omega f,\omega g\right> &=\left<f, g\right>  \\
\omega\,\p_\lambda &= \deg(\lambda) \p_\lambda & (7.7.5)\\
\omega\,\s_{\lambda/\nu} &= \s_{\lambda'/\nu'} & (7.15.6)
\end{align*}
\bpoint{Duale Basen} $\{\mathbf{u}_\lambda\},\ \{\mathbf{v}_\lambda\}$ seien zwei duale Basen f�r symmetrische Funktionen, d. h. 
$ \left<\mathbf{u}_\lambda ,\mathbf{v}_\nu\right> = \delta_{\lambda\nu} $.
Dann gilt:
\begin{gather*}  \sum_\lambda \mathbf{u}_\lambda(x)\mathbf{v}_\lambda(y) = \prod_{i,j}\frac{1}{1-x_iy_j} \\
\sum_\lambda \mathbf{u}_\lambda(x)\,\omega_y\mathbf{v}_\lambda(y) = \prod_{i,j}{1+x_iy_j}
\end{gather*}

\section{Beziehungen zwischen den Basen}
\bpoint{Darstellung durch $\m_\lambda$} Siehe \cite[7.4.1, 7.5.1., 7.7.1.]{stanley2}
\begin{gather}
\s_\lambda = \sum_{\mu} K_{\lambda\mu}\m_\mu
\end{gather}
\bpoint{Durch Potenzsummen} Siehe \cite[7.7.6.]{stanley2}
\bpoint{Durch Schur} \cite[7.12.4, 7.15.3, 7.17.3]{stanley2}
\begin{gather}
\s_\nu\h_\mu = \sum_{\lambda}K_{\lambda/\nu\,\mu}\;\s_\lambda \\
\s_\nu\e_\mu = \sum_{\lambda}K_{\lambda'/\nu'\,\mu}\;\s_\lambda \\
\label{Murnagham}\s_\nu\p_\mu = \sum_{\lambda}\chi^{\lambda/\nu}(\mu)\;\s_\lambda\\
\s_\nu\s_\mu = \sum_{\lambda}C_{\nu\mu}^\lambda\s_\lambda
\end{gather}
Gleichung \ref{Murnagham} hei�t Murnagham-Nakayama Regel. $\chi$ wird in \cite[7.17.3]{stanley2} definiert. Dort auch Border-Strip-Tableaus.
\bpoint{Durch Matrizen}\label{matrizen} Siehe \cite{macdonald} S. 56. $\Ko$ ist die Matrix aus Kostka--Zahlen. $M^\top$ bedeutet Transposition, $M^{-\top}$ bedeutet Transposition plus Inversion. $J_{\lambda\mu}=\delta_{\lambda'\mu}$.
$$
\renewcommand\arraystretch{1.4}
\begin{tabular}{|c|c|c|c|c|}
\hline     & $\e$ & $\h$ & $\m$ & $\s$ \\
\hline $\e$& $1 $ & $\Ko^{\top}J\Ko^{-\top}$&$\Ko^{\top}J\Ko $&$\Ko^{\top}J$  \\ 
\hline $\h$&$\Ko^{\top}J\Ko^{-\top}$&$ 1 $&$\Ko^{\top}\Ko$&$\Ko^{\top}$\\ 
\hline $\m$&$\Ko^{-1}J\Ko^{-\top}$&$\Ko^{-1}\Ko^{-\top}$&$1$&$\Ko^{-1}$\\ 
\hline $\s$&$J\Ko^{-\top}$&$ \Ko^{-\top} $&$ \Ko $&$ 1 $\\ 
\hline 
\end{tabular} 
$$
\bpoint{Jacobi-Trudy} (Stanley 7.16.1)
\[\s_{\lambda/\mu} = \det\left(\h_{\lambda_i-\mu_j+i-j}\right)
\]

\section{Plethysmen}
\bpoint{Definition} F�r $f$ eine symmetrische Funktion ist der Plethysmus mit einer Potenzsumme definiert durch: $f[\p_n](x_1,x_2,\ldots) = f(x_1^n,x_2^n,\ldots) = \p_n[f](x_1,x_2,\ldots)$ und der Forderung, da� $f[g]$ ein Ringhomomorphismus in $f$ ist. In $g$ hat man nicht mal Linearit�t. Plethysmen haben was mit Verkettung zu tun. Man kann sich auch den Plethysmus mit  $\e_n$ bzw. $\h_n$ so vorstellen, da� man die Variablen $x_1,x_2,\ldots$ durch die Monome ersetzt, die in $\e_n,\ \h_n$ vorkommen. Das gilt nicht f�r jede beliebige symmetrische Funktion, nur wenn man einen Funktor von Darstellungen finden kann, welcher die Charaktere entsprechend transformiert, siehe auch \ref{SPleth}.
\bpoint{Plethystische Identit�ten} Siehe \cite[S. 447ff]{stanley2}.
\begin{gather}
\h_n[-\p_1] = (-1)^n \e_n \\
f[-\p_1] = (-1)^n \omega(f)\\
\sum_n \h_n[\e_1+\e_2] = \sum_{\lambda}\s_\lambda
\end{gather}

\section{Hopfalgebren}
Der Ring der symmetrischen Funktionen tr�gt mehrere Hopfalgebren-Strukturen.
\bpoint{Klassische Hopfalgebrenstruktur} \cite{hopfSym} Mit den Setzungen
$$ \Delta(\e_n) := \sum \e_i \otimes \e_{n-i}, \qquad \varepsilon(\e_n) := \delta_{0,n} $$ 
und der bekannten Involution als Antipode. Man hat f�r die Komultiplikation:
$$ \Delta(\h_n)= \sum \h_i \otimes \h_{n-i}, \quad \Delta(\s_\mu) = \sum \s_\lambda\otimes \s_{\mu/\lambda}, \quad
\Delta(\p_n)=1\otimes \p_n + \p_n\otimes 1  
$$
Die $\p_n$ spannen die primitiven Elemente der Komultiplikation. S. \cite[Kap. 10]{hazewinkel}.
Au�erdem hat man die Adjunktionsformel $\left<x\otimes y, \Delta(z)\right> = \left<x y, z\right>$.
\bpoint{Produkt--Bialgebra} \cite[1]{hazewinkel}
Mit der Setzung 
$$ \Delta(\p_n) := \p_n\otimes\p_n,\ \ \text{bzw.} \ \  \Delta(\h_n)=\sum_{\|\lambda\|=n} \h_\lambda\otimes\m_\lambda=\sum_{\|\lambda\|=n} \s_\lambda\otimes\s_\lambda
$$
ist eine andere Bialgebren-Struktur erkl�rt.

\bpoint{Fa� di Bruno Algebra} \cite{bultel} Auch die Setzung
$$
\Delta(\h_n):=\sum_k \h_k\otimes\h_{n-k}\left[(1+k)\p_1\right], \quad \varepsilon(\h_n) := \delta_{0,n}, \quad \psi(\h_n) := \frac{\h_n\left[-(1+n)\p_n\right]}{1+n}
$$
definiert eine Hopfalgebra.
