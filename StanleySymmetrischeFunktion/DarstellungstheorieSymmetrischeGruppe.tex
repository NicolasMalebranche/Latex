\documentclass[11pt]{article}
\usepackage[T1]{fontenc}
\usepackage[latin1]{inputenc}
\usepackage[german]{babel}
\usepackage{fourier}  % Use the Adobe Utopia font for the document
\usepackage{amsmath,amsthm,amssymb,amscd,color,graphicx,hyperref}
%Struktur
\newcommand{\point}{\vspace{3mm}\par \noindent \refstepcounter{subsection}{\bf \thesubsection.} }
	\numberwithin{equation}{subsection}
\newcommand{\tpoint}[1]{\vspace{3mm}\par \noindent \refstepcounter{subsection}{\bf \thesubsection.} 
  \numberwithin{equation}{subsection} {\em #1. ---} }
\newcommand{\epoint}[1]{\vspace{3mm}\par \noindent \refstepcounter{subsection}{\bf \thesubsection.} 
  \numberwithin{equation}{subsection} {\em #1.} }
\newcommand{\bpoint}[1]{\vspace{3mm}\par \noindent \refstepcounter{subsection}{\bf \thesubsection.} 
  \numberwithin{equation}{subsection} {\bf\em #1.} }
  
%Abk�rzungen
\newcommand{\N}{\mathbb{N}}
\newcommand{\Z}{\mathbb{Z}}
\newcommand{\Q}{\mathbb{Q}}
\newcommand{\C}{\mathbb{C}}
\renewcommand{\O}{\mathcal{O}}
\renewcommand{\S}{\mathfrak{S}}
\newcommand{\e}{\mathbf{e}}
\newcommand{\p}{\mathbf{p}}
\newcommand{\h}{\mathbf{h}}
\newcommand{\m}{\mathbf{m}}
\newcommand{\s}{\mathbf{s}}
\newcommand{\diff}[1]{\frac{\partial}{\partial #1}}
\newcommand{\End}{\mathrm{End}}
\newcommand{\Sym}{\mathrm{Sym}}
\newcommand{\Hom}{\mathrm{Hom}}
\newcommand{\sgn}{\mathrm{sgn}}
\newcommand{\Schur}{\mathbb{S}}
\newcommand{\LR}{N}
\newcommand{\Ko}{K}
\renewcommand{\i}{\mathbf{i}}
\newcommand{\ch}{\mathrm{ch}}
\newcommand{\z}{\mathrm{z}}

% Title Page
\title{Memo: Symmetrische Darstellungstheorie}
\author{Simon Kapfer}

\begin{document}
\maketitle
\part{Symmetrische Funktionen}
Was im Stanley \cite{stanley2} dr�ber steht.
\section{Definitionen}
Wir arbeiten �ber $\Q[x_1,x_2,\ldots, x_n]$. Dabei ist $n$ beliebig, aber hinreichend gro�. Es sollen $\lambda, \nu, \mu$ Partitionen sein. Partitionen in Multiindex--Schreibweise werden fett $\i,\mathbf{j}$ notiert. Permutationen werden mit $\sigma, \tau \in \S_n$ bezeichnet. 
\bpoint{Sonstige Bezeichnungen}\label{notation}
\begin{gather*}
\delta := (n-1,n-2,\ldots, 0), \quad \text{$n$ ist Anzahl der Variablen} \\
\z_\lambda \; :=\; \z_\i \; :=\; \prod_k k^{i_k} i_k! \\
\Delta_\lambda(x) := \det(x_i^{\lambda_j})_{ij} \quad\text{(schiefsymmetrisch in $x_i$)}\\
\Delta(x):=\Delta_\delta(x)=\prod_{i<j}(x_i-x_j) \quad\text{Vandermonde--Determinante}\\
\Ko_{\lambda\mu}\quad\text{Kostka-Zahlen} \\
\LR_{\lambda\mu}^\nu \quad\text{Littlewood-Richardson Zahlen}
\end{gather*}

\bpoint{Standardbasen} 
\begin{itemize}
\item \textbf{Monomial} symmetrische Funktionen werden �ber $\m_\lambda = (x^\lambda)^\text{Sym}$ definiert. 
\item \textbf{Schurpolynome} sind �ber Determinanten definiert, \cite[7.15]{stanley2}:
$$\s_\lambda :=\frac{\Delta_{\lambda+\delta}}{\Delta_\delta}$$
\end{itemize}
Die anderen Basen werden �ber Produkte definiert:
$$ \e_\lambda = \prod_i \e_{\lambda_i}, \quad
 \h_\lambda = \prod_i \h_{\lambda_i}, \quad
  \p_\lambda = \prod_i \p_{\lambda_i}.
$$
\begin{itemize}
\item \textbf{Elementar--} und \textbf{vollst�ndige} symmetrische Funktionen $\e_k$ und $\h_k$:
$$\e_k = \sum_{i_1<i_2<\ldots <i_k} x_{i_1}x_{i_2}\ldots x_{i_k}, \qquad \h_k = \sum_{i_1\leq i_2\leq\ldots \leq i_k} x_{i_1}x_{i_2}\ldots x_{i_k}$$
\item \textbf{Potenzsummen} $\p_k := x_1^k + x_2^k+\ldots$
\end{itemize}
\bpoint{Erzeugende Funktionen} 
\begin{gather*}
\E(t)=\sum_{k\geq 0}{\e_kt^k} = \prod_i(1+x_it) =\exp\left(
\sum_{k\geq 1}\tfrac{-(-1)^k}{k}\p_kt^k \right)\\
\H(t)=\sum_{k\geq 0}{\h_kt^k}= \prod_i\frac{1}{1-x_it} =\exp\left(
\sum_{k\geq 1}\tfrac{1}{k}\p_kt^k \right)
\end{gather*}

\section{Skalarprodukt und Involution}
\bpoint{Skalarprodukt} Das Skalarprodukt wird so definiert, da� gilt:
\begin{gather*}\left<\m_\lambda,\h_\mu\right> = \delta_{\lambda\mu} = \left<\s_\lambda,\s_\mu\right> \\
\left<\p_\lambda,\p_\mu\right> = \delta_{\lambda\mu} \z_\lambda 
\end{gather*}
\bpoint{Adjungierte Multiplikationsperatoren} (7.15.2, und \cite{macdonald}, S. 44.)
\[\left<\s_\nu f,\s_\lambda\right>=\left<f,\s_{\lambda/\nu}\right> \]
Bezeichne mit $D(\_)$ den adjungierten Operator zur Multiplikation. Dann: 
\[ D(\p_n)\  =\ \sum_{r\geq 0}\h_r \diff{\h_{n+r}}\ =\ (-1)^{n-1} \sum_{r\geq 0}\e_r \diff{\e_{n+r}}\ =\ n\diff{\p_n}
\]
\bpoint{Differentialoperatoren} Man kann alternativ auch schreiben:
$$ 
\diff{\p_n} \H(t) = \frac{1}{k}t^n \H(t)\quad \text{und} \quad \diff{\p_n} \E(t) = (-1)^{n-1}t^n \E(t)
$$
\bpoint{Involution} Definiere eine Involution $\omega$ durch 
\[\omega\, \e_\lambda = \h_\lambda. \] Dann hat $\omega$ folgende Eigenschaften:
\begin{align*}
\omega^2 &= \text{id}\\
\left<\omega f,\omega g\right> &=\left<f, g\right>  \\
\omega\,\p_\lambda &= \deg(\lambda) \p_\lambda & (7.7.5)\\
\omega\,\s_{\lambda/\nu} &= \s_{\lambda'/\nu'} & (7.15.6)
\end{align*}
\bpoint{Duale Basen} $\{\mathbf{u}_\lambda\},\ \{\mathbf{v}_\lambda\}$ seien zwei duale Basen f�r symmetrische Funktionen, d. h. 
$ \left<\mathbf{u}_\lambda ,\mathbf{v}_\nu\right> = \delta_{\lambda\nu} $.
Dann gilt:
\begin{gather*}  \sum_\lambda \mathbf{u}_\lambda(x)\mathbf{v}_\lambda(y) = \prod_{i,j}\frac{1}{1-x_iy_j} \\
\sum_\lambda \mathbf{u}_\lambda(x)\,\omega_y\mathbf{v}_\lambda(y) = \prod_{i,j}{1+x_iy_j}
\end{gather*}

\section{Beziehungen zwischen den Basen}
\bpoint{Darstellung durch $\m_\lambda$} Siehe \cite[7.4.1, 7.5.1., 7.7.1.]{stanley2}
\begin{gather}
\s_\lambda = \sum_{\mu} K_{\lambda\mu}\m_\mu
\end{gather}
\bpoint{Durch Potenzsummen} Siehe \cite[7.7.6.]{stanley2}
\bpoint{Durch Schur} \cite[7.12.4, 7.15.3, 7.17.3]{stanley2}
\begin{gather}
\s_\nu\h_\mu = \sum_{\lambda}K_{\lambda/\nu\,\mu}\;\s_\lambda \\
\s_\nu\e_\mu = \sum_{\lambda}K_{\lambda'/\nu'\,\mu}\;\s_\lambda \\
\label{Murnagham}\s_\nu\p_\mu = \sum_{\lambda}\chi^{\lambda/\nu}(\mu)\;\s_\lambda\\
\s_\nu\s_\mu = \sum_{\lambda}C_{\nu\mu}^\lambda\s_\lambda
\end{gather}
Gleichung \ref{Murnagham} hei�t Murnagham-Nakayama Regel. $\chi$ wird in \cite[7.17.3]{stanley2} definiert. Dort auch Border-Strip-Tableaus.
\bpoint{Durch Matrizen}\label{matrizen} Siehe \cite{macdonald} S. 56. $\Ko$ ist die Matrix aus Kostka--Zahlen. $M^\top$ bedeutet Transposition, $M^{-\top}$ bedeutet Transposition plus Inversion. $J_{\lambda\mu}=\delta_{\lambda'\mu}$.
$$
\renewcommand\arraystretch{1.4}
\begin{tabular}{|c|c|c|c|c|}
\hline     & $\e$ & $\h$ & $\m$ & $\s$ \\
\hline $\e$& $1 $ & $\Ko^{\top}J\Ko^{-\top}$&$\Ko^{\top}J\Ko $&$\Ko^{\top}J$  \\ 
\hline $\h$&$\Ko^{\top}J\Ko^{-\top}$&$ 1 $&$\Ko^{\top}\Ko$&$\Ko^{\top}$\\ 
\hline $\m$&$\Ko^{-1}J\Ko^{-\top}$&$\Ko^{-1}\Ko^{-\top}$&$1$&$\Ko^{-1}$\\ 
\hline $\s$&$J\Ko^{-\top}$&$ \Ko^{-\top} $&$ \Ko $&$ 1 $\\ 
\hline 
\end{tabular} 
$$
\bpoint{Jacobi-Trudy} (Stanley 7.16.1)
\[\s_{\lambda/\mu} = \det\left(\h_{\lambda_i-\mu_j+i-j}\right)
\]

\section{Plethysmen}
\bpoint{Definition} F�r $f$ eine symmetrische Funktion ist der Plethysmus mit einer Potenzsumme definiert durch: $f[\p_n](x_1,x_2,\ldots) = f(x_1^n,x_2^n,\ldots) = \p_n[f](x_1,x_2,\ldots)$ und der Forderung, da� $f[g]$ ein Ringhomomorphismus in $f$ ist. In $g$ hat man nicht mal Linearit�t. Plethysmen haben was mit Verkettung zu tun. Man kann sich auch den Plethysmus mit  $\e_n$ bzw. $\h_n$ so vorstellen, da� man die Variablen $x_1,x_2,\ldots$ durch die Monome ersetzt, die in $\e_n,\ \h_n$ vorkommen. Das gilt nicht f�r jede beliebige symmetrische Funktion, nur wenn man einen Funktor von Darstellungen finden kann, welcher die Charaktere entsprechend transformiert, siehe auch \ref{SPleth}.
\bpoint{Plethystische Identit�ten} Siehe \cite[S. 447ff]{stanley2}.
\begin{gather}
\h_n[-\p_1] = (-1)^n \e_n \\
f[-\p_1] = (-1)^n \omega(f)\\
\sum_n \h_n[\e_1+\e_2] = \sum_{\lambda}\s_\lambda
\end{gather}

\section{Hopfalgebren}
Der Ring der symmetrischen Funktionen tr�gt mehrere Hopfalgebren-Strukturen.
\bpoint{Klassische Hopfalgebrenstruktur} \cite{hopfSym} Mit den Setzungen
$$ \Delta(\e_n) := \sum \e_i \otimes \e_{n-i}, \qquad \varepsilon(\e_n) := \delta_{0,n} $$ 
und der bekannten Involution als Antipode. Man hat f�r die Komultiplikation:
$$ \Delta(\h_n)= \sum \h_i \otimes \h_{n-i}, \quad \Delta(\s_\mu) = \sum \s_\lambda\otimes \s_{\mu/\lambda}, \quad
\Delta(\p_n)=1\otimes \p_n + \p_n\otimes 1  
$$
Die $\p_n$ spannen die primitiven Elemente der Komultiplikation. S. \cite[Kap. 10]{hazewinkel}.
Au�erdem hat man die Adjunktionsformel $\left<x\otimes y, \Delta(z)\right> = \left<x y, z\right>$.
\bpoint{Produkt--Bialgebra} \cite[1]{hazewinkel}
Mit der Setzung 
$$ \Delta(\p_n) := \p_n\otimes\p_n,\ \ \text{bzw.} \ \  \Delta(\h_n)=\sum_{\|\lambda\|=n} \h_\lambda\otimes\m_\lambda=\sum_{\|\lambda\|=n} \s_\lambda\otimes\s_\lambda
$$
ist eine andere Bialgebren-Struktur erkl�rt.

\bpoint{Fa� di Bruno Algebra} \cite{bultel} Auch die Setzung
$$
\Delta(\h_n):=\sum_k \h_k\otimes\h_{n-k}\left[(1+k)\p_1\right], \quad \varepsilon(\h_n) := \delta_{0,n}, \quad \psi(\h_n) := \frac{\h_n\left[-(1+n)\p_n\right]}{1+n}
$$
definiert eine Hopfalgebra.

\part{Darstellungstheorie}
Orientiert sich an Fulton--Harris, \cite{fh}. Im Stanley, \cite[7.18, 7.A2]{stanley2}, steht auch was.
\section{Allgemeines �ber Charaktere}
Folgendes gilt f�r beliebige (endliche) Gruppen $G$, welche auf $\C$--Vektorr�umen wirken.
\bpoint{Definition} Eine Darstellung ist ein Algebrenhomomorphismus: $\C[G] \longrightarrow \End{V}$. Der Charakter $\chi_V$ ist die Verkettung der Darstellung mit der Spurbildung, also eine lineare Abbildung: $\C[G]\longrightarrow \C$.
\\
Charaktere sind Klassenfunktionen, d.~h.~ der Wert des Charakters h�ngt nur von der Konjugationsklasse ab. Zwei Darstellungen sind gleich, falls ihre Charaktere gleich sind.
\bpoint{Rechenregeln f�r Charaktere} F�r die induzierten Darstellungen gilt:
\begin{gather}
\chi_{V\oplus W} = \chi_V +\chi_W\\
\chi_{V\otimes W} = \chi_V\chi_W \\
\chi_{\Hom(V,W)} = \overline{\chi_V}\chi_W \\
\sum_{k\geq 0} \chi_{\Sym^kV}(g)t^k = \exp\left(\sum_{j\geq 1}\tfrac{1}{j}\chi_V(g^j)t^j\right)\\
\sum_{k\geq 0} \chi_{\Lambda^kV}(g)t^k = \exp\left(\sum_{j\geq 1}\tfrac{(-1)^{j+1}}{j}\chi_V(g^j)t^j\right)
\end{gather}
Man beweist das mit Potenzsummen, vollst�ndigen und elementarsymmetrischen Polynomen in den Eigenwerten der darstellenden Matrizen.
\bpoint{Komposition von Darstellungen} Wirkt eine Gruppe $G$ auf $V$ und die Gruppe $\mathrm{GL}(V)$ auf $W$, so gilt f�r die induzierte Wirkung auf $W$ (siehe \cite[S. 448]{stanley2}):
$$ \chi_{G,W} = \chi_{\mathrm{GL}(V), W }[\chi_{G,V}]$$
\bpoint{Ring der Darstellungen} Durch Hinzuf�gen formaler additiver Inverser werden die Darstellungen einer festen Gruppe mit $\oplus,\;\otimes$ ein Ring mit der trivialen Darstellung als 1. Die Abbildungen $\chi_V(g)\mapsto\chi_V(g^k)$ sind Ringhomomorphismen und hei�en auch Adams--Operationen.
\bpoint{Irreduzible Darstellungen} Es gibt genauso viele irreduzible Darstellungen wie Konjugationsklassen. Die Charaktere der irreduziblen Darstellungen bilden eine Orthonormalbasis der Klassenfunktionen bez�glich des Skalarprodukts:
$$ \left<\alpha,\beta\right> := \frac{1}{|G|}\sum_{g\in G} \overline{\alpha(g)}\beta(g) $$
Jede Darstellung von $G$ zerf�llt in eine direkte Summe von irreduziblen. \cite[2.13]{fh}
\bpoint{Darstellungen, die Namen haben und Konstruktionen} \begin{itemize}
\item Die \textbf{triviale} Darstellung: Eindimensional, irreduzibel, jedes Gruppenelement wirkt wie die Identit�t. $V^G$ ist isomorph zu einer direkten Summe trivialer Darstellungen.
\item Die \textbf{regul�re} Darstellung ist $\C[G]$ mit Linksmultiplikation. Jede irreduzible Darstellung taucht in der Zerlegung der regul�ren mit einer Vielfachheit auf, die gleich ihrer Dimension ist.
\item Wirkt $H$ auf $W$ und $G$ auf $V$, so wirkt $H\times G$ auf $W\otimes V$. Diese Konstruktion hei�t �u�eres Tensorprodukt und wird $W \boxtimes V$ geschrieben.
\item Sei $H\leq G$ eine Untergruppe, $W$ eine Darstellung von $H$. Die \textbf{induzierte} Darstellung von $G$ ist 
\begin{equation}
\mathrm{Ind}_H^G W = \bigoplus_{\gamma \in G/H} \gamma\cdot W
\end{equation} und ist adjungiert zur Einschr�nkung der Darstellung bez�glich $\Hom_G,\Hom_H$, sowie des Skalarprodukts von Klassenfunktionen im Sinne von \cite[3.20]{fh}:
\begin{equation}\left<\chi_{\mathrm{Ind}_H^G W }\,,\,\chi_V\right>\;=\;
\left<\chi_W\,,\,\chi_{\mathrm{Res}_H^G V }\right> \end{equation}
\end{itemize}
\bpoint{Invarianten} Der Mittelungs--Operator $$\frac{1}{|G|}\sum_{g\in G} g$$ projiziert $V$ auf $V^G$, den Teil der unter $G$--Wirkung invariant bleibt. Das entspricht dem Summand, der von der trivialen (irreduziblen) Darstellung kommt.

\section{Irreduzible Darstellungen von $\S_d$}
\bpoint{Young--Symmetrisierer} Die Projektion auf die irreduzible Darstellung, die einer (beliebig numerierten) Partition $\lambda$ zugeordnet wird, lautet: $c_\lambda = a_\lambda b_\lambda \in \C[\S_d]$ wobei
$a_\lambda = \sum \sigma$ und $b_\lambda = \sum \sgn(\sigma)\sigma$. Die erste Summe durchl�uft die Permutationen, die die Reihen, die zweite die, die die Spalten auf sich abbilden. \\
Wenn man die Reihenfolge von $a_\lambda$ und $b_\lambda$ vertauscht, so erh�lt man f�r jeden Summanden sein Inverses (Antipode).
\bpoint{Frobenius Abbildung} \cite[S. 351]{stanley2} Bilde Klassenfunktionen auf symmetrische Polynome ab durch: 
\begin{gather*}
\ch:\; \bigoplus\text{CF}(\S_n) \longrightarrow \Lambda \\\
\ch f = \frac{1}{n!} \sum_{\sigma\in\S_n} f(\sigma) \,\p_{\lambda(\sigma)} = \sum_{\lambda}\frac{1}{\z_\lambda}f(\lambda)\,\p_\lambda
\end{gather*}
Diese Abbildung ist linear und bez�glich der Skalarprodukte eine Isometrie. F�r Darstellungen $V,\;W$ von $\S_n,\;\S_m$ definiert  $(V,W)\mapsto \mathrm{Ind}_{\S_n\times\S_m}^{\S_{n+m}}(V\boxtimes W)$ eine Multiplikation auf $\bigoplus \text{CF}(\S_n)$, bez�glich der $\ch$ ein Isomorphismus von Ringen wird.
\bpoint{Frobenius Formel} Bezeichnen $\lambda=(\lambda_1\geq\lambda_2\geq\ldots)$ und $\mathbf{i}$ Partitionen, wobei $\mathbf{i}$ durch Multiplizit�ten gegeben ist. Dann gilt mit den Bezeichnungen aus \ref{notation}:
$$ \p_\mathbf{i}=\sum_\lambda\chi_\lambda(C_\mathbf{i})\,\s_\lambda 
\quad\text{bzw.}\quad \chi_\lambda(C_\mathbf{i})\ =\  \left[x^{\lambda+\delta}\right]\ \Delta(x)\p_\mathbf{i}(x)$$
\bpoint{Hakenl�ngenformel} F�r die irreduzible Darstellung zur Partition $\lambda$ gilt:
$$\dim V_\lambda = \frac{d!}{\prod(\text{L�ngen der Haken})}$$
\bpoint{Standard--Darstellung} $\C^d = 1 + V_{(d-1,1)}$, direkte Summe aus trivialer Darstellung und sog.~Standarddarstellung. Die �u�ere Potenz $\Lambda^kV_{(d-1,1)} = V_{(d-k,1,1,\ldots)}$ ergibt die Darstellung, die zu einem Haken geh�rt. \cite[4.6]{fh} \\
Siehe auch \cite{marin}.
\bpoint{Regel von Young} Seien $U_\lambda = \C[\S_d]a_\lambda,\ V_\lambda =\C[\S_d]c_\lambda$. Dann ist $V_\lambda$ irreduzibel und (vergleiche mit \ref{matrizen}): 
$$U_\lambda = \sum_{\mu}\Ko_{\mu\lambda} V_\mu$$

\section{Schur--Funktoren}
\bpoint{Definition} Auf $V^{\otimes d}$ wirkt $\S_d$ durch Vertauschung der Faktoren. Dann ist der Schurfunktor definiert durch $\Schur^\lambda V:= c_\lambda (V^{\otimes d}) $. Insbesondere also: $\Schur^{(1^d)} = \Lambda^d$ und $\Schur^{(d)} = \Sym^d$. Man hat:
$$ V^{\otimes d} = \bigoplus\; (\Schur^\lambda V)^{\oplus \dim V_\lambda}$$
\bpoint{Link zu symmetrischen Funktionen} Sei $G$ Gruppe, die auf $V$ wirkt. Dann ist $\chi_{\Schur^\lambda V} $ das Schurpolynom $s_\lambda$ in den Eigenwerten der korrespondierenden Matrix. Insbesondere hat man die Formeln:
\begin{gather}
\dim\Schur^\lambda V = s_\lambda(1,1,\ldots) = \prod_{1\leq i<j\leq\dim V} \frac{\lambda_i-\lambda_j + j-i}{j-i} \\
\Schur^\lambda V\otimes \Schur^\mu V = \bigoplus_\nu \LR_{\lambda\mu}^\nu\Schur^\nu V
\end{gather} 
\bpoint{Weitere Analogien} Die Funktoren
\begin{align*}
V \longmapsto a_\lambda(V^{\otimes d}) =& \Sym^{\lambda_1}V\otimes\Sym^{\lambda_2}V\otimes\ldots \\
V \longmapsto b_{\lambda'}(V^{\otimes d}) =& \Lambda^{\lambda_1}V\otimes\Lambda^{\lambda_2}V\otimes\ldots
\end{align*}
verhalten sich wie die vollst�ndigen und elementarsymmetrischen Polynome:
\begin{gather}
\Lambda^{\lambda_1}V\otimes \ldots\otimes\Lambda^{\lambda_r}V = \bigoplus\; \Ko_{\mu\lambda}\Schur^{\mu '}V
\end{gather}
und die analogen Identit�ten gelten auch.
\bpoint{Plethysmen} \label{SPleth}Es gilt (?): Die Charaktere von $\Schur^\lambda(\Schur^{\mu}V)$ sind als symmetrische Funktionen in den Eigenwerten gleich $\s_\lambda[\s_\mu]$.
\bpoint{Andere Rechenregeln} Siehe \cite[S. 79ff]{fh}. F�r das �u�ere Produkt von Darstellungen hat man:
\begin{gather}
\Schur^\nu(V\oplus W) = \bigoplus\; \LR_{\lambda\mu}^\nu \left(\Schur^\lambda V \boxtimes \Schur^\mu W\right) \\
\Schur^\nu(V\boxtimes W) = \bigoplus\; C_{\lambda\mu\nu} \left(\Schur^\lambda V \boxtimes \Schur^\mu W\right)\\
\Sym^d(V\boxtimes W) = \bigoplus_{\lambda\vdash d} \Schur^\lambda V\boxtimes\Schur^\lambda W 
\end{gather}

\bibliographystyle{plain}
\bibliography{bibl}
\end{document}          
