\documentclass{amsart}

\usepackage{amsmath,amssymb,amsfonts}
\usepackage[all]{xy}
\usepackage{appendix,listings,hyperref}

\DeclareMathOperator{\rank}{rk}
\DeclareMathOperator{\trace}{tr}
\DeclareMathOperator{\Aut}{Aut}
\DeclareMathOperator{\End}{End}
\DeclareMathOperator{\id}{id}
\DeclareMathOperator{\Hom}{Hom}
\DeclareMathOperator{\Sym}{Sym}
\DeclareMathOperator{\Hilb}{Hilb}

\newcommand{\hilb}[1]{^{[#1]}}
\newcommand{\ie}{{\it i.e. }}
\newcommand{\eg}{{\it e.g. }}
\newcommand{\loccit}{{\it loc. cit. }}
\newcommand{\vac}{|0\rangle}
\newcommand{\odd}{{\rm{odd}}}
\newcommand{\even}{{\rm{even}}}
\newcommand{\tors}{{\rm{tors}}}

\newcommand{\p}[2]{p_{#1}^{#2}\;\!\!}
\renewcommand{\L}{\mathcal{L}}

\newcommand{\coloneqq}{:=}
\newcommand{\bra}{\left<\!\!\!\:\left<}
\newcommand{\ket}{\right>\!\!\!\:\right>}


%%%%%%%%%%%%%%%%%%%%%%%%%%%%%%

\newcommand{\C}{\mathbb{C}}
\newcommand{\R}{\mathbb{R}}
\newcommand{\Q}{\mathbb{Q}}
\newcommand{\Z}{\mathbb{Z}}
\renewcommand{\S}{\mathbb{S}}
\renewcommand{\H}{\mathbb{H}}

%%%%%%%%%%%%%%%%%%%%%%%%%%%%%

\newcommand{\kS}{\mathfrak{S}}

%%%%%%%%%%%%%%%%%%%%%%%%%%%%%%

\newcommand{\lra}{\longrightarrow}
\newcommand{\ra}{\rightarrow}

%%%%%%%%%%%%%%%%%%%%%%%%%%%%%

\theoremstyle{plain}
\newtheorem{theorem}{Theorem}[section]
\newtheorem{lemma}[theorem]{Lemma}
\newtheorem{proposition}[theorem]{Proposition}
\newtheorem{corollary}[theorem]{Corollary}
\theoremstyle{definition}
\newtheorem{definition}[theorem]{Definition}
\newtheorem{notation}[theorem]{Notation}
\theoremstyle{remark}
\newtheorem{remark}[theorem]{Remark}
\newtheorem{example}[theorem]{Example}


%%%%%%%%%%%%%%%%%%%%%%%%%%%%%

\begin{document}

\title[Symmetric Powers, Hom.~Orth.~Polynomials, Hyperk\"ahlers]{Symmetric Powers of Symmetric Bilinear Forms, Homogeneous Orthogonal Polynomials on the Sphere and an Application in Compact Hyperk\"ahler Manifolds}


\author{Simon Kapfer}
\address{Simon Kapfer, Laboratoire de Math\'ematiques et Applications, UMR CNRS 6086, Universit\'e de Poitiers, T\'el\'eport 2, Boulevard Marie et Pierre Curie, F-86962 Futuroscope Chasseneuil}
\email{simon.kapfer@math.univ-poitiers.fr}
%\urladdr{http://www.math.uni-augsburg.de/alg/}


\date{\today}

%\keywords{}

\begin{abstract} 
We do technical stuff!
\end{abstract}

\maketitle

%%%%%%%%%%%%%%%%%%%%%%%%%%%%%%%%%%%%%%%%%%%%%%%%%%%%%%%%%%%%%%%%%%
%%%%%%%%%%%%%%%%%%%%%%%%%%%%%%%%%%%%%%%%%%%%%%%%%%%%%%%%%%%%%%%%%%
%%%%%%%%%%%%%%%%%%%%%%%%%%%%%%%%%%%%%%%%%%%%%%%%%%%%%%%%%%%%%%%%%%

\section{Symmetric Bilinear Forms on Symmetric Powers}
Let $V$ be a vector space (or, more generally, a free module) over $K$ of rank $d$ with basis $\{e_0,\ldots,e_{d-1}\}$, equipped with a symmetric bilinear form $\left<\,\ ,\ \right>: V\times V \rightarrow K$. There are
are at least two possibilities to define an induced bilinear form on the symmetric product $\Sym^kV$. The first one is to define on the induced basis of $\Sym^kV$
\begin{equation}\label{Garr}
\left< e_{i_1}\ldots e_{i_k}, e_{j_1}\ldots e_{i_k}\right>_k \coloneqq \sum_\sigma  
\prod_{n=1}^k \left< e_{i_n},e_{j_{\sigma(n)}}\right>,
\end{equation}
the sum being over all permutations of $\{1,\ldots,k\}$, as studied by McGarraghy in \cite{McGarr}. However, we will \emph{not} consider 
this construction. Instead, we make the following
\begin{definition} On the basis $\{e_{i_1}\ldots e_{i_k}\;|\;0\leq i_1\leq\ldots\leq i_k<d\}$ of $\Sym^kV$, we define a symmetric bilinear form $\bra\ \,,\ \ket_k$ by: 
\begin{equation}
\label{mydef}
\bra e_{i_1}\ldots e_{i_k}\,,\,e_{i_{k+1}}\ldots e_{i_{2k}} \ket_k \coloneqq \sum_{\mathcal{P}} \prod_{\{n,m\}\in\mathcal{P}} \left<e_{i_n},e_{i_m}\right>,
\end{equation}
where the sum is over all partitions $\mathcal{P}$ of $\{1,\ldots,2k\}$ in pairs.
\end{definition}
\begin{example}
To contrast the two definitions, observe that 
\begin{align}
\left<ab,cd\right>_2 &= \left<a,c\right>\left<b,d\right>+\left<a,d\right>\left<b,c\right> \\
\bra ab,cd\ket_2 &= \left<a,c\right>\left<b,d\right>+\left<a,d\right>\left<b,c\right> + \left<a,b\right>\left<c,d\right> 
\end{align}
\end{example}
\begin{remark}
Note that (\ref{Garr}) does not require symmetry of the bilinear form $\left<\,\ ,\ \right>$ on $V$. Indeed, the definition would equally be valid for a completely arbitrary bilinear form~$: V\times W \rightarrow K$, yielding a bilinear form $:\Sym^kV\times\Sym^kW\rightarrow K$. On the other hand, if the form on $V$ is not symmetric, then (\ref{mydef}) is not well-defined.
\end{remark}




\section{Homogeneous Orthogonal Polynomials}
In this section we will construct a basis for the space of homogeneous polynomials of degree $k$ in $d+1$ variables, $\R[x_0,\ldots,x_d]$, that is orthogonal with respect to the bilinear form given by
\begin{equation*}
 f\,,\,g\, \longmapsto \int_{\R^{d+1}}f(x)g(x)e^{-\frac{1}{2}\|x\|^2} dx.
\end{equation*}
In view of Lemma \ref{homosphere}, we could equivalently integrate the homogeneous polynomials over the unit sphere $\S^d$. That is the reason why we call them orthogonal on the sphere.
\begin{remark}
 If the homogeneity constraint was dropped, the answer to the problem would be much simpler: A basis of polynomials orthogonal with respect to that form is given by products $H_{\alpha_0}\!(x_0)\ldots H_{\alpha_d}\!(x_d)$ of Hermite polynomials in one variable.
\end{remark}
\subsection{Some formulas}
We give a few basic definitions and recall some facts on elementary calculus. The reader may also consult \cite{Folland}.

Denote $\Gamma(x)$ the gamma function.
\begin{lemma}
We have $n!=\Gamma(n+1)$ and $(2n-1)!! = 2^{n}\pi^{-\frac{1}{2}}\Gamma\left(n+\frac{1}{2}\right)$. Furthermore,
\begin{equation}
 \int_0^\infty r^se^{-\frac{1}{2}r^2} dr = 2^{\frac{s-1}{2}}\Gamma\left(\tfrac{s+1}{2}\right).
\end{equation}
\end{lemma}
It follows immediately, using multi-index notation: $x^\alpha = x_0^{\alpha_0}\ldots x_d^{\alpha_d}$, that:
\begin{align}    
 \int_{\R^{d+1}}x^\alpha x^\beta e^{-\frac{1}{2}\|x\|^2} dx &= \ \ 
\displaystyle \prod_{i=0}^d \int_{-\infty}^\infty x_i^{\alpha_i+\beta_i} e^{-\frac{1}{2}x_i^2} dx_i \\
 &= \left\{
\begin{array}{*2{l}p{5cm}}
 \prod_{i=0}^d (\alpha_i+\beta_i-1)!! &\text{if all }\alpha_i+\beta_i\text{ are even}, \vspace{0.2cm} \\ 
 0 &\text{otherwise}.
\end{array}
  \right.\nonumber
\end{align}

\begin{lemma}\label{homosphere}
Let $f:\R^{n+1}\rightarrow\R$ be an integrable homogeneous function of degree $k$, that is $f(sx) = s^kf(x)$. Using polar coordinates $(r,\omega) = (\|x\|,\frac{x}{\|x\|})$, we get:
\begin{align*}
\int_{\R^{d+1}}f(x)e^{-\frac{1}{2}\|x\|^2} dx &= \int_{\S^d}\!\int_0^\infty\! f(r\omega) r^d e^{-\frac{1}{2}r^2}dr d\omega \\
&= 2^{\frac{k+d-1}{2}}\Gamma\left(\tfrac{k+d+1}{2}\right)\int_{\S^d}f(\omega)d\omega .
\end{align*}
\end{lemma}


\subsection{A polynomial family}
\begin{definition} Let $n,\ m$ be integers with $0\leq 2n\leq m+1$. We define polynomials $\p{n}{m}$ of degree $n$:
\begin{equation}
\p{n}{m}(x) \coloneqq \sum_{\substack{l=0\\ n-l\ \text{even}}}^n \frac{n!\,(m-2n)!!\,(-1)^{\frac{n-l}{2}}}{l!\,(m-n-l)!!\,(n-l)!!}\;x^l.
\end{equation}
\end{definition}
It is straightforward to check that the following two identities hold.
\begin{lemma} \label{trigonometric}
For $n\geq 1$, we have a trigonometric differential relation:
\begin{equation}
\frac{d}{d\omega} \Big[\p{n-1}{m-2}\big(\tan(\omega)\big)\cos(\omega)^{m-1} \Big]= (n-m)\, \p{n}{m}\big(\tan(\omega)\big)\cos(\omega)^{m-1}
\end{equation} \qed
\end{lemma}
\begin{proposition} \label{threeterm} For $0\leq 2n\leq m+1$, we have a three-term recurrence:
\begin{equation}
\p{0}{m}(x) = 1,\qquad \p{1}{m}(x) = x, \qquad \p{n+1}{m}(x) = x\p{n}{m}(x) -d_n^m \p{n-1}{m}(x),
\end{equation} 
where $d_n^m\coloneqq \frac{n(m-n+1)}{(m-2n)(m-2n+2)}$.\qed
\end{proposition}
\begin{theorem} We define, for $m\geq 1$, a linear functional $\L$ on the vector space of polynomials of degree less than $m$, by setting
\begin{equation}
\L: f \longmapsto \int_0^\infty\!\!\! \int_{-\infty}^\infty y^{m-1}f\left(\frac{x}{y}\right)dxdy.
\end{equation}
Then the $\p{n}{m}$ form a set of orthogonal polynomials with respect to the induced bilinear form, \ie $\L(\p{n}{m}\p{n'}{m})=0$ for $n\neq n'$ and 
\begin{equation}
\L(\p{n}{m}\p{n}{m}) = 2^{\frac{3}{2}m-\frac{1}{2}-2n}  \frac{n!}{(m-n)!}
\Gamma\left(\frac{m}{2}-n\right)\Gamma\left(\frac{m}{2}-n+1\right)\Gamma\left(\frac{m+1}{2}\right).
\end{equation}
\end{theorem}
\begin{proof}
Since $\p{n}{m}$ satisfies the three-term relation in Prop.~\ref{threeterm}, by Favard's theorem, there exists a unique functional $\L'$ with $\L'(1)=1$, such that the $\p{n}{m}$ form an orthogonal basis with respect to the bilinear form induced by $\L'$. We will show that $\L$ is a scalar multiple of $\L'$, namely $\L=2^{\frac{m-1}{2}}\Gamma\left(\frac{m+1}{2}\right) \L'$. Since $(\p{n}{m})_{n\leq\frac{m-1}{2}} $ is a basis of our vector space, all we must do is showing that, for $n\geq 1$, $\L(\p{n}{m}) = \L(\p{n}{m}\p{0}{m}) =0$.
Using polar coordinates $(x,y) = (r\cos\omega,r\sin\omega)$ and Lemma \ref{trigonometric}, we get:
\begin{align*}
\L(\p{n}{m}) &= \int_0^\infty r^m e^{-\frac{r^2}{2}}dr \int_{\frac{\pi}{2}}^{\frac{3\pi}{2}} \cos(\omega)^{m-1} p_{n}^{m}\big(\tan(\omega)\big)d\omega\\
&= \Gamma\left(\tfrac{m+1}{2}\right)2^{\frac{m-1}{2}}\Big[p_{n-1}^{m-2}\big(\tan(\omega)\big)\cos(\omega)^{m-1} \Big]_{\frac{\pi}{2}}^{\frac{3\pi}{2}} =0.
\end{align*}
\end{proof}



\bibliographystyle{amsplain}
\bibliography{SymBil.tex}


\end{document}
