\documentclass{amsart}

\usepackage{amsmath,amssymb,amsfonts}
\usepackage[all]{xy}
\usepackage{appendix,listings,hyperref}

\DeclareMathOperator{\rank}{rk}
\DeclareMathOperator{\trace}{tr}
\DeclareMathOperator{\Aut}{Aut}
\DeclareMathOperator{\End}{End}
\DeclareMathOperator{\id}{id}
\DeclareMathOperator{\Hom}{Hom}
\DeclareMathOperator{\Sym}{Sym}
\DeclareMathOperator{\Hilb}{Hilb}
\DeclareMathOperator{\len}{len}

\newcommand{\hilb}[1]{^{[#1]}}
\newcommand{\ie}{{\it i.e. }}
\newcommand{\eg}{{\it e.g. }}
\newcommand{\loccit}{{\it loc. cit. }}
\newcommand{\vac}{|0\rangle}
\newcommand{\odd}{{\rm{odd}}}
\newcommand{\even}{{\rm{even}}}
\newcommand{\tors}{{\rm{tors}}}

\newcommand{\p}[2]{p_{#1}^{#2}\;\!\!}
\renewcommand{\L}{\mathcal{L}}

\newcommand{\coloneqq}{:=}
\newcommand{\bra}{\left<\!\!\!\:\left<}
\newcommand{\ket}{\right>\!\!\!\:\right>}
\newcommand{\myeq}[1]{\mathrel{\overset{\makebox[0pt]{\text{\tiny #1}}}{=}}}


%%%%%%%%%%%%%%%%%%%%%%%%%%%%%%

\newcommand{\G}{\mathbb{G}}
\newcommand{\R}{\mathbb{R}}
\newcommand{\Q}{\mathbb{Q}}
\newcommand{\Z}{\mathbb{Z}}
\renewcommand{\S}{\mathbb{S}}
\renewcommand{\H}{\mathbb{H}}

%%%%%%%%%%%%%%%%%%%%%%%%%%%%%

\newcommand{\kS}{\mathfrak{S}}

%%%%%%%%%%%%%%%%%%%%%%%%%%%%%%

\newcommand{\lra}{\longrightarrow}
\newcommand{\ra}{\rightarrow}

%%%%%%%%%%%%%%%%%%%%%%%%%%%%%

\theoremstyle{plain}
\newtheorem{theorem}{Theorem}[section]
\newtheorem{lemma}[theorem]{Lemma}
\newtheorem{proposition}[theorem]{Proposition}
\newtheorem{corollary}[theorem]{Corollary}
\theoremstyle{definition}
\newtheorem{definition}[theorem]{Definition}
\newtheorem{notation}[theorem]{Notation}
\theoremstyle{remark}
\newtheorem{remark}[theorem]{Remark}
\newtheorem{example}[theorem]{Example}


%%%%%%%%%%%%%%%%%%%%%%%%%%%%%

\begin{document}

\title[Symmetric Powers, Hom.~Orth.~Polynomials, Hyperk\"ahlers]{Symmetric Powers of Symmetric Bilinear Forms, Homogeneous Orthogonal Polynomials on the Sphere and an Application in Compact Hyperk\"ahler Manifolds}


\author{Simon Kapfer}
\address{Simon Kapfer, Laboratoire de Math\'ematiques et Applications, UMR CNRS 6086, Universit\'e de Poitiers, T\'el\'eport 2, Boulevard Marie et Pierre Curie, F-86962 Futuroscope Chasseneuil}
\email{simon.kapfer@math.univ-poitiers.fr}
%\urladdr{http://www.math.uni-augsburg.de/alg/}


\date{\today}

\keywords{Symmetric Bilinear Forms on Symmetric Powers, Orthogonal Polynomials in Several Variables, Homogeneous Orthogonal Polynomials, Hermite Polynomials, Spherical Harmonics, Hankel matrices, Hyperk\"ahler Manifolds, Beauville--Bogomolov Form, Beauville--Fujiki relation}

\begin{abstract} The Beauville--Fujiki relation for a compact Hyperk\"ahler $X$ manifold of dimension $2k$ allows to equip the symmetric power $\Sym^kH^2(X)$ with a symmetric bilinear form induced by the Beauville--Bogomolov form. We study some of its properties and compare it to the form given by the Poincar\'e pairing.

The construction generalizes to a definition for an induced symmetric bilinear form on the symmetric power of any free module equipped with a symmetric bilinear form. We point out that our construction is related to the theory of orthogonal polynomials in several variables.
Finally, we construct a basis of homogeneous polynomials that are orthogonal when integrated over the unit sphere $\S^d$, or equivalently, over $\R^{d+1}$ with a Gaussian kernel.
\end{abstract}

\maketitle

%%%%%%%%%%%%%%%%%%%%%%%%%%%%%%%%%%%%%%%%%%%%%%%%%%%%%%%%%%%%%%%%%%
%%%%%%%%%%%%%%%%%%%%%%%%%%%%%%%%%%%%%%%%%%%%%%%%%%%%%%%%%%%%%%%%%%
%%%%%%%%%%%%%%%%%%%%%%%%%%%%%%%%%%%%%%%%%%%%%%%%%%%%%%%%%%%%%%%%%%

\section{Introduction}
Our motivation originated in Hyperk\"ahler theory. The Beauville--Bogomolov--Fujiki form $q$ for a compact Hyperk\"ahler manifold $X$ is a quadratic form on the integral cohomology group $H^2$ which is defined by an equation of the structure
\begin{equation} \label{initialeq}
q(x)^k = I(x^{2k}),
\end{equation}
where $x^{2k}$ means a power in the cohomology ring, and $I$ is a linear form (in fact, a scaled integral). 

Now every quadratic form $q$ has an associated symmetric bilinear form $\left<\ ,\;\right>$, obtained by polarization: $2\left<x,y\right> = q(x+y)-q(x)-q(y)$. This allows us to retrieve some information about $I$ from $\left<\ ,\;\right>$,
%On the other hand, the map $(x^\alpha,y^\beta) \mapsto I(x^\alpha y^\beta)$ clearly defines a symmetric bilinear form on the symmetric product $\Sym^kH^2$. So we may ask for the relation between these two bilinear forms. 
%For instance, by comparing coefficients in the equality $I\left((x+y)^{2k}\right) = q(x+y)^k = \left( q(x)+2\left<x,y\right>+q(y) \right)^k$, we obtain:
%\begin{equation*}
%\textstyle \binom{2k}{2j} I\!\left(x^{2j}y^{2k-2j} \right) = \sum\limits_{i=0}^j \frac{k!\,2^{2j-2i}}{i!\,(2j-2i)!\,(k-2j+i)!} \, q(x)^i \left< x,y\right>^{2j-2i}q(y)^{2k-2j+i}
%\end{equation*}
by comparing coefficients in the equality
\begin{equation*}
I\!\left((x_1+\ldots+x_{2k})^{2k}\right) = q(x_1+\ldots+x_{2k})^{k} = \left(\sum\limits_{i=1}^{2k} q(x_i) +\!\!\!\sum\limits_{1\leq i<j\leq 2k}\!\!2 \left<x_i,x_j\right>  \right)^k.
\end{equation*}
If we look at the summands belonging to $x_1\ldots x_{2k}$, we obtain a seemingly more general but in fact equivalent version of (\ref{initialeq}):
\begin{equation} \label{initialpolar}
 (2k)!\; I\!\left(x_1\ldots x_{2k}\right) = 2^k k! \!\!\prod\limits_{1\leq i<j\leq 2k}\!\!\left<x_i,x_j\right>.
\end{equation}
This observation was already made by O'Grady, see \cite[Eq.~3.2.4]{OGrady}. Let us now change a bit our point of view. The map $(f,g) \mapsto I(fg)$ clearly defines a symmetric bilinear form on the symmetric product $\Sym^kH^2$.
Equation (\ref{initialpolar}) gives now a redefinition of this form by means of a bilinear form on $H^2$. So we liberate ourselves from the initial setting and take the right hand side of (\ref{initialpolar}) as a general recipe to construct a symmetric bilinear form $\bra\ ,\;\ket$ on $\Sym^kV$ from a symmetric bilinear form on an appropriate space $V$. This is carried out in Section~\ref{symSection}. Our main result, Theorem~\ref{maintheorem}, gives a formula for the determinant of the Gram matrix of $\bra\ ,\;\ket$. 

It is not surprising that we can get back our linear form $I$. However, for real vector spaces $V$, there is a notable description in terms of an analytic integral given in Prop.~\ref{intequiv}: After some simplifications this amounts to integrating homogeneous polynomials over a sphere. Since for computing determinants it is good to have diagonal matrices, we have to look for polynomials that are mutually orthogonal on the sphere. Surprisingly we could not find this topic in the literature, so we constructed them in Section~\ref{polynomialSection}.

After doing that, we come back to our starting point and apply our results to Hyperk\"ahler manifolds. The bilinear form on $\Sym^kH^2$ allows us to compare $\Sym^kH^2$ with $H^{2k}$. We give some results on the quotient $\frac{H^{2k}}{\Sym^kH^2}$ in Section \ref{hyper}, similar to those the author studied in \cite{Kapfer}.



\section{Terminology and helper formulas} \label{boring}
In this section 
we give a few standard definitions and recall some facts on elementary calculus. We also mention technical formulas needed for our proofs.
\begin{definition}\label{multiindex}
For a multi-index $\alpha=(\alpha_0,\ldots,\alpha_d)$ of length $\len(\alpha)\coloneqq d+1$ we define: $x^\alpha \coloneqq x_0^{\alpha_0}\ldots x_d^{\alpha_d}$. The degree is defined by $|\alpha |\coloneqq\sum\alpha_i$, the factorial is $\alpha! \coloneqq \prod \alpha_i!$. Further, we set
$\alpha'\coloneqq(\alpha_0,\ldots,\alpha_{d-1})$. We introduce the lexicographical ordering on multi-indices: $\alpha < \beta$ iff $\alpha_d < \beta_d$ or $(\alpha_d=\beta_d) \wedge (\alpha'<\beta')$.
\end{definition}
\begin{definition}
The binomial coefficient for nonnegative integers $k$ and arbitrary $z$ is defined as:
$\binom{z}{k} \coloneqq \frac{z(z-1)\ldots(z-k+1)}{k!}$. Thus we have $\binom{-z}{k}=(-1)^k\binom{z+k-1}{k}$. For negative $k$ we set $\binom{z}{k}\coloneqq 0$.
\end{definition}
We introduce the difference operator $\Delta f(n) \coloneqq f(n+1)-f(n)$. It has the following properties similar to the differential operator:
\begin{align}
 \sum_{i=0}^n \Delta(f) & = f\,\Big|_0^{n+1}= f(n+1)-f(0) && \text{(telescoping sum)} \\
 \Delta (fg)(n) &= f(n+1) \Delta g(n) + g(n)\Delta f(n) &&\text{(product rule)} \\
\label{sumbyparts}  \sum_{i=0}^n g(i)\Delta f(i) & = (fg)\Big|_0^{n+1} - \sum_{i=0}^n f(i+1)\Delta g(i) && \text{(summation by parts)}
\end{align}
This often applies to the binomial coefficient, since we have: 
\begin{equation} \label{binomdiff}
\textstyle \Delta \binom{n}{k}=\binom{n+1}{k}-\binom{n}{k} = \binom{n}{k-1}.
\end{equation}
%The same recurrence holds for the cardinality of the set $\{|\alpha| =k\}$ of multi-indices of length $d+1$ and degree $k$, because $\{ |\alpha| = k \} = \bigcup_{j=0}^k  \{ |\alpha'| = j \} \times \{ k-j \}  $

For instance, let $r_{d,k} = \rank \left(\Sym^k K^{d+1} \right)$ be the rank of the symmetric power. 
Because of the decomposition $\Sym^k\! K^{d+1} \cong \Sym^k\! K^d \oplus \,\Sym^{k-1}\! K^{d+1}\!\otimes\! K$, we have the recurrence $r_{d,k} = r_{d-1,k} +r_{d,k-1}$. So we deduce:
\begin{equation} \label{binomcount} \textstyle
\binom{k+d}{d} =\binom{k+d}{k} = r_{d,k}=\rank \left(\Sym^k K^{d+1} \right) = \text{card}\big(\{|\alpha| =k\}\big).
\end{equation}

The following identity for integers $d,k,m\geq 0$ is proven by induction over $k$:
\begin{align} \label{facprod1}
\prod_{j=0}^k (k-j)!^{\binom{j+d-1}{d-1}} &= \ \prod_{i=1}^k i^{\binom{k-i+d}{d}},
% \\ \label{facprod2}
 %\prod_{j=0}^k (j+m)!^{\binom{j+d-1}{d-1}} &= (k+m)!^{\binom{k+d}{d}}, \prod_{i=m+1}^{k+m} i^{ -\binom{i-m+d-1}{d}}.
\end{align}
where the induction step $k\rightarrow k+1$ produces a factor $\prod\limits_{i=1}^{k+1} i^{\binom{k-i+d}{d-1} }$ on both sides.

We will also need the identity:
\begin{equation} \label{evensum}
\sum_{\substack{i=1\\i\text{ even}}}^{2k+d+1}\textstyle \binom{k-i+d}{d-1} = \left\{ 
 \begin{array}{*2{c}p{5cm}}0 &\text{if }d\text{ is even}, \vspace{1mm}\\
 \binom{k+d}{d} &\text{if }d\text{ is odd},
\end{array}\right.
\end{equation}
which is proven by splitting the sum into:
$$
\sum_{\substack{i=1\\i\text{ even}}}^{k+1} \textstyle\binom{k-i+d}{d-1} + \displaystyle\sum_{\substack{i=k+d+1\\i\text{ even}}}^{2k+d+1}\textstyle \binom{k-i+d}{d-1}
= \displaystyle\sum_{\substack{i=1\\k-i\text{ even}}}^{k+1}\!\! \textstyle\binom{i+d-2}{d-1} +(-1)^{d-1} \!\!\!\!
 \displaystyle\sum_{\substack{i=1\\k+d+i\text{ even}}}^{k+1} \!\!\!\! \textstyle\binom{i+d-2}{d-1} .
$$

\begin{definition}\label{doublefactorial}
We define the double factorial for $ n\geq -1$ by 
$$n!! \,\coloneqq \prod_{i=0}^{\left\lfloor\!\frac{n-1}{2}\!\right\rfloor }(n-2i)=n(n-2)(n-4)\ldots $$
Clearly, $(n-1)!!\,n!! = n!$ and $(2n)!! = 2^n n!$.
\end{definition}
\begin{proposition} \label{partitioncount}
The number of partitions of the set $\{1,\ldots,2k\}$ into pairs equals $(2k-1)!! = \frac{(2k)!}{2^kk!}$.
\end{proposition}
\begin{proof}
Given such a partition, look at the pair that contains the element $1$. There are $2k-1$ possible partners for this element; removing the pair leaves a partition of a set of cardinality $(2k-2)$ into pairs. Then proceed by induction.
\end{proof}



Denote $\Gamma(t) \coloneqq \int_{0}^{\infty}r^{t-1}e^{-r}dr$ the gamma function. It satisfies:
\begin{align}
\label{ffgamma}
n!&=\Gamma(n+1),\qquad (2n-1)!!\sqrt{\pi} =2^{n}\Gamma\left(n+\tfrac{1}{2}\right), \\
\label{doublegamma}
n!&\sqrt{\pi}=2^{n}\Gamma\left(\tfrac{n}{2}+1\right)\Gamma\left(\tfrac{n+1}{2}\right) ,\\
\int_0^\infty& r^se^{-\frac{1}{2}r^2} dr = 2^{\frac{s-1}{2}}\Gamma\left(\tfrac{s+1}{2}\right).
\end{align}
It follows, that:
\begin{align}    \label{monoint}
 \int_{\R^{d+1}}x^\alpha x^\beta & e^{-\frac{1}{2}\|x\|^2} dx = \ \ 
\displaystyle \prod_{i=0}^d \int_{-\infty}^\infty x_i^{\alpha_i+\beta_i} e^{-\frac{1}{2}x_i^2} dx_i \\
 &= \left\{
\begin{array}{*2{l}p{5cm}}
(2 \pi)^{\frac{d+1}{2}}\prod_{i=0}^d (\alpha_i+\beta_i-1)!! &\text{if all }\alpha_i+\beta_i\text{ are even}, \vspace{0.2cm} \\ 
 0 &\text{otherwise}.
\end{array}
  \right.\nonumber
\end{align}
The reader may also consult \cite{Folland} for that kind of calculus. In particular, \cite[Eq.~(4)]{Folland} yields:
\begin{lemma}\label{homosphere}
Let $f:\R^{n+1}\rightarrow\R$ be a continuous homogeneous function of degree $k$, that is $f(sx) = s^kf(x)\; \forall s\in\!\R$. Then, using polar coordinates $(r,\omega) = (\|x\|,\frac{x}{\|x\|})$:
\begin{align*}
\int_{\R^{d+1}}f(x)e^{-\frac{1}{2}\|x\|^2} dx &= \int_{\S^d}\!\int_0^\infty\! f(r\omega) r^d e^{-\frac{1}{2}r^2}dr d\omega \\
&= 2^{\frac{k+d-1}{2}}\Gamma\!\left(\tfrac{k+d+1}{2}\right)\int_{\S^d}f(\omega)d\omega .
\end{align*}
\end{lemma}



\section{Symmetric Bilinear Forms on Symmetric Powers} \label{symSection}
Let $V$ be a vector space (or a free module) over a field (resp.~a commutative ring) $K$ of rank $d+1$ with basis $\{x_0,\ldots,x_{d}\}$, equipped with a symmetric bilinear form $\left<\,\ ,\ \right>: V\times V \rightarrow K$. We will freely identify the symmetric power $\Sym^kV$ with the space $K[x_0,\ldots,x_d]_k$ of homogeneous polynomials of degree $k$. 

There are at least two possibilities to define an induced bilinear form on $\Sym^kV$. We will use the following
\begin{definition} \label{formdef} On the basis $\{x_{n_1}\ldots x_{n_k}\;|\;0\leq n_1\leq\ldots\leq n_k\leq d\}$ of $\Sym^kV$, we define a symmetric bilinear form $\bra\ \,,\ \ket$ by: 
\begin{equation}
\label{mydef}
\bra x_{n_1}\ldots x_{n_k}\,,\,x_{n_{k+1}}\ldots x_{n_{2k}} \ket \coloneqq \sum_{\mathcal{P}} \prod_{\{i,j\}\in\mathcal{P}} \left<x_{n_i},x_{n_j}\right>,
\end{equation}
where the sum is over all partitions $\mathcal{P}$ of $\{1,\ldots,2k\}$ into pairs.
\end{definition}

We emphasize that this is not the only possibility. One could alternatively define
\begin{equation}\label{Garr}
\left(\!\left( x_{n_1}\ldots x_{n_k}\,,\, x_{m_1}\ldots x_{m_k}\right)\!\right) \coloneqq \sum_\sigma  
\prod_{i=1}^k \left< x_{n_i},x_{m_{\sigma(i)}}\right>,
\end{equation}
the sum being over all permutations $\sigma$ of $\{1,\ldots,k\}$, as studied by McGarraghy in \cite{McGarr}. However, this is a different construction that we will \emph{not} consider in this article.

If $U\in O(V)$ is an orthogonal transformation, then the induced diagonal action of $U^{\otimes k}$ on $\Sym^kV$ is orthogonal in both cases. This shows that the two definitions are independent of the choice of the base of $V$. 
\begin{example}
To contrast the two definitions, observe that in the case $k=2$
\begin{align}
\left(\!\left( ab,cd \right)\!\right) &= \left<a,c\right>\left<b,d\right>+\left<a,d\right>\left<b,c\right>, \\
\bra ab,cd\ket &= \left<a,c\right>\left<b,d\right>+\left<a,d\right>\left<b,c\right> + \left<a,b\right>\left<c,d\right>.
\end{align}
\end{example}
\begin{remark}
Note that (\ref{Garr}) does not require symmetry of the bilinear form $\left<\,\ ,\ \right>$ on $V$. Indeed, the definition would also be valid for an arbitrary bilinear form~$: V\times W \rightarrow K$, yielding a bilinear form $:\Sym^kV\times\Sym^kW\rightarrow K$. On the other hand, if the form on $V$ is not symmetric, then (\ref{mydef}) is not well-defined.
\end{remark}
\begin{remark}
The defining equation (\ref{mydef}) works equally well, if the two arguments have different degree. So we can easily extend our definition to a symmetric bilinear form~$\bra\ \,,\ \ket:\Sym^*V\times\Sym^*V \rightarrow K$. Then we have: $\bra a,bc\ket =\bra ab,c\ket$. Note that $\Sym^kV$ is in general not orthogonal to $\Sym^lV$ unless $k-l$ is an odd number.
\end{remark}
We wish to investigate some properties of this construction. Let $G$ be the Gram matrix of $\left< \ ,\;\right>$, \ie $G_{ij} = \left<x_i,x_j\right>$ and
let $\G$ be the Gram matrix of $\bra\ ,\;\ket$. We use multi-index notation, cf.~Definition~\ref{multiindex}.
\begin{proposition} \label{intequiv}Assume $K=\R$ and $G$ is positive definite, so its inverse $G^{-1} $ exists. Then $\bra\ ,\;\ket$ takes an analytic integral form:
\begin{equation*}
\bra x^\alpha, x^\beta \ket = \frac{1}{c}\int_{\R^{d+1}} x^\alpha x^\beta d\mu(x),
\end{equation*}
where the integration measure is $d\mu(x) = \exp\left(-\frac{1}{2}\sum_{i,j} G^{-1}_{ij}x_ix_j\right)dx$ and the normalization constant is $c=\int_{\R^{n+1}} d\mu(x)=\sqrt{(2\pi)^{d+1}\det G}$.
\end{proposition}
\begin{proof} Note that we need positive definiteness of $G$ to make the integral converge. We make use of the content in Section \ref{boring}.
First, observe that both sides of the equation are invariant under orthogonal transformations of the base space $\R^{d+1}$. We may therefore assume that $G= \text{diag}\left(a_0,\ldots,a_d\right)$ is a diagonal matrix. Then the integral splits nicely:
\begin{align*}
\frac{1}{c}\int_{\R^{d+1}}& x^\alpha x^\beta d\mu(x)= \frac{1}{c}\prod_{i=0}^d \int_{-\infty}^\infty x_i^{\alpha_i+\beta_i} e^{-\frac{1}{2a_i}x_i^2}dx_i \\=&\: 
\frac{1}{c}\prod_{i=0}^d a_i^{\frac{\alpha_i+\beta_i+1}{2}}\int_{-\infty}^\infty x^{\alpha_i+\beta_i} e^{-\frac{1}{2}x^2}dx\\
\myeq{(\ref{monoint})}\  &\left\{
\begin{array}{*2{l}p{5cm}}\displaystyle \prod_{i=0}^d a_i^{\frac{\alpha_i+\beta_i}{2}}(\alpha_i+\beta_i-1)!! &\text{if all }\alpha_i+\beta_i\text{ are even}, \vspace{0.2cm} \\ 
 0 &\text{otherwise}.
\end{array}
 \right.
\end{align*}
On the other hand, if $G$ is diagonal, then every partition into pairs in Equation (\ref{mydef}) that contains a pair of two different numbers will not contribute to the sum. But the number of partitions of the multiset $\{n|\,0\!\leq\!n\!\leq\! d,\text{ multiplicity of } n = \alpha_n+\beta_n\}$ into pairs, such that every pair consists of two equal numbers, is by Prop.~\ref{partitioncount} evidently equal to $\prod_{i=0}^d (\alpha_i+\beta_i-1)!!$ if all $\alpha_i+\beta_i$ are even and $0$ otherwise. So we obtain the same formula for $\bra x^\alpha ,x^\beta\ket$.
\end{proof}

The next theorem gives a formula for the determinant of $\G$. This is of particular interest when $K=\Z$, because in this case we are in the setting of lattice theory, and $\det \G$ is an important lattice-theoretic invariant, called the discriminant.
\begin{theorem} \label{maintheorem}
The determinant of the Gram matrix $\G$ of $\bra\ ,\;\ket$ is:
\begin{equation}
\det(\G)= \det(G)^{\binom{d+k}{d+1}}\,\theta_{d,k}
\end{equation}
where $\theta_{d,k}$ is a combinatorial factor given by:
\begin{equation} \label{thetaDef}
\theta_{d,k} = \left\{
 \begin{array}{*2{l}p{5cm}}
 \displaystyle\prod_{i=1}^k i^{\binom{k-i+d}{d}d}\prod_{\substack{i=1 \\ i\ \text{odd}\\\ }}^{2k+d-1}i^{\binom{k-i+d}{d}} &\text{if }d\text{ is even}, \\
 \displaystyle\prod_{i=1}^k i^{\binom{k-i+d}{d}d}\prod_{i=1}^{k+\frac{d-1}{2}} i^{\binom{k-i+d}{d} - \binom{k-2i+d}{d}} &\text{if }d\text{ is odd}.
\end{array}
\right.
\end{equation}
\end{theorem}
\begin{remark} If $d$ or $k$ is small, this simplifies as follows:
\begin{gather*}
\theta_{d,0}=\theta_{d,1} =1,\qquad \theta_{d,2} = 2^{d}(d+3), \\
\theta_{0,k} = (2k-1)!!, \qquad \theta_{1,k} = (k!)^{k+1}.
\end{gather*}
\end{remark}
\begin{remark}
We didn't mention the base ring $K$ in the theorem. In fact, if the formula holds for $K=\Z$, then automatically for any $K$, because $\det\G$ is defined only by sums and products that don't depend on $K$. However, for the proof we work with $\R$-valued coefficients. It is clear that the formula holds for $K=\R$ if and only if it holds for $K=\Z$. We require further that $G$ is positive definite, because we want to use Prop.~\ref{intequiv}. That this means no loss of generality, is seen by the following argument: Let $Q\subset \R^{(d+1)\times(d+1)}$ be the set of real symmetric square matrices of size $d+1$. The subset $R\subset Q$ of all matrices $G\in Q$ that satisfy our formula is Zariski-closed. But on the other hand, the positive definite matrices form a nonempty subset $P\subset Q$ which is open in the analytic topology. So if $P\subset R$, then necessarily $R=Q$.
\end{remark}
\begin{proof}
As one may imagine, finding the factor $\theta_{d,k}$ is the hard part. We have to postpone this to Section \ref{hsection}. We will therefore reduce the statement to the case when $G$ is the identity matrix, which is proven in Theorem \ref{thetaCor}. Since any orthogonal transformation $U\in O(V)$ induces a  transformation $U^{\otimes k} \in O(\Sym^k V)$ and thus doesn't affect determinants, we may assume that $G$ is a diagonal matrix. So let us check, what happens if we apply a coordinate transformation $x\mapsto\tilde x$ that changes the last coordinate by $\tilde{x}_d = \gamma x_d$ and leaves the other coordinates invariant. Let $\tilde{G}$ and $\tilde{\G}$ be the Gram matrices corresponding to the new coordinates. We clearly have: $\tilde{x}^\alpha = \gamma^{\alpha_d} x^\alpha$. Extracting the factor $\gamma$ from the Leibniz determinant formula, which is of the form $\det \tilde{\G}=\sum\limits_\sigma\pm\prod\limits_{|\alpha|=k} \bra \tilde{x}^{\alpha},\tilde{x}^{\sigma(\alpha)}\ket$, we get: 
\vspace{-2mm}
$$
\frac{\det \tilde{\G}}{\det\G} = \prod_{|\alpha|=k}\gamma^{2\alpha_d} = \prod_{i=0}^k\ \prod_{|\alpha '|=k-i} \gamma^{2i} \; \myeq{(\ref{binomcount})}\;\prod_{i=0}^k\gamma^{2i\binom{k-i+d-1}{d-1}} \; \myeq{(\ref{sumbyparts})} \; \gamma^{2\binom{d+k}{d+1}}.
$$
Now, if $G= \text{diag}\left(a_0,\ldots,a_d\right)$, we apply successively coordinate transformations that map $x_i$ to $\frac{x_i}{\sqrt{a_i}}$. We get a factor $(a_0\ldots a_d)^{\binom{d+k}{d+1}} = \det G^{\binom{d+k}{d+1}}$.
\end{proof}





\section{Homogeneous Orthogonal Polynomials on the sphere} \label{polynomialSection}
In this section we will construct a basis for the space of homogeneous polynomials of degree $k$ in $d+1$ variables, $\R[x_0,\ldots,x_d]_k$, that is orthogonal with respect to the bilinear form given by
\begin{equation*}
 \bra f,g\ket = \int_{\R^{d+1}}f(x)g(x) d\mu(x),
\end{equation*}
where the measure is $d\mu(x) = (2\pi)^{-\frac{d+1}{2}}e^{-\frac{1}{2}\|x\|^2}dx$. In order to do this, we wish to apply the Gram-Schmidt process to the (lexicographically ordered) monomial basis $(x^\alpha)_{|\alpha |=k}$. Our result is stated in Subsection \ref{hsection}. 
\begin{remark}
Although our definition of $\bra\ ,\;\ket$ doesn't mention the sphere, in view of Lemma \ref{homosphere}, we could equivalently integrate the homogeneous polynomials over the unit sphere $\S^d$:
$$
 \bra f,g\ket = c \int_{\S^{d}}f(\omega)g(\omega) d\omega, \qquad c = 2^{\frac{k}{2}-1}\pi^{-\frac{d+1}{2}}\Gamma\!\left(\tfrac{k+d+1}{2}\right).
$$ 
This is the reason why we call our polynomials orthogonal on the sphere. However, we prefer to integrate over $\R^{d+1}$, since this avoids the unwanted constant $c$.
\end{remark}
\begin{remark}
 If the homogeneity constraint was dropped, the answer to the problem would be much simpler: A basis of $ \bra\ ,\;\ket$-orthogonal polynomials is given by products $H_{\alpha_0}\!(x_0)\ldots H_{\alpha_d}\!(x_d)$ of Hermite polynomials in one variable, see also \cite[Sect.~2.3.4]{Dunkl}.
\end{remark}



\subsection{Generalities on orthogonal polynomials in one variable}
Given a nondegenerate symmetric bilinear form on the space of polynomials $K[x]$, one may ask for a basis of polynomials $(p_n)_n$ that are mutually orthogonal with respect to that form. To find such a basis, one could start with the monomial basis $(x^n)_n$ and apply some version of the Gram--Schmidt algorithm. The result will be an infinite lower triangular matrix $T$ such that $p_n =\sum_j T_{nj} x^j$. We prefer to normalize such that the diagonal elements of $T$ are equal to $1$. If our bilinear form now depends only on the product of its two arguments, the procedure simplifies as follows:

Let $\L$ be a linear functional such that the induced bilinear form $(f,g)=\L(fg)$ is nondegenerate when restricted to $K[x]_{\leq n}$, the space of polynomials of bounded degree, for all $n\geq 0$. Let $(p_n)_n$ be the associated sequence of monic orthogonal polynomials, \ie the leading term of $p_n(x)$ is $x^n$ and $(p_k,p_{n})=0$ for $k\neq n$. Then we have
\begin{theorem} \cite[Thm.~4.1]{Chihara} There are constants $c_n,\: d_n$ such that
\begin{equation*}
 p_0(x) = 1,\qquad  p_{n+1}(x) = (x-c_n)p_n(x) - d_np_{n-1}(x).
\end{equation*}
\end{theorem}
But also the converse is true:
\begin{theorem}[Favard's theorem] \cite[Thm.~4.4]{Chihara} \label{favard}
Let $(p_n)_n$ be a sequence of polynomials, such that $\deg p_n =n$ and the following three-term recurrence holds:
$$p_0(x) = 1,\qquad  p_{n+1}(x) = (x-c_n)p_n(x) - d_np_{n-1}(x).
$$
Then there exists a unique linear functional $\L$ such that $\L(1)=1$ and $\L(p_kp_{n})=0$ for $k\neq n$. 
\end{theorem}
\begin{theorem}\cite[Thm.~4.2]{Chihara} \label{generalLnorm}
Under the conditions of the above theorem, we have for $n\geq 1$: $$\L(p_n^2) = d_n\L(p_{n-1}^2).$$
\end{theorem}
\begin{remark} \label{finiteFarvard}Since we shall deal with finite polynomial families, we need a little modification of Favard's theorem:
If $(p_n)_{n\leq N}$ is a finite sequence that satisfies a three-term recurrence as above, then we can always extend it to an infinite sequence by choosing arbitrary constants $c_n$, $d_n$ for $n\geq N$. But for every such extension, the resulting functional $\L$ from Favard's theorem will satisfy $\L(1) =1$ and $\L(p_n)=\L(p_np_0)=0$ for $n\geq 1$. So $\L$ will always be uniquely determined on $K[x]_{\leq N}$, the space of degree-bounded polynomials.
\end{remark}






\subsection{A polynomial family}
\begin{definition} Let $n,\ m$ be integers with $0\leq 2n\leq m+1$, a condition that we always will assume silently. We define polynomials $\p{n}{m}$ of degree $n$:
\begin{equation*}
\p{n}{m}(x) \coloneqq \sum_{\substack{j=0\\ n-j\ \text{even}}}^n (-1)^{\frac{n-j}{2}} \frac{n!\,(m-2n)!!}{j!\,(m-n-j)!!\,(n-j)!!}\:x^j.
\end{equation*}
\end{definition}
\begin{lemma} \label{trigonometric}
For $n\geq 1$, we have a trigonometric differential relation:
\begin{equation*}
\frac{d}{d\omega} \Big[\p{n-1}{m-2}\big(\tan(\omega)\big)\cos(\omega)^{m-1} \Big]= (n-m)\, \p{n}{m}\big(\tan(\omega)\big)\cos(\omega)^{m-1}.
\end{equation*} 
\end{lemma}
\begin{proof} This is straightforward.
Firstly, we calculate $ \frac{d}{d\omega}\! \left[\sin(\omega)^j \cos(\omega)^{m-j-1}\right] = j \sin(\omega)^{j-1} \cos(\omega)^{m-j}-(m\!-\!j\!-\!1)\sin(\omega)^{j+1} \cos(\omega)^{m-j-2}$, and so
\begin{align*}% \textstyle
\frac{d}{d\omega} &\Big[\p{n-1}{m-2}\big(\tan(\omega)\big)\cos(\omega)^{m-1} \Big] 
\\&=\!\! \textstyle \sum\limits_{\substack{j=0\\ n-j\ \text{odd}}}^{n-1} \!\!\!\frac{(-1)^{\frac{n-j-1}{2}}  (n-1)!\,(m-2n)!!}{j!\,(m-n-j-1)!!\,(n-j-1)!!}\frac{d}{d\omega}\! \left[\sin(\omega)^j \cos(\omega)^{m-j-1}\right]
\\ &=\!\!\textstyle\sum\limits_{\substack{j=0\\ n-j\ \text{even}}}^{n-2} \!\!\!\frac{(-1)^{\frac{n-j-2}{2}}(n-1)!\,(m-2n)!!}{j!\,(m-n-j-2)!!\,(n-j-2)!!}\sin(\omega)^{j} \cos(\omega)^{m-j-1} \\[-3mm]
&\hspace{3cm}-\textstyle\sum\limits_{\substack{j=1\\ n-j\ \text{even}}}^{n}\!\!\! \frac{(-1)^{\frac{n-j}{2}}(n-1)!\,(m-2n)!!\,(m-j)}{(j-1)!\,(m-n-j)!!\,(n-j)!!}\sin(\omega)^{j} \cos(\omega)^{m-j-1} \\
&=\!\!\textstyle\sum\limits_{\substack{j=0\\ n-j\ \text{even}}}^{n} \!\!\!\frac{(-1)^{\frac{n-j}{2}}(n-1)!\,(m-2n)!!}{j!\,(m-n-j)!!\,(n-j)!!}\underbrace{\big[(j\!-\!n)(m\!-\!n\!-\!j)-j(m\!-\!j)\big] }_{=n(n-m)}\tan(\omega)^{j} \cos(\omega)^{m-1}
\\ &= \textstyle(n-m)\, \p{n}{m}\big(\tan(\omega)\big)\cos(\omega)^{m-1}.
\end{align*}
\end{proof}

\begin{proposition} \label{threeterm} For $0\leq 2n\leq m-1$, we have a three-term recurrence:
\begin{equation*}
\p{0}{m}(x) = 1,\qquad \p{1}{m}(x) = x, \qquad \p{n+1}{m}(x) = x\p{n}{m}(x) -d_n^m \p{n-1}{m}(x),
\end{equation*} 
where $d_n^m\coloneqq \frac{n(m-n+1)}{(m-2n)(m-2n+2)}$.
\end{proposition}
\begin{proof}
We start from the right: $ x\p{n}{m}(x) -d_n^m \p{n-1}{m}(x)$ gives
\begin{align*}
 &\textstyle \sum\limits_{\substack{j=1\\ n-j\ \text{odd}}}^{n+1} \!\!\!\frac{(-1)^{\frac{n-j+1}{2}} n!\,(m-2n)!!}{(j-1)!\,(m-n-j+1)!!\,(n-j+1)!!}\: x^j 
\ - \sum\limits_{\substack{j=0\\ n-j\ \text{odd}}}^{n-1} \!\!\!d_n^m\frac{(-1)^{\frac{n-j-1}{2}} (n-1)!\,(m-2n+2)!!}{j!\,(m-n-j+1)!!\,(n-j-1)!!}\: x^j 
\\=&\textstyle \sum\limits_{\substack{j=0\\ n-j\ \text{odd}}}^{n+1} \!\!\!\frac{(-1)^{\frac{n-j+1}{2}} (n+1)!\,(m-2n-2)!!}{j!\,(m-n-j-1)!!\,(n-j+1)!!}\underbrace{\textstyle
\frac{j(m-2n)+(m-n+1)(n-j+1)}{(n+1)(m-n-j+1)}}_{=1} \,x^j = \p{n+1}{m}(x) .
\end{align*}
\end{proof}

\begin{theorem} \label{pthm}We define, for $m\geq 1$, a linear functional $\L$ on the vector space of polynomials of degree less than $m$, by setting
\begin{equation}
\L: f \longmapsto \int_0^\infty\!\!\! \int_{-\infty}^\infty z^{m-1}f\left(\frac{y}{z}\right) e^{-\frac{y^2+z^2}{2}} dydz.
\end{equation}
Then the $\p{n}{m}$ form a set of orthogonal polynomials with respect to the induced bilinear form, \ie for $k\neq n,\ k\!+\! n\leq m\!-\!1$ we have $\L(\p{k}{m}\p{n}{m})=0$ and for $2n\leq m\!-\!1$:
\begin{equation} \label{Lpn2}
\L(\p{n}{m}\p{n}{m}) = 2^{\frac{3}{2}m-2n-\frac{1}{2}}  \frac{n!}{(m-n)!}
\Gamma\left(\frac{m}{2}-n\right)\Gamma\left(\frac{m}{2}-n+1\right)\Gamma\left(\frac{m+1}{2}\right).
\end{equation}
\end{theorem}
\begin{proof}
Since $\p{n}{m}$ satisfy the three-term relation in Prop.~\ref{threeterm}, by Favard's theorem and Remark \ref{finiteFarvard}, there exists a unique functional $\L'$ with $\L'(1)=1$, such that the $\p{n}{m}$ form an orthogonal basis with respect to the bilinear form induced by $\L'$. We claim that $\L$ is a scalar multiple of $\L'$.
Since $(\p{n}{m})_{n} $ is a basis of the space of polynomials, we must show that, for $n\geq 1$, $\L(\p{n}{m}) = \L(\p{n}{m}\p{0}{m}) =0$.
Using polar coordinates $(y,z) = (r\cos\omega,r\sin\omega)$ and Lemma \ref{trigonometric}, we get:
\begin{align*}
\L(\p{n}{m}) &= \int_0^\pi\!\!\int_0^\infty\p{n}{m}\!\left(\tfrac{\cos\omega}{\sin\omega}\right)\sin(\omega)^{m-1} r^m e^{-\frac{r^2}{2}}dr d\omega \\
&=\int_0^\infty r^m e^{-\frac{r^2}{2}}dr \int_{\frac{\pi}{2}}^{\frac{3\pi}{2}} (-1)^n \p{n}{m}\big(\tan(\omega)\big)\cos(\omega)^{m-1}d\omega\\
&= 2^{\frac{m-1}{2}}\Gamma\left(\tfrac{m+1}{2}\right)\Big[\tfrac{(-1)^n}{n-m}p_{n-1}^{m-2}\big(\tan(\omega)\big)\cos(\omega)^{m-1} \Big]_{\frac{\pi}{2}}^{\frac{3\pi}{2}} =0,
\end{align*}
while $\L(1) = 2^{\frac{m-1}{2}}\sqrt{\pi}\,\Gamma\!\left(\frac{m}{2}\right)=2^{\frac{3m-1}{2}}  \frac{1}{m!}
\Gamma\left(\frac{m}{2}\right)\Gamma\left(\frac{m}{2}+1\right)\Gamma\left(\frac{m+1}{2}\right)$ by (\ref{monoint}) and (\ref{doublegamma}). To verify that equation (\ref{Lpn2}) holds for $n\geq 1$, too, we must show that the right hand side satisfies the recurrence from Theorem~\ref{generalLnorm}, but this is immediate:
$$ 
\frac{2^{\frac{3}{2}m\!-\!2n\!-\!\frac{1}{2}}  \frac{n!}{(m\!-\!n)!}
\Gamma\!\left(\frac{m}{2}\!-\!n\right)\Gamma\!\left(\frac{m}{2}\!-\!n\!+\!1\right)\Gamma\!\left(\frac{m+1}{2}\right)}{
2^{\frac{3}{2}m\!-\!2n\!+\!\frac{3}{2}}  \frac{(n\!-\!1)!}{(m\!-\!n\!+\!1)!}
\Gamma\!\left(\frac{m}{2}\!-\!n\!+\!1\right)\Gamma\!\left(\frac{m}{2}\!-\!n\!+\!2\right)\Gamma\!\left(\frac{m+1}{2}\right)} = \frac{n(m\!-\!n\!+\!1)}{(m\!-\!2n)(m\!-\!2n\!+\!2)} =d_n^m.
$$
\end{proof}
\begin{corollary} \label{pcor}
 $\L(x^{k}\p{n}{m}) = 0$ for $k < n$ and $\L(x^n\p{n}{m})= \L(\p{n}{m}\p{n}{m})$.
\end{corollary}
\begin{proof} By the theorem, we have $\L(x^{0}\p{n}{m}) = 0$ for $n>0$, so the case $k=0$ holds true.
Now the three-term recurrence from Proposition \ref{threeterm} allows us to inductively conclude that $\L(x^k\p{n}{m}) = \L(x^{k-1}\p{n+1}{m})+d_n^m\L(x^{k-1}\p{n-1}{m}) =0$.
The second assertion, $\L(x^n\p{n}{m})= \L(\p{n}{m}\p{n}{m})$ is trivial in the case $n\leq 1$. For $n\geq 1$, the three-term recurrence yields now $\L(x^n\p{n}{m}) = \L(x^{n-1}\p{n+1}{m})+d_n^m\L(x^{n-1}\p{n-1}{m})=d_n^m\L(x^{n-1}\p{n-1}{m})$, so $\L(x^n\p{n}{m})$ and $\L(\p{n}{m}\p{n}{m})$ (by Theorem \ref{generalLnorm}) satisfy the same recurrence relation and therefore must be equal.
\end{proof}



\subsection{Homogeneous orthogonal polynomials} \label{hsection}
We are now ready to give the desired basis of homogeneous polynomials that are orthogonal on the sphere. Inspired by the procedure for spherical harmonics, see \cite[p.~35]{Dunkl}, where homogeneous polynomials were defined by recursion over $d$, we make the following
\begin{definition} \label{hdef}
For multi-indices $\alpha=(\alpha_0,\ldots,\alpha_d)$ we recursively define homogeneous polynomials $h_\alpha$ of degree $|\alpha |$ by $h_{(\alpha_0)}(x) \coloneqq x_0^{\alpha_0}$ and, for $d\geq 1$,
$$
h_\alpha(x) \coloneqq \p{\alpha_d}{2|\alpha |+d}\left(\frac{x_d}{r}\right) r^{\alpha_d}h_{\alpha'}(x'),
$$
where we have set $r=\sqrt{x_0^2 +\ldots+x_{d-1}^2}=\|x'\|$. Note that the definition of $\p{n}{m}$ implies that $\p{n}{m}(\frac{1}{y})y^n$ is an even polynomial, so all square roots vanish.
\end{definition}
\begin{theorem}
For all multi-indices $\alpha,\;\beta$ of length $d+1$ and degree $k$ we have:
\begin{align} \label{hth1}
\bra h_\alpha ,h_\alpha \ket &= \alpha_d!\,\frac{\left(2|\alpha '|\!+\!d\right)!!\,\left(2|\alpha |\!+\!d\!-\!1\right)!!}{\left(|\alpha '|+|\alpha |+d\right)!}\bra h_{\alpha '},h_{\alpha '}\ket ,\\
\label{hth2}
\bra h_\alpha,h_\beta \ket &= 0 \quad\text{for }\alpha\neq\beta,\\
\label{hth3}
\bra x^\alpha, h_\alpha \ket &= \bra h_\alpha ,h_\alpha \ket ,\\
\label{hth4}
\bra x^\alpha,h_\beta \ket &= 0 \quad\text{for }\alpha < \beta.
\end{align}
\end{theorem}
\begin{remark}
This means that the $h_\alpha(x),\ |\alpha|=k$ form an orthogonal basis of $\R[x_0,\ldots,x_d]_k$ that comes from a Gram--Schmidt process applied to the monomials (in lexicographic order). Indeed, equations (\ref{hth1}) and (\ref{hth2}) say that the $h_\alpha(x)$ are orthogonal, while equations (\ref{hth3}) and (\ref{hth4}) imply that the transition matrix $T^{-1}$, defined by $ x^\alpha = \sum_\beta T^{-1}_{\alpha\beta}h_\beta,\ T^{-1}_{\alpha\beta} \coloneqq \frac{\bra x^\alpha,h_\beta\ket}{\bra h_\beta,h_\beta\ket} $ is lower triangular with all diagonal elements equal to $1$.
\end{remark}
\begin{proof}
For equation (\ref{hth1}), we use polar coordinates on $\R ^d$ to compute $\bra h_\alpha ,h_\alpha \ket$:
\begin{align*}
 &\int_{\R^{d+1}}\left[\p{\alpha_d}{2|\alpha|+d}\left(\frac{x_d}{r}\right) r^{\alpha_d}h_{\alpha'}(x)\right]^2 e^{-\frac{1}{2}\|x\|^2} dx \\
 = & \int_0^\infty\!\!\int_{\R} \left[\p{\alpha_d}{2|\alpha|+d}\left(\frac{x_d}{r}\right)\right]^2 r^{2|\alpha'|+2\alpha_d+d-1} e^{-\frac{r^2+x_d^2}{2} }dx_ddr\int_{\S^{d-1}}\left[h_{\alpha'}(\omega)\right]^2 d\omega \\
\myeq{(\ref{Lpn2})}& \  \frac{\alpha_d!\, 2^{2|\alpha '|+|\alpha|+\frac{3}{2}d-\frac{1}{2}} }{\left(|\alpha|+|\alpha'|+d\right)!} { \textstyle
\Gamma\!\left( |\alpha'| \!+\!\frac{d}{2}\right)\Gamma\!\left( |\alpha'|\!+\! \frac{d}{2}\!+\! 1\right) \Gamma\!\left( |\alpha| \!+\! \frac{d+1}{2}\right) }\int_{\S^{d-1}}\!\!\left[h_{\alpha'}(\omega)\right]^2 d\omega \\
\myeq{Lemma~\ref{homosphere}}&\quad\quad \alpha_d!\,2^{|\alpha '|+|\alpha |+d+\frac{1}{2}}\,
\frac{\Gamma\!\left(|\alpha '|\!+\!\frac{d}{2}\!+\!1\right)\Gamma\!\left(|\alpha |\!+\!\tfrac{d+1}{2}\right)}{\left(|\alpha '|+|\alpha |+d\right)!} \int_{\R^d} \left[h_{\alpha'}(x')\right]^2 dx' \\
\myeq{(\ref{ffgamma})}& \quad\; 
\alpha_d!\,\frac{\left(2|\alpha '|\!+\!d\right)!!\,\left(2|\alpha |\!+\!d\!-\!1\right)!!}{\left(|\alpha '|+|\alpha |+d\right)!}\, \sqrt{2\pi}\int_{\R^d} \left[h_{\alpha'}(x')\right]^2 dx' .
\end{align*}
For the proof of (\ref{hth2}), we may assume that $\alpha_d \neq \beta_d$. Then we use the calculation above to see that Thm. \ref{pthm} now implies the vanishing of the integral. Equations (\ref{hth3}) and (\ref{hth4}) follow from Corollary \ref{pcor} in the same way.
\end{proof}
\begin{theorem}\label{thetaCor} Let $D(d,k):=\det\limits_{|\alpha|,|\beta|=k}\bra x^\alpha ,x^\beta \ket $ be the determinant of the Gram matrix of $\bra\ ,\;\ket$. Then:
$$
D(d,k) = \theta_{d,k}
$$
where $\theta_{d,k}$ is defined as in Equation (\ref{thetaDef}).
\end{theorem}
\begin{proof} We do a double induction over $k$ and $d$. First check that $D(d,0) = \theta_{d,0}=1$ and $D(0,k) = \theta_{0,k} =(2k-1)!!$.
From the above theorem it is clear that $D(d,k)=\prod_{|\alpha | =k}\bra h_\alpha ,h_\alpha \ket$. Since $\{ |\alpha| = k \} = \bigcup_{j=0}^k  \{ |\alpha'| = j \} \times \{ k-j \}  $, we have from Equation (\ref{hth1}):
$$
D(d,k) = \prod_{j=0}^k D(d\!-\!1,j) \prod_{|\alpha'| = j} (k-j)!\,\frac{(2j+d)!!\,(2k+d-1)!!}{(j+k+d)!},
$$
hence 
\begin{align*}
R(d,k) \;&\coloneqq\; \frac{D(d,k)}{\prod_{j=0}^k D(d\!-\!1,j) } \;=\; \prod_{j=0}^k \left[\frac{(2j+d)!!\,(2k+d-1)!!}{(j+k+d)!} (k-j)!\right]^{\binom{j+d-1}{d-1}} \\
&\qquad\myeq{(\ref{facprod1})}\; \prod_{j=0}^k\left[ \frac{(2j+d)!!}{(j\!+\!k\!+\!d)!!}\right]^{\binom{j+d-1}{d-1}}(2k\!+\!d\!-\!1)^{\binom{k+d}{d}} \;\prod_{i=1}^k i^{\binom{k-i+d}{d}} .
\end{align*}
% R(d,k)=\frac{\displaystyle\prod_{i=k+d+1}^{2k+d} i^{\binom{i-k-1}{d} }}{\displaystyle \prod_{\substack{i=d+1 \\ i+d\text{ even}}}^{2k+d} i^{\binom{\frac{i+d}{2}-1}{d}
We will now show the principal inductive step: $\frac{D(d,k+1)}{D(d,k)D(d-1,k+1)}=\frac{\theta_{d,k+1}}{\theta_{d,k}\theta_{d-1,k+1}}$. The left hand side clearly equals
\begin{align*}
 \frac{R(d,k+1)}{R(d,k)} &= \frac{(2k\!+\!d\!+\!2)!!^{\binom{k+d}{d-1}}\,(2k\!+\!d\!+\!1)^{\binom{k+d+1}{d}}\,(2k\!+\!d\!+\!1)!!^{\binom{k+d}{d-1}}}{(2k\!+\!d\!+\!2)!^{\binom{k+d}{d-1}} \prod\limits_{j=0}^k(j\!+\!k\!+\!d\!+\!1)^{\binom{j+d-1}{d-1}}} \prod_{i=1}^{k+1} i^{\binom{k-i+d}{d-1}}\\
&= \frac{(2k+d+1)^{\binom{k+d}{d}}}{\prod\limits_{i=k+d+1}^{2k+d+1}i^{ \binom{k-i+d}{d-1}} }\prod_{i=1}^{k+1} i^{\binom{k-i+d}{d-1}}.
\end{align*}
If we split $\theta_{d,k} = A(d,k)B(d,k)$ with $A(d,k)\coloneqq\prod_{i=1}^k i^{\binom{k-i+d}{d}d}$, we see that
\begin{align*}
\frac{A(d,k+1)}{A(d,k)A(d-1,k+1)} &=
 \prod_{i=1}^{k+1}i^{\binom{k-i+d+1}{d}d-\binom{k-i+d}{d}d -\binom{k-i+d}{d-1}(d-1)} \ 
 \myeq{(\ref{binomdiff})}\ \prod_{i=1}^{k+1} i^{\binom{k-i+d}{d-1}},
\end{align*}
while the other factor $B(d,k)$ gives, for even $d$,
\begin{gather*}
\frac{B(d,k+1)}{B(d,k)B(d-1,k+1)} = \frac{ (2k\!+\!d\!+\!1)^{\binom{-k-1}{d}} \prod\limits_{\substack{i=1 \\ i\ \text{odd} }}^{2k+d+1}i^{\binom{k-i+d+1}{d}-\binom{k-i+d}{d}}  }{\prod\limits_{i=1}^{k+\frac{d}{2}} i^{\binom{k-i+d}{d-1}} \prod\limits_{\substack{i=1 \\ i\ \text{even} }}^{2k+d} \left(\frac{i}{2}\right)^{- \binom{k-i+d}{d-1}} } \\
\myeq{(\ref{binomdiff})}\ \frac{ (2k\!+\!d\!+\!1)^{\binom{k+d}{d}}  \prod\limits_{i=1}^{2k+d+1}i^{\binom{k-i+d}{d-1}}  }{ \prod\limits_{i=1}^{k+\frac{d}{2}} i^{\binom{k-i+d}{d-1}} \prod\limits_{\substack{i=1 \\ i\ \text{even} }}^{2k+d} 2^{ \binom{k-i+d}{d-1}} } 
\ \myeq{(\ref{evensum})} \ \frac{ (2k\!+\!d\!+\!1)^{\binom{k+d}{d}} }{ \prod\limits_{i=k+d+1}^{2k+d+1}i^{ \binom{k-i+d}{d-1}} },
\end{gather*}
but also for odd $d$,
\begin{gather*}
\frac{B(d,k+1)}{B(d,k)B(d-1,k+1)} = \frac{ (k+\tfrac{d+1}{2})^{-\binom{-k-1}{d}} \prod\limits_{i=1}^{k+\frac{d+1}{2}} i^{\binom{k-i+d}{d-1}-\binom{k-2i+d}{d-1}}  }{ \prod\limits_{\substack{i=1 \\ i\ \text{odd} }}^{2k+d+1}i^{\binom{k-i+d}{d-1}} } \\
= \frac{ (k+\tfrac{d+1}{2})^{\binom{k+d}{d}} \prod\limits_{i=1}^{k+\frac{d+1}{2}} i^{\binom{k-i+d}{d-1}}  \prod\limits_{\substack{i=1 \\ i\ \text{even} }}^{2k+d} 2^{ \binom{k-i+d}{d-1}}}{ \prod\limits_{i=1}^{2k+d+1}i^{\binom{k-i+d}{d-1}}} 
\ \myeq{(\ref{evensum})} \ \frac{ (2k\!+\!d\!+\!1)^{\binom{k+d}{d}} }{ \prod\limits_{i=k+d+1}^{2k+d+1}i^{ \binom{k-i+d}{d-1}} }.
\end{gather*}
\end{proof}



\section{Application in Hyperk\"ahler geometry} \label{hyper}
Let $X$ be a compact Hyperk\"ahler manifold of complex dimension $2k$. These objects are also called Irreducible Holomorphic Symplectic manifolds. The second cohomology group $H^2(X,\Z)$ comes with an integral quadratic form, called the Beauville--Bogomolov form $q_X$, which can be computed by an integration over some cup--product power, see \cite[Subsection~2.3]{OGrady}:
\begin{equation} \label{fujiki}
\int_X \alpha ^{2k} = (2k-1)!!\,c_X q_X(\alpha)^k,\qquad \alpha\in H^2(X,\Z).
\end{equation}
This equation is referred to as the Beauville--Fujiki relation. The constant $c_X\in\Q$ is chosen such that the quadratic form $q_X$ is indivisible and its signum is such that $q_X(\sigma + \bar{\sigma}) > 0$ for a holomorphic two-form $\sigma$ with $\int_X\sigma\bar{\sigma} = 1$. There is an alternative description, as shown in \cite[Chap.~23]{Huybrechts}. Up to a scalar factor $\tilde{c}$, $q_X$ is equal to:
\begin{equation}\label{bb}
 \tilde{c}\,q_X(\alpha) = \frac{k}{2}\int_X \alpha^2 (\sigma\bar{\sigma})^{k-1} + (1-k)\left(\int_X\alpha\,\sigma^{n-1}\bar{\sigma}^{n}\right)\left(\int_X\alpha\,\sigma^{n}\bar{\sigma}^{n-1}\right).
\end{equation}
Now $q_X$, by polarisation, gives rise to a symmetric bilinear form $\left<\ ,\;\right>$ on $H^2(X,\Z)$, namely $2\left<\alpha,\beta\right> \coloneqq q_X(\alpha+\beta)-q_X(\alpha) -q_X(\beta)$. On the other hand, from (\ref{fujiki}) one deduces (again by polarisation, cf.~\cite[Eq.~3.2.4]{OGrady}) that:
\begin{equation}
 \int_X \alpha_1\wedge\ldots\wedge\alpha_{2k} = c_X \bra \alpha_1\ldots\alpha_k\,,\,\alpha_{k+1}\ldots\alpha_{2k}\ket,
\end{equation}
with the induced form $\bra\ ,\;\ket$ on $\Sym^kH^2(X,\Z)$, according to Definition \ref{formdef}. Since the Poincar\'e pairing $(\beta_1,\beta_2)_X \coloneqq \int_X\beta_1\wedge\beta_2$ gives $H^{2k}(X,\Z)$ the structure of an unimodular lattice, we have got an imbedding of lattices:
\begin{equation}
\Big( \Sym^kH^2(X,\Z),\: c_X\!\bra\ ,\;\ket\Big) \longrightarrow \Big(H^{2k}(X,\Z),\;(\ ,\;)_X\Big).
\end{equation}
From this observation and Theorem \ref{maintheorem}, we deduce some interesting corollaries:
\begin{corollary}
Let $X$ be a compact Hyperk\"ahler manifold of complex dimension $2k$. Denote $h_2$ resp.~$d_2$ the rank and the discriminant of $H^2(X,\Z)$. Then the torsion part of the quotient
$$
\frac{H^{2k}(X,\Z)}{\Sym^kH^2(X,\Z)}
$$
contains no prime factors that are bigger than $2k+h_2-2$ and don't divide neither $c_X$ nor $d_2$. \qed
\end{corollary}
For the known examples of compact Hyperk\"ahler manifolds, we can refine this a bit, using \cite[Table~1]{OGrady}:
\begin{corollary}
 The torsion part of the quotient
$$
\frac{H^{2k}(X,\Z)}{\Sym^kH^2(X,\Z)}
$$
contains no prime factors bigger than
\begin{itemize}
\item $2k+21$, if $X$ is $S^{[k]}$, the Hilbert scheme of $k$ points on a K3 surface $S$,
\item $2k+5$, if $X$ is $K^{[[k]]}$, the generalized Kummer variety of a torus,
\item $16$, if $X$ is the 10--dimensional O'Grady manifold,
\item $6$, if $X$ is the 6--dimensional O'Grady manifold.
\end{itemize} \qed
\end{corollary}
\begin{remark}
The cases $X=S^{[2]}$ and $X=S^{[3]}$ were already studied in \cite[Prop.~6.6]{BNS} and \cite[Prop.~2.4]{Kapfer}, using explicit calculations. The case $X=S^{[2]}$ is particularly nice, because $\Sym^2H^2(S^{[2]},\Z)$ and $H^4(S^{[2]},\Z)$ have the same rank. Since the rank and the discriminant of $H^2(S^{[2]},\Z)$ are $23$ and $-2$, Theorem \ref{maintheorem} implies that the cardinality of the quotient is precisely $\sqrt{2^{24}\,2^{22}(22+3)}$.
\end{remark}

\emph{Acknowledgements.} We thank Samuel Boissi\`ere, Cl\'ement Chesseboeuf and K\'evin Tari for useful conversations and the University of Poitiers for its hospitality. The author was supported by a DAAD grant.

\bibliographystyle{amsplain}
\bibliography{SymBil.tex}


\end{document}
