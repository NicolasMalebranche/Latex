\documentclass{amsart}

\usepackage{amsmath,amssymb,amsfonts}
\usepackage[all]{xy}
\usepackage{appendix,listings,hyperref}

\DeclareMathOperator{\rank}{rk}
\DeclareMathOperator{\trace}{tr}
\DeclareMathOperator{\Aut}{Aut}
\DeclareMathOperator{\End}{End}
\DeclareMathOperator{\id}{id}
\DeclareMathOperator{\Hom}{Hom}
\DeclareMathOperator{\Sym}{Sym}
\DeclareMathOperator{\Hilb}{Hilb}

\newcommand{\hilb}[1]{^{[#1]}}
\newcommand{\ie}{{\it i.e. }}
\newcommand{\eg}{{\it e.g. }}
\newcommand{\loccit}{{\it loc. cit. }}
\newcommand{\vac}{|0\rangle}
\newcommand{\odd}{{\rm{odd}}}
\newcommand{\even}{{\rm{even}}}
\newcommand{\tors}{{\rm{tors}}}

\newcommand{\p}[2]{p_{#1}^{#2}\;\!\!}
\renewcommand{\L}{\mathcal{L}}

\newcommand{\coloneqq}{:=}


%%%%%%%%%%%%%%%%%%%%%%%%%%%%%%

\newcommand{\C}{\mathbb{C}}
\newcommand{\R}{\mathbb{R}}
\newcommand{\Q}{\mathbb{Q}}
\newcommand{\Z}{\mathbb{Z}}
\renewcommand{\H}{\mathbb{H}}

%%%%%%%%%%%%%%%%%%%%%%%%%%%%%

\newcommand{\kS}{\mathfrak{S}}

%%%%%%%%%%%%%%%%%%%%%%%%%%%%%%

\newcommand{\lra}{\longrightarrow}
\newcommand{\ra}{\rightarrow}

%%%%%%%%%%%%%%%%%%%%%%%%%%%%%

\theoremstyle{plain}
\newtheorem{theorem}{Theorem}[section]
\newtheorem{lemma}[theorem]{Lemma}
\newtheorem{proposition}[theorem]{Proposition}
\newtheorem{corollary}[theorem]{Corollary}
\theoremstyle{definition}
\newtheorem{definition}[theorem]{Definition}
\newtheorem{notation}[theorem]{Notation}
\theoremstyle{remark}
\newtheorem{remark}[theorem]{Remark}
\newtheorem{example}[theorem]{Example}


%%%%%%%%%%%%%%%%%%%%%%%%%%%%%

\begin{document}

\title[Symmetric Powers, Hom.~Orth.~Polynomials, Hyperk\"ahlers]{Symmetric Powers of Symmetric Bilinear Forms, Homogeneous Orthogonal Polynomials on the Sphere and an Application in Compact Hyperk\"ahler Manifolds}


\author{Simon Kapfer}
\address{Simon Kapfer, Laboratoire de Math\'ematiques et Applications, UMR CNRS 6086, Universit\'e de Poitiers, T\'el\'eport 2, Boulevard Marie et Pierre Curie, F-86962 Futuroscope Chasseneuil}
\email{simon.kapfer@math.univ-poitiers.fr}
%\urladdr{http://www.math.uni-augsburg.de/alg/}


\date{\today}

%\keywords{}

\begin{abstract} 
We do technical stuff!
\end{abstract}

\maketitle

%%%%%%%%%%%%%%%%%%%%%%%%%%%%%%%%%%%%%%%%%%%%%%%%%%%%%%%%%%%%%%%%%%
%%%%%%%%%%%%%%%%%%%%%%%%%%%%%%%%%%%%%%%%%%%%%%%%%%%%%%%%%%%%%%%%%%
%%%%%%%%%%%%%%%%%%%%%%%%%%%%%%%%%%%%%%%%%%%%%%%%%%%%%%%%%%%%%%%%%%

\section{Homogeneous Orthogonal Polynomials}
\subsection{A polynomial family}
\begin{definition} Let $n,\ m$ be integers with $0\leq 2n\leq m+1$. We define polynomials $\p{n}{m}$ of degree $n$:
\begin{equation}
\p{n}{m}(x) \coloneqq \sum_{\substack{l=0\\ n-l\ \text{even}}}^n \frac{n!\,(m-2n)!!\,(-1)^{\frac{n-l}{2}}}{l!\,(m-n-l)!!\,(n-l)!!}\;x^l.
\end{equation}
\end{definition}
It is straightforward to check that the following two identities hold.
\begin{lemma} \label{trigonometric}
For $n\geq 1$, we have a trigonometric differential relation:
\begin{equation}
\frac{d}{d\omega} \Big[\p{n-1}{m-2}\big(\tan(\omega)\big)\cos(\omega)^{m-1} \Big]= (n-m)\, \p{n}{m}\big(\tan(\omega)\big)\cos(\omega)^{m-1}
\end{equation} \qed
\end{lemma}
\begin{proposition} \label{threeterm} For $0\leq 2n\leq m+1$, we have a three-term recurrence:
\begin{equation}
\p{0}{m}(x) = 1,\qquad \p{1}{m}(x) = x, \qquad \p{n+1}{m}(x) = x\p{n}{m}(x) -d_n^m \p{n-1}{m}(x),
\end{equation} 
where $d_n^m\coloneqq \frac{n(m-n+1)}{(m-2n)(m-2n+2)}$.\qed
\end{proposition}
\begin{theorem} We define, for $m\geq 1$, a linear functional $\L$ on the vector space of polynomials of degree less than $m$, by setting
\begin{equation}
\L: f \longmapsto \int_0^\infty\!\!\! \int_{-\infty}^\infty y^{m-1}f\left(\frac{x}{y}\right)dxdy.
\end{equation}
Then the $\p{n}{m}$ form a set of orthogonal polynomials with respect to the induced bilinear form, \ie $\L(\p{n}{m}\p{n'}{m})=0$ for $n\neq n'$ and 
\begin{equation}
\L(\p{n}{m}\p{n}{m}) = 2^{\frac{3}{2}m-\frac{1}{2}-2n}  \frac{n!}{(m-n)!}
\Gamma\left(\frac{m}{2}-n\right)\Gamma\left(\frac{m}{2}-n+1\right)\Gamma\left(\frac{m+1}{2}\right).
\end{equation}
\end{theorem}
\begin{proof}
Since $\p{n}{m}$ satisfies the three-term relation in Prop.~\ref{threeterm}, by Favard's theorem, there exists a unique functional $\L'$ with $\L'(1)=1$, such that the $\p{n}{m}$ form an orthogonal basis with respect to the bilinear form induced by $\L'$. We will show that $\L$ is a scalar multiple of $\L'$, namely $\L=2^{\frac{m-1}{2}}\Gamma\left(\frac{m+1}{2}\right) \L'$. Since $(\p{n}{m})_{n\leq\frac{m-1}{2}} $ is a basis of our vector space, all we must do is showing that, for $n\geq 1$, $\L(\p{n}{m}) = \L(\p{n}{m}\p{0}{m}) =0$.
Using polar coordinates $(x,y) = (r\cos\omega,r\sin\omega)$ and Lemma \ref{trigonometric}, we get:
\begin{align*}
\L(\p{n}{m}) &= \int_0^\infty r^m e^{-\frac{r^2}{2}}dr \int_{\frac{\pi}{2}}^{\frac{3\pi}{2}} \cos(\omega)^{m-1} p_{n}^{m}\big(\tan(\omega)\big)d\omega\\
&= \Gamma\left(\tfrac{m+1}{2}\right)2^{\frac{m-1}{2}}\Big[p_{n-1}^{m-2}\big(\tan(\omega)\big)\cos(\omega)^{m-1} \Big]_{\frac{\pi}{2}}^{\frac{3\pi}{2}} =0.
\end{align*}
\end{proof}



\bibliographystyle{amsplain}
\bibliography{SymBil.tex}


\end{document}
