
%----------------------------------------------------------------------------------------
%	PACKAGES AND OTHER DOCUMENT CONFIGURATIONS
%----------------------------------------------------------------------------------------

\documentclass[final]{beamer}

\usepackage[size=a1,scale=1.04]{beamerposter} % Use the beamerposter package for laying out the poster

\usetheme{confposter} % Use the confposter theme supplied with this template

\setbeamercolor{block title}{fg=ngreen,bg=white} % Colors of the block titles
\setbeamercolor{block body}{fg=black,bg=white} % Colors of the body of blocks
\setbeamercolor{block alerted title}{fg=white,bg=dblue!70} % Colors of the highlighted block titles
\setbeamercolor{block alerted body}{fg=black,bg=dblue!10} % Colors of the body of highlighted blocks
% Many more colors are available for use in beamerthemeconfposter.sty

%-----------------------------------------------------------
% Define the column widths and overall poster size
% To set effective sepwid, onecolwid and twocolwid values, first choose how many columns you want and how much separation you want between columns
% In this template, the separation width chosen is 0.024 of the paper width and a 4-column layout
% onecolwid should therefore be (1-(# of columns+1)*sepwid)/# of columns e.g. (1-(4+1)*0.024)/4 = 0.22
% Set twocolwid to be (2*onecolwid)+sepwid = 0.464
% Set threecolwid to be (3*onecolwid)+2*sepwid = 0.708

\newlength{\sepwid}
\newlength{\onecolwid}
\setlength{\paperwidth}{36in} % A0 width: 46.8in
\setlength{\paperheight}{24in} % A0 height: 33.1in
\setlength{\sepwid}{0.024\paperwidth} % Separation width (white space) between columns
\setlength{\onecolwid}{0.3\paperwidth} % Width of one column
\setlength{\topmargin}{-0.5in} % Reduce the top margin size
%-----------------------------------------------------------

\usepackage{graphicx}  % Required for including images

\usepackage{booktabs} % Top and bottom rules for tables

\DeclareMathOperator{\rank}{rk}
\DeclareMathOperator{\trace}{tr}
\DeclareMathOperator{\Aut}{Aut}
\DeclareMathOperator{\im}{im}
\DeclareMathOperator{\id}{id}
\DeclareMathOperator{\Hom}{Hom}
\DeclareMathOperator{\Sym}{Sym}
\DeclareMathOperator{\Hilb}{Hilb}
\DeclareMathOperator{\len}{len}
\DeclareMathOperator{\discr}{discr}

\newcommand{\hilb}[1]{^{[#1]}}
\newcommand{\ie}{{\it i.e. }}
\newcommand{\eg}{{\it e.g. }}
\newcommand{\loccit}{{\it loc. cit. }}
\newcommand{\vac}{|0\rangle}
\newcommand{\odd}{{\rm{odd}}}
\newcommand{\even}{{\rm{even}}}
\newcommand{\tors}{{\rm{tors}}}

\newcommand{\p}[2]{p_{#1}^{#2}\;\!\!}
\renewcommand{\L}{\mathcal{L}}

\newcommand{\coloneqq}{:=}
\newcommand{\bra}{\left<\!\!\!\:\left<}
\newcommand{\ket}{\right>\!\!\!\:\right>}
\newcommand{\myeq}[1]{\mathrel{\overset{\makebox[0pt]{\text{\tiny #1}}}{=}}}
\newcommand{\stareq}{\myeq{*}}

%%%%%%%%%%%%%%%%%%%%%%%%%%%%%%

\newcommand{\G}{\mathbb{G}}
\newcommand{\R}{\mathbb{R}}
\newcommand{\Q}{\mathbb{Q}}
\newcommand{\Z}{\mathbb{Z}}
\renewcommand{\S}{\mathbb{S}}
\renewcommand{\H}{\mathbb{H}}

%----------------------------------------------------------------------------------------
%	TITLE SECTION 
%----------------------------------------------------------------------------------------

\title{Aspects of the Beauville--Fujiki relation} % Poster title

\author{Simon Kapfer} % Author(s)

\institute{Universit\'e de Poitiers} % Institution(s)

%----------------------------------------------------------------------------------------

\begin{document}

\addtobeamertemplate{block end}{}{\vspace*{2ex}} % White space under blocks
\addtobeamertemplate{block alerted end}{}{\vspace*{2ex}} % White space under highlighted (alert) blocks

\setlength{\belowcaptionskip}{2ex} % White space under figures
\setlength\belowdisplayshortskip{2ex} % White space under equations

\begin{frame}[t] % The whole poster is enclosed in one beamer frame

\begin{columns}[t] % The whole poster consists of three major columns, the second of which is split into two columns twice - the [t] option aligns each column's content to the top

\begin{column}{\sepwid}\end{column} % Empty spacer column

\begin{column}{\onecolwid} % The first column

\setbeamercolor{block alerted title}{fg=black,bg=norange} % Change the alert block title colors
\setbeamercolor{block alerted body}{fg=black,bg=white} % Change the alert block body colors
\begin{alertblock}{Summary}
For $X$ a compact Hyperk\"ahler manifold, $\dim X=2n$, we construct a form  $\bra\ ,\;\ket$ on $\Sym^nH^2(X)$ from the Beauville--Bogomolov form on $H^2(X)$, such that the
evident embedding: $\Sym^n\!H^2(X) \rightarrow H^{2n}(X)$ becomes metric.
\end{alertblock}

\setbeamercolor{block alerted title}{fg=white,bg=dblue!70} % Colors of the highlighted block titles
\setbeamercolor{block alerted body}{fg=black,bg=dblue!10} % Colors of the body of highlighted blocks

%----------------------------------------------------------------------------------------
%	INTRODUCTION
%----------------------------------------------------------------------------------------


\begin{block}{Introduction}
Let $X$ be a compact Hyperk\"ahler manifold of dimension $2n$. The Beauville--Fujiki relation expresses an integral symmetric bilinear form on $H^2(X,\Z)$, called the Beauville--Bogomolov form, in terms of the Poincar\'e pairing on $H^{2n}(X,\Z)$:
$$
\left< \alpha,\alpha\right> \stareq \left(\int_X \alpha^{2n} \right) ^{\frac{1}{n}} \vspace{6pt}
$$
\emph{Question:} Is there a way to invert this procedure? \vspace{6pt}\\
\emph{Answer:} Yes, on the image of $\Sym^n\!H^2(X)$ in $ H^{2n}(X)$.\vspace{6pt}

\begin{alertblock}{B--F relation, polarized version:}
$$
\int_X \alpha_1\wedge\ldots \wedge\alpha_{2n} \stareq\sum_{\mathcal{P}} \prod_{\{i,j\}\in\mathcal{P}} \left<\alpha_i,\alpha_j\right>.
$$
The sum is over all partitions $\mathcal{P}$ of $\{1,\ldots,2n\}$ into pairs.
\end{alertblock}
We can take this as a general recipe to generate symmetric bilinear forms on symmetric powers!
\vspace{55mm}
\begin{flushright}
${}^*$all equations are meant to hold only up to a constant factor
\end{flushright}
\end{block}

%------------------------------------------------



%\begin{figure}
%\includegraphics[width=0.8\linewidth]{placeholder.jpg}
%\caption{Figure caption}
%\end{figure}

%----------------------------------------------------------------------------------------

\end{column} % End of the first column


\begin{column}{\sepwid}\end{column} % Empty spacer column

\begin{column}{\onecolwid} % Begin a column which is two columns wide (column 2)

\begin{block}{Generalized setting}
Let $V$ be a free module with basis $(x_i)_{0\leq i\leq d}$, equipped with a symmetric bilinear form $\left<\ ,\;\right>$.
On the induced basis of $\Sym^nV$, we define a symmetric bilinear form $\bra\ \,,\ \ket$ by: 
\begin{equation*}
\label{mydef}
\bra x_{k_1}\ldots x_{k_n}\,,\,x_{k_{n+1}}\ldots x_{k_{2n}} \ket \coloneqq \sum_{\mathcal{P}} \prod_{\{i,j\}\in\mathcal{P}} \left<x_{k_i},x_{k_j}\right>,
\end{equation*}
where the sum is over all partitions $\mathcal{P}$ of $\{1,\ldots,2n\}$ into pairs.
\end{block}

\begin{block}{Link to real analysis}
There is an alternative description, for $V=\R^{d+1}$ with the standard scalar product.
For two homogeneous polynomials $h_1(x),\;h_2(x)$ in $d+1$ variables, we have
$$
\bra h_1(x), h_2(x) \ket \stareq \int_{\S^d}  h_1(\omega) h_2(\omega) d\omega ,
$$ 
with an analytic integral over the unit sphere $\S^d$.

Finding a basis of homogeneous polynomials orthogonal on the sphere amounts to understanding a portion of the structure of the Beauville--Fujiki relation! A such orthogonal basis
\begin{itemize}
 \item can be constructed recursively,
 \item admits a computation of the discriminant of $\bra\ ,\;\ket$ in closed form.
\end{itemize}
\end{block}
\begin{alertblock}{Theorem}
Let $a$ be the discriminant of $\left<\ ,\;\right>$ on $V$, where $\rank V=d+1$.
Then the discriminant of $\bra\ ,\;\ket$ on $\Sym^nV $ equals
$$
a^{\binom{d+n}{n}}\, \theta
$$ 
where the factor $\theta$ is integral and contains only prime numbers smaller than $2n+d$.
\end{alertblock}


\end{column} % End of the second column

\begin{column}{\sepwid}\end{column} % Empty spacer column

\begin{column}{\onecolwid} % The third column


\begin{block}{Consequence for compact HK manifolds}
Seen as a lattice, $\Sym^n\!H^2(X,\Z)$ is embedded in the unimodular Poincar\'e lattice $H^{2n}(X,\Z)$. Its discriminant is composed of factors coming from:
\begin{itemize}
 \item The discriminant of the Beauville--Bogomolov form,
\item the Fujiki constant,
\item the combinatorial factor $\theta$.
\end{itemize}
\vspace{-5mm}
\end{block}
\begin{alertblock}{Corollary}
For all known examples $X $ of compact HKM, the quotient
$$
\frac{H^{2n}(X,\Z)}{\Sym^n\!H^2(X,\Z)}
$$
contains no prime torsion factors greater than $2n +b_2-2$.
\end{alertblock}


%----------------------------------------------------------------------------------------
%	ADDITIONAL INFORMATION
%----------------------------------------------------------------------------------------


%----------------------------------------------------------------------------------------
%	REFERENCES
%----------------------------------------------------------------------------------------

\begin{block}{References}

\begin{itemize}
\small
\item S.~Kapfer, \emph{Symmetric Powers of Symmetric Bilinear Forms, Homogeneous Orthogonal Polynomials on the sphere and an application in Compact Hyperk\"ahler Manifolds}, preprint 2015.
\item K.~O'Grady, \emph{Compact Hyperk\"ahler manifolds: general theory} (2013), lecture notes.
\end{itemize}
%\vspace{-1cm}
\end{block}

%----------------------------------------------------------------------------------------
%	ACKNOWLEDGEMENTS
%----------------------------------------------------------------------------------------

%\setbeamercolor{block title}{fg=red,bg=white} % Change the block title color

%----------------------------------------------------------------------------------------
%	CONTACT INFORMATION
%----------------------------------------------------------------------------------------

\setbeamercolor{block title}{fg=red,bg=white} 
\begin{block}{Contact information}
Email: \href{mailto:simon.kapfer@math.univ-poitiers.fr}{simon.kapfer@math.univ-poitiers.fr}
\end{block}

\begin{center}
\begin{tabular}{ccc}
\includegraphics[height=45mm]{logo-Dept-Math-1.png} & \hspace{4cm} & \includegraphics[height=45mm]{logo-cnrs2.jpg}
\end{tabular}
\end{center}

%----------------------------------------------------------------------------------------

\end{column} % End of the third column

\end{columns} % End of all the columns in the poster
\end{frame} % End of the enclosing frame

\end{document}
