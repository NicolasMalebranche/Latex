\documentclass[a0,portrait]{a0poster} 


\usepackage[french]{babel}
\usepackage[T1]{fontenc}
\usepackage[utf8]{inputenc}


%\documentstyle[poster, landscape=true]{article}
\usepackage[nologos]{edposter} 
\usepackage{times}
%\usepackage[ansinew]{inputenc}
\usepackage{overpic}
\usepackage{pstricks}
\usepackage{wrapfig}
\usepackage{pst-grad}         
\usepackage{pst-slpe}
\usepackage[]{graphics}
%\usepackage{fancybox}
%\usepackage[latin1]{inputenc}
\usepackage[usenames]{color}
\usepackage{graphicx}
\usepackage{amsfonts,amssymb,latexsym}
\usepackage{amsmath}
\usepackage{amsthm}
\usepackage{url}
\usepackage{fancybox}
\usepackage{shadow} 
\usepackage{pifont}
\pagestyle{headings}

\newtheorem {definition} {Definition}
\newtheorem {Th} {Theorem}
\newtheorem {Prop} {Proposition}
\newtheorem {Lem} {Lemma}
\newtheorem {Cor} {Corollary}
\newtheorem {req} {Remark}
\newtheorem {ex} {Exemple}
\newtheorem {motivation} {Motivation}
\newcommand\Z{{\mathbb Z}}
\newcommand\N{{\mathbb N}}
\newcommand\R{{\mathbb R}}
\newcommand\C{{\mathbb C}}
\newcommand\m{{\arrowvert}}
\newcommand\n{{\Arrowvert}}
\widowpenalty=10000
\clubpenalty=10000
\raggedbottom

\newcommand{\subtitle}[1]{\section*{#1}}
\newcommand{\norme}[2]{\Vert #1 \Vert_{#2}}


\newcommand{\arrow}[0]{
\raisebox{-0.5ex}{
\resizebox{1.5em}{1em}{
\input{arr2.pic}}} }

\newcommand{\dwnarrow}[0]{\centerline{\raisebox{-0.5ex}{\scalebox{0.5}{\rotatebox{270}{\input{arr2.pic}}}}}}

\title{\color{Sepia}�tude de la respiration humaine}


\def\colsepcolor{RawSienna}
\def\authorcol{RawSienna}
\def\secheadcol{BrickRed}
%\def\SeminarBackgroundCompositeColorSecond{Dandelion}
%\def\SeminarBackgroundCompositeColorFirst{yellow}

\definecolor{firstgrad}{rgb}{0.650,.900,.900}
\definecolor{lastgrad}{rgb}{1,1,1}
%{.950,.990, .990}
\definecolor{titlecolor}{rgb}{0.5,0,0}
\definecolor{brown1}{rgb}{0.5,0.25,0}
\definecolor{brown2}{rgb}{0.5,0.25,0.25}
\def\exbordercolor{Tan}
\def\bordercolor{brown1}



\begin{document}

\begin{pspicture}(0,0)(78,108)
\psframe[linestyle=none,fillstyle=gradient, gradbegin=lastgrad, gradend=lastgrad, gradmidpoint=1](-4.2,-8)(80,111)
\psframe[linewidth=2mm,linecolor=\bordercolor,framearc=0.1](-1.5,-4.7)(78,107.6)  
\end{pspicture}


\vspace{-107cm}
\hspace{1cm}
\begin{minipage}[H]{8cm}\vspace{1cm}\includegraphics[height=6cm]{logo_S2iM.eps}\end{minipage}
\hspace{8cm}
\begin{minipage}[H]{40cm}\vspace{1cm}
\begin{center}
{\VeryHuge{\color{\authorcol}\bf{Modélisation des incertitudes en imagerie médicale.}} }
\vskip.8cm
{\LARGE {\color{\authorcol} C.Chesseboeuf} } 
\vskip.4cm
{\LARGE \it \color{\authorcol} Université de Poitiers, LMA} 
\vskip.2cm
{\large \color{\authorcol} clement.chesseboeuf@math.univ-poitiers.fr} 
\end{center}
\end{minipage}
\hspace{1cm}
\begin{minipage}[H]{5cm}\vspace{0.5cm}
\begin{tabular}{ccc}
\includegraphics[height=4cm]{logo_math.eps} & \hspace{1cm}  &\includegraphics[height=4cm]{logo_poitiers.eps}
\end{tabular}
\end{minipage}




%\maketitle
\vspace{2cm}
%%%%%%%%%%%%%%%%

%% PRIMA LIGNA

\vskip1cm
\large
\begin{multicols}{2}

\begin{center}
\subtitle{Problématique : Appariement, comparaison, d'images obtenues par résonnance magnétique (IRM).}
\textbf{En médecine :}
\begin{itemize}
\item Comparaison pré-opératoire, post-opératoire.
\item Recalage de différentes images dans un même référentiel pour la comparaison.
\item Prédiction des déformations post-opératoires.
\item modélisation de structures en mouvement
\end{itemize}

\end{center}

%%SECONDA COLONNA

\begin{center}
\subtitle{Données brutes : Images IRM.}
\end{center}

\begin{minipage}[H]{15cm}
\begin{center}
\begin{tabular}{ccc}
\includegraphics[height=9cm]{avant.eps} &  \hspace{4cm} & \includegraphics[height=9cm]{apres.eps} \\[.3cm]
{\footnotesize \large Avant} & \hspace{4cm} &{\footnotesize \large Après} \\[.3cm]  
\end{tabular}
\end{center}
\end{minipage}
\begin{minipage}[H]{30cm}
\begin{center}
$[$ \small données de R.Guillevin.\\
    CHU de Poitiers. $]$
\end{center}
\end{minipage}
\\
\textbf{Question:} Peut-on modéliser la transformation qui a eu lieu?
\end{multicols}



%%separation line
\vskip.10cm\noindent
{\color{titlecolor}\rule{\linewidth}{2mm}}
\vskip.7cm

%%%%%%%%%%%%%%%%%%%%
%%SECONDA LIGNA
\vskip.2cm

\begin{multicols}{2}

\subtitle{Représentation des images / Informations utiles. }

Pour construire un problème mathématiques "bien posé" les images doivent être plongées dans un espaces correctement défini. 
\vspace{1em}
\begin{itemize}
\item Courbe et surface. Etape de régionnement/segmentation, extraction de zones connexes et de leurs contours.
\item Représentation de l'image 2D par "élévation" ,\textit{ie} une fonction $f :\R^2 \rightarrow \R$. Méthodes Level-Set
\item Mesures sur $\R^2$ dans $\R^+$ (niveau de gris). Histogramme couleur (perte de l'information spatiale).
\item Réalisation d'un processus aléatoire.
\end{itemize}

\end{multicols}

\vspace{2em}
\begin{center}
\begin{tabular}{cccc}
\includegraphics[height=11cm]{seg.eps} & \hspace{6cm} \includegraphics[height=11cm]{peaks2.eps} \hspace{6cm} & \includegraphics[height=9cm]{histo2.eps} \\[.3cm]
{\footnotesize \large Segmentation} & {\footnotesize \large  Elévation}  &{\footnotesize \large histogramme} \\[.3cm]  
\end{tabular}
\end{center}


%%separation line
\vskip.10cm\noindent
{\color{titlecolor}\rule{\linewidth}{2mm}}
\vskip.7cm

%%%%%%%%%%%%%%%%%%%%
%%SECONDA LIGNA
\vskip.2cm

\begin{multicols}{2}

\subtitle{Déplacer/Déformer les objets. }

Définition d'un ensemble de transformations admissibles : applications flot d'un ensemble de champ de vecteur :
$$
\mathcal{H} = \{ v : [0,1]\times\R^d \rightarrow \R^d \quad \text{ $L^2$ en temps, $\mathcal{C}^1$ en espace}\}.
$$
Equation différentielle décrivant les trajectoires :
$$
\Phi^v(t,x) = x + \int_0^t v(s,\Phi^v(s,x))ds \quad t \in [0,1]\,\, , \,\, x \in \R^d \,\, , \,\, v\in \mathcal{H}.
$$
Génération d'un groupe de difféomorphismes. Contrôle sur l'énergie nécessaire à la déformation :
$$
d(Id,\Phi)=\norme{v}{\mathcal{H}} \quad \text{ ou $v$ est tel que $\Phi=\Phi^v(1,.)$.} 
$$

\vspace{2em}

\subtitle{Modéliser et comparer les images. }

\begin{itemize}
\item Comparer deux nuages de points : distance euclidienne $\sum_{i=1}^n\norme{x_i - y_i}{2}$.
\item Comparer deux mesures : distance de Wasserstein
$$
W(\mu,\nu) = inf_{\gamma \in M(\mu,\nu)} \int \norme{x-y}{2} \,\, \gamma(dx,dy).
$$
\item Mesures de courant : \begin{minipage}[H]{10cm}  
\hspace{5cm}
\begin{center}
\begin{tabular}{c}
 \includegraphics[height=10cm]{circu.eps}\\[.3cm] %\hspace{6cm} & \includegraphics[height=8cm]{nuage.eps} \\[.3cm] 
\end{tabular}
\end{center}      \end{minipage}

\end{itemize}
%\hspace{3cm}
%\begin{center}
%\begin{tabular}{c}
% \includegraphics[height=10cm]{circu.eps}\\[.3cm] %\hspace{6cm} & \includegraphics[height=8cm]{nuage.eps} \\[.3cm]
%{\footnotesize \large Mesure de courant }%  &{\footnotesize \large Nuage de points} \\[.3cm]  
%\end{tabular}
%\end{center}
\end{multicols}


%%separation line
\vskip.10cm\noindent
{\color{titlecolor}\rule{\linewidth}{2mm}}
\vskip.7cm

%%%%%%%%%%%%%%%%%%%%
%%SECONDA LIGNA
\vskip.2cm

\begin{multicols}{2}
\subtitle{Un exemple de problème bien posé.}
Pour l'appariement de deux objet S et C, on minimise une fonctionnelle comportant deux termes, l'un pour l'énergie de la déformation, l'autre pour la qualité de l'appariement :
$$
J(v) = \int_0^1 \norme{v_t}{V}^2 dt \,\, + \,\, d(\Phi_1^v(S),C).
$$

\subtitle{Source.}
J.Glaunès : \textit{ Transport par difféomorphismes de points, de
mesures et de courants pour la comparaison
de formes et l’anatomie numérique.}

\end{multicols}




%%separation line
\vskip.10cm\noindent
{\color{titlecolor}\rule{\linewidth}{2mm}}
%\vskip.7cm

%%%%%%%%%%%%%%%%%%%%
%%SECONDA LIGNA
%%\vskip0.2cm

\subtitle{Méthode numérique (RKHS).}
Soit S un ensemble de données. Si $\,\,k : S \times S \rightarrow \R\,$  est un noyau \textbf {de type positif}, alors  : 
$$
k(x,y) \,\,\, \text{peut toujours s'interpréter comme un produit scalaire entre des représentants de x et y dans un espace de Hilbert.}
$$
\textbf{Exemple : Théorème de Mercer (S métrique compact).}
$$
k(x,y)= \sum_{i \in \N} \lambda_i e_i(x)e_i(y). \quad \text{avec $(e_i)_{i \in \N}$ des vecteurs propres associés à l'opérateur intégrale.}
$$
\begin{center}La méthode est très générale et on profite des propriétés des espaces de Hilbert, gradient, projection...\end{center}






\end{document}
