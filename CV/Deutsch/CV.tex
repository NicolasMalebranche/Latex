\documentclass[11pt,a4paper,sans,english]{moderncv}        % possible options include font size ('10pt', '11pt' and '12pt'), paper size ('a4paper', 'letterpaper', 'a5paper', 'legalpaper', 'executivepaper' and 'landscape') and font family ('sans' and 'roman')
\moderncvstyle{casual}                             % style options are 'casual' (default), 'classic', 'oldstyle' and 'banking'
\moderncvcolor{blue}                               % color options 'blue' (default), 'orange', 'green', 'red', 'purple', 'grey' and 'black'
%\nopagenumbers{}                                  % uncomment to suppress automatic page numbering for CVs longer than one page
\usepackage[utf8]{inputenc}                       % if you are not using xelatex ou lualatex, replace by the encoding you are using
\usepackage[scale=0.75,a4paper]{geometry}
\usepackage[ngerman]{babel}
\usepackage{amsmath,amssymb}
\setlength{\hintscolumnwidth}{2.7cm}
%----------------------------------------------------------------------------------
%            personal data
%----------------------------------------------------------------------------------

\firstname{Simon}
\familyname{Kapfer}
\title{Lebenslauf}
\address{Dorfstr. 46}{89443 Schwenningen}
\mobile{0151/ 5 66 33 67 6}
\email{kapfer.simon@freenet.de}
\photo[4.6cm]{../Bild.jpg}
\begin{document}
\maketitle
\section{Persönliches}
\cventry{Geboren}{2. September 1988}{Lauingen a. d. Donau}{}{}{}
\cventry{Familienstand}{ledig}{}{}{}{}

%\cventry{}{}{}{}{}{}

\section{Akademische Ausbildung}
\cventry{seit Oktober 2013}{Promotionsstudium Mathematik}{Universität Augsburg}{Thema: Automorphismen irreduzibler holomorph symplektischer Mannigfaltigkeiten}{}{}
\cventry{Oktober 2010 -- Dezember 2012}{Masterstudium Mathematik (Gesamtnote: 1,10)}{Universität Augsburg}{Thema der Abschlussarbeit: Berechnungen im Kohomologiering der Hilbertschemata von Punkten auf Flächen}{Nebenfach: Physik}{} % arguments 3 to 6 are optional
\cventry{Oktober 2007 -- September 2010}{Bachelorstudium Mathematik (Gesamtnote: 1,09)}{Universität Augsburg}{Thema der Abschlussarbeit: Die Nullhomotopie der 720--Grad--Drehung}{Nebenfach: Physik}{}
\cventry{Juli 2007}{Allgemeine Hochschulreife (Abiturnote: 1,2)}{Johann--Michael--Sailer--Gymnasium}{Dillingen a. d. Donau}{}{}
\section{Konferenzen und Workshops}
\cventry{Februar 2015}{Workshop on geometry and arithmetic of hyperkähler manifolds}{Hannover}{}{}{}
\cventry{November 2014}{GAGC}{Luminy, Frankreich}{}{}{}
\cventry{Juni 2014}{Géométrie Algébrique en Liberté XXII}{Triest, Italien}{}{}{}
\cventry{Februar 2014}{Winter School: Higher Structures in Algebraic Analysis}{Padua, Italien}{}{}{}
\cventry{Dezember 2013} {Andrejewski Day: Random Partitions in Mathematics and Physics}{Augsburg}{}{}{}
\cventry{Oktober 2013}{Master class: Around Torelli's theorem for K3 surfaces}{Straßburg}{}{}{}
\cventry{September 2013}{Autumn School: Rational curves on algebraic varieties}{Poitiers, Frankreich}{Vortrag: Parametrizing morphisms with extra structures}{}{}
\cventry{Juni 2011}{Blockseminar zu geometrischen Ungleichungen}{Hedersleben}{Vortrag:  Brunn-Minkowski-Ungleichung auf $\mathbb{S}^n$ und $\mathbb{H}^n$, Steiner-Symmetrisierung}{}{}
\section{Lehrerfahrung}
\cventry{WS 13/14}{Tutorium Analysis III}{bei PD Dr. Quast}{}{}{}
\cventry{}{Betreuung und Hilfestellung für Studenten bei der Übungsaufgabenbearbeitung}{}{}{sog. offener Matheraum}{ \url{http://www.math.uni-augsburg.de/matheraum/}}
\cventry{September 2012}{Klausurvorbereitungskurs Analysis I}{bei Prof. Dr. Peter}{}{}{}
\cventry{SS 2012}{Tutorium Analysis I}{bei Prof. Dr. Peter}{}{}{}
\cventry{WS 11/12}{Tutorium Gewöhnliche Differentialgleichungen}{bei Prof. Dr. Blömker}{}{}{}
\cventry{SS 2011}{Tutorium Funktionalanalysis}{bei Prof. Dr. Blömker}{}{}{}
\cventry{WS 10/11}{Tutorium Analysis III}{bei Prof. Dr. Blömker}{}{}{}
\cventry{SS 2010}{Tutorium Analysis II}{bei Prof. Dr. Blömker}{}{}{}
\cventry{WS 09/10}{Tutorium Analysis I}{bei Prof. Dr. Blömker}{}{}{}
\cventry{SS 2009}{Tutorium Elemente der Mathematik}{bei Prof. Dr. Eschenburg}{}{}{}

\section{Praktika und andere Berufserfahrung}
\cventry{April 2013 -- August 2013}{CAD--Softwareentwicklung}{Compositence GmbH}{Leonberg}{}{}
\cventry{Februar 2011 -- April 2011}{Simulation "`Crowd Control"'}{Siemens AG}{München}{}{}
\cventry{Juli 2007 -- September 2007}{VBA--Entwicklung}{B/S/H}{Dillingen a. d. Donau}{}{}

\section{Ehrenamtliches}
\cventry{seit November 2013}{Leitung eines Matheschülerzirkels}{}{Angebot der Uni für Schüler der umliegenden Gymnasien}{interessante Mathematik abseits des Lehrplans}{\url{http://www.math.uni-augsburg.de/schueler/mathezirkel/}}
\cventry{seit Oktober 2013}{Projekt Best MINT}{
Engagement als Mentor}{Angebot der Uni für Erstsemester}{}{}
\cventry{seit Februar 2013}{Pizzaseminar}{Vorträge, ab Februar 2014 auch Organisation}{Seminar von Studenten für Studenten auf freiwilliger Basis}{}{\url{http://pizzaseminar.speicherleck.de/}}
\section{Sprachkenntnisse}
\cventry{Programmier-sprachen}{C++, Haskell, Java, Maple, R, VBA, Python und andere}{}{}{}{}
\cventry{Lebende Sprachen}{Englisch, Französisch}{}{}{}{}
\cventry{Tote Sprachen}{Althebräisch}{(Hebraicum 2010)}{}{}{}
\section{Hobbies}
\cventry{Sport}{Taekwondo}{seit November 2012}{6. Kup}{}{}
\cventry{Musik}{Klavier}{seit 2005}{}{}{}
\cventry{}{Violine}{seit 1998}{Schulorchester von 2000 bis 2007}{}{}
%\makelettertitle
%\makeletterclosing
\end{document}