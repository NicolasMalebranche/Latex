\documentclass[11pt,a4paper,sans,english]{moderncv}        % possible options include font size ('10pt', '11pt' and '12pt'), paper size ('a4paper', 'letterpaper', 'a5paper', 'legalpaper', 'executivepaper' and 'landscape') and font family ('sans' and 'roman')
\moderncvstyle{casual}                             % style options are 'casual' (default), 'classic', 'oldstyle' and 'banking'
\moderncvcolor{blue}                               % color options 'blue' (default), 'orange', 'green', 'red', 'purple', 'grey' and 'black'
%\nopagenumbers{}                                  % uncomment to suppress automatic page numbering for CVs longer than one page
\usepackage[utf8]{inputenc}                       % if you are not using xelatex ou lualatex, replace by the encoding you are using
\usepackage[scale=0.75,a4paper]{geometry}
\usepackage[ngerman]{babel}
\usepackage{amsmath,amssymb}
\setlength{\hintscolumnwidth}{2.7cm}
%----------------------------------------------------------------------------------
%            personal data
%----------------------------------------------------------------------------------

\firstname{Simon}
\familyname{Kapfer}
\title{Dr.~rer.~nat.}
\address{Dorfstr. 46}{89443 Schwenningen}
\mobile{0151/ 5 66 33 67 6}
\email{kapfer.simon@freenet.de}
\social[github][https://github.com/nicolasmalebranche]{github.com/nicolasmalebranche}
\photo[4.6cm]{../Bild.jpg}
\begin{document}
\maketitle
\section{Persönliches}
\cventry{Geboren}{2. September 1988}{Lauingen a. d. Donau}{}{}{}
\cventry{Familienstand}{ledig}{}{}{}{}

%\cventry{}{}{}{}{}{}

\section{Akademische Ausbildung}
\cventry{Oktober 2013 -- Juni 2016}{Promotionsstudium Mathematik (Gesamtnote: 0,7)}{Universität Augsburg, Universit\'e de Poitiers}{Thema der Doktorarbeit: Cohomology of irreducible holomorphically symplectic manifolds}{}{}
\cventry{Oktober 2010 -- Dezember 2012}{Masterstudium Mathematik (Gesamtnote: 1,10)}{Universität Augsburg}{Thema der Abschlussarbeit: Berechnungen im Kohomologiering der Hilbertschemata von Punkten auf Flächen}{Nebenfach: Physik}{} % arguments 3 to 6 are optional
\cventry{Oktober 2007 -- September 2010}{Bachelorstudium Mathematik (Gesamtnote: 1,09)}{Universität Augsburg}{Thema der Abschlussarbeit: Die Nullhomotopie der 720--Grad--Drehung}{Nebenfach: Physik}{}
\cventry{Juli 2007}{Allgemeine Hochschulreife (Abiturnote: 1,2)}{Johann--Michael--Sailer--Gymnasium}{Dillingen a. d. Donau}{}{}

\section{Berufserfahrung und Praktika}
\cventry{2013--2016}{Universität Augsburg}{}{Betreuung und Hilfestellung für Studenten bei der Übungsaufgabenbearbeitung}{sog. offener Matheraum}{\url{http://www.math.uni-augsburg.de/matheraum/}}
\cventry{WS 15/16}{Universität Augsburg}{}{Vorlesung Algebraische Geometrie (interimsweise)}{}{}
\cventry{April 2013 -- August 2013}{Compositence GmbH}{Leonberg}{CAD--Softwareentwicklung mit Open CASCADE, Karbonfasern entlang geodätischer Bahnen}{}{}
\cventry{2009--2016}{Universität Augsburg}{}{Tutorien für verschiedene Mathematik-Vorlesungen}{}{}
\cventry{Februar 2011 -- April 2011}{Siemens AG}{München}{Simulation "`Crowd Control"', Verbesserung bestehender Algorithmen in Java, Visualisierung von Personenströmen}{}{}
\cventry{Juli 2007 -- September 2007}{B/S/H}{Dillingen a. d. Donau}{VBA--Entwicklung und Prüfvorschriften}{}{}


\section{Publikationen}
\cventry{2016}{Integral cohomology of the generalized Kummer fourfold}{zusammen mit G.~Menet}{arXiv:1607.03431}{}{}
\cventry{2015}{Symmetric Powers of Symmetric Bilinear Forms, Homogeneous Orthogonal Polynomials on the Sphere and an Application to Compact Hyperk\"ahler Manifolds}{Communications in Contemporary Mathematics}{arXiv:1507.00157}{}{}
\cventry{2014}{Computing Cup-Products in integral cohomology of Hilbert schemes of points on K3 surfaces}{LMS J.~of Computation and Math.~\textbf{19} (2016)}{arXiv:1410.8398}{}{}


\section{Sprachkenntnisse}
\cventry{Gesprochene Sprachen}{Deutsch (Muttersprache), Englisch (fließend), Französisch (C1), Italienisch (A2), Althebräisch (Hebraicum)}{}{}{}{}
\cventry{Programmier-sprachen}{C++, Haskell, Java, Maple, R, VBA, Python und andere}{}{}{}{}

%\section{Konferenzen und Workshops}
%\cventry{Juni 2016}{Varieties with trivial canonical bundle}{Bedlewo, Polen}{Vortag: Integral cohomology of the generalized Kummer fourfold}{}{}
%\cventry{April 2016}{British Algebraic Geometry meeting}{Edinburgh, Schottland}{}{}{}
%\cventry{März 2016}{Journées de Géométrie Algébrique}{Poitiers, Frankreich}{}{}{}
%\cventry{Juni 2015}{Géométrie Algébrique en Liberté XXIII}{Löwen, Belgien}{}{}{}
%\cventry{Februar 2015}{Workshop on geometry and arithmetic of hyperkähler manifolds}{Hannover}{}{}{}
%\cventry{November 2014}{GAGC}{Luminy, Frankreich}{}{}{}
%\cventry{Juni 2014}{Géométrie Algébrique en Liberté XXII}{Triest, Italien}{Vortrag: Integer cohomology of compact Hyperkähler manifolds}{}{}
%\cventry{Februar 2014}{Winter School: Higher Structures in Algebraic Analysis}{Padua, Italien}{}{}{}
%\cventry{Dezember 2013} {Andrejewski Day: Random Partitions in Mathematics and Physics}{Augsburg}{}{}{}
%\cventry{Oktober 2013}{Master class: Around Torelli's theorem for K3 surfaces}{Straßburg}{}{}{}
%\cventry{September 2013}{Autumn School: Rational curves on algebraic varieties}{Poitiers, Frankreich}{Vortrag: Parametrizing morphisms with extra structures}{}{}
%\cventry{Juni 2011}{Blockseminar zu geometrischen Ungleichungen}{Hedersleben}{Vortrag:  Brunn-Minkowski-Ungleichung auf $\mathbb{S}^n$ und $\mathbb{H}^n$, Steiner-Symmetrisierung}{}{}
%\section{Lehrerfahrung}
%\cventry{SS 2016}{Tutorium Algebraische Geometrie}{bei Prof. Dr. Smirnov}{}{}{}x
%\cventry{SS 2014}{Tutorium Elemente der Mathematik}{bei Prof. Dr. Eschenburg}{}{}{}
%\cventry{WS 13/14}{Tutorium Analysis III}{bei PD Dr. Quast}{}{}{}
%\cventry{September 2012}{Klausurvorbereitungskurs Analysis I}{bei Prof. Dr. Peter}{}{}{}
%\cventry{SS 2012}{Tutorium Analysis I}{bei Prof. Dr. Peter}{}{}{}
%\cventry{WS 11/12}{Tutorium Gewöhnliche Differentialgleichungen}{bei Prof. Dr. Blömker}{}{}{}
%\cventry{SS 2011}{Tutorium Funktionalanalysis}{bei Prof. Dr. Blömker}{}{}{}
%\cventry{WS 10/11}{Tutorium Analysis III}{bei Prof. Dr. Blömker}{}{}{}
%\cventry{SS 2010}{Tutorium Analysis II}{bei Prof. Dr. Blömker}{}{}{}
%\cventry{WS 09/10}{Tutorium Analysis I}{bei Prof. Dr. Blömker}{}{}{}
%\cventry{SS 2009}{Tutorium Elemente der Mathematik}{bei Prof. Dr. Eschenburg}{}{}{}


\section{Stipendien}
\cventry{Oktober 2014 -- August 2015}{Jahresstipendium des DAAD}{Universit\'e de Poitiers}{Frankreich}{}{}

\section{Ehrenamtliches}
\cventry{seit November 2013}{Leitung eines Matheschülerzirkels}{}{Angebot der Uni für Schüler der umliegenden Gymnasien}{interessante Mathematik abseits des Lehrplans}{\url{http://www.math.uni-augsburg.de/schueler/mathezirkel/}}
\cventry{2013 -- 2016}{Projekt Best MINT}{
Engagement als Mentor}{Angebot der Uni für Erstsemester}{}{}
\cventry{2013 -- 2015}{Pizzaseminar}{Vorträge, ab Februar 2014 auch Organisation}{Seminar von Studenten für Studenten auf freiwilliger Basis}{}{\url{http://pizzaseminar.speicherleck.de/}}

\section{Hobbies}
\cventry{Sport}{Taekwondo}{seit November 2012}{4. Kup}{}{}
\cventry{Musik}{Klavier}{seit 2005}{}{}{}
\cventry{}{Violine}{seit 1998}{Schulorchester von 2000 bis 2007}{}{}
\vspace{14pt}
Schwenningen, den \today
%\makelettertitle
%\makeletterclosing
\end{document}
