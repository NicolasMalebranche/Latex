During my PhD, I found several interesting topics that deserve further research.

\begin{enumerate}
\item In \cite{Symmetric}, I studied an algebraic property of the Fujiki formula of IHS manifolds, that led to a lattice-theoretic observation. Let $L$ be a lattice and consider its symmetric power $\Sym^k (L)$. The Fukjiki formula provides a way to naturally define a symmetric bilinear form on this space, giving $\Sym^k(L)$ a lattice structure induced by $L$. I developed a discriminant formula for the discriminant of $\Sym^k(L)$. However, there are other natural symmetric bilinear forms on $\Sym^k(L)$ that have been studied, so the question arises, how to classify them. Naturality here means, that the symmetric form should be compatible with the action of the orthogonal group $O(L)$ of the basic lattice on $\Sym^k(L)$. It is clear that this involves represenation theory of the orthogonal group. Over $\Q$, the irreducible representations of $O(n)$ on $\Sym^k(\Q^n) \cong \Q[x_1,\ldots, x_n]_k$ can be described by means of harmonic polynomials. More precisely, let $\mathcal H^k_n $ be the space of homogeneous harmonic polynomials of degree $k$. Then there there is a decomposition into irreducible representations of $O(n$:
\[
\Sym^k(\Q^n) = \mathcal H^k_n \oplus r^2\mathcal H^{k-2}_n \oplus r^4\mathcal H^{k-4}_n \oplus \ldots,
\]
where $r^2= x_1^2+\ldots+x_n^2$.
This means that giving a natural symmetric bilinear form on $\Sym^k(\Q^n)$ is equivalent to giving a sequence of numbers, corresponding to the direct summands in the decomposition.
If we intersect each summand with $\Sym^k(\Z^n)$, however, the right hand side will not give back the entire $\Sym^k(\Z^n)$, but rather a non-primitive subspace. So the task is to clearly work out the structure of $\Sym^k(\Z^n)$ and to give a generalized discriminant formula, analogous to \cite{Symmetric}. I estimate four months of work for this project.

\item The cohomology of Hilbert schemes of points on surfaces is still not worked out, if one takes coefficients in the integers. Even it is assumed that there is no torsion, a complete answer is missing. A standard tool to describe rational cohomology of Hilbert schemes is given by Nakajima's operators and it is possible to use them for integral coefficients, too. Qin and Wang worked out a description of $H^*(S\hilb{n},\Z)$ in the case that the odd cohomology vanishes and obtain also a partial description for the general case. In \cite{GS} we were able to describe the cohomology $H^*(S\hilb{2},\Z)$ in the case that $S$ is a complex torus. It turned out, that it was only necessary to determine a few divisibility properties for Nakajima operators of odd classes to complete the picture. Since the technique used is completely independent from the fact that $S$ was a torus, there is hope that with a little greater effort it would be possible to get enough information for all $S\hilb{n}$, $n\geq 2$. I think that having a closer look at the standard stratification of the Hilbert scheme will be the right way to attack the problem, if we are only interested in the torsion-free part of cohomology. 
Determining the torsion factors will be considerably harder and could be considered as a second step.

\item In \cite{Universal}, multiplicative classes of Hilbert schemes (such as Chern classes) are studied. 
\end{enumerate}
