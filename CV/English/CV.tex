\documentclass[11pt,a4paper,sans,english]{moderncv}        % possible options include font size ('10pt', '11pt' and '12pt'), paper size ('a4paper', 'letterpaper', 'a5paper', 'legalpaper', 'executivepaper' and 'landscape') and font family ('sans' and 'roman')
\moderncvstyle{casual}                             % style options are 'casual' (default), 'classic', 'oldstyle' and 'banking'
\moderncvcolor{blue}                               % color options 'blue' (default), 'orange', 'green', 'red', 'purple', 'grey' and 'black'
%\nopagenumbers{}                                  % uncomment to suppress automatic page numbering for CVs longer than one page
\usepackage[utf8]{inputenc}                       % if you are not using xelatex ou lualatex, replace by the encoding you are using
\usepackage[scale=0.75,a4paper]{geometry}
%\usepackage[ngerman]{babel}
\usepackage{amsmath,amssymb}
\setlength{\hintscolumnwidth}{2.7cm}
%----------------------------------------------------------------------------------
%            personal data
%----------------------------------------------------------------------------------

\firstname{Simon}
\familyname{Kapfer}
\title{Curriculum vitae}
%\address{Dorfstr. 46}{89443 Schwenningen}
\mobile{+49 151 5 66 33 67 6}
\email{simon.kapfer@math.uni-augsburg.de}
\photo[4.6cm]{../Bild.jpg}
\begin{document}
\maketitle
\section{Personal information}
\cventry{Nationality}{German}{}{}{}{}
\cventry{Date of birth}{2 September 1988}{Lauingen (Donau)}{Germany}{}{}
\cventry{Marital status}{Single}{}{}{}{}

%\cventry{}{}{}{}{}{}

\section{Formation}
\cventry{October 2013 -- Spring 2016}{PhD in Mathematics}{Augsburg University, University of Poitiers}{Subject of PhD thesis: Automorphisms of irreducible holomorphically symplectic manifolds}{}{}
\cventry{October 2010 -- December 2012}{M.~Sc.~in Mathematics}{Augsburg University, final grade: 1.10 (ECTS: A)}{Subject of master thesis: Berechnungen im Kohomologiering der Hilbertschemata von Punkten auf Flächen}{Subsidiary subject: Physics}{} % arguments 3 to 6 are optional
\cventry{October 2007 -- September 2010}{B.~Sc.~in Mathematics}{Augsburg University, final grade: 1.09 (ECTS: A)}{Subject of bachelor thesis: Die Nullhomotopie der 720--Grad--Drehung}{Subsidiary subject: Physics}{}
%\cventry{July 2007}{Allgemeine Hochschulreife (Abiturnote: 1,2)}{Johann--Michael--Sailer--Gymnasium}{Dillingen a. d. Donau}{}{}

\section{Publications}
\cventry{2015}{Symmetric Powers of Symmetric Bilinear Forms, Homogeneous Orthogonal Polynomials on the Sphere and an Application to Compact Hyperk\"ahler Manifolds}{to appear in Communications in Contemporary Mathematics}{arXiv:1507.00157}{}{}
\cventry{2014}{Computing Cup-Products in integral cohomology of Hilbert schemes of points on K3 surfaces}{to appear in LMS Journal of Computation and Mathematics}{arXiv:1410.8398}{}{}

\section{Conferences and Workshops}
\cventry{June 2015}{Géométrie Algébrique en Liberté XXIII}{Leuven, Belgium}{}{}{}
\cventry{February 2015}{Workshop on geometry and arithmetic of hyperkähler manifolds}{Hannover, Germany}{}{}{}
\cventry{November 2014}{GAGC}{Luminy, France}{}{}{}
\cventry{June 2014}{Géométrie Algébrique en Liberté XXII}{Trieste, Italy}{Talk: Integer cohomology of compact Hyperkähler manifolds}{}{}
\cventry{February 2014}{Winter School: Higher Structures in Algebraic Analysis}{Padova, Italy}{}{}{}
\cventry{December 2013} {Andrejewski Day: Random Partitions in Mathematics and Physics}{Augsburg, Germany}{}{}{}
\cventry{October 2013}{Master class: Around Torelli's theorem for K3 surfaces}{Strasbourg, France}{}{}{}
\cventry{September 2013}{Autumn School: Rational curves on algebraic varieties}{Poitiers, France}{Talk: Parametrizing morphisms with extra structures}{}{}
%\cventry{June 2011}{Blockseminar zu geometrischen Ungleichungen}{Hedersleben, Germany}{Talk: Brunn-Minkowski-Ungleichung auf $\mathbb{S}^n$ und $\mathbb{H}^n$, Steiner-Symmetrisierung}{}{}

\section{Teaching}
\cventry{Winter 2015}{Lectures on Algebraic Geometry (on interim basis)}{}{}{}{}
%\cventry{}{Betreuung und Hilfestellung für Studenten bei der Übungsaufgabenbearbeitung}{}{}{sog. offener Matheraum}{ \url{http://www.math.uni-augsburg.de/matheraum/}}
%\cventry{September 2012}{Klausurvorbereitungskurs Analysis I}{bei Prof. Dr. Peter}{}{}{}
\cventry{2009 -- 2016}{Tutorials for various math lectures (regularly)}{including Calculus, ODEs, Functional analysis, Algebraic Geometry}{at Augsburg University}{}{}

\section{Other professional experience}
\cventry{April 2013 -- August 2013}{CAD software development}{Compositence GmbH}{Leonberg, Germany}{}{}
\cventry{February 2011 -- April 2011}{Simulation "Crowd Control"}{Siemens AG}{Munich, Germany}{}{}
%\cventry{Juli 2007 -- September 2007}{VBA--Entwicklung}{B/S/H}{Dillingen a. d. Donau}{}{}

\section{Scholarships}
\cventry{October 2014 -- August 2015}{DAAD Jahresstipendium}{full scholarship}{spent in Poitiers, France}{}{}

\section{Honorary positions}
\cventry{Since 2013}{Direction of a "Mathesch\"ulerzirkel"}{}{special offer of Augsburg University for pupils}{interesting math not taught at school}{\url{http://www.math.uni-augsburg.de/schueler/mathezirkel/}}
\cventry{Since 2013}{Project "Best MINT"}{}{special offer of Augsburg University for freshmen}{}{}
\cventry{2014 -- 2015}{S\'eminaire des doctorants}{}{mathematical seminar organised by PhD students}{}{}
\cventry{2013 -- 2014}{Pizzaseminar}{talks and organisation}{mathematical seminar by volunteers}{}{\url{http://pizzaseminar.speicherleck.de/}}
\section{Languages}
\cventry{Spoken languages}{English, French, German}{}{}{}{}
\cventry{Programming languages}{C++, Haskell, Java, Maple, R, VBA, Python}{}{}{}{}
%\cventry{Tote Sprachen}{Althebräisch}{(Hebraicum 2010)}{}{}{}
%\section{Hobbies}
%\cventry{Sports}{Taekwondo}{since 2012}{}{}{}
%\cventry{Music}{Piano}{since 2005}{}{}{}
%\cventry{}{Violin}{since 1998}{School orchestra from 2000 to 2007}{}{}
%\makelettertitle
%\makeletterclosing
\end{document}