\documentclass[12pt,notheorems,compress,handout,ngerman]{beamer}
\usepackage[ngerman,english]{babel} 
\usepackage{ragged2e}
\usepackage{kurier}
\usepackage[utf8]{inputenc}
\usepackage[T1]{fontenc}
\usepackage{nicefrac}
\usepackage{amsmath,amssymb,mathtools}
\usepackage[protrusion=true,expansion=false]{microtype}
%\usepackage[framed,amsmath,thmmarks,hyperref]{ntheorem}
\RequirePackage[ngerman=ngerman-x-latest]{hyphsubst}
\usepackage{booktabs}
\usepackage{tabto}
\usepackage{tikz-cd}
%\usetikzlibrary{matrix,arrows,decorations.pathmorphing}
\usepackage{array}
\usepackage{textpos}


%\usepackage[natbib=true,style=numeric]{biblatex}
%\usepackage[babel]{csquotes}
%\bibliography{lit}

%\usepackage{hyperref}

\setlength\parskip{\medskipamount}
\setlength\parindent{0pt}

%\theoremseparator{:}
\theoremstyle{plain}  %nonumberplain
%\newtheorem{beh}{Behauptung}
\newtheorem{proposition}{Proposition}
\newtheorem{korollar}{Korollar}
\newtheorem{frage}{Frage}
\newtheorem{theorem}{Theorem}
\newtheorem{satz}{Satz}
\newtheorem{vermutung}{Vermutung}
\theoremstyle{definition}
\newtheorem{definition}{Definition}
\newtheorem{beispiel}{Beispiel}
\newtheorem{bemerkung}{Bemerkung}
\newtheorem{beweis}{Beweis}
%\theorembodyfont{\normalfont}
%\theoremsymbol{\ensuremath{\openbox}}

\newcommand{\lra}{\longrightarrow}
\newcommand{\lhra}{\ensuremath{\lhook\joinrel\relbar\joinrel\rightarrow}}
\newcommand{\thlra}{\relbar\joinrel\twoheadrightarrow}

\newcommand{\Z}{\mathbb{Z}}
\newcommand{\C}{\mathbb{C}}
\newcommand{\N}{\mathbb{N}}
\newcommand{\R}{\mathbb{R}}
\newcommand{\Q}{\mathbb{Q}}
\newcommand{\Hom}{\mathrm{Hom}}
\newcommand{\id}{\mathrm{id}}
\newcommand{\Aut}[1]{\operatorname{Aut}(#1)}
\newcommand{\GL}[1]{\operatorname{GL}(#1)}
\newcommand{\freist}{\_{}\_{}}
\newcommand{\Set}{\mathrm{Set}}
\newcommand{\Grp}{\mathrm{Grp}}
\newcommand{\Vect}{\mathrm{Vect}}
\newcommand{\e}{\'{e}}
\newcommand{\hilb}[1]{^{[#1]}}
\newcommand{\kum}[1]{^{[[#1]]}}
\DeclareMathOperator{\im}{im}
\DeclareMathOperator{\supp}{supp}
\renewcommand{\S}{\mathfrak{S}}

\def\longleadsto{\mathrel{-}\joinrel\leadsto}
\DeclareMathOperator{\Sym}{Sym}
\DeclareMathOperator{\PD}{PD}
\DeclareMathOperator{\Sp}{Sp}
\DeclareMathOperator{\discr}{discr}
\newcommand{\op}{\mathrm{op}}

\title[Ganzzahlige Kohomologie von IHS Varietäten]{Ganzzahlige Kohomologie von irreduziblen holomorph symplektischen Varietäten}
\author[Simon Kapfer,\ \ Universit\"at Augsburg]{Simon Kapfer\vspace{-0.5cm}}
\institute{Universit\"at Augsburg}
\date{29.~Juni 2016\vspace{0.6cm}\\
\includegraphics[scale=0.07]{LogoInstitut}}

%\usetheme{Warsaw}  %Warsaw, Berkeley?
\usetheme{Darmstadt}
\useoutertheme{split}
\usecolortheme{whale}
%\usecolortheme[named=Peach]{structure}
\usefonttheme{serif}
%\usepackage{palatino}
\useinnertheme{rectangles}
%\usepackage{bookman}
%\setbeamercovered{transparent}

\setbeamertemplate{navigation symbols}{}
%\setbeamertemplate{footline}{}
%\setbeamertemplate{headline}{}

%\beamertemplateboldcenterframetitle
%\setbeamerfont{frametitle}{size={\Large}}

\newcommand*\oldmacro{}%
\let\oldmacro\insertshorttitle%


\newenvironment{changemargin}[2]{%
  \begin{list}{}{%
    \setlength{\topsep}{0pt}%
    \setlength{\leftmargin}{#1}%
    \setlength{\rightmargin}{#2}%
    \setlength{\listparindent}{\parindent}%
    \setlength{\itemindent}{\parindent}%
    \setlength{\parsep}{\parskip}%
  }%
  \item[]}{\end{list}}

\newcommand{\slogan}[1]{%
  \begin{center}%
    \setlength{\fboxrule}{2pt}%
    \setlength{\fboxsep}{-3pt}%
    {\usebeamercolor[fg]{item}\fbox{\usebeamercolor[fg]{normal
    text}\parbox{0.9\textwidth}{\begin{center}#1\end{center}}}}%
  \end{center}%
}

\makeatletter
    \newenvironment{withoutheadline}{
        \setbeamertemplate{headline}[default]
        \def\beamer@entrycode{\vspace*{-\headheight}}
    }{}
\makeatother

\newcommand{\hil}[1]{{\usebeamercolor[fg]{item}{#1}}}

\hyphenation{K\"ah-ler-man-nig-fal-tig-keit Hy-per-k\"ah-ler Koh-ho-mo-lo-gie Hodge-struk-tur}

\begin{document}

\setbeameroption{show notes}
\setbeamertemplate{note page}[plain]
\begin{withoutheadline}
\frame{\vspace{4mm}
\titlepage
}

\end{withoutheadline}

\addtobeamertemplate{frametitle}{}{%
\begin{textblock*}{100mm}(0.88\textwidth,-1.52cm)
\includegraphics[scale=0.18]{UniLogoNeg}
\end{textblock*}}
\addtocounter{framenumber}{-1}
\renewcommand*\insertshorttitle{%
  \oldmacro\hfill\insertframenumber\,/\,\inserttotalframenumber\hfill}



\frame[t]{\frametitle{Überblick}
\tableofcontents
}

\section{IHS Mannigfaltigkeiten}
\subsection{Einführung}
\frame[t]{\frametitle{Kählermannigfaltigkeiten mit trivialem kanonischen Geradenbündel}
\begin{satz}[Berger 1955] \justifying
Jede kompakte Kählermannigfaltigkeit mit verschwindender erster Chernklasse läßt sich bis auf endliche Überlagerungen schreiben als Produkt von
\begin{itemize}
 \item komplexen Tori,
 \item Calabi--Yau Mannigfaltigkeiten,
 \item Hyperkählermannigfaltigkeiten.
\end{itemize}
\end{satz}
\pause 
\begin{bemerkung}\justifying
Die Unterscheidung erfolgt anhand der Holonomie\-grup\-pen der zugehörigen Riemannschen Metrik. Der Hyperkähler-Fall entspricht der symplektischen Gruppe $\Sp(n)$.
\end{bemerkung}
}

\frame[t,shrink=11]{\frametitle{IHS Mannigfaltigkeiten}\justifying
Die quaternionale Interpretation der Holonomiegruppe $\Sp(n)$ führt auf die Existenz einer $\mathbb S^2$-Schar komplexer Strukturen, die alle mit der Metrik verträglich sind.
Dies rechtfertigt die Bezeichnung ``Hyperkähler'' für solche Mannigfaltigkeiten.
\pause
\begin{definition}
Eine Kählermannigfaltigkeit $X$ heißt IHS (irreduzibel holomorph symplektisch), wenn sie einfach zusammenhängend ist und $H^0(X,\Omega_X^2)$ von einer nichtdegenerierten holomorphen 2-Form $\sigma$ aufgespannt wird.
\end{definition}
\begin{satz}
Die IHS- und die Hyperkählereigenschaft sind äquivalent.
\end{satz}
\begin{beispiel}
Die zweidimensionalen IHSM sind genau die K3 Flächen.
\end{beispiel}

}

\frame[t]{\frametitle{Hodge-Diamant}
Der Hodge-Diamant einer IHSM hat folgende Gestalt:
$$
\begin{array}{ccccccccc}
 & & & & 1 & & &  &\\
 & & &0&   &0& &  &\\
 & &1& & *& &1& & \\
 &\reflectbox{$\ddots$}& &\vdots&   &\vdots& &\ddots& \\
1&&*& &\hspace{-1mm}*\hspace{-1mm}& &*&&1\\
 &\ddots& &\vdots&   &\vdots& &\reflectbox{$\ddots$}& \\
 & &1& & *& &1& & \\
 & & &0&   &0& &  &\\
 & & & & 1 & & &  &  
\end{array} 
$$
Insbesondere ist die komplexe Dimension stets geradzahlig.
}

\subsection{Beauville--Bogomolov Form}
\frame[t,shrink=11]{\frametitle{Warum sich überhaupt für ganzzahlige Kohomologie interessieren?}
\begin{satz}[Beauville--Bogomolov]\justifying
Die Gruppe $H^2(X,\Z)$ ist frei und mit einer $\Z$-wertigen nicht\-ent\-arteten quadratischen Form $q_X$ ausgestattet.
\end{satz}\justifying
Diese Form ist von überragender Bedeutung für die Theorie:
\begin{itemize}
\item Lokales Torelli-Theorem: Infinitesimale Deformationen von $X$ entsprechen einer bestimmten Klasse von infinitesimalen Deformationen der Gitterstruktur relativ zur Hodgestruktur von $H^2(X,\Z)\subset H^*(X,\C)$
\item Automorphismen von $X$ von endlicher Primordnung können über ihre Wirkungen auf $H^2(X,\Z)$ studiert werden
\item Auch die höheren Kohomologiegruppen werden benötigt
\end{itemize}
% Quotienten (Grégoire)
}


\frame[t]{\frametitle{Beauville--Bogomolov Form}
Die Beauville--Bogomolov Form $q_X$ kann über ein Integral ausgedrückt werden:
\begin{satz}[Fujiki]
$$
q_X(\alpha)^n = c_X \int_X \alpha^{2n},
$$
wobei die Fujiki-Konstante $c_X\in \mathbb R$ nur von $X$ abhängt.
\end{satz}

\pause
\begin{korollar}
Wir erhalten ein Untergitter
$$
\Sym^n (H^2(X,Z)) \subset H^{2n}(X,\Z),
$$
das im allgemeinen aber nicht primitiv ist.
\end{korollar}
}
\subsection{Diskriminantenformel}
\frame[t]{\frametitle{Diskriminantenformel}
\begin{theorem}\justifying
Seien $d+1$ der Rang von $H^2(X,\Z)$ und $c_X$ die Fujiki-Konstante.
Die Diskriminante von $\Sym^n\!H^2(X,\Z)$ ist gleich
\begin{gather*}
\left(\discr \left(H^2(X,\Z)\right)\right)^{\binom{d+n}{d+1}}\cdot c_X^{\binom{d+n}{d}} \cdot \prod_{i=1}^n i^{\binom{n-i+d}{d}d} 
\cdot C, \\
\qquad \text{mit } \ 
C=
\left\{
 \begin{array}{*2{l}p{5cm}}
 \displaystyle\prod_{\substack{i=1 \\ i\ \text{odd}\\\ }}^{2n+d-1}i^{\binom{n-i+d}{d}} &\text{für }d\!+\! 1\text{ ungerade}, \\
 \displaystyle\prod_{i=1}^{n+\frac{d-1}{2}} i^{\binom{n-i+d}{d} - \binom{n-2i+d}{d}} &\text{für }d\! +\! 1\text{ gerade}.
\end{array}
\right.
\end{gather*}
\end{theorem}

}



\section{Beispiele}
\frame[t,shrink=11]{\frametitle{Die bekannten Beispiele}
Nur relativ wenige Beispiele von IHSM sind bisher bekannt:
\begin{itemize}
\item Hilbertschemata $K3\hilb{n}$ von Punkten auf K3 Flächen, $ n \in \N$,
\item verallgemeinerte Kummersche Varietäten $A\kum n$, $n\in \N$,
\item zwei weitere Beispiele in Dimension $6$ bzw.~$10$,
\end{itemize}
beziehungsweise deren Deformationsklassen.

\pause
Es gibt Ansätze, das Konzept IHSM in verschiedene Richtungen zu verallgemeinern:
\begin{itemize}
\item Virtuelle IHSM, die nur als derivierte Kategorie existieren
\item IS Varietäten mit Singularitäten
\end{itemize}

}




\subsection[Hilbertschemata]{Hilbertschemata von Punkten auf Flächen}
\frame[t]{\frametitle{Was sind Hilbertschemata?}
\begin{definition}[oder zumindest eine Idee davon]
Sei $X$ ein $\C$-Schema. Das Hilbertschema $X\hilb n$ von $n$ Punkten auf $X$ ist der Modulraum aller endlichen Unterschemata von $X$ der Länge $n$. 
\end{definition}
Wichtige Fakten:
\begin{itemize}
\item Ist $X$ zweidimensional und nichtsingulär, so ist $X\hilb n$ $2n$-dimensional und nichtsingulär.
\item Damit ist $X\hilb n$ eine (krepante) Auflösung der Singularitäten von $\Sym^n(X)$.
\item Es existiert eine gute Beschreibung der Kohomologie durch Nakajima-Operatoren.
\end{itemize}
}


\frame[t,shrink=11]{\frametitle{Nakajima-Operatoren} \justifying
Sei $\Xi \subset X\hilb n\times X\times X\hilb{n+m}$ das Inzidenzschema $\{(\xi_1,x,\xi_2)\;|\; \xi_1\subset\xi_2,\allowbreak \supp(\xi_2\backslash\xi_1) = x\}$.
Dann sind die Nakajima-Operatoren $\mathfrak q_m(\alpha)$ für $\alpha\in H^*(X,\Q)$ über eine Korrespondenz definiert:

\begin{center}
\begin{tikzcd}[ampersand replacement=\&]
\Xi \arrow[hook]{rr} \& \&  X\hilb n\times X\times X\hilb{n+m} 
\arrow{dl}{pr_1}
\arrow{d}{pr_2}
\arrow{dr}{pr_3} \\
\& X\hilb n \& X \& X\hilb{n+m}
\end{tikzcd}
\end{center}
\begin{align*}
\mathfrak q_m(\alpha) : H^*(X\hilb n, \Q) &\longrightarrow H^{*+2m-2+|\alpha|}(X\hilb{n+m}, \Q) \\
	 y &\longmapsto \PD\left( pr_{3*}\left(\Xi \cap pr_1^*(y)\cdot pr_2^*(\alpha) \right)\right) 
\end{align*}
Mittels dieser Operatoren läßt sich jede Klasse in $H^*(X\hilb n, \Q)$ erzeugen.
}


\frame[t,shrink=13]{\frametitle{Cup-Produkte}
Es gibt effektive Algorithmen, um Multiplikationen in $H^*(X\hilb n, \Q)$ durch Nakajima--Operatoren auszudrücken \cite{lehnsorger}.
Falls $K_X=0$, erhält man elegante Beschreibungen
\begin{itemize}
 \item mit dem Gruppenring der Permutationen (Lehn, Sorger)
 \item oder mittels W-Algebren (Li, Qin und Wang).
\end{itemize}
\begin{bemerkung}
Bei den von uns betrachteten Beispielen ist $H^*(X\hilb n, \Z)$ torsionsfrei, daher genügt es, mit $\Q$-Koeffizienten zu rechnen.
\end{bemerkung}
Falls $X$ eine K3-Fläche ist, so ist $X\hilb n$ eine IHSM.
Von besonderem Interesse sind Produkte von Grad-2-Klassen und man studiert die durch das Cup-Produkt gegebene Einbettung
$$
\Sym^kH^2(X\hilb n,\Z) \subset H^{2k}(X\hilb n,\Z),
$$
etwa, um Aussagen über Automorphismen zu treffen.
}

\frame[t,shrink=8]{\frametitle{Resultate über $H^*(K3\hilb{n},\Z)$}
\begin{beispiel}[Boissi\`ere--Nieper-Wißkrichen--Sarti \cite{BNS}]
$$
\frac{H^4(K3\hilb{2},\Z)}{\Sym^2 H^2(K3\hilb{2},\Z)}  \cong \left(\frac{\Z}{2\Z}\right)^{\oplus 23} \oplus \frac{\Z}{5\Z}
$$
\end{beispiel}
\begin{beispiel}[]
$$
\frac{H^6(K3\hilb{3},\Z)}{\Sym^3 H^2(K3\hilb{3},\Z)}  \cong \Z^{\oplus 254} \oplus\left(\frac{\Z}{3\Z}\right)^{\oplus 230} \oplus \left(\frac{\Z}{36\Z}\right)^{\oplus 22} \oplus \frac{\Z}{72\Z}
$$
\end{beispiel}
\begin{satz}
Für alle $n\geq k+2$ ist der Quotient
$$
 \frac{H^{2k}(K3\hilb{n},\Z)}{\Sym^k H^{2}(K3\hilb{n},\Z)}
$$
ein freier $\Z$-Modul.
\end{satz}
}

\subsection[Verallgemeinerte Kummersche]{Verallgemeinerte Kummersche Varietäten}

\frame[t]{\frametitle{Verallgemeinerte Kummer-Varietäten}
\begin{definition}\justifying
Sei $A$ ein zweidimensionaler komplexer Torus und $A\hilb{n}$ das Hilbertschema. Bezeichne $\Sigma : A\hilb{n}\rightarrow A$ die Summationsabbildung. Dies ist eine Faserung über $A$ und 
die verallgemeinerte Kummersche Varietät $A\kum n$ ist dann definiert als Faser eines Punktes (und daher von Dimension $2n-2$).
\begin{center}
\vspace{-4pt}
\begin{tikzcd}[ampersand replacement=\&]
A\kum n
\arrow[hook]{r}{\theta}
\arrow{d}
\& A\hilb n \arrow{d}{\Sigma}\\
0 \arrow[hook]{r} \& A  
\end{tikzcd}
\end{center}
\end{definition}
\pause
\begin{beispiel}
Für $n=2$ erhält man eine K3-Fläche.
\end{beispiel}
}

\frame[t,shrink=12]{\frametitle{Ein etwas größeres Diagramm}
Der Morphismus $\theta$ paßt in ein etwas größeres Faserdiagramm:
\begin{center}
\begin{tikzcd}[ampersand replacement=\&]
A\kum n
\arrow[hook]{r}{0\times \id}
\arrow[hook,bend left]{rr}{\theta}
\arrow{d}\&
 A\times A\kum n \arrow{d}{pr_1} 
\arrow{r}{\Theta}
\& A\hilb n \arrow{d}{\Sigma}\\
0 \arrow[hook]{r}
\& A  \arrow{r}{n\cdot}  \& A  
\end{tikzcd}
\end{center}
wobei der Mophismus $\Theta$ eine $n^4$-fache Überlagerung wird.

\pause
Zwei Zutaten, um $H^*(A\kum 3,\Z)$ zu bestimmen:
\begin{itemize}
\item Kohomologie der Hilbertschemata und Rückzug entlang $\theta$
\item Ein paar zusätzliche Klassen mit Träger an den $3$-Torsionspunkten von $A$
\end{itemize}
}

\frame[t]{\frametitle{Rückzug vom Hilbertschema}
\begin{satz}
Der Kern von $\theta^* : H^*(A\hilb n ,\Z) \rightarrow H^*(A\kum n, \Z)$
ist das (zweiseitige) Ideal $\mathcal I$ in $H^*(A\hilb n,\Z) $ erzeugt von $H^1$.
\end{satz}
\pause
\begin{beweis} Es genügt, $\Q$-Koeffizienten zu betrachten.
Klarerweise wird $H^1$ auf Null abgebildet, also $\mathcal I  \subset \ker \theta^*$. 
Weil nun der Morphismus 
$$
\Theta :  A\times A\kum n \rightarrow A\hilb n
$$
eine endliche Überlagerung ist, muß $\ker \theta^*$ der Annihilator der Klasse $A\kum n$ in $H^*(A\hilb n,\Q)$ sein. 
Dann kann man zeigen, daß dies gleich dem Ideal erzeugt von $H^1$ ist. \qed
\end{beweis}
}

\frame[t]{\frametitle{Zur Surjektivität des Rückzugs}
\begin{satz}
Für $n=3$ ist $\theta^* : H^*(A\hilb n ,\Z) \rightarrow H^*(A\kum n, \Z)$ surjektiv in allen Graden außer dem mittleren. Dort ist das Bild von $\theta^*$ ein primitives Untergitter von Kodimension $n^4-1$.
\end{satz}
\pause
\begin{vermutung}
Obiger Satz gilt für alle $n$, die Primzahlen sind.
\end{vermutung}
\pause
\begin{proposition}
$$
\frac{H^4(A\kum 3,\Z)}{\Sym^2H^2(A\kum 3,\Z)} \cong \Z^{\oplus 80}\oplus \left(\frac{\Z}{2\Z}\right)^{\oplus 7} \oplus \left(\frac{\Z}{3\Z}\right)^{\oplus 8}
$$
\end{proposition}
Insbesondere ist also $\Sym^2H^2(A\kum 3,\Z) \subset \theta^*(H^4(A\hilb 3,\Z))$ ein Untergitter von vollem Rang.
}


\frame[t]{\frametitle{Weitere Klassen mittleren Grades}
\begin{proposition}[Hassett und Tschinkel]
Die Brian\c con Schemata mit Träger an einem $3$-Torsionspunkt $\tau \in A[3]$ liefern $81$ weitere Klassen $W_\tau \in H^4(A\kum 3,\Z)$, die zusammen mit den schon betrachteten ganz $ H^4(A\kum 3,\Q)$ aufspannen.
\end{proposition}
Darüber hinaus bestimmen Hassett und Tschinkel einige zusätzliche Klassen in $ H^4(A\kum 3,\Q)$.
\pause
\begin{proposition}[mit Gr.~Menet]
Durch die Wirkung der Monodromiegruppe $\Sp A[3] \rtimes A[3]$ auf diese zusätzliche Klassen erhält man eine Basis von $ H^4(A\kum 3,\Z)$ .
\end{proposition}

}

\section[Singuläres]{Singuläre IS Varietäten}
\frame[t,shrink=11]{\frametitle{Anwendung auf Quotienten}

Die Abbildung $-\id$ auf $A$ induziert eine Involution $\iota$ auf $A\kum 3$, welche (gem.~Tari) eine K3-Fläche und $36$ Punkte fixiert.
Sei 
$$
K' \longrightarrow A\kum 3 / \iota
$$
die Aufblasung der K3-Fläche im Quotienten.
\begin{satz}[Menet]
Es ist $K'$ eine irreduzible symplektische V-Mannigfaltigkeit mit Singularitäten von Kodimension vier. Der Isomorphietyp des Beauville--Bogomolov-Gitters $H^2(K',\Z)$ ist $U(3)^{3}\oplus\left(
\begin{array}{cc}
-5 & -4\\
-4 & -5 
\end{array} \right)$, und die Fujiki-Konstante $c_{K'}$ ist gleich $8$.
\end{satz}
}


\appendix

\frame[t]{\frametitle{Das Ende des Vortrags}
\vspace{64pt}
\slogan{Danke für eure Aufmerksamkeit!}
}

  \addtocounter{framenumber}{-1}
%\backupstart
\renewcommand*\insertshorttitle{%
  \oldmacro\hfill}

\frame[t,shrink=8]{\frametitle{Zum Weiterlesen}
%\cite{qinintegral2004}\cite{lehnsorger}\cite{gross2003calabi}\cite{boissiereniepersarti}
\bibliographystyle{plain}
\bibliography{bibl}
}

\end{document}
