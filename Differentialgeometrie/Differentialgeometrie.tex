\documentclass[12pt]{article}
\usepackage[T1]{fontenc}
\usepackage[latin1]{inputenc}
\usepackage[german]{babel}
\usepackage{amsmath}
\usepackage{amsfonts} 
\usepackage{graphicx}
\usepackage{hyperref}
\newcommand*{\R}{\ensuremath{\mathbb{R}}}
\newcommand*{\diff}[1]{\ensuremath{\frac{\partial}{\partial #1}}}
\newcommand*{\ddiff}[1]{\ensuremath{\frac{\partial^2}{\partial #1 ^2}}}

\title{Differentialgeometrie}
\author{Simon Kapfer}
\date{10. Januar 2022}

\begin{document}
\maketitle
\section{Koordinatentransformationen}
\subsection{Polarkoordinaten}
\begin{align*}
E: \R^2 &\longrightarrow \R^2\\
(r,\phi) &\longmapsto (r\cos\phi, r\sin\phi)
\end{align*}
$E$ ist surjektiv, diffbar, eingeschr\"ankt auf den Streifen
$$
\R^{>0}\times (-\pi,\pi) \longrightarrow \R^2 \setminus \{(x,0), x \leq 0\} 
$$
ein Diffeomorphismus auf die geschlitzte Ebene mit Umkehrung 
$$
L : (x,y) \longmapsto \left(\sqrt{x^2+y^2},\ \left\{
\begin{array}{cl}
\arctan(\frac{y}{x}), & x>0\\
\arctan(\frac{y}{x})+ \frac{\pi}{2}, & x<0, y>0\\
\arctan(\frac{y}{x})-\frac{\pi}{2}, & x<0, y<0\\
\frac{\pi}{2} , & x=0, y>0\\
-\frac{\pi}{2} , & x=0, y<0
\end{array}
\right\}\right).
$$
Differentiale / Jacobi-Matrizen:
\begin{align*}
& DE: & \R^2 &\longrightarrow \R^{2\times 2}\\
& & (r,\phi) &\longmapsto \left(
\begin{array}{cc}
\cos\phi & -r\sin\phi\\
\sin\phi & r \cos\phi
\end{array}
\right)\\\\
& DL: & \R^2 \setminus \{(x,0), x \leq 0\}  &\longrightarrow \R^{2\times 2}\\ 
& & (x,y) &\longmapsto\left(
\begin{array}{cc}
\frac{x}{\sqrt{x^2+y^2}} & \frac{y}{\sqrt{x^2+y^2}}\\
\frac{-y}{x^2+y^2} & \frac{x}{x^2+y^2}
\end{array}
\right)
\end{align*}
$DE$ hat vollen Rang f\"ur $r\neq 0$ und $DL$ ist fortsetzbar auf $\R^2\setminus 0$.
Die lokalen Umkehrungen sind:
\begin{align*}
\left(DL\right)^{-1} &= DE\circ L: & (x,y) &\longmapsto  \left(
\begin{array}{cc}
\cos\left(\arctan(\frac{y}{x})\right)& -\sqrt{x^2+y^2}\sin\left(\arctan(\frac{y}{x})\right)\\
\sin\left(\arctan(\frac{y}{x})\right)& \sqrt{x^2+y^2}\cos\left(\arctan(\frac{y}{x})\right)
\end{array}
\right) \\
\left(DE\right)^{-1} &= DL\circ E: & (r,\phi) &\longmapsto  \left(
\begin{array}{cc}
\cos\phi& \sin\phi\\
-\frac{\sin\phi}{r} & \frac{\cos\phi}{r}
\end{array}
\right) 
\end{align*}
\subsection{Pushforward und Pullback}
\begin{align*}
\diff{r}E^*f &= \diff{r} f(r\cos\phi, r\sin\phi) = f_1 \cos\phi + f_2\sin\phi \\
\left(E_*\diff{r}\right) &= \cos\phi \diff{x} + \sin\phi\diff{y} = \frac{x}{\sqrt{x^2+y^2}}\diff{x}+ \frac{y}{\sqrt{x^2+y^2}}\diff{y}
\end{align*}
Pullback von Funktionen geht anhand beliebiger Morphismen.
Pushforward von Differentialoperatoren geht offensichtlich nur dort, wo $E$ ein Diffeo-morphismus ist (siehe die Wurzel im Nenner), sonst unklar, was das Ergebnis sein soll. 
Wenn der Morphismus nicht injektiv ist, ist unklar, was das Ergebnis auf dem Bild sein soll, wenn er nicht surjektiv ist, fehlt was, um ein komplettes Vektorfeld zu erhalten.
Richtungsableitungen an einzelnen Punkten hingegen kann man beliebig nach vorne schieben. Offensichtlich gilt f\"ur eine Richtungsableitung $A$ die Projektionsformel:
$$
E^*\left((E_*A)f\right) = A(E^* f).
$$
H\"ohere Ableitungen gehen im Prinzip genauso, nur die Kompatibilit\"at mit der Komposition von Differentialoperatoren ist kaputt:
\begin{align*}
E_*\left. \ddiff{r}\right|_p &= \cos^2\phi\ddiff{x} + 2 \sin\phi\cos\phi \diff{x}\diff{y} + \sin^2 \phi \ddiff{y}
\\
&= \left.\left(E_*\diff{r}\right)^2\right|_p - \left. \frac{\sin^2\phi\cos\phi}{r}\diff{x} \right|_p + \left. \frac{\sin\phi\cos^2\phi}{r}\diff{y}\right|_p 
\end{align*}
Gradienten (Richtung des steilsten Anstiegs) sind auch inkompatibel mit Koordinatentransformationen:
$$
E_*\left(\nabla E_* x\right) = E_*\left(r\cos\phi\right) = E_*\left[\begin{array}{c} \cos\phi\\-r\sin\phi\end{array}\right]= \left[\begin{array}{c} 
\cos^2\phi +r^2\sin^2\phi\\
(1-r^2) \sin\phi\cos\phi
\end{array}\right] 
\neq \left[\begin{array}{c}1\\0\end{array}\right]
$$
Nur wenn die Jacobische der Transformation orthogonal ist (hier dort, wo $r=0$ ist), bleibt der Gradient erhalten.
\end{document}