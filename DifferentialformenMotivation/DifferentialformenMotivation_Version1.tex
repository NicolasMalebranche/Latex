\documentclass[11pt]{article}
\usepackage[T1]{fontenc}
\usepackage[latin1]{inputenc}
\usepackage[german]{babel}
\usepackage{fourier}  % Use the Adobe Utopia font for the document
\usepackage{amsmath,amsthm,amssymb,amscd,color,graphicx}

% Geschwungene Kleinbuchstaben im Mathemodus (benutze \mathpzc)
\DeclareFontFamily{OT1}{pzc}{}
\DeclareFontShape{OT1}{pzc}{m}{it}{<-> s * [1.10] pzcmi7t}{}
\DeclareMathAlphabet{\mathpzc}{OT1}{pzc}{m}{it}


%Struktur
\newcommand{\point}{\vspace{3mm}\par \noindent \refstepcounter{subsection}{\bf \thesubsection.} }
	\numberwithin{equation}{subsection}
\newcommand{\tpoint}[1]{\vspace{3mm}\par \noindent \refstepcounter{subsection}{\bf \thesubsection.} 
  \numberwithin{equation}{subsection} {\em #1. ---} }
\newcommand{\epoint}[1]{\vspace{3mm}\par \noindent \refstepcounter{subsection}{\bf \thesubsection.} 
  \numberwithin{equation}{subsection} {\em #1.} }
\newcommand{\bpoint}[1]{\vspace{3mm}\par \noindent \refstepcounter{subsection}{\bf \thesubsection.} 
  \numberwithin{equation}{subsection} {\bf\em #1.} }
  
%Abk�rzungen
\newcommand{\N}{\mathbb{N}}
\newcommand{\Z}{\mathbb{Z}}
\newcommand{\Q}{\mathbb{Q}}
\newcommand{\C}{\mathbb{C}}
\newcommand{\R}{\mathbb{R}}
\newcommand{\Cinf}{C^\infty}
\newcommand{\id}{\text{id}}
\renewcommand{\O}{\mathcal{O}\!}
\renewcommand{\S}{\mathfrak{S}}
\newcommand{\T}{\mathcal{T}}
\newcommand{\E}{\mathcal{E}}
\newcommand{\A}{\mathcal{A}}
\newcommand{\Hom}{\mathcal{H}\!\!\!\mathpzc{om}}
\newcommand{\End}{\mathcal{E}\!\!\mathpzc{nd}}
\renewcommand{\d}{d\!}
\newcommand{\del}{\partial}
\newcommand{\delbar}{\overline{\partial}}
\newcommand{\dzbar}{\d\overline{z}}
\newcommand{\diff}[1]{\frac{\partial}{\partial #1}}
\newcommand{\vac}{\left|0\right>}

%\binoppenalty=7000
%\relpenalty=5000 

% Title Page
\title{Warum definiert man das �u�ere Differential so wie man es tut?}
\author{Simon Kapfer}

\begin{document}
\maketitle

%TODO: f�r die Vorstellung: wir machen multilineare Algebra mit einem Extra-Parameter (dem Punkt auf der Mannigfaltigkeit)
%TODO: Koordinatenunabh�ngigkeit im Sinne von "bis auf Diffeomorphie"
\section{Koordinatenunabh�ngigkeit}
F�r $M$ eine abstrakte $\Cinf$--Mannigfaltigkeit hat man keine kanonische Wahl von Koordinatenfunktionen wie im $\R^n$. Jede Wahl von Koordinaten sollte also gleichwertig sein, sofern nur lokal jeder Punkt eindeutig durch seine Koordinaten definiert ist. F�r eine offene Menge im $\R^2$, die hier stets als Beispiel dienen wird, sollen die Koordinaten $x,\ y$ gegen�ber anderen, etwa $u,\ v$ gleichberechtigt sein.

\section{Wie kann man Formen ableiten?}
Aus Analysis II sollte klar sein, da� die Ableitung einer $\Cinf$--Funktion eine Linearform ergibt. Wie geht es nun weiter?
\\Sei der Einfachheit halber $M=\R^2$. Wir definieren eine kleine Verschiebung 
$$ \phi_\epsilon:M \longrightarrow M,\quad p\longmapsto p +\epsilon v $$
f�r eine beliebige Richtung $v \in \R^2$. Die partielle Ableitung einer 1--Form $\omega$ nach $v$ definieren wir durch:
$$ \diff{v} \omega\ :=\  \lim\limits_{\epsilon\rightarrow 0} \frac{\phi_\epsilon^*(\omega) - \omega}{\epsilon}$$
wobei der Pullback f�r $\omega =f \d x + g\d y$ gegeben ist durch:
$$\phi_\epsilon^*(\omega) \ :=\ (f\circ \phi_\epsilon) d(x\circ \phi_\epsilon) + (g\circ \phi_\epsilon)d(y\circ \phi_\epsilon)$$
F�r die speziellen Koordinaten $x$ und $y$ stellen wir fest, da� $ \phi_\epsilon^* (\d x) = \d x$ und damit $\diff{v} \d x = 0 = \diff{v} \d y $ f�r beliebiges $v$. \\
Au�erdem sieht man ein, da� $\diff{v}\omega$ linear von $v$ abh�ngt. Beim �bergang von der partiellen zur totalen Ableitung werden wir deshalb etwas erhalten, was bilinear von zwei Tangentialvektoren abh�ngt. Man rechnet nach:
$$ d\omega = \frac{\del f}{\del x} \d x \odot \d x + \frac{\del f}{\del y}\d y \odot \d x +  \frac{\del g}{\del x} \d x \odot \d y +\frac{\del g}{\del y} \d y \odot \d y $$
(Es steht $\odot$ dabei f�r eine bilineare Verkn�pfung, die wir noch genauer bestimmen werden.)

\section{Die Gleichung $d^2 = 0$}
Die zweite Ableitung der Koordinatenfunktion $x$ ist offensichtlich konstant Null, d. h. es gilt $d (\d x) =0$. Allerdings hatten wir vorher ja vereinbart, da� Koordinatenwahlen keine Rolle spielen d�rfen. Wir m�ssen die Gleichung $d(\d u)=0$ demnach f�r beliebige Koordinaten $u$ fordern. Da eine Koordinate nichts anderes ist als eine $\Cinf$--Funktion (mit nichtverschwindender Ableitung, aber das ist eine offene Bedingung), mu� die Gleichung $d^2=0$ von allen $\Cinf$--Funktionen erf�llt werden. 

\section{Antikommutativit�t}
Beispielsweise k�nnte man $u$ und $v$ so w�hlen, da� $u = xy$. Dann ist $\d u = y\d x + x \d y$ und dementsprechend 
$$ d^2 u \ \ =\ \  0\ \ =\ \ \d y\odot \d x + \d x \odot \d y$$
Analog erh�lt man f�r die Wahl $u = \frac{1}{2}x^2$ die Bedingung $\d x \odot \d x =0$. \\Die Definition der �u�eren Ableitung ergibt sich daher aus der Forderung, nur das zu betrachten, was koordinatenunabh�ngig ist.
\end{document}          
