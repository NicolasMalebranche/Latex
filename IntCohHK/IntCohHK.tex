\documentclass{amsart}

\usepackage{amsmath,amssymb,amsfonts}
\usepackage[all]{xy}

\DeclareMathOperator{\rank}{rank}
\DeclareMathOperator{\trace}{tr}
\DeclareMathOperator{\Tor}{Tor}
\DeclareMathOperator{\Ext}{Ext}
\DeclareMathOperator{\Aut}{Aut}
\DeclareMathOperator{\End}{End}
\DeclareMathOperator{\id}{id}
\DeclareMathOperator{\Hom}{Hom}
\DeclareMathOperator{\im}{Im}
\DeclareMathOperator{\Ker}{Ker}
\DeclareMathOperator{\Sym}{Sym}
\DeclareMathOperator{\Hilb}{Hilb}

\newcommand{\hilb}[1]{^{[#1]}}
\newcommand{\ie}{{\it i.e. }}
\newcommand{\eg}{{\it e.g. }}
\newcommand{\loccit}{{\it loc. cit. }}
\newcommand{\ii}{{\rm i}}
\newcommand{\dual}[1]{{#1}\spcheck}
\newcommand{\abs}[1]{|{#1}|}
\newcommand{\Kummer}[2]{{#2}^{\llbracket#1\rrbracket}}
\newcommand{\vac}{|0\rangle}
\newcommand{\odd}{{\rm{odd}}}
\newcommand{\even}{{\rm{even}}}
\newcommand{\tors}{{\rm{tors}}}

\newcommand{\aG}{{\rm{a}}_G}
\newcommand{\mG}{{\rm{m}}_G}

\newcommand{\One}{1}

% for spectral sequences

\newcommand{\coloneqq}{:=}

\makeatletter
\newcommand{\rmnum}[1]{\romannumeral #1}
\newcommand{\Rmnum}[1]{\expandafter\@slowromancap\romannumeral #1@}
\makeatother


\newcommand{\FI}{F_{\text{\Rmnum{1}}}}
\newcommand{\FII}{F_{\text{\Rmnum{2}}}}
\newcommand{\EI}{\text{}^{\text{\Rmnum{1}}}\!E}
\newcommand{\EII}{\text{}^{\text{\Rmnum{2}}}\!E}
\newcommand{\dI}{\text{}^{\text{\Rmnum{1}}}\!d}
\newcommand{\dII}{\text{}^{\text{\Rmnum{2}}}\!d}

% for total derived functors and hypercohomology

\newcommand{\HH}{\mathbf{H}}
\newcommand{\LL}{\mathbf{L}}
\newcommand{\RR}{\mathbf{R}}

% for equivariant cohomology

\newcommand{\BG}{BG}
\newcommand{\EG}{EG}


%%%%%%%%%%%%%%%%%%%%%%%%%%%%%%

\newcommand{\IC}{\mathbb{C}}
\newcommand{\IR}{\mathbb{R}}
\newcommand{\IQ}{\mathbb{Q}}
\newcommand{\IZ}{\mathbb{Z}}


%%%%%%%%%%%%%%%%%%%%%%%%%%%%%

\newcommand{\kS}{\mathfrak{S}}

\newcommand{\km}{\mathfrak{m}}
\newcommand{\kq}{\mathfrak{q}}

%%%%%%%%%%%%%%%%%%%%%%%%%%%%%%

\newcommand{\lra}{\longrightarrow}
\newcommand{\ra}{\rightarrow}

%%%%%%%%%%%%%%%%%%%%%%%%%%%%%

\theoremstyle{plain}
\newtheorem{theorem}{Theorem}[section]
\newtheorem{lemma}[theorem]{Lemma}
\newtheorem{proposition}[theorem]{Proposition}
\newtheorem{corollary}[theorem]{Corollary}
\theoremstyle{definition}
\newtheorem{definition}[theorem]{Definition}
\newtheorem{notation}[theorem]{Notation}
\theoremstyle{remark}
\newtheorem{remark}[theorem]{Remark}
\newtheorem{example}[theorem]{Example}




%%%%%%%%%%%%%%%%%%%%%%%%%%%%%

\begin{document}

\title[Products in $H^\ast(\Hilb^n(K3), \IZ)$]{Computing Cup-Products in integer cohomology of Hilbert schemes of points on K3 surfaces}


\author{Simon Kapfer}
\address{Simon Kapfer, Lehrstuhl f\"ur Algebra und Zahlentheorie, Universit\"ats\-stra{\ss}e~14, D-86159 Augsburg}
\email{simon.kapfer@math.uni-augsburg.de}
%\urladdr{http://www.math.uni-augsburg.de/alg/}


\date{\today}

%\subjclass{Primary 14J50, Secondary 14C50, 55T10}

%\keywords{}

\begin{abstract} 

We study the images of cup 
products in integer cohomology of the Hilbert scheme of $n$ points on a K3 surface. 
\end{abstract}

\maketitle


%%%%%%%%%%%%%%%%%%%%%%%%%%%%%%%%%%%%%%%%%%%%%%%%%%%%%%%%%%%%%%%%%%
%%%%%%%%%%%%%%%%%%%%%%%%%%%%%%%%%%%%%%%%%%%%%%%%%%%%%%%%%%%%%%%%%%
%%%%%%%%%%%%%%%%%%%%%%%%%%%%%%%%%%%%%%%%%%%%%%%%%%%%%%%%%%%%%%%%%%
\section{Preliminaries}
\begin{definition}
Let $n$ be a natural number. A partition of $n$ is a sequence $\lambda = (\lambda_1\geq\ldots\geq\lambda_k>0)$ of natural numbers such that $\sum_i \lambda_i =n$. It is convenient to write $\lambda = (1^{m_1},2^{m_2},\ldots)$ as a sequence of multiplicities. We define the weight $\|\lambda\| :=\sum m_i i =n$ and the length $|\lambda| := \sum_i m_i =k$. We also define $z_\lambda \coloneqq\prod_i i^{m_i} m_i!$. 
\end{definition}
\begin{definition} \label{SymFun}
Let $\Lambda$ be the ring of symmetric functions. Let $m_\lambda$ and $p_\lambda$ denote the monomial and the power sum symmetric functions. They are both indexed by partitions and form a basis of $\Lambda$. They are linearly related by $p_\lambda = \sum_{\mu} \psi_{\lambda\mu}m_\mu$, the sum being over partitions with the same weight as $\lambda$. It turns out that the base change matrix $(\psi_{\lambda\mu})$ has integral entries, but its inverse $(\psi_{\lambda\mu}^{-1})$ has not. For example, $p_{(2,1,1)} = 2m_{(2,1,1)} + 2m_{(2,2)}+2m_{(3,1)}+m_{(4)} $ but $ m_{(2,1,1)} = \frac{1}{2}p_{(2,1,1)} -\frac{1}{2}m_{(2,2)} -p_{(3,1)}+p_{(4)}$. A method to determine the coefficients $(\psi_{\lambda\mu})$ is given in \cite[Sect. 3.7]{Lascoux}.
\end{definition}
\begin{definition}
For our purposes, a lattice $L$ is a free $\IZ$-module of finite rank, equipped with a non-degenerate symmetric integral bilinear form $B$. The lattice $L$ is called odd, if there exists a $v\in L$, such that $B(v,v)$ is odd, otherwise it is called even. Choosing a base $\{e_i\}_i$ of our lattice, we can write $B$ as a symmetric matrix. $L$ is called unimodular, if the matrix $B$ has determinant $\pm 1$. The difference between the number of positive eigenvalues and the number of negative eigenvalues of $B$ (regarded over $\IR$) is called the signature. If $B$ has both positive and negative eigenvalues, the lattice is called indefinite. For a classification theorem for indefinite lattices, see \cite{milnor1973symmetric}.
\end{definition}
\begin{definition}
Let $S$ be a projective K3 surface. We fix integral bases $\One$ of $H^0(S,\IZ)$, $x$ of $H^4(S,\IZ)$ and $\alpha_1,\ldots ,\alpha_{22}$ of $H^2(S,\IZ)$. The cup product induces a symmetric bilinear form $B_{H^2}$ on $H^2(S,\IZ)$ and thus the structure of a unimodular lattice.
We may extend $B_{H^2}$ to a symmetric non-degenerate bilinear form $B$ on $H^\ast(S,\IZ)$ by setting $ B(\One,\One) = 0,\ B(\One,\alpha_i) = 0,\ B(\One,x) = 1, \ B(x,x) = 0$.
\end{definition}
It turns out that $H^2(S,\IZ)$ has signature -16 and, by the classification theorem for indefinite unimodular lattices, is isomorphic to $U^{\oplus 3}\oplus E_8(-1)^{\oplus 2}$, \ie three times the hyperbolic lattice and two times the negative $E_8$ lattice.
\begin{definition}
$B$ induces a form $B\otimes B$ on $\Sym^2H^\ast(S,\IZ)$. So the cup-product 
\begin{equation*}
\mu : \Sym ^2H^{*}(S,\IZ) \longrightarrow H^\ast(S,\IZ) 
\end{equation*}
induces an adjoint comultiplication $\Delta$ that is coassociative, given by:
\begin{equation*}
\Delta : H^\ast(S,\IZ) \longrightarrow \Sym^2H^\ast(S,\IZ),\quad \Delta = (B\otimes B)^{-1}\mu^TB
\end{equation*}
Note that this does not define a bialgebra structure.
The image of $\One$ under the composite map $\mu(\Delta(\One)) = B(\Delta(\One),\Delta(\One)) = 24 x$, denoted by $e$ is called the Euler Class.
\end{definition}
We denote by $S\hilb{n}$ the Hilbert scheme of $n$ points on $S$, \ie the classifying space of all zero-dimensional closed subschemes of length $n$. $S\hilb{0}$ consists of a single point and $S\hilb{1}=S$. Fogarty proved, that the Hilbert scheme is a smooth variety.
A theorem by Nakajima gives an explicit description of the vector space structure of $H^\ast(S\hilb{n},\IQ)$ in terms of creation operators
$$
\kq_l(\beta) :  H^\ast(S\hilb{n},\IQ) \longrightarrow  H^{\ast+k+l-1}(S\hilb{n+l},\IQ)
,$$ 
where $\beta\in H^k(S,\IQ)$, acting on the direct sum 
$\mathbb{H}:=\bigoplus_n H^\ast(S\hilb{n},\IQ)$. The operators $\kq_l(\beta)$ are linear, commute with each other, and the images of the vacuum vector $\vac$, defined as the generator of $H^0(S\hilb{0},\IQ)\cong\IQ$, under the polynomial algebra generated by the creation operators span $\mathbb{H}$ as a vector space. It is convenient to abbreviate $\kq_{l_1}(\beta)\ldots\kq_{l_k}(\beta)=:\kq_\lambda(\beta)$, where the partition $\lambda$ is composed by the $l_i$. 

An integral basis for $H^\ast(S\hilb{n},\IZ)$ in terms of Nakajima's operators was given by Qin--Wang:
\begin{theorem} \cite[Thm. 5.4.]{QinWang} Let $\km_{\nu,\alpha} := \sum_\rho \psi_{\nu\rho}^{-1}\,\kq(\alpha)$, with coefficients $ \psi_{\nu\rho}^{-1}$ as in Definition \ref{SymFun}. The classes
$$ \frac{1}{z_\lambda} \kq_\lambda(1)\kq_\mu(x)\km_{\nu^1,\alpha_1}\ldots\km_{\nu^{22},\alpha_{22}}\vac,\quad \|\lambda\| +\|\mu\| + \sum_{i=1}^{22}\|\nu^i\| = n
$$ 
form an integral basis for $H^\ast(S\hilb{n},\IZ)$. Here,
$\lambda,\; \mu,\; \nu^i$ are partitions.
\end{theorem}
\begin{notation}
To enumerate the basis of $H^\ast(S\hilb{n},\IZ)$, we introduce the following abbreviation:
$$ \boldsymbol\alpha^{\boldsymbol\lambda} :=
\One^{\lambda^0} \alpha_1^{\lambda^1}\ldots\alpha_{22}^{\lambda^{22}}x^{\lambda^{23}} :=
\frac{1}{z_{\widetilde{\lambda^0}} }
\kq_{\widetilde{\lambda^0}}(\One)\kq_{\lambda^{23}}(x)\km_{\lambda^1,\alpha_1}\ldots\km_{\lambda^{22},\alpha_{22}}\vac
$$
where the partition $\widetilde{\lambda^0}$ is built from $\lambda^0$ by appending sufficiently many Ones, such that $\left\|\widetilde{\lambda^0}\right\| +\sum_{i\geq 1}\left\|\lambda^i\right\| = n $. If $\sum_{n\geq 0}\left\|\lambda^i\right\| > n, $ we put $\boldsymbol\alpha^{\boldsymbol\lambda}=0$. Thus we can interpret $\boldsymbol\alpha^{\boldsymbol\lambda}$ as an element of $H^\ast(S\hilb{n},\IZ)$ for arbitrary $n$. We say that the symbol $\boldsymbol\alpha^{\boldsymbol\lambda}$ is reduced, if $\lambda^0$ contains no Ones. We define also $\left\|\boldsymbol\lambda\right\| := \sum_{n\geq 0}\left\|\lambda^i\right\|$. 
\end{notation}
\begin{lemma}\label{degBound}
Let $\boldsymbol\alpha^{\boldsymbol\lambda}$ represent a class of cohomological degree $2k$. If $\boldsymbol\alpha^{\boldsymbol\lambda}$ is reduced, then $\frac{k}{2}\leq\left\|\boldsymbol{\lambda}\right\| \leq 2k$.
\begin{proof} This is a simple combinatorial observation. The lower bound is witnessed by $x^{(\frac{k}{2})}$ (if $k$ is even)  and the upper bound is witnessed by $\One^{(2^k)}$.
\end{proof}
\end{lemma}
The ring structure of $H^\ast(S\hilb{n}, \IQ)$ has been studied by Lehn and Sorger in \cite{LehnSorger}, where an explicit algebraic model is constructed, which we recall briefly:
\begin{definition} \cite[Sect. 2]{LehnSorger}
Let $\pi$ be a permutation of $n$ letters, written as a sum of disjoint cycles. To each cycle we may associate an element of $A:=H^\ast(S,\IQ)$. This defines an element in $A^{\otimes m}$, $m$ being the number of cycles. So these mappings span a vector space over $\IQ$. The space obtained by taking the direct sum over all $\pi \in S_n$ will be denoted by $A\{S_n\}$. 

To define a ring structure, take two permutations $\pi,\tau$, together with mappings. The result of the multiplication will be the permutation $\pi\tau$, together with a mapping of cycles. To construct the mappings to $A$, look first at the orbit space of the group of permutations $\left<\pi,\tau\right>$, generated by $\pi$ and $\tau$. For each cycle of $\pi, \tau$ contained in one orbit $B$ of $\left<\pi,\tau\right>$, multiply with the associated element of $A$. Also multiply with a certain power of the Euler class $e^g$. Afterwards, apply the comultiplication $\Delta$ repeatedly on the product to get a mapping from the cycles of $\pi\tau$ contained in $B$ to $A$. 
Here the "graph defect" $g$ is defined as follows: Let $u,v,w$ be the number of cycles contained in $B$ of $\pi,\;\tau,\;\pi\tau$, respectively. Then $g:=\frac{1}{2}\left(|B| + 2- u-v-w\right)$. Now follow this procedure for each orbit $B$. 

\end{definition}
The symmetric group $S_n$ acts on $A\{S_n\}$ by conjugation. This action preserves the ring structure. Therefore the space of invariants $A\hilb{n} := \left(A\{S_n\} \right)^{S_n}$ becomes a subring. The main theorem of \cite{LehnSorger} can now be stated:
\begin{theorem} \cite[Thm. 3.2.]{LehnSorger}
The following map is an isomorphism of rings:
\begin{align*}
H^\ast(S\hilb{n},\IQ) & \longrightarrow A\hilb{n} \\
\kq_{n_1}(\beta_1)\ldots \kq_{n_k}(\beta_k) \vac &\longmapsto \sum_{\sigma\in S_n} \sigma a \sigma^{{-}1} 
\end{align*}
with $n_1+\ldots + n_k =n$ and $a\in A\{S_n\} $ corresponds to an arbitrary permutation with $k$ cycles of lengths $n_1,\ldots,n_k$ that are associated to the classes $\beta_1,\ldots,\beta_k \in H^\ast(S,\IQ)$, respectively.
\end{theorem}

Since $H^\text{odd}(S\hilb{n},\IZ) = 0$ and $H^\text{even}(S\hilb{n},\IZ)$ is torsion-free by \cite{Markman}, we can apply these results to $H^\ast(S\hilb{n}, \IZ)$ to determine the multiplicative structure of cohomology with integer coefficients. It turns out, that it is somehow independent of $n$. More precisely, we have the following stability theorem, derived from Li, Qin and Wang:
\begin{theorem} \cite[Thm.~2.1]{QinWang}
Let $Q_1,\ldots,Q_s$ be products of creation operators, \ie $Q_i = \prod_j \kq_{\lambda_{i,j}}(\beta_{i,j}) $ for some partitions $\lambda_{i,j}$ and classes $\beta_{i,j}\in H^\ast(S,\IZ)$. Set $n_i := \sum_j \left\|\lambda_{i,j}\right\|$.
Then the cup product 
$ \prod_{i=1}^s \left(\frac{1}{(n-n_i)!} \kq_{n-n_i}(1) \,Q_i \,\vac \right)$ is equal to a finite linear combination of classes of the form $\frac{1}{(n-m)!}\kq_{n-m}(1)\prod_j \kq_{\mu_{j}}(\gamma_{j})\vac$, with $\gamma\in H^\ast(S,\IZ)$, $m=\sum_j\left\|\mu_j\right\|$, whose coefficients are independent of $n$. We have the upper bound $m\leq\sum_i n_i$. 
\end{theorem}
\begin{corollary} \label{stabCor} Let $\boldsymbol{\alpha}^{\boldsymbol{\lambda}},\boldsymbol{\alpha}^{\boldsymbol{\mu}},\boldsymbol{\alpha}^{\boldsymbol{\nu}}$ be reduced. Assume $n\geq\left\|\boldsymbol\lambda\right\|,\left\|\boldsymbol\mu\right\| $. Then the coefficients $c^{\boldsymbol{\lambda\mu}}_{\boldsymbol{\nu}}$ of the cup product in $H^\ast(S\hilb{n},\IZ)$
$$\boldsymbol{\alpha}^{\boldsymbol{\lambda}} \smile
\boldsymbol{\alpha}^{\boldsymbol{\mu}}
= \sum_{\boldsymbol{\nu}} c^{\boldsymbol{\lambda\mu}}_{\boldsymbol{\nu}} \boldsymbol{\alpha}^{\boldsymbol{\nu}}
$$  
are polynomials in $n$ of degree at most $ \left\|\boldsymbol\lambda\right\|+\left\|\boldsymbol\mu\right\|-\left\|\boldsymbol\nu\right\|$.
\end{corollary}
\begin{proof} Set $Q_{\boldsymbol{\lambda}}:=  \kq_{\lambda^0}(1)\kq_{\lambda^{23}}(x)\prod_{1\leq j\leq 22}\kq_{\lambda^j}(\alpha_j)$ and $n_{\boldsymbol\lambda}:=\left\|\boldsymbol\lambda\right\|$. Then we have:
$\boldsymbol{\alpha}^{\boldsymbol{\lambda}} = \frac{1}{(n-n_{\boldsymbol\lambda})!\,z_{\lambda^0}}\kq_{n-n_{\boldsymbol\lambda}}(1)Q_{\boldsymbol\lambda}\vac$ and $
\boldsymbol{\alpha}^{\boldsymbol{\mu}}=\frac{1}{(n-n_{\boldsymbol\mu})!\,z_{\mu^0}}\kq_{n-n_{\boldsymbol\mu}}(1)Q_{\boldsymbol\mu}\vac $. 
Thus the coefficient $c^{\boldsymbol{\lambda\mu}}_{\boldsymbol{\nu}}$ in the product expansion is a constant, which depends on $ \left\|\boldsymbol\lambda\right\|,\left\|\boldsymbol\mu\right\|,\left\|\boldsymbol\nu\right\|$, but not on $n$, multiplied with $\frac{(n-n_{\boldsymbol\nu})!}{(n-m)!}$ for a certain $m\leq n_{\boldsymbol\lambda}+n_{\boldsymbol\mu}$. 
This is a polynomial of degree $m-n_{\boldsymbol\nu}\leq n_{\boldsymbol\lambda}+n_{\boldsymbol\mu}-n_{\boldsymbol\nu} =\left\|\boldsymbol\lambda\right\|+\left\|\boldsymbol\mu\right\|-\left\|\boldsymbol\nu\right\| $.
\end{proof}
\begin{remark}
The above condition,  $n\geq\left\|\boldsymbol\lambda\right\|,\left\|\boldsymbol\mu\right\| $,
seems to be unnecessary. This implies, that if $n<\left\|\boldsymbol\lambda\right\|$ or $n<\left\|\boldsymbol\nu\right\|$ but $n\geq\left\|\boldsymbol\nu\right\|$, the polynomial $ c^{\boldsymbol{\lambda\mu}}_{\boldsymbol{\nu}}$ is zero at $n$.
\end{remark}
\begin{example}\label{example} We now give some explicit examples for illustration.
\begin{enumerate} \item $
\One^{(2,2)}\smile \alpha_i^{(2)} = 2\cdot \One^{(2)}\alpha_i^{(1)}x^{(1)} + \One^{(2,2)}\alpha_i^{(2)} + 2\cdot\One^{(2)}\alpha_i^{(3)} +\alpha_i^{(4)} $
\item \label{oddWitness} Let $i,j$ be indices, such that $B(\alpha_i,\alpha_j)=1,\ B(\alpha_i,\alpha_i)=0=B(\alpha_j,\alpha_j)$ and let $k\geq 0$. Set $\boldsymbol{\alpha}^{\boldsymbol{\lambda}} := \alpha_i^{(1)}\alpha_j^{(1)}x^{(1^k)}$ 
and $\boldsymbol{\alpha}^{\boldsymbol{\nu}}:= x^{(1^{2k})}$. Then $c^{\boldsymbol{\lambda\lambda}}_{\boldsymbol{\nu}} =1$.
\end{enumerate}
\end{example}
\section{Computational results} 
We now give some results in low degrees, obtained by computing multiplication matrices with respect to the integral basis and a then reducing to Smith normal form (both done by a computer).
\begin{remark}
Denote $h^k(S\hilb{n})$ the rank of $H^k(S\hilb{n},\IZ)$. We have:
\begin{itemize}
\item $h^2(S\hilb{n}) = 23 $ for $n\geq 2$.
\item $h^4(S\hilb{n}) = 276,\; 299,\; 300$ for $n=2,3, \geq 4$ resp.
\item $h^6(S\hilb{n}) = 23,\; 2554,\; 2852,\; 2875,\; 2876$ for $n=2,3,4,5,\geq6$ resp.
\end{itemize}
\end{remark}
The algebra generated by classes of degree 2 is an interesting object to study. For cohomology with complex coefficients, Verbitsky has proven in \cite{Verbitsky} that the cup product mapping from $\Sym^k H^2(S\hilb{n},\IC)$ to $H^{2k}(S\hilb{n},\IC)$ is injective for $k\leq n$. One concludes that this also holds for integral coefficients.
\begin{proposition} We identify $\Sym^2H^2(S\hilb{n},\IZ)$ with its image in $H^4(S\hilb{n},\IZ)$ under the cup product mapping. Then: 
\begin{align}
\frac{H^4(S\hilb{2},\IZ)}{\Sym^2 H^2(S\hilb{2},\IZ)} & \cong \left(\frac{\IZ}{2\IZ}\right)^{\oplus 23} \oplus \frac{\IZ}{5\IZ},\\
\frac{H^4(S\hilb{3},\IZ)}{\Sym^2 H^2(S\hilb{3},\IZ)} & \cong \frac{\IZ}{3\IZ} \oplus \IZ^ {\oplus 23}, \\
\frac{H^4(S\hilb{n},\IZ)}{\Sym^2 H^2(S\hilb{n},\IZ)} & \cong  \IZ^ {\oplus 24}, \quad \text{for }n\geq 4.
\end{align}
The 3-torsion part is generated by the integral class $1^{(3)}$.
\end{proposition}
\begin{remark}
The torsion in the case $n=2$ was also computed by Boissi\`{e}re, Nieper-Wi\ss kirchen and Sarti, \cite[Prop. 3]{BNS} using similar techniques.
The result for $n=3$ seems to be new.
The freeness result for $n\geq 4$ was already proven by Markman, \cite[Thm. 1.10]{Markman2}, using a completely different method. 
\end{remark}
\begin{proposition} For triple products of $H^2(S\hilb{n},\IZ)$, we have:
$$
\frac{H^6(S\hilb{2},\IZ)}{\Sym^3 H^2(S\hilb{2},\IZ)} \cong 
\frac{\IZ}{2\IZ}
$$
The quotient is generated by the integral class $\One^{(2)}$.
$$
\frac{H^6(S\hilb{3},\IZ)}{\Sym^3 H^2(S\hilb{3},\IZ)} \cong  \left(\frac{\IZ}{2\IZ}\right)^{\oplus 230}\oplus \left(\frac{\IZ}{36\IZ}\right)^{\oplus 22}\oplus \frac{\IZ}{72\IZ} \oplus \IZ^{\oplus 507}
$$
$$
\frac{H^6(S\hilb{4},\IZ)}{\Sym^3 H^2(S\hilb{4},\IZ)} \cong  \frac{\IZ}{2\IZ} \oplus \IZ^{\oplus 552}
$$
For $n\geq 5$, the quotient is free.
\end{proposition}
We study now cup products between classes of degree 2 and 4. The case of $S\hilb{3}$ is of particular interest.
\begin{proposition} \label{p24}The cup product mapping $ : H^2(S\hilb{n},\IZ)\otimes H^4(S\hilb{n},\IZ) \rightarrow H^6(S\hilb{n},\IZ) $ is neither injective (unless $n=0$) nor surjective (unless $n\leq 2$). We have:
\begin{align} 
\frac{H^6(S\hilb{3},\IZ)}{H^2(S\hilb{3},\IZ)\smile H^4(S\hilb{3},\IZ)} &\cong \left(\frac{\IZ}{3\IZ}\right)^{\oplus 22} \oplus \frac{\IZ}{3\IZ}
\\
\frac{H^6(S\hilb{4},\IZ)}{H^2(S\hilb{4},\IZ)\smile H^4(S\hilb{4},\IZ)} &\cong  \left(\frac{\IZ}{6\IZ}\right)^{\oplus 22}\oplus\frac{\IZ}{108\IZ} \oplus\frac{\IZ}{2\IZ} 
\\
\frac{H^6(S\hilb{5},\IZ)}{H^2(S\hilb{5},\IZ)\smile H^4(S\hilb{5},\IZ)} &\cong 
 \IZ^{\oplus 22} \oplus \IZ
\\
\frac{H^6(S\hilb{n},\IZ)}{H^2(S\hilb{n},\IZ)\smile H^4(S\hilb{n},\IZ)} &\cong 
 \IZ^{\oplus 22} \oplus \IZ\oplus\IZ, \ n\geq 6.
\end{align}
In each case, the first 22 parts of the quotient are generated by the integral classes 
 $$
\alpha_i^{(1,1,1)} -3\cdot \alpha_i^{(2,1)} + 3\cdot \alpha_i^{(3)}+ 3 \cdot \One^{(2)}\alpha_i^{(1,1)} -6\cdot \One^{(2)}\alpha_i^{(2)}+6\cdot \One^{(2,2)}\alpha_i^{(1)}-3\cdot \One^{(3)}\alpha_i^{(1)},
$$ 
for $ i=1\ldots 22$. Now define an integral class
\begin{align*}
K:=&\;\sum_{i\neq j} B(\alpha_i,\alpha_j)\left[\alpha_i^{(1,1)}\alpha_j^{(1)} - 2\cdot\alpha_i^{(2)}\alpha_j^{(1)}+\frac{3}{2}\cdot \One^{(2)}\alpha_i^{(1)}\alpha_j^{(1)} \right] +\\
+&\;\sum_{i}B(\alpha_i,\alpha_i)\left[\alpha_i^{(1,1,1)} - 2\cdot\alpha_i^{(2,1)} + \frac{3}{2}\cdot \One^{(2)}\alpha_i^{(1,1)} \right]+  x^{(2)}-\One^{(2)}x^{(1)}.
\end{align*} 
In the case $n=3$, the last part of the quotient is generated by $K$. 
\\In the case $n=4$, the class $ \One^{(4)}$ generates the 2-torsion part and $K+38\cdot\One^{(4)}$ generates the 108-torsion part.
\\In the case $n=5$, the last part of the quotient is generated by $K + 16\cdot \One^{(4)} - 21\cdot \One^{(3,2)}$.\\
If $n\geq 6$, the two last parts of the quotient are generated by some multiples of $K -\frac{4}{3}(45-n)\One^{(2,2,2)} + (48-n)\One^{(3,2)}$ and $K-\frac{1}{2}(40-n)\One^{(2,2,2)}+ \frac{1}{4}(48-n)\One^{(4)}$.
\end{proposition}
\begin{proof} The last assertion for arbitrary $n$ follows from Corollary \ref{stabCor}. First observe that for $\boldsymbol{\alpha}^{\boldsymbol{\lambda}}\!\in\! H^2,\  \boldsymbol{\alpha}^{\boldsymbol{\mu}}\!\in\! H^4,\  \boldsymbol{\alpha}^{\boldsymbol{\nu}}\!\in\! H^6 $, we have $\left\| \boldsymbol\lambda\right\| \leq 2$, $\left\| \boldsymbol\mu\right\| \leq 4$ and $\left\| \boldsymbol\nu\right\| \geq 2,$ according to Lemma \ref{degBound}.
The coefficient of the cup product martix are thus polynomials of degree at most $2+4-2 =4$ and it suffices to compute only a finite number of instances for $n$. It turns out, that the maximal degree is 1 and its cokernel is given as stated.
\end{proof}
\begin{remark}
Observe that the generators of the quotients are independent of the choice of the base of $H^2(S,\IZ)$.

\end{remark}
We give now some computational results for the middle cohomology group. Since $S\hilb{n}$ is a projective variety of complex dimension $2n$, Poincar\'{e} duality gives $H^{2n}(S\hilb{n},\IZ)$ the structure of an unimodular lattice.  
\begin{proposition} Let $L$ denote the lattice $H^{2n}(S\hilb{n},\IZ)$. We have:
\begin{enumerate}
\item For $n=2$, $L$ is an odd lattice of rank 276 and signature 124.
\item For $n=3$, $L$ is an even lattice of rank 2554 and signature -640.
\item For $n=4$, $L$ is an odd lattice of rank 19298 and signature ...
\end{enumerate}
The oddity of the lattice for $n$ even is witnessed by Example \ref{example} (\ref{oddWitness}).
\end{proposition}
\bibliographystyle{amsplain}
\bibliography{BiblioAutIHS}


\end{document}
