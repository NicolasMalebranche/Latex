\documentclass{amsart}

\usepackage{amsmath,amssymb,amsfonts}
\usepackage[all]{xy}

\DeclareMathOperator{\rank}{rank}
\DeclareMathOperator{\trace}{tr}
\DeclareMathOperator{\Tor}{Tor}
\DeclareMathOperator{\Ext}{Ext}
\DeclareMathOperator{\Aut}{Aut}
\DeclareMathOperator{\End}{End}
\DeclareMathOperator{\id}{id}
\DeclareMathOperator{\Hom}{Hom}
\DeclareMathOperator{\im}{Im}
\DeclareMathOperator{\Ker}{Ker}
\DeclareMathOperator{\Sym}{Sym}
\DeclareMathOperator{\Hilb}{Hilb}

\newcommand{\hilb}[1]{^{[#1]}}
\newcommand{\ie}{{\it i.e. }}
\newcommand{\eg}{{\it e.g. }}
\newcommand{\loccit}{{\it loc. cit. }}
\newcommand{\ii}{{\rm i}}
\newcommand{\dual}[1]{{#1}\spcheck}
\newcommand{\abs}[1]{|{#1}|}
\newcommand{\Kummer}[2]{{#2}^{\llbracket#1\rrbracket}}
\newcommand{\vac}{|0\rangle}
\newcommand{\odd}{{\rm{odd}}}
\newcommand{\even}{{\rm{even}}}
\newcommand{\tors}{{\rm{tors}}}

\newcommand{\aG}{{\rm{a}}_G}
\newcommand{\mG}{{\rm{m}}_G}

% for spectral sequences

\newcommand{\coloneqq}{:=}

\makeatletter
\newcommand{\rmnum}[1]{\romannumeral #1}
\newcommand{\Rmnum}[1]{\expandafter\@slowromancap\romannumeral #1@}
\makeatother


\newcommand{\FI}{F_{\text{\Rmnum{1}}}}
\newcommand{\FII}{F_{\text{\Rmnum{2}}}}
\newcommand{\EI}{\text{}^{\text{\Rmnum{1}}}\!E}
\newcommand{\EII}{\text{}^{\text{\Rmnum{2}}}\!E}
\newcommand{\dI}{\text{}^{\text{\Rmnum{1}}}\!d}
\newcommand{\dII}{\text{}^{\text{\Rmnum{2}}}\!d}

% for total derived functors and hypercohomology

\newcommand{\HH}{\mathbf{H}}
\newcommand{\LL}{\mathbf{L}}
\newcommand{\RR}{\mathbf{R}}

% for equivariant cohomology

\newcommand{\BG}{BG}
\newcommand{\EG}{EG}


%%%%%%%%%%%%%%%%%%%%%%%%%%%%%%

\newcommand{\IC}{\mathbb{C}}
\newcommand{\IR}{\mathbb{R}}
\newcommand{\IQ}{\mathbb{Q}}
\newcommand{\IZ}{\mathbb{Z}}


%%%%%%%%%%%%%%%%%%%%%%%%%%%%%

\newcommand{\kS}{\mathfrak{S}}

\newcommand{\km}{\mathfrak{m}}
\newcommand{\kq}{\mathfrak{q}}

%%%%%%%%%%%%%%%%%%%%%%%%%%%%%%

\newcommand{\lra}{\longrightarrow}
\newcommand{\ra}{\rightarrow}

%%%%%%%%%%%%%%%%%%%%%%%%%%%%%

\theoremstyle{plain}
\newtheorem{theorem}{Theorem}[section]
\newtheorem{lemma}[theorem]{Lemma}
\newtheorem{proposition}[theorem]{Proposition}
\newtheorem{corollary}[theorem]{Corollary}
\theoremstyle{definition}
\newtheorem{definition}[theorem]{Definition}
\theoremstyle{remark}
\newtheorem{remark}[theorem]{Remark}
\newtheorem{example}[theorem]{Example}




%%%%%%%%%%%%%%%%%%%%%%%%%%%%%

\begin{document}

\title[Products in $H^\ast(\Hilb^n(K3), \IZ)$]{Computing Cup-Products in integer cohomology of Hilbert schemes of points on K3 surfaces}


\author{Simon Kapfer}
\address{Simon Kapfer, Lehrstuhl f\"ur Algebra und Zahlentheorie, Universit\"ats\-stra{\ss}e~14, D-86159 Augsburg}
\email{simon.kapfer@math.uni-augsburg.de}
%\urladdr{http://www.math.uni-augsburg.de/alg/}


\date{\today}

%\subjclass{Primary 14J50, Secondary 14C50, 55T10}

%\keywords{}

\begin{abstract} 
We study the cokernels of the cup 
product in integer cohomology of the Hilbert scheme of $n$ points on a K3 surface. The main difference to cohomology with rational coefficients is the presence of torsion, depending on $n$. We particularly discuss the case $n=3$.
\end{abstract}

\maketitle


%%%%%%%%%%%%%%%%%%%%%%%%%%%%%%%%%%%%%%%%%%%%%%%%%%%%%%%%%%%%%%%%%%
%%%%%%%%%%%%%%%%%%%%%%%%%%%%%%%%%%%%%%%%%%%%%%%%%%%%%%%%%%%%%%%%%%
%%%%%%%%%%%%%%%%%%%%%%%%%%%%%%%%%%%%%%%%%%%%%%%%%%%%%%%%%%%%%%%%%%
\section{Preliminaries}
Let $S$ be a K3 surface. We fix integral bases $1$ of $H^0(S,\IZ)$, $x$ of $H^4(X,\IZ)$ and $\alpha_1,\ldots ,\alpha_{22}$ of $H^2(S,\IZ)$ such that the cup product pairing on $H^2(X,\IZ)$, written as a symmetric matrix with respect to this basis, looks like
$$
\left(\begin{array}{ccccc}
U&&&& \\
&U&&& \\
&&U&& \\
&&&E& \\
&&&&E \end{array}\right),
$$
where $U$ stands for the intersection matrix of the hyperbolic lattice and $E$ stands for the negative matrix of the $E_8$ lattice, \ie 
\begin{equation*}
U = 
\left(\begin{array}{cc}
0&1\\1&0 \end{array}\right),
\ \ E = \left(\begin{array}{cccccccc}
-2&1&0&0&0&0&0&0\\1&-2&1&0&0&0&0&0\\0&1&-2&1&0&0&0&0\\0&0&1&-2&1&0&0&0\\0&0&0&1&-2&1&1&0\\0&0&0&0&1&-2&0&1\\0&0&0&0&1&0&-2&0\\0&0&0&0&0&1&0&-2
\end{array}\right).
\end{equation*}
We denote by $S\hilb{n}$ the Hilbert scheme of $n$ points on $S$. An integral basis for $H^\ast(S\hilb{n},\IZ)$ in terms of Nakajima's operators was given by Qin--Wang:
\begin{theorem} \cite[Thm. 5.4.]{QinWang} The classes
$$ \frac{1}{z_\lambda} \kq_\lambda(1)\kq_\mu(x)\km_{\nu^1,\alpha_1}\ldots\km_{\nu^{22},\alpha_{22}}\vac,\quad \|\lambda\| +\|\mu\| + \sum_{i=1}^{22}\|\nu^i\| = n
$$ 
form an integral basis for $H^\ast(S\hilb{n},\IZ)$. Here,
$\lambda,\; \mu,\; \nu^i$ are partitions, $\|\cdot\|$ means the weight of a partition \ie $\|\lambda\| = \sum_i m_i i$ and $z_\lambda \coloneqq\prod_i i^{m_i} m_i!$, if $\lambda = (1^{m_1},2^{m_2},\ldots)$. The symbol $\kq$ stands for Nakajima's creation operator. 
The relation of $\km_{\nu,\alpha}$ to $\kq_{\tilde{\nu}}(\alpha)$ is the same as the monomial symmetric functions
$m_\nu$ to the power sum symmetric functions 
$p_{\tilde{\nu}}$.
\end{theorem}

\section{Computational results}
The ring structure of $H^\ast(S\hilb{n}, \IQ)$ has been studied in \cite{LehnSorger}. 
Since $H^\text{odd}(S\hilb{n},\IZ) = 0$ and $H^\text{even}(S\hilb{n},\IZ)$ is torsion-free by \cite{Markman}, we can also apply these results to  $H^\ast(S\hilb{n}, \IZ)$. A basis for cohomology with integer coefficients was given by Qin--Wang in \cite{QinWang} which allows us to compute explicitly the image of $H^k(S\hilb{n}, \IZ) \cup H^l(S\hilb{n}, \IZ)$ in $H^{k+l}(S\hilb{n}, \IZ)$. 
We obtain:
\begin{proposition}
If X is deformation equivalent to $S\hilb{3}$, then:
$$
\frac{H^4(X,\IZ)}{\Sym^2 H^2(X,\IZ)} \cong \frac{\IZ}{3\IZ} \oplus \IZ^ {\oplus 23} 
$$
The torsion part of the quotient is generated by the integral class $\frac{1}{3}\kq_{(3)}(1)\vac$.
$$
\frac{H^6(X,\IZ)}{H^2(X,\IZ)\cup H^4(X,\IZ)} \cong \left(\frac{\IZ}{3\IZ}\right)^{\oplus 12} 
$$
This quotient is generated by the 12 integral classes $\km_{(1^3),\alpha_i}\vac$, where \\$ i \in \{ 1,2,3,4,5,6,8,9,11,16,17,19 \}$.
\end{proposition}
\begin{proposition}
\begin{align*}
\frac{H^6(S\hilb{4},\IZ)}{H^2(S\hilb{4},\IZ)\cup H^4(S\hilb{4},\IZ)} &\cong \left(\frac{\IZ}{2\IZ}\right)^{\oplus 23} \oplus \left(\frac{\IZ}{3\IZ}\right)^{\oplus 12} \\
\frac{H^6(S\hilb{5},\IZ)}{H^2(S\hilb{5},\IZ)\cup H^4(S\hilb{5},\IZ)} &\cong \left(\frac{\IZ}{2\IZ}\right)^{\oplus 23} \oplus \left(\frac{\IZ}{3\IZ}\right)^{\oplus 12} \oplus \left(\frac{\IZ}{5\IZ}\right)^{\oplus 3} \\
\frac{H^6(S\hilb{6},\IZ)}{H^2(S\hilb{6},\IZ)\cup H^4(S\hilb{6},\IZ)} &\cong \left(\frac{\IZ}{2\IZ}\right)^{\oplus 23} \oplus \left(\frac{\IZ}{3\IZ}\right)^{\oplus 12} \oplus \left(\frac{\IZ}{5\IZ}\right)^{\oplus 2}\oplus \IZ
\end{align*}
The free summand is generated by $\left[\frac{10}{48}\kq_{(2^3)}(1) - \frac{12}{6}\kq_{(3,2,1)}(1) + \frac{3}{8}\kq_{(4,1^2)}(1) \right]\vac $.
\end{proposition}
\begin{proposition}
$$
\frac{H^6(S\hilb{3},\IZ)}{\Sym^3 H^2(S\hilb{3},\IZ)} \cong  \left(\frac{\IZ}{2\IZ}\right)^{\oplus 243}\oplus \left(\frac{\IZ}{4\IZ}\right)^{\oplus 10}\oplus \left(\frac{\IZ}{3\IZ}\right)^{\oplus 3} \oplus \IZ^{\oplus 507}
$$
$$
\frac{H^6(S\hilb{4},\IZ)}{\Sym^3 H^2(S\hilb{4},\IZ)} \cong  \left(\frac{\IZ}{2\IZ}\right)^{\oplus 23}\oplus \left(\frac{\IZ}{3\IZ}\right)^{\oplus 2} \oplus \IZ^{\oplus 575}
$$
$$
\frac{H^6(S\hilb{5},\IZ)}{\Sym^3 H^2(S\hilb{5},\IZ)} \cong  \left(\frac{\IZ}{2\IZ}\right)^{\oplus 22} \oplus \IZ^{\oplus 597}
$$
$$
\frac{H^6(S\hilb{n},\IZ)}{\Sym^3 H^2(S\hilb{n},\IZ)} \cong  \left(\frac{\IZ}{2\IZ}\right)^{\oplus 22} \oplus \IZ^{\oplus 598} \quad \text{for } n\geq 6.
$$
\end{proposition}

\bibliographystyle{amsplain}
\bibliography{BiblioAutIHS}


\end{document}
