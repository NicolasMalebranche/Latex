\documentclass{amsart}

\usepackage{amsmath,amssymb,amsfonts}
\usepackage[all]{xy}

\DeclareMathOperator{\rank}{rank}
\DeclareMathOperator{\trace}{tr}
\DeclareMathOperator{\Tor}{Tor}
\DeclareMathOperator{\Ext}{Ext}
\DeclareMathOperator{\Aut}{Aut}
\DeclareMathOperator{\End}{End}
\DeclareMathOperator{\id}{id}
\DeclareMathOperator{\Hom}{Hom}
\DeclareMathOperator{\im}{Im}
\DeclareMathOperator{\Ker}{Ker}
\DeclareMathOperator{\Sym}{Sym}
\DeclareMathOperator{\Hilb}{Hilb}

\newcommand{\hilb}[1]{^{[#1]}}
\newcommand{\ie}{{\it i.e. }}
\newcommand{\eg}{{\it e.g. }}
\newcommand{\loccit}{{\it loc. cit. }}
\newcommand{\ii}{{\rm i}}
\newcommand{\dual}[1]{{#1}\spcheck}
\newcommand{\abs}[1]{|{#1}|}
\newcommand{\Kummer}[2]{{#2}^{\llbracket#1\rrbracket}}
\newcommand{\vac}{|0\rangle}
\newcommand{\odd}{{\rm{odd}}}
\newcommand{\even}{{\rm{even}}}
\newcommand{\tors}{{\rm{tors}}}

\newcommand{\aG}{{\rm{a}}_G}
\newcommand{\mG}{{\rm{m}}_G}

% for spectral sequences

\newcommand{\coloneqq}{:=}

\makeatletter
\newcommand{\rmnum}[1]{\romannumeral #1}
\newcommand{\Rmnum}[1]{\expandafter\@slowromancap\romannumeral #1@}
\makeatother


\newcommand{\FI}{F_{\text{\Rmnum{1}}}}
\newcommand{\FII}{F_{\text{\Rmnum{2}}}}
\newcommand{\EI}{\text{}^{\text{\Rmnum{1}}}\!E}
\newcommand{\EII}{\text{}^{\text{\Rmnum{2}}}\!E}
\newcommand{\dI}{\text{}^{\text{\Rmnum{1}}}\!d}
\newcommand{\dII}{\text{}^{\text{\Rmnum{2}}}\!d}

% for total derived functors and hypercohomology

\newcommand{\HH}{\mathbf{H}}
\newcommand{\LL}{\mathbf{L}}
\newcommand{\RR}{\mathbf{R}}

% for equivariant cohomology

\newcommand{\BG}{BG}
\newcommand{\EG}{EG}


%%%%%%%%%%%%%%%%%%%%%%%%%%%%%%

\newcommand{\IC}{\mathbb{C}}
\newcommand{\IR}{\mathbb{R}}
\newcommand{\IQ}{\mathbb{Q}}
\newcommand{\IZ}{\mathbb{Z}}


%%%%%%%%%%%%%%%%%%%%%%%%%%%%%

\newcommand{\kS}{\mathfrak{S}}

\newcommand{\km}{\mathfrak{m}}
\newcommand{\kq}{\mathfrak{q}}

%%%%%%%%%%%%%%%%%%%%%%%%%%%%%%

\newcommand{\lra}{\longrightarrow}
\newcommand{\ra}{\rightarrow}

%%%%%%%%%%%%%%%%%%%%%%%%%%%%%

\theoremstyle{plain}
\newtheorem{theorem}{Theorem}[section]
\newtheorem{lemma}[theorem]{Lemma}
\newtheorem{proposition}[theorem]{Proposition}
\newtheorem{corollary}[theorem]{Corollary}
\theoremstyle{definition}
\newtheorem{definition}[theorem]{Definition}
\newtheorem{notation}[theorem]{Notation}
\theoremstyle{remark}
\newtheorem{remark}[theorem]{Remark}
\newtheorem{example}[theorem]{Example}




%%%%%%%%%%%%%%%%%%%%%%%%%%%%%

\begin{document}

\title[Products in $H^\ast(\Hilb^n(K3), \IZ)$]{Computing Cup-Products in integer cohomology of Hilbert schemes of points on K3 surfaces}


\author{Simon Kapfer}
\address{Simon Kapfer, Lehrstuhl f\"ur Algebra und Zahlentheorie, Universit\"ats\-stra{\ss}e~14, D-86159 Augsburg}
\email{simon.kapfer@math.uni-augsburg.de}
%\urladdr{http://www.math.uni-augsburg.de/alg/}


\date{\today}

%\subjclass{Primary 14J50, Secondary 14C50, 55T10}

%\keywords{}

\begin{abstract} 
We study the images of cup 
products in integer cohomology of the Hilbert scheme of $n$ points on a K3 surface. 
\end{abstract}

\maketitle


%%%%%%%%%%%%%%%%%%%%%%%%%%%%%%%%%%%%%%%%%%%%%%%%%%%%%%%%%%%%%%%%%%
%%%%%%%%%%%%%%%%%%%%%%%%%%%%%%%%%%%%%%%%%%%%%%%%%%%%%%%%%%%%%%%%%%
%%%%%%%%%%%%%%%%%%%%%%%%%%%%%%%%%%%%%%%%%%%%%%%%%%%%%%%%%%%%%%%%%%
\section{Preliminaries}
\begin{definition}
Let $S$ be a projective K3 surface. We fix integral bases $1$ of $H^0(S,\IZ)$, $x$ of $H^4(X,\IZ)$ and $\alpha_1,\ldots ,\alpha_{22}$ of $H^2(S,\IZ)$. The cup product induces a symmetric bilinear form $B_{H^2}$ on $H^2(X,\IZ)$, written as a symmetric matrix with respect to this basis, looks like
$$ B_{H^2} =
\left(\begin{array}{ccccc}
U&&&& \\
&U&&& \\
&&U&& \\
&&&E& \\
&&&&E \end{array}\right),
$$
where $U$ stands for the intersection matrix of the hyperbolic lattice and $E$ stands for the negative matrix of the $E_8$ lattice, \ie 
\begin{equation*}
U = 
\left(\begin{array}{cc}
0&1\\1&0 \end{array}\right),
\ \ E = \left(\begin{array}{cccccccc}
-2&1&0&0&0&0&0&0\\1&-2&1&0&0&0&0&0\\0&1&-2&1&0&0&0&0\\0&0&1&-2&1&0&0&0\\0&0&0&1&-2&1&1&0\\0&0&0&0&1&-2&0&1\\0&0&0&0&1&0&-2&0\\0&0&0&0&0&1&0&-2
\end{array}\right).
\end{equation*}
We may extend $B_{H^2}$ to a symmetric non-degenerate bilinear form on $H^\ast(S,\IZ)$ by setting $ B(1,1) = 0,\ B(1,\alpha_i) = 0,\ B(1,x) = 1, \ B(x,x) = 0$.
\end{definition}
\begin{definition}
$B$ induces a form $B\otimes B$ on $\Sym^2H^\ast(S,\IZ)$. So the cup-product 
\begin{equation*}
\mu : \Sym ^2H^{*}(S,\IZ) \longrightarrow H^\ast(S,\IZ) 
\end{equation*}
has an adjoint comultiplication $\Delta$, given by:
\begin{equation*}
\Delta : H^\ast(S,\IZ) \longrightarrow \Sym^2H^\ast(S,\IZ),\quad \Delta = (B\otimes B)^{-1}\mu^TB
\end{equation*}
The image of 1 under the composite map $\mu(\Delta(1)) = B(\Delta(1),\Delta(1)) = 24 x$, denoted by $e$ is called the Euler Class.
\end{definition}
We denote by $S\hilb{n}$ the Hilbert scheme of $n$ points on $S$, \ie the classifying space of all zero-dimensional closed subschemes of length $n$, which is smooth. 
A classical result by Nakajima gives an explicit description of $H^\ast(S\hilb{n},\IQ)$ in terms of creation operators
$\kq_l(\beta)$, 
$\beta\in H^\ast(S,\IQ)$, acting on the direct sum 
$\bigoplus_n H^\ast(S\hilb{n},\IQ)$. 
An integral basis for $H^\ast(S\hilb{n},\IZ)$ in terms of Nakajima's operators was given by Qin--Wang:
\begin{theorem} \cite[Thm. 5.4.]{QinWang} The classes
$$ \frac{1}{z_\lambda} \kq_\lambda(1)\kq_\mu(x)\km_{\nu^1,\alpha_1}\ldots\km_{\nu^{22},\alpha_{22}}\vac,\quad \|\lambda\| +\|\mu\| + \sum_{i=1}^{22}\|\nu^i\| = n
$$ 
form an integral basis for $H^\ast(S\hilb{n},\IZ)$. Here,
$\lambda,\; \mu,\; \nu^i$ are partitions, $\|\cdot\|$ means the weight of a partition \ie $\|\lambda\| = \sum_i m_i i$ and $z_\lambda \coloneqq\prod_i i^{m_i} m_i!$, if $\lambda = (1^{m_1},2^{m_2},\ldots)$. The symbol $\kq$ stands for Nakajima's creation operator. 
The relation of $\km_{\nu,\alpha}$ to $\kq_{\tilde{\nu}}(\alpha)$ is the same as the monomial symmetric functions
$m_\nu$ to the power sum symmetric functions 
$p_{\tilde{\nu}}$.
\end{theorem}
The ring structure of $H^\ast(S\hilb{n}, \IQ)$ has been studied in \cite{LehnSorger}, where an explicit algebraic model is constructed.
Since $H^\text{odd}(S\hilb{n},\IZ) = 0$ and $H^\text{even}(S\hilb{n},\IZ)$ is torsion-free by \cite{Markman}, we can also apply these results to  $H^\ast(S\hilb{n}, \IZ)$ to determine the multiplicative structure of cohomology with integer coefficients.


\section{Computational results}
With the help of a computer, we are able to compute arbitrary products in $H^\ast(S\hilb{n},\IZ)$.
We give some results in low degrees. 
The algebra generated by classes of degree 2 is an interesting object to study. For cohomology with complex coefficients, Verbitsky has proven that the algebra generated by $H^2(X,\IC)$
\begin{notation}
To enumerate the basis of $H^\ast(S\hilb{n},\IZ)$, we introduce the following abbreviation:
$$ 
1^\lambda \alpha_1^{\nu_1}\ldots\alpha_{22}^{\nu_{22}}x^\mu :=
\frac{1}{z_{\tilde{\lambda}} }
\kq_{\tilde{\lambda}}(1)\kq_\mu(x)\km_{\nu^1,\alpha_1}\ldots\km_{\nu^{22},\alpha_{22}}\vac
$$
where the partition $\tilde{\lambda}$ is built from $\lambda$ by appending sufficiently many Ones, such that $\|\tilde{\lambda}\| +\|\mu\| + \sum\|\nu^i\| = n $.
\end{notation}
By computing multiplication matrices with respect to the integral basis and a reduction to Smith normal form (with the help of a computer), images of cup products can be explored.
\begin{proposition} Studying the image of $\Sym^2H^2$ in $H^4$, we obtain: 
$$
\frac{H^4(S\hilb{2},\IZ)}{\Sym^2 H^2(S\hilb{2},\IZ)}  \cong \left(\frac{\IZ}{2\IZ}\right)^{\oplus 23} \oplus \frac{\IZ}{5\IZ}
$$
This was already known to Boissi\`{e}re, Nieper-Wi\ss kirchen and Sarti, \cite[Prop. 3]{BNS}.
$$
\frac{H^4(S\hilb{3},\IZ)}{\Sym^2 H^2(S\hilb{3},\IZ)} \cong \frac{\IZ}{3\IZ} \oplus \IZ^ {\oplus 23} 
$$
The torsion part of the quotient is generated by the integral class $\frac{1}{3}\kq_{(3)}(1)\vac$.
$$
\frac{H^4(S\hilb{n},\IZ)}{\Sym^2 H^2(S\hilb{n},\IZ)} \cong  \IZ^ {\oplus 24}, \quad \text{for }n\geq 4.
$$
This was already proven by Markman, \cite[Thm. 1.10]{Markman2}.
\end{proposition}
\begin{proposition} Comparing $H^2(S\hilb{n},\IZ) \cup H^4(S\hilb{n},\IZ) $ with $H^6(S\hilb{n},\IZ) $, we obtain:
\begin{align} 
\frac{H^6(S\hilb{2},\IZ) }{H^2(S\hilb{2},\IZ)\cup H^4(S\hilb{2},\IZ)} &= 0 
\\
\frac{H^6(S\hilb{3},\IZ)}{H^2(S\hilb{3},\IZ)\cup H^4(S\hilb{3},\IZ)} &\cong \left(\frac{\IZ}{3\IZ}\right)^{\oplus 12} 
\\
\frac{H^6(S\hilb{4},\IZ)}{H^2(S\hilb{4},\IZ)\cup H^4(S\hilb{4},\IZ)} &\cong \left(\frac{\IZ}{2\IZ}\right)^{\oplus 23} \oplus \left(\frac{\IZ}{3\IZ}\right)^{\oplus 12} 
\\
\frac{H^6(S\hilb{5},\IZ)}{H^2(S\hilb{5},\IZ)\cup H^4(S\hilb{5},\IZ)} &\cong \left(\frac{\IZ}{2\IZ}\right)^{\oplus 23} \oplus \left(\frac{\IZ}{3\IZ}\right)^{\oplus 12} \oplus \left(\frac{\IZ}{5\IZ}\right)^{\oplus 3} 
\\
\frac{H^6(S\hilb{n},\IZ)}{H^2(S\hilb{n},\IZ)\cup H^4(S\hilb{n},\IZ)} &\cong \left(\frac{\IZ}{2\IZ}\right)^{\oplus 22} \oplus \left(\frac{\IZ}{3\IZ}\right)^{\oplus 12} \oplus \left(\frac{\IZ}{5\IZ}\right)^{\oplus 2}\oplus \IZ, \ n\geq 6.
\end{align}
\begin{itemize}
\item The 3-torsion part is generated by the 12 integral classes $\alpha_i^{(1,1,1)}\in H^6$, where $ i = 1,2,3,4,5,6,8,9,11,16,17,19$.
\item The 2-torsion part is generated by the 22 integral classes $\alpha_i^{(1,1,1)} +\alpha_i^{(2,1)} + \alpha_i^{(3)}+ 1^{(2)}\alpha_i^{(1,1)}+1^{(3)}\alpha_i^{(1)} $, $i=1,\ldots,22$ and, in the cases $n=4,5$, by the integral class $1^{(4)}\in H^6$.
\item The 5-torsion part is generated by the 2 integral classes
$\alpha_i^{(1,1,1)}+2\alpha_i^{(2,1)}+3\alpha_i^{(3)}+ 4\cdot 1^{(2)}\alpha_i^{(1,1)} +  2\cdot 1^{(2)} \alpha_i^{(2)}+ 2\cdot 1^{(3)}\alpha_i^{ (1)} + 3\cdot 1^{(2,2)}\alpha_i^{(1)},\ i =13,21$ 
and, in the case $n=5$, by the integral class $ 1^{(4)}+1^{(3,2)}$.
\item The free summand is generated by the class $3\cdot 1^{(4)} - 12\cdot 1^{(3,2)}+10\cdot 1^{(2,2,2)}$.
\end{itemize}
\end{proposition}
\begin{proposition}
$$
\frac{H^6(S\hilb{2},\IZ)}{\Sym^3 H^2(S\hilb{2},\IZ)} \cong 
\frac{\IZ}{2\IZ}
$$
The quotient is generated by the integral class $\frac{1}{2} \kq_{(2)}(1)\vac$.
$$
\frac{H^6(S\hilb{3},\IZ)}{\Sym^3 H^2(S\hilb{3},\IZ)} \cong  \left(\frac{\IZ}{2\IZ}\right)^{\oplus 230}\oplus \left(\frac{\IZ}{36\IZ}\right)^{\oplus 22}\oplus \frac{\IZ}{72\IZ} \oplus \IZ^{\oplus 507}
$$
$$
\frac{H^6(S\hilb{4},\IZ)}{\Sym^3 H^2(S\hilb{4},\IZ)} \cong  
$$
$$
\frac{H^6(S\hilb{5},\IZ)}{\Sym^3 H^2(S\hilb{5},\IZ)} \cong  
$$
$$
\frac{H^6(S\hilb{n},\IZ)}{\Sym^3 H^2(S\hilb{n},\IZ)} \cong   n\geq 6.
$$
\end{proposition}

\bibliographystyle{amsplain}
\bibliography{BiblioAutIHS}


\end{document}
