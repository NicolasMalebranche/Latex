\section{Super algebras} \label{SuperSection}
Let us recall some material on super algebras, which will be useful in Sections \ref{Section_Hilbert} and \ref{Section_GeneralKummer} to understand the cohomology structure of the Hilbert schemes of points on surfaces.
\begin{definition}
 A super vector space $V$ over a field $k$ is a vector space with a $\Z/2\Z$-graduation, that is a decomposition
$$
 V = V^{+} \oplus V^{-},
$$
called the even and the odd part of $V$. Elements of $V^{+}$ are called homogeneous of even degree, elements of $V^{-}$ are called homogeneous of odd degree.
The degree of a homogeneous element $v$ is denoted by $|v|\in \Z/2\Z$.
Direct sum and tensor product of two super vector spaces $V$ and $W$ yield again super vector spaces:
\begin{align*}
 (V\oplus W)^{+} &= V^{+}\oplus W^{+}, & (V\oplus W)^{-} &= V^{-}\oplus W^{-}, \\
 (V\otimes W)^{+} &= (V^{+}\otimes W^{+})\oplus (V^{-}\otimes W^{-}), & (V\otimes W)^{-} &=(V^{+}\otimes W^{-})\oplus (V^{-}\otimes W^{+}).
\end{align*}
\end{definition}
\begin{definition}
A superalgebra $R$ is a unital associative $k$-algebra which carries a super vector space structure. Define the supercommutator by setting for homogeneous elements $u,v \in R$:
\begin{align*}
[u,v] := uv - (-1)^{|u||v|} v u.
\end{align*}
$R$ is called supercommutative, if $[u,v]=0$ for all $u,v\in R$. Note that a graded commutative algebra $R = \bigoplus\limits_n R^n$ is supercommutative in a natural way, by setting $R^{+}=\bigoplus\limits_{n\text{ even}} R^n$, $R^{-}=\bigoplus\limits_{n\text{ odd}} R^n$.

For a supercommutative algebra $R$, the tensor power $R^{\otimes n}$ is again a supercommutative algebra, if we set for the product:
$$
(u_1\otimes\cdots\otimes u_n)\cdot(v_1\otimes\cdots\otimes v_n) 
=  (-1)^{\sum\limits_{i>j}|u_i||v_j|} u_1v_1\otimes\cdots\otimes u_nv_n.
%,\quad \text{where } k= {\sum_{j<i}|u_i||v_j|}.
$$
\end{definition}
\begin{definition}
Let $V$ be a super vector space over $k$ and $n\geq 0$. Then the supersymmetric power $\SSym^n(V)$ of $V$ is a super vector space, given by
\begin{align*}
\SSym^n(V) &= \bigoplus_{p+q=n} \Sym^p(V^{+}) \otimes \Lambda^q(V^{-}), \\
\SSym^n(V)^{+} &= \bigoplus_{\substack{p+q=n \\ q\text{ even} }} \Sym^p(V^{+}) \otimes \Lambda^q(V^{-}), &
\SSym^n(V)^{-} &= \bigoplus_{\substack{p+q=n \\ q\text{ odd} }} \Sym^p(V^{+}) \otimes \Lambda^q(V^{-}).
\end{align*}
The supersymmetric algebra $\SSym^*(V):= \bigoplus\limits_n \SSym^n(V)$ on $V$ is a supercommutative algebra over $k$, where the product of two elements $s\otimes e \in \Sym^p(V^{+}) \otimes \Lambda^q(V^{-})$ and $s'\otimes e' \in \Sym^{p'}(V^{+}) \otimes \Lambda^{q'}(V^{-})$ is given by 
$$
(s\otimes e)\diamond (s'\otimes e') = (s s')\otimes (e\wedge e') \ \  \in \Sym^{p+p'}(V^{+}) \otimes \Lambda^{q+q'}(V^{-}).
$$
\end{definition}
\begin{remark}
The supersymmetric power $\SSym^n (V)$ can be realized as a quotient of $V^{\otimes n}$ by an action of the symmetric group $\mathfrak S_n$. This action can be described as follows: If $\tau\in \mathfrak S_n$ is a transposition that exchanges two numbers $i<j$, then $\tau$ permutes the corresponding tensor factors in $v_1\otimes  \cdots\otimes v_n$ introducing a sign
$(-1)^{|v_i||v_j|+(|v_i|+|v_j|)\sum_{i<k<j} |v_k|}$.
\end{remark}


Now let $U$ be a vector space over a field $k$ of characteristic $0$ and look at the exterior algebra $H:= \Lambda^* U$. 
Since $H$ is a super vector space, we can construct the supersymmetric power $S^n:= \SSym^n( H)$.
We may identify $S^n$ with the space of $\mathfrak S_n$-invariants in $H^{\otimes n}$ by means of the linear projection operator
$$
\pr : H^{\otimes n} \longrightarrow S^n , \quad \pr = \frac{1}{n!}\sum_{\pi \in\mathfrak S_n} \pi.
$$
The multiplication in $H^{\otimes n}$ induces a multiplication on the subspace of invariants, which makes $S^n$ a supercommutative algebra. Of course, it is different from the product $\diamond$ discussed above.

Since $H$ is generated as an algebra by $U=\Lambda^1(U)\subset H$, we may define a homomorphism of algebras:
$$ s : H \longrightarrow S^n, \quad s(u) = \pr( u \otimes 1\otimes\cdots\otimes 1)\text{ for }u\in U, $$
so $S^n$ becomes an algebra over $H$.
\begin{lemma}
\label{SuperFree}
The morphism $s$ turns $S^n$ into a free module over $H$, for $n\geq 1$.
\end{lemma}
\begin{proof}
We start with the tensor power $H^{\otimes n}$ and the ring homomorphism 
$$
\iota : H \longrightarrow H^{\otimes n},\quad h\longmapsto h\otimes 1\otimes\cdots\otimes 1
$$
that makes $H^{\otimes n}$ a free $H$-module. Note that $\pr \iota \neq s$, since $\pr$ is not a ring homomorphism.
(For example, $\pr\iota(h)\neq s(h)$ for any nonzero $h\in\Lambda^2(U)$.)
We therefore modify the $H$-module structure of $H^{\otimes n}$:

For some $u\in U$, denote $u^{(i)} := 1^{\otimes i-1}\otimes u\otimes 1^{\otimes n-i+1} \in H^{\otimes n}$. Then $H^{\otimes n}$ is generated as a $k$-algebra by the elements $\{u^{(i)}\,,\,u\in U\}$. Now consider the ring automorphism
$$
\sigma : H^{\otimes n} \longrightarrow H^{\otimes n}, \quad u^{(1)} \longmapsto u^{(1)} +u^{(2)} + \ldots + u^{(n)}, \quad
u^{(i)} \longmapsto u^{(i)} \text{ for } i>1.
$$
Then we have $\sigma\iota = s$ on $S^n$. On the other hand, if $\{b_i\}$ is a $k$-basis of $V$, then $\{b_i^{(j)},\,j>1\} $ is a $\iota$-basis for $H^{\otimes n}$, and $\{\sigma(b_i^{(j)})\}$ is a $\sigma\iota$-basis for $H^{\otimes n}$.
Now if we project the basis elements, we get a set $\{\pr(\sigma(b_i^{(j)}))\}$ that spans $S^n$. Eliminating linear dependent vectors (this is possible over the rationals), we get a $s$-basis of $S^n$.
\end{proof}

