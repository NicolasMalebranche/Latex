\section{Introduction}
\subsection{Deutsch}
Irreduzible holomorph symplektische Mannigfaltigkeiten wurden erstmals von Beauville \cite{Beauville} eingeführt, der sie als einfach zusammenh\"angende kompakte K\"ahlermannigfaltigkeiten mit bis auf Vielfache eindeutiger nirgends entarteter holomorpher 2-Form definiert. 
Eine alternative Beschreibung 
\subsection{Englisch}
Irreducible holomorphically symplectic (IHS) manifolds have been introduced by Beauville \cite{Beauville} as simply-connected compact K\"ahler manifolds admitting a everywhere non-degenerate holomorohic two-form, unique up to scalar. 
Alternatively, they can be described in a differential geometric setting as compact Riemannian manifolds with holonomy group isomorphic to the symplectic group $\Sp(n,\R)$. This implies the existence of a set of complex structures, parametrized by imaginary quaternions of unit norm, such that the metric is K\"ahler with respect to all of these. 
Another name, compact Hyperk\"ahler manifolds, is therefore common to emphasize this aspect of our class of manifolds. We will use the two names interchangeably.

It can be shown that any such manifold must have even complex dimension. IHS manifolds in dimension 2 are the K3 surfaces, and the concept of an IHS manifold can be seen as a generalization of them. The two main example series are given by the deformation classes of Hilbert schemes of points on K3 surfaces and Generalized Kummer varieties. Both were identified by Beauville \cite{Beauville}. Apart from that, only two further examples due to O'Grady are known up to now.

An important structure of any IHS manifold $X$ is the so-called Beauville-Bogomolov form, a non-degenerate quadratic form on $H^2(X,\Z)$ that can be described with the help of the map of the symmetric power of $H^2(X,\Z)$ to the middle cohomology group via the cup product, that relates the Beauville-Bogomolov form with the form given by Poincar\'e duality. This is called the Fujiki relation. It implies that the map $\Sym^k H^2(X,\Z)\rightarrow H^{2k}(X,\Z)$ is an embedding of lattices. We study the algebraic properties of that situation in part \textbf{XY}.

An automorphism of an IHS manifold $X$ induces a lattice automorphism on $H^2(X,\Z)$.
In recent years lattice theoretic methods have been used by
Boiss\`ere, Camere, Joumaah, Menet, Mongardi, Nieper-Wi\ss kirchen, Oguiso, Sarti, Tari, Wandel and others to give results on
automorphisms of finite order. 


Generalized Kummer: Beauville, Britze, Tari

Torelli: Verbitsky
Monodromy: Markman


This work is divided into three parts. 