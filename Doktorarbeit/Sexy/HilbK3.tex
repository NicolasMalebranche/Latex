\begin{lstlisting}
-- implements the cup product according to Lehn-Sorger and Qin-Wang
module HilbK3 where

import Data.Array
import Data.MemoTrie
import Data.Permute hiding (sort,sortBy)
import Data.List
import qualified Data.IntMap as IntMap
import qualified Data.Set as Set
import Data.Ratio
import K3
import Partitions
import SymmetricFunctions

-- elements in A^[n] are indexed by partitions, with attached elements of the base K3
-- is also used for indexing H^*(Hilb, Z)
type AnBase = (PartitionLambda Int, [K3Domain])

-- elements in A{S_n} are indexed by permutations, in cycle notation,
-- where to each cycle an element of the base K3 is attached, see L-S (2.5)
type SnBase = [([Int],K3Domain)]

-- an equivalent to partZ with painted partitions
-- counts multiplicites that occur, when the symmetrization operator is applied
anZ :: AnBase -> Int
anZ (PartLambda l, k) = comp 1 (0,undefined) 0 $ zip l k where
	comp acc old m (e@(x,_):r) | e==old = comp (acc*x) old (m+1) r
		| otherwise = comp (acc*x*factorial m) e 1 r
	comp acc _ m [] = factorial m * acc

-- injection of A^[n] in A{S_n}, see L-S 2.8
-- returns a symmetrized vector of A{S_n}
toSn :: AnBase -> ([SnBase],Int)
toSn = makeSn where
	allPerms = memo p where 
		p n = map (array (0,n-1). zip [0..]) (permutations [0..n-1]) 
	shape l = (map (forth IntMap.!) l, IntMap.fromList $ zip [1..] sl) where
		sl = map head$ group $ sort l; 
		forth = IntMap.fromList$ zip sl [1..]
	symmetrize :: AnBase -> ([[([Int],K3Domain)]],Int)
	symmetrize (part,l) = (perms, toInt $ factorial n % length perms)  where 
		perms = nub [sortSn$ zipWith (\c cb ->(ordCycle $ map(p!)c, cb) ) cyc l 
			| p <- allPerms n]
		cyc = sortBy ((.length).flip compare.length) $ cycles $ partPermute part
		n = partWeight part
	ordCycle cyc = take l $ drop p $ cycle cyc where
		(m,p,l) = foldl findMax (-1,-1,0) cyc
		findMax (m,p,l) ce = if m<ce then (ce,l,l+1) else (m,p,l+1)
	sortSn = sortBy	compareSn  where
		compareSn (cyc1,class1) (cyc2,class2) = let
			cL = compare l2 $ length cyc1 ; l2 = length cyc2
			cC = compare class2 class1
			in if cL /= EQ then cL else 
				if cC /= EQ then cC else compare cyc2 cyc1  
	mSym = memo symmetrize
	makeSn (part,l) = ([ [(z,im IntMap.! k) | (z,k) <- op ]|op <- res],m)  where
		(repl,im) = shape l
		(res,m) = mSym (part,repl)

-- multiplication in A{S_n}k, see L-S, Prop 2.13
multSn :: SnBase -> SnBase -> [(SnBase,Int)]
multSn l1 l2 = tensor $ map m cmno where
	-- determines the orbits of the group generated by pi, tau
	commonOrbits :: Permute -> Permute -> [[Int]]
	commonOrbits pi tau = Data.List.sortBy ((.length).compare.length) orl where
		orl = foldr (uni [][]) (cycles pi) (cycles tau) 
		uni i ni c []  = i:ni
		uni i ni c (k:o) = if Data.List.intersect c k == [] 
			then uni i (k:ni) c o else uni (i++k) ni c o
	pi1 = cyclesPermute n $ cy1 ; cy1 = map fst l1; n = sum $ map length cy1
	pi2 = cyclesPermute n $ map fst l2
	set1 = map (\(a,b)->(Set.fromList a,b)) l1; 
	set2 = map (\(a,b)->(Set.fromList a,b)) l2
	compose s t = swapsPermute (max (size s) (size t)) (swaps s ++ swaps t)
	tau = compose pi1 pi2
	cyt = cycles tau ; 
	cmno = map Set.fromList $ commonOrbits pi1 pi2; 
	m or = fdown where
		sset12 = [xv | xv <-set1++set2, Set.isSubsetOf (fst xv) or]
		-- fup and fdown correspond to the images of the maps described in L-S (2.8)
		fup = cupLSparse $ map snd sset12 ++ replicate def xK3
		t = [c | c<-cyt, Set.isSubsetOf (Set.fromList c) or]
		fdown = [(zip t l,v*w*24^def)| (r,v) <- fup, (l,w)<-cupAdLSparse(length t) r] 
		def = toInt ((Set.size or + 2 - length sset12 - length t)%2)

-- tensor product for a list of arguments
tensor :: Num a =>  [[([b],a)]] -> [([b],a)]
tensor [] = [([],1)]
tensor (t:r) = [(y++x,w*v) |(x,v)<-tensor r, (y,w) <- t ]

-- multiplication in A^[n]
multAn :: AnBase -> AnBase -> [(AnBase,Int)]
multAn a = multb where
	(asl,m) = toSn a
	toAn sn =(PartLambda l, k) where 
		(l,k)= unzip$ sortBy (flip compare)$ map (\(c,k)->(length c,k)) sn
	multb (pb,lb) = map ungroup$ groupBy ((.fst).(==).fst) $sort elems where
		ungroup g@((an,_):_) = (an, m*(sum $ map snd g) )
		bs = zip (sortBy ((.length).flip compare.length) $cycles $ partPermute pb) lb
		elems = [(toAn cs,v) | as <- asl, (cs,v) <- multSn as bs]

-- integer base to ordinary base, see Q-W, Thm 1.1
intCrea :: AnBase -> [(AnBase,Ratio Int)]
intCrea = map makeAn. tensor. construct where
	memopM = memo pM
	pM pa = [(pl,v)| p@(PartLambda pl)<-map partAsLambda$ partOfWeight (partWeight pa), 
		let v = powerMonomial p pa, v/=0]
	construct pl = onePart pl : xPart pl : 
		[ [(zip l $ repeat a,v)| (l,v)<- memopM (subpart pl a)] |a<-[1..22]] 
	onePart pl = [(zip l$ repeat oneK3, 1%partZ p)] where 
		p@(PartLambda l) = subpart pl oneK3
	xPart pl = [(zip l$ repeat xK3, 1)] where 
		(PartLambda l) = subpart pl xK3
	makeAn (list,v) = ((PartLambda x,y),v) where 
		(x,y) = unzip$ sortBy (flip compare) list 

-- ordinary base to integer base, see Q-W, Thm 1.1
creaInt :: AnBase -> [(AnBase, Int)]
creaInt = map makeAn. tensor. construct where
	memomP = memo mP
	mP pa = [(pl,v)| p@(PartLambda pl)<-map partAsLambda$ partOfWeight (partWeight pa), 
		let v = monomialPower p pa, v/=0]
	construct pl = onePart pl : xPart pl : 
		[ [(zip l $ repeat a,v)| (l,v)<- memomP (subpart pl a)] |a<-[1..22]] 
	onePart pl = [(zip l$ repeat oneK3, partZ p)] where 
		p@(PartLambda l) = subpart pl oneK3
	xPart pl = [(zip l$ repeat xK3, 1)] where 
		(PartLambda l) = subpart pl xK3
	makeAn (list,v) = ((PartLambda x,y),v) where 
		(x,y) = unzip$ sortBy (flip compare) list 

-- cup product for integral classes
cupInt :: AnBase -> AnBase -> [(AnBase,Int)]
cupInt a b = [(s,toInt z)| (s,z) <- y] where
	ia = intCrea a; ib = intCrea b
	x = sparseNub [(e,v*w*fromIntegral z) | (p,v) <- ia, 
		let m = multAn p, (q,w) <- ib, (e,z)<- m q] 
	y = sparseNub [(s,v*fromIntegral w) | (e,v) <- x, (s,w) <- creaInt e]

-- helper function, adds duplicates in a sparse vector
sparseNub :: (Num a) => [(AnBase, a)] -> [(AnBase,a)] 
sparseNub = map (\g->(fst$head g, sum $map snd g)).groupBy ((.fst).(==).fst). 
	sortBy ((.fst).compare.fst)

-- cup product for integral classes from a list of factors
cupIntList :: [AnBase] -> [(AnBase,Int)]
cupIntList = makeInt. ci . cL where
	cL [b] = intCrea b
	cL (b:r) = x where
		ib = intCrea b
		x = sparseNub [(e,v*w*fromIntegral z) | 
			(p,v) <- cL r, let m = multAn p, (q,w) <- ib, (e,z)<-m q]
	makeInt l = [(e,toInt z) | (e,z) <- l]
	ci l = sparseNub [(s,v*fromIntegral w) | (e,v) <- l, (s,w) <- creaInt e]

-- degree of a base element of cohomology
degHilbK3 :: AnBase -> Int
degHilbK3 (lam,a) = 2*partDegree lam + sum [degK3 i | i<- a]

-- base elements in Hilb^n(K3) of degree d 
hilbBase :: Int -> Int -> [AnBase]
hilbBase = memo2 hb where
	hb n d = sort $map ((\(a,b)->(PartLambda a,b)).unzip) $ hilbOperators n d  

-- all possible combinations of creation operators of weight n and degree d
hilbOperators :: Int -> Int -> [[ (Int,K3Domain) ]]
hilbOperators = memo2 hb where 
	hb 0 0 = [[]] -- empty product of operators
	hb n d = if n<0 || odd d || d<0 then [] else 
		nub $ map (Data.List.sortBy (flip compare)) $ f n d
	f n d = [(nn,oneK3):x | nn <-[1..n], x<-hilbOperators(n-nn)(d-2*nn+2)] ++
		[(nn,a):x | nn<-[1..n], a <-[1..22], x<-hilbOperators(n-nn)(d-2*nn)] ++
		[(nn,xK3):x | nn <-[1..n], x<-hilbOperators(n-nn)(d-2*nn-2)] 

-- helper function
subpart :: AnBase -> K3Domain -> PartitionLambda Int
subpart (PartLambda pl,l) a = PartLambda $ sb pl l where
	sb [] _ = []
	sb pl [] = sb pl [0,0..]
	sb (e:pl) (la:l) = if la == a then e: sb pl l else sb pl l

-- converts from Rational to Int
toInt :: Ratio Int -> Int
toInt q = if n ==1 then z else error "not integral" where 
	(z,n) =(numerator q, denominator q)
\end{lstlisting}