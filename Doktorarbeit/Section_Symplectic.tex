\section{Actions of the symplectic group over finite fields}
Let $V$ be a symplectic vector space of dimension $n\in 2\mathbb{N}$ over a field $k$ with a nondegenerate symplectic form $\omega : \Lambda^2 V \rightarrow k$. A line is a one-dimensional subspace ov $V$, a plane is a two-dimensional subspace of $V$. A plane $P\subset V$ is called isotropic, if $\omega (x,y)=0$ for any $x,y\in P$, otherwise non-isotropic.  The symplectic group $\Sp V$ is the set of all linear maps $\phi : V\rightarrow V$ with the property $\omega(\phi(x),\phi(y)) = \omega(x,y)$ for all $ x,y\in V$.
\begin{proposition}
The symplectic group $\Sp V$ acts transitively on the set of non-isotropic planes as well as on the set of isotropic planes.
\end{proposition}
\begin{proof}
Given two planes $P_1$ and $P_2$, we may choose vectors $v_1,v_2,w_1,w_2$ such that $v_1,v_2$ span $P_1$, $w_1,w_2$ span $P_2$ and $\omega(u_1,u_2) =\omega(w_1,w_2)$. We complete $\{v_1,v_2\}$ as well as $\{w_1,w_2\}$ to a symplectic basis of $V$.
Then define $\phi(v_1)=w_1$ and $\phi(v_2)=w_2$. 
It is now easy to see that the definition of $\phi$ can be extended to the remaining basis elements to give a symplectic morphism.
\end{proof}
\begin{remark}
The set of planes in $V$ can be identified with the simple tensors in $\Lambda^2V$ up to multiples. Indeed, given a simple tensor $v\wedge w \in \Lambda^2 V$, the span of $v$ and $w$ yields the corresponding plane. Conversely, any two spanning vectors $v$ and $w$ of a plane give the same element $v\wedge w$ (up to multiples).
\end{remark}
\begin{proposition}
If $\phi\in\Sp V$ acts through multiplication of a scalar, $\phi(v) = \alpha v$, then $\alpha = \pm 1$ (this is immediate from the definition). Moreover, if $\phi(v)\wedge \phi(w) = \alpha v\wedge w$, then $\alpha=1$.
\end{proposition}
\begin{proof}
We may assume that $V$ is two-dimensional, generated by $v$ and $w$. Our condition on $\phi$ reads then $\det\phi = \alpha$. But the condition on $\phi$ being symplectic is $\det\phi = 1$, because on a two-dimensional vector space there is only one symplectic form up to scalar multiple. 
\end{proof}
\begin{remark} \label{PlaneTriple}
 If $k$ is the field with two elements, then the set of planes in $V$ can be identified with the set $\{\{x,y,z\}\;|\;x,y,z\in V\backslash\{0\},\,x+y+z=0\}$. Observe that for such a $\{x,y,z\}$, $\omega(x,y)=\omega(x,y)=\omega(y,x)$ and this value gives the criterion for isotropy.
\end{remark}

\begin{proposition}\label{OrbitesSp}
Assume that $k$ is finite of cardinality $q$.
\begin{align}
&\text{The number of lines in $V$ is }\frac{q^n-1}{q-1}, \\
&\text{the number of planes in $V$ is }\frac{(q^n-1)(q^{n-1}-1)}{(q^2-1)(q-1)}, \\
&\text{the number of isotropic planes in $V$ is }\frac{(q^n-1)(q^{n-2}-1)}{(q^2-1)(q-1)}, \\
&\text{the number of non-isotropic planes in $V$ is }\frac{q^{n-2}(q^n-1)}{q^2-1}.
\end{align}
\end{proposition}
\begin{proof}
A line $\ell$ in $V$ is determined by a nonzero vector. There are $q^n - 1$ nonzero vectors in $V$ and $q-1$ nonzero vectors in $\ell$. A plane $P$ is determined by a line $\ell_1 \subset V$ and a unique second line $\ell_2\in V/\ell_1$. We have $\frac{q^2-1}{q-1}$ choices for $\ell_1$ in $P$. The number of planes is therefore
$$
\frac{ \frac{q^n-1}{q-1} \cdot\frac{q^{n-1}-1}{q-1}}{\frac{q^2-1}{q-1} } = \frac{(q^n-1)(q^{n-1}-1)}{(q^2-1)(q-1)}.
$$
For an isotropic plane we have to choose the second line from $\ell_1^\perp/\ell_1$. This is a space of dimension $n-2$, hence the formula. The number of non-isotropic planes is the difference of the two previous numbers.
\end{proof}

Assume now that $V$ is a four-dimensional vector space over $k=\mathbb F_q$. Consider the free $k$-module $k[V]$ with basis $\{X_i \,|\, i\in V\}$. It carries a natural $k$-algebra structure, given by
$X_i\cdot X_j := X_{i+j}$ with unit $1=X_0$. This algebra is local with maximal ideal $\mathfrak m$ generated by all elements of the form $(X_i-1)$.

We introduce an action of $\Sp (4,k)$ on $k[V]$ by setting $\phi(X_i) = X_{\phi(i)}$. Furthermore, the underlying additive group of $V$ acts on $k[V]$ by $v( X_i) = X_{i+v} =X_iX_v$. 
\begin{definition} \label{SymplecticIdeal}
We define a subset of $k[V]$:
$$
N  := \left\{\sum_{i\in P}X_i \,|\, P\subset V \text{ non-isotropic plane}\right\}.
$$
Denote by $\left< N \right>$ and by $(N)$ the linear span of $N$ and the ideal generated by $N$, respectively. Note that $(N) $ is the linear span of $ \{ v\cdot b \,|\, b\in N, v\in V \}$.
Further, let $D$ be the linear span of $\{v(b) - b \,|\, b\in N, v\in V \}$. Then $D$ is in fact an ideal, namely the product of ideals $\mathfrak m\cdot (N)$.
\end{definition}
The following table illustrates the dimensions of these objects for some fields $k$:
\vspace{2mm}
\begin{center}
\begin{tabular}{c||c|c|c}
 $k$ & $\dim_k \left<N\right>$ & $\dim_k(N)$ & $\dim_k D$ \\
\hline
$\mathbb F_2$ & 10 & 11 &  5  \\
$\mathbb F_3$ & 30 & 50 & 31  \\
$\mathbb F_5$ &121 &355 &270
\end{tabular}
\end{center}
Let us now consider some special orthogonal sums.
Set $S:=\Sym^2 (\Lambda^2V)$. Take two vectors $v,w\in V$ with $\omega(v,w)=1$ and set $x:= (v\wedge w)^2\in S$. Denote $P$ the plane spanned by $v$ and $w$ and set $y:= \sum_{i\in P}X_i\in  k[V]$.
We set $Y':=y\cdot \mathfrak{m} = \{\sum_{i\in P} X_{i+j}-X_i\,|\, j\in V \} $.

We consider now the action of $\Sp V$ on $S\oplus k[V]$. 
Denote $O_1$ the vector space spanned by the elements $\phi(x)\oplus \phi(z),$ for $\phi \in \Sp V,$ $z \in (y)$ and
denote $O_2$ the vector space spanned by the elements $\phi(x)\oplus \phi(y'),$ for $\phi \in \Sp V,$ $y' \in Y'$.
Then we have:
\vspace{2mm}
\begin{center}
\begin{tabular}{c||c|c}
 $k$ & $\dim_k O_1$ & $\dim_k O_2 $  \\
\hline
$\mathbb F_2$ & 11 & 10  \\
$\mathbb F_3$ & 51 & 50  \\
$\mathbb F_5$ & 375 & 289
\end{tabular}
\end{center}
