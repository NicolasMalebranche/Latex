
\section{Source Code for the operator model}\label{VertexCode}
We give the source code for our tool implementing the rational cohomology of Hilbert schemes of points on surfaces using Nakajima operators. 
We follow the notational conventions of~\cite{LehnSorger}. The description given there in Section 3 allows to deduce an algorithm for operator actions on $\H$. The Chern classes of  tangent bundles of Hilbert schemes is computed with the help of the description from~\cite[Section 3]{Boissiere}.
%It is available online under \url{https://github.com/s--kapfer/HilbK3}. We used the language Haskell, compiled with the \textsc{GHC} software, version 7.6.3.

In contrast to the code in the previous section, we are not restricted to K3 surfaces. Indeed, the surface may have cohomology in odd degree as well as a non-vanishing canonical class. On the other hand, the implemention of the cup product in general is slower than the model for K3 surfaces. 

\subsection{How to use the code}
The main module is \verb|LS_Operators.hs|, which can be opened with ghci.
We implemented the actions of the following operators:
\begin{itemize}
 \item The Nakajima operators $\mathfrak p_n(a)$ and $L_n(a)$ from~\cite{LehnSorger} are given by \verb|P n a| and \verb|L n a|. 
 \item The differential operator $\partial$ is given by \verb|Del|.
 \item The multiplication operators $\G_k(a)$ from~\cite{LiQinWang} related to Chern characters correspond to \verb|Ch k a|.
 \item The Chern character $\mathfrak{ch}T$ of the tangent bundle from~\cite{Boissiere} in degree $k$ corresponds to \verb|ChT k|.
\end{itemize}
\begin{example}
To evaluate the action of a single operator product on the vacuum, use the command \verb|nakaState| to show the result in terms of Nakajima operators:
\begin{verbatim}
*LS_Operators>  let a = P2 0 in nakaState [P (-4) a, P (-2) a]
1 % 1 *         p_4(P2 0) p_2(P2 0) |0>
*LS_Operators>  let a = P2 0 in nakaState [Del,L(-3) a]
6 % 1 *         p_3(P2 2) |0> +
3 % 1 *         p_2(P2 1) p_1(P2 2) |0> +
3 % 1 *         p_2(P2 2) p_1(P2 1) |0> +
(-3) % 1 *      p_1(P2 0) p_1(P2 2) p_1(P2 2) |0> +
(-3) % 1 *      p_1(P2 1) p_1(P2 1) p_1(P2 2) |0>
*LS_Operators>  let a = P2 0 in 
                       nakaState [ChT 2, P(-1) a, P(-1) a, P(-1) a]
2 % 1 *         p_3(P2 0) |0> +
18 % 1 *        p_2(P2 1) p_1(P2 0) |0> +
(-9) % 1 *      p_1(P2 0) p_1(P2 1) p_1(P2 1) |0>
\end{verbatim}
\end{example}

\subsection{What the code does}
An important observation is that we do not need to know explicitly the commutator of $\G_k(a)$ with $\mathfrak p_n(b)$ to compute the action of $\G_k(a)$ on $\H$. 
Indeed, every element of $\H$ can be written as the action on the vacuum of either a polynomial in Nakajima operators $\mathfrak p_n(a)$ or of a polynomial in the operators $\partial$ and $\mathfrak p_{-1}(a)$. We call the two representations \verb|nakaState| and \verb|delState|, respectively. For the action of a Nakajima operator, the first one is more appropriate, while a multiplication operator acts better on the second one (multiplication commutes with $\partial$ and the commutators with $\mathfrak p_n(a)$ are known). 
In addition, the necessary commutation relations to switch between the two representations are known. This is the guiding philosophy for the algorithms contained in \verb|LS_Operators.hs|.

The other module, \verb|LS_Frobenius.hs| contains nothing but the definition of a graded Frobenius algebra according to~\cite[Section 2.1]{LehnSorger} and some instances, namely the cohomologies of K3 surfaces, complex tori and projective space $\C\mathbb P^2$.

The datatype that models $\H$ is called \verb|State|. It consists of linear combinations of ordered operator products, implemented as lists of pairs, containing the product (as a list) and the scalar. 

\subsection{Module for graded Frobenius algebras} 
\input{Sexy/LS_Frobenius.tex}
\subsection{Module for the commutator algebra} 
\input{Sexy/LS_Operators.tex}
