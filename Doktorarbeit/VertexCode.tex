
\section{Source Code for the operator model}\label{VertexCode}
We give the source code for our tool implementing the rational cohomology of Hilbert schemes of points on surfaces using Nakajima operators. 
We follow the notational conventions of \cite{LehnSorger}. The description given there in Section 3 allows to deduce an algorithm for operator actions on $\H$. The Chern classes of  tangent bundles of Hilbert schemes is computed with the help of the description from \cite[Section 3]{Boissiere}.
%It is available online under \url{https://github.com/s--kapfer/HilbK3}. We used the language Haskell, compiled with the \textsc{GHC} software, version 7.6.3.

In contrast to the code in the previous section, we are not restricted to K3 surfaces. Indeed, the surface may have cohomology in odd degree as well as a non-vanishing canonical class. On the other hand, the implemention of the cup product in general is slower than the model for K3 surfaces. 

\subsection{How to use the code}
The main module is \verb|LS_Operators.hs|, which can be opened with ghci.
We implemented the actions of the following operators:
\begin{itemize}
 \item The Nakajima operators $\mathfrak p_n(a)$ and $L_n(a)$ from \cite{LehnSorger} are given by \verb|P n a| and \verb|L n a|. 
 \item The differential operator $\partial$ is given by \verb|Del|.
 \item The multiplication operators $\G_k(a)$ from \cite{LiQinWang} related to Chern characters correspond to \verb|Ch k a|.
 \item The Chern character $\mathfrak{ch}T$ of the tangent bundle from \cite{Boissiere} in degree $k$ corresponds to \verb|ChT k|.
\end{itemize}
\begin{example}
To evaluate the action of a single operator product on the vacuum, use the command \verb|nakaState| to show the result in terms of Nakajima operators:
\begin{verbatim}
*LS_Operators>  let a = P2 0 in nakaState [P (-4) a, P (-2) a]
1 % 1 *         p_4(P2 0) p_2(P2 0) |0>
*LS_Operators>  let a = P2 0 in nakaState [Del,L(-3) a]
6 % 1 *         p_3(P2 2) |0> +
3 % 1 *         p_2(P2 1) p_1(P2 2) |0> +
3 % 1 *         p_2(P2 2) p_1(P2 1) |0> +
(-3) % 1 *      p_1(P2 0) p_1(P2 2) p_1(P2 2) |0> +
(-3) % 1 *      p_1(P2 1) p_1(P2 1) p_1(P2 2) |0>
*LS_Operators>  let a = P2 0 in 
                       nakaState [ChT 2, P(-1) a, P(-1) a, P(-1) a]
2 % 1 *         p_3(P2 0) |0> +
18 % 1 *        p_2(P2 1) p_1(P2 0) |0> +
(-9) % 1 *      p_1(P2 0) p_1(P2 1) p_1(P2 1) |0>
\end{verbatim}
\end{example}

\subsection{What the code does}
An important observation is that we do not need to know explicitly the commutator of $\G_k(a)$ with $\mathfrak p_n(b)$ to compute the action of $\G_k(a)$ on $\H$. 
Indeed, every element of $\H$ can be written as the action on the vacuum of either a polynomial in Nakajima operators $\mathfrak p_n(a)$ or of a polynomial in the operators $\partial$ and $\mathfrak p_{-1}(a)$. We call the two representations \verb|nakaState| and \verb|delState|, respectively. For the action of a Nakajima operator, the first one is more appropriate, while a multiplication operator acts better on the second one (multiplication commutes with $\partial$ and the commutators with $\mathfrak p_n(a)$ are known). 
In addition, the necessary commutation relations to switch between the two representations are known. This is the guiding philosophy for the algorithms contained in \verb|LS_Operators.hs|.

The other module, \verb|LS_Frobenius.hs| contains nothing but the definition of a graded Frobenius algebra according to \cite[Section 2.1]{LehnSorger} and some instances, namely the cohomologies of K3 surfaces, complex tori and projective space $\C\mathbb P^2$.

The datatype that models $\H$ is called \verb|State|. It consists of linear combinations of ordered operator products, implemented as lists of pairs, containing the product (as a list) and the scalar. 

\subsection{Module for graded Frobenius algebras} 
\begin{lstlisting}
{-# LANGUAGE GeneralizedNewtypeDeriving, TypeOperators, TypeFamilies #-}
module LS_Frobenius 
	where

import Data.Array
import Data.List
import Data.MemoTrie

-- the d. We are dealing with surfaces, so d=2.
gfa_d = 2 :: Int

class (Ix k,Ord k)=> GradedFrobeniusAlgebra k where
	gfa_deg :: k -> Int
	gfa_base :: [k]
	gfa_baseOfDeg :: Int -> [k]
	gfa_1 :: Num a => [(k,a)]
	gfa_K :: Num a => [(k,a)]
	gfa_T :: Num a => k -> a
	gfa_mult :: Num a => k -> k -> [(k,a)]
	gfa_bilinear :: Num a => k -> k -> a
	gfa_bilinear i j = sum [ gfa_T k * x | (k,x) <- gfa_mult i j ]
	gfa_bilinearInverse :: Num a => k -> [(k, a)]

-- Tensor product
instance (GradedFrobeniusAlgebra k, GradedFrobeniusAlgebra k') 
		=> GradedFrobeniusAlgebra (k,k') where
	gfa_deg (i,j) = gfa_deg i + gfa_deg j
	gfa_base = [(i,j) | i<-gfa_base, j<-gfa_base]
	gfa_baseOfDeg n = [(i,j) | i<-gfa_base, j<-gfa_baseOfDeg (n-gfa_deg i) ]
	gfa_1 = [((i,j),x*y) | (i,x) <- gfa_1, (j,y) <- gfa_1]
	gfa_K = [ ((i,j),x*y) | (i,x) <- gfa_K, (j,y) <- gfa_1] ++ 
			[ ((i,j),x*y) | (i,x) <- gfa_1, (j,y) <- gfa_K] 
	gfa_T (i,j) = gfa_T i * gfa_T j
	gfa_mult (i,j) (k,l) = [((m,n),ep*x*y) | (m,x)<-gfa_mult i k, (n,y)<-gfa_mult j l] where
		ep = if odd (gfa_deg j * gfa_deg k) then (-1) else 1
	gfa_bilinearInverse (i,j) = [ ((k,l),ep k *x*y) | 
		(k,x) <- gfa_bilinearInverse i, (l,y) <-gfa_bilinearInverse j] where
			ep k = if odd (gfa_deg k * gfa_deg j) then (-1) else 1

-- base for Symmetric tensors
gfa_symBase n = cs n gfa_base where
	f [] = []
	f l@(a:b) = (a, if odd (gfa_deg a) then b else l ) : f b
	cs 0 _ = [[]]
	cs k b@(a:r) = [ x:t | (x,r) <- f b, t <- cs (k-1) r] 
-- power is n, degree is k
gfa_symBaseOfDeg n k = csd n k gfa_base where
	f [] = []
	f l@(a:b) = (a, d, if odd d then b else l) : f b where d = gfa_deg a
	csd 0 0 _ = [[]]
	csd n k b@(a:r) = if n<= 0 then [] else
		[ x:t | (x,d,r) <- f b, t <- csd (n-1) (k-d) r] 
	
	
gfa_multList [] = gfa_1
gfa_multList [i] = [(i,1)]
gfa_multList [i,j] = gfa_mult i j
gfa_multList (i:r) = sparseNub [ (k,y*x) | (j,x)<-gfa_multList r, (k,y)<-gfa_mult i j ]

gfa_adjoint f = adj where
	b i = [(j,x) | j<-gfa_base, let x = gfa_bilinear j i, x/=0]
	ftrans = accumArray (flip (:)) [] (head gfa_base, last gfa_base)
			[ (j,(i,x)) | i <-gfa_base, (j,x) <- f i] 
	ftb i = sparseNub [ (k,y*x) | (j,x) <- b i, (k,y) <- ftrans!j]
	adj i = sparseNub [ (k,y*x) | (j,x) <- ftb i, (k,y) <- gfa_bilinearInverse j]

gfa_comult :: (GradedFrobeniusAlgebra k, Ix k,Num a,Eq a) => k -> [((k,k),a)]
gfa_comult = gfa_adjoint (uncurry gfa_mult)

gfa_comultN 0 a = [([a],1)]
gfa_comultN n a = let
	rec = gfa_comultN (n-1) a
	in sparseNub [ (c:d:r, x*y) | (b:r,x) <- rec, ((c,d),y) <- gfa_comult b]

gfa_euler :: (GradedFrobeniusAlgebra k, Ix k, Num a) => [(k,a)]
gfa_euler = [(k,fromIntegral x) | (k,x)<-e] where 
	e = sparseNub [ (k,y*x*v) | (u,v) <- gfa_1, (ij,x) <- gfa_comult u, (k,y) <- uncurry gfa_mult ij] 

sparseNub [] = []
sparseNub l = sn (head sl) (tail sl) where
	sl = sortBy ((.fst).compare.fst) l
	sn (i,x) ((j,y):r) = if i==j then sn (i,x+y) r else app (i,x) $ sn (j,y) r
	sn ix [] = app ix []
	app (i,x) r = if x==0 then r else (i,x) : r

scal 0 _ = []
scal a l = [ (p,a*x) | (p,x) <- l]

-- Cohomology of K3 surfaces
newtype K3Domain = K3 Int deriving (Enum,Eq,Num,Ord,Ix)
instance Show K3Domain where show (K3 i) = show i
instance GradedFrobeniusAlgebra K3Domain where
	gfa_deg (K3 0) = -2
	gfa_deg (K3 23) = 2
	gfa_deg i = if 1<=i && i<=22 then 0 else undefined
	
	gfa_1 = [(0,1)]
	gfa_K = []
	
	gfa_T (K3 23) = -1
	gfa_T _ = 0
	
	gfa_base = [0..23]
	gfa_baseOfDeg 0 = [1..22]
	gfa_baseOfDeg (-2) = [0]
	gfa_baseOfDeg 2 = [23]
	gfa_baseOfDeg _ = []
	
	gfa_mult (K3 0) i = [(i,1)]
	gfa_mult i (K3 0) = [(i,1)]
	gfa_mult (K3 23) _ = []
	gfa_mult _ (K3 23) = []
	gfa_mult (K3 i) (K3 j) =  [(23, fromIntegral $ bilK3_func i j)]
	
	gfa_bilinearInverse (K3 i) = [(K3 j,fromIntegral x) | j<-[0..23], let x =bilK3inv_func i j, x/=0]

delta i j = if i==j then 1 else 0

e8 = array ((1,1),(8,8)) $
	zip [(i,j) | i <- [1..8],j <-[1..8]] [
	-2, 1, 0, 0, 0, 0, 0, 0,
	1, -2, 1, 0, 0, 0, 0, 0,
	0, 1, -2, 1, 0, 0, 0, 0,
	0, 0, 1, -2, 1, 0, 0, 0,
	0, 0, 0, 1, -2, 1, 1, 0,
	0, 0, 0, 0, 1, -2, 0, 1,
	0, 0, 0, 0, 1, 0, -2, 0,
	0, 0, 0, 0, 0, 1, 0, -2 :: Int]

inve8= array ((1,1),(8,8)) $
	zip [(i,j) | i <- [1..8],j <-[1..8]] 
	    [-2, -3, -4, -5, -6, -4, -3, -2, -3, -6, -8, -10, -12, -8, -6, -4,
		-4, -8, -12, -15, -18, -12, -9, -6, -5, -10, -15, -20, -24, -16, -12,
		-8, -6, -12, -18, -24, -30, -20, -15, -10, -4, -8, -12, -16, -20,
		-14, -10, -7, -3, -6, -9, -12, -15, -10, -8, -5, -2, -4, -6, -8,
		-10, -7, -5, -4 :: Int]

-- Bilinear form on K3 surfaces
bilK3_func ii jj = let 
	(i,j) = (min ii jj, max ii jj) 
	u 1 2 = 1
	u 2 1 = 1
	u 1 1 = 0
	u 2 2 = 0
	u i j = undefined	
	in 
	if (i < 0) || (j > 23) then undefined else
	if (i == 0) then delta j 23 else
	if (i >= 1) && (j <= 2) then u i j else
	if (i >= 3) && (j <= 4) then u (i-2) (j-2) else
	if (i >= 5) && (j <= 6) then u (i-4) (j-4) else
	if (i >= 7) && (j <= 14) then e8 ! ((i-6), (j-6)) else
	if (i >= 15) && (j<= 22) then e8 ! ((i-14), (j-14))  else
	0 :: Int

-- Inverse bilinear form
bilK3inv_func ii jj = let 
	(i,j) = (min ii jj, max ii jj) 
	u 1 2 = 1
	u 2 1 = 1
	u 1 1 = 0
	u 2 2 = 0
	u i j = undefined	
	in 
	if (i < 0) || (j > 23) then undefined else
	if (i == 0) then delta j 23 else
	if (i >= 1) && (j <= 2) then u i j else
	if (i >= 3) && (j <= 4) then u (i-2) (j-2) else
	if (i >= 5) && (j <= 6) then u (i-4) (j-4) else
	if (i >= 7) && (j <= 14) then inve8 ! ((i-6), (j-6)) else
	if (i >= 15) && (j<= 22) then inve8 ! ((i-14), (j-14))  else
	0 :: Int

-- Cohomology of projective space
newtype P2Domain = P2 Int deriving (Show,Eq,Ord,Ix,Num)
instance GradedFrobeniusAlgebra P2Domain where
	gfa_deg (P2 0) = -2
	gfa_deg (P2 1) = 0
	gfa_deg (P2 2) = 2
	
	gfa_1 = [(P2 0,1)]
	gfa_K = [(P2 1,-3)]
	
	gfa_T (P2 2) = -1
	gfa_T _ = 0
	
	gfa_base = [0,1,2]
	gfa_baseOfDeg 0 = [1]
	gfa_baseOfDeg (-2) = [0]
	gfa_baseOfDeg 2 = [2]
	gfa_baseOfDeg _ = []
	
	gfa_mult (P2 0) i = [(i,1)]
	gfa_mult i (P2 0) = [(i,1)]
	gfa_mult (P2 2) _ = []
	gfa_mult _ (P2 2) = []
	gfa_mult (P2 1) (P2 1) =  [(2, 1)]

	gfa_bilinearInverse i =  [(2-i, 1)]


-- Cohomology of complex torus
newtype TorusDomain = Tor Int deriving (Enum,Eq,Num,Ord,Ix)
instance Show TorusDomain where show (Tor i) = show i
instance GradedFrobeniusAlgebra TorusDomain where
	gfa_deg (Tor i) = 
		if i<0 then undefined else
		if i<=0 then -2 else
		if i<=4 then -1 else
		if i<=10 then 0 else
		if i<=14 then 1 else
		if i==15 then 2 else undefined
		
	gfa_1 = [(0,1)]
	gfa_K = []
	
	gfa_T (Tor 15) = -1
	gfa_T _ = 0
	
	gfa_base = [0..15]
	gfa_baseOfDeg (-2) = [0]
	gfa_baseOfDeg (-1) = [1..4]
	gfa_baseOfDeg 0 = [5..10]
	gfa_baseOfDeg 1 = [11..14]
	gfa_baseOfDeg 2 = [15]
	gfa_baseOfDeg _ = []
	
	gfa_mult i j = if k<0 then [] else [(k,fromIntegral x)] where (k,x)= torusMultArray!(i,j)
	
	gfa_bilinearInverse i = [(15-i,gfa_bilinear i (15-i))]

torusMultArray = listArray ((0,0),(15,15)) mults where  
	toLists = listArray (0,15) list
	list = [[],[1],[2],[3],[4],[1,2],[1,3],[1,4],[2,3],[2,4],
			[3,4],[1,2,3],[1,2,4],[1,3,4],[2,3,4],[1,2,3,4::Int]]
	fromLists x = findIndex (x==) list
	mults = [ check (toLists!i ++ toLists!j)  | i<-[0..15],j<-[0..15]]
	check ab = case fromLists $ sort ab of
		Nothing -> (-1,0)
		Just i -> (Tor i, sign ab)
	sign [] = 1; sign (i:r) = sign r * signum (product [j-i | j<-r])
\end{lstlisting}

\subsection{Module for the commutator algebra} 
\begin{lstlisting}
{-# LANGUAGE GeneralizedNewtypeDeriving, ParallelListComp #-}
module LS_Operators
	where

import LS_Frobenius

data VertexOperator k = P Int k | L Int k | Del | Ch Int k | GV Int k | ChT Int deriving (Show,Eq,Ord)

newtype State a k = Vak { unVak:: [([VertexOperator k],a)] }

weight (P n _) = -n
weight (L n _) = -n
weight Del = 0
weight (Ch _ _) = 0
weight (GV _ _) = 0
weight (ChT _) = 0

degree (P n k) = gfa_deg k
degree (L n k) = gfa_deg k + gfa_d
degree Del = 2
degree (Ch n k) = gfa_deg k + 2*n
degree (GV n k) = gfa_deg k + 2*n
degree (ChT n) = 2*n


data ActsOn = DelState | Both | NakaState deriving (Show,Eq)


actsOn (P _ _) = Both
actsOn Del = Both
actsOn (L _ _) = NakaState
actsOn (Ch _ _) = DelState
actsOn (GV _ _) = DelState
actsOn (ChT _) = DelState


actOnNakaVac p@(P n _) = Vak $ if n<0 then [([p],1)] else []
actOnNakaVac (L n k) = Vak $ sparseNub [(o,y*x/2) | 
	nu <- [n+1 .. -1], ((a,b),x) <- gfa_comult k, (o,y) <-unVak$nakaState $op nu a b ] where
	op nu a b = if n-nu > 0 then [P nu a, P (n-nu) b] else [P (n-nu) a, P nu b]
actOnNakaVac Del = Vak []

actOnDelVac p@(P n k) = Vak $ if n >= 0 then [] else scal ( 1/ fac) $ rec n where
	fac = fromIntegral $ product [n+1 .. -1] 
	rec (-1) = [([P(-1) k],1)]
	rec n = sparseNub [ t | (o,x) <- rec (n+1), (oo,y) <- p', t<-[(oo++o,x*y),(o++oo,-x*y)]]   
	p' = [ ([Del,P(-1) a], x) | (a,x) <-gfa_1 ] ++ [ ([P(-1) a,Del], -x) | (a,x) <-gfa_1 ] 
actOnDelVac Del = Vak  []
actOnDelVac (Ch _ _) = Vak  []
actOnDelVac (GV _ _) = Vak  []
actOnDelVac (ChT _) = Vak []

-- ad(Del)^n(op)/n!
facDiff n op = let 
	bins 0 = [1]
	bins n = zipWith (-) (b++[0]) (0:b) where b = bins (n-1)
	ders = scanr (:) [] $ replicate (fromIntegral n) Del
	dels = scanl (flip (:)) [] $ replicate (fromIntegral n) Del
	fac = product [1..fromIntegral n] 
	in [ (d1++[op]++d2,fromIntegral b/fac) | d1<-ders | d2 <- dels | b <- bins n]

ad n u v = let
	rec = ad (n-1) u v
	--com = [ ([Del,P(-1) 0],1) , ([P(-1) 0,Del], -1)]
	new = [ z | (x,a) <- rec, (y,b) <- u, z <- [(x++y,-a*b),(y++x,a*b)]]
	in if n==0 then v else sparseNub new

commutator (P n a) (P m b) = if n+m==0 then [ ([], gfa_bilinear a b*fromIntegral n) ] else []
commutator Del (P n a) = ([L n a], fromIntegral n) : 
	[ ([P n c],x*y*sc) | (b,x) <- gfa_K, (c,y) <- gfa_mult b a] where
	sc = fromIntegral $ (-n*(abs n - 1) ) `div` 2
commutator p@(P _ _) Del = scal (-1) $ commutator Del p
commutator (L n a) (P m b) = [ ([P (n+m) c], x*fromIntegral(-m)) | (c,x) <- gfa_mult a b ]
commutator (Ch _ _) Del = []
commutator (Ch n a) (P (-1) y) = [ (c,x*fromRational z) | (b,x) <- gfa_mult a y, (c,z) <- facDiff n (P (-1) b) ]
commutator (GV _ _) Del = []
commutator (GV n a) (P (-1) y) = if odd n then [(s,negate x) | (s,x) <-csn] else csn where
	csn = commutator (Ch n a) (P (-1) y) 
commutator (ChT _) Del = []
commutator (ChT n) p@(P (-1) y) =  sparseNub $ first ++ second ++ third ++ fourth ++ fifth where
	k2= [(c,2*x*xx*z) | (a,x) <-gfa_K, (b,xx) <-gfa_K, (c,z) <- gfa_mult a b]
	todd_Inv_y = [[(y,1)], [(b,x*xx/2) | (a,x)<-gfa_K, (b,xx) <- gfa_mult a y],  
		sparseNub [(b,x*xx) | (a,x)<-scal (1/6) k2 ++ scal (1/12) gfa_euler, (b,xx) <- gfa_mult a y]]
	exp_K_y = [[(y,1)], [ (b,-x*xx) | (a,x) <- gfa_K, (b,xx) <- gfa_mult a y] ,
		[ (b,x*xx/2)| (a,x) <-k2, (b,xx) <- gfa_mult a y ] ]
	expTodd_y = zipWith scal [1,-1,1] todd_Inv_y
	first = [ (c,x) | (c,x) <-facDiff n p ]
	second = [ ( o++[GV gn b2], x*xx*z) | k <- [0..2] , (a,x) <- todd_Inv_y!!k, ((b1,b2),xx) <- gfa_comult a, 
		nu <- [0..n-k-2], let gn = n-nu-k-2, (o,z) <- facDiff nu (P (-1) b1) ]
	third = [ (c,x*xx*(-1)^nu) | nu<-[max (n-2) 0..n], 
		(a,x) <- exp_K_y !! (n-nu), (c,xx) <-facDiff nu (P (-1) a) ]
	fourth = [ ( o++[Ch gn b2], x*xx*z*(-1)^nu) | k <- [0..2] , (a,x) <- expTodd_y!!k, ((b1,b2),xx) <- gfa_comult a, 
		nu <- [0..n-k-2], let gn = n-nu-k-2, (o,z) <- facDiff nu (P (-1) b1) ]
	fifth = if n==2 then [ ([P(-1) b], x*xx) | (a,x) <- gfa_euler, (b,xx) <- gfa_mult a y] else []


showOperatorList [] = "|0>"
showOperatorList (Del:r) = "D " ++ showOperatorList r
showOperatorList (P n k:r) = sh ++ showOperatorList r where
	sh = (if n<0 then "p_"++show(-n)else "p"++show n)++"("++show k++") "
showOperatorList (L n k:r) = sh ++ showOperatorList r where
	sh = (if n<0 then "L_"++show(-n)else "L"++show n)++"("++show k++") "
showOperatorList (Ch n k:r) = "ch"++show n++"["++show k++ "] " ++ showOperatorList r 

instance (Show a, Show k) => Show (State a k) where
	show (Vak []) = "0"
	show (Vak [(l,x)]) = show x ++ " * \t"++showOperatorList l
	show (Vak ((l,x):r)) = show x ++ " * \t"++showOperatorList l ++ " +\n"++show(Vak r) 



-- Operator product acting on Vacuum. Result is given in terms of deltas and P(-1) operators.
delState :: (GradedFrobeniusAlgebra k, Ord k) => [VertexOperator k] -> State Rational k

delState [] = Vak [([],1)] 
delState (o:r) = if actsOn o == NakaState then toDel $ nakaState (o:r) else result where
	result = Vak $ sparseNub[ (q,x*y) | (s,x) <-unVak$ delState r, (q,y) <- unVak $ commuteIn s]
	commuteIn [] = actOnDelVac o
	commuteIn (pd:r) = case (o,pd) of 
		(Del,_) -> Vak [ (Del:pd:r,1) ]
		(P (-1) _, Del) -> Vak [ (o:pd:r,1) ]
		(P (-1) a, P (-1) b) -> if a <= b then Vak [(o:pd:r,1)] else Vak cI
		_ -> Vak cI
		where
		cI = case comm of [] -> ted; _ -> sparseNub $ ted ++ comm
		ted = [(pd:q,x*sign) | (q,x) <- unVak $ commuteIn r]
		comm =  [ (ds,x*y) | (q,x) <- commutator o pd, (ds,y) <- unVak $ delState (q++r) ]
		sign= if odd (degree pd) && odd (degree o) then -1 else 1


-- Operator product acting on Vacuum. Result is given in terms of creation operators.
nakaState :: (GradedFrobeniusAlgebra k, Ord k) => [VertexOperator k] -> State Rational k

nakaState [] = Vak [([],1)]
nakaState (o:r) = if actsOn o == DelState then toNaka $ delState (o:r) else result where
	result = Vak $ sparseNub[ (q,x*y) | (s,x) <- unVak $ nakaState r, (q,y) <- unVak $ commuteIn s]
	nakaSort p [] = ([p],1)
	nakaSort p (q:r) = case (odd (degree p)&& odd (degree q), compare p q) of
		(True,EQ) -> (p:q:r,0)
		(v, GT) -> (q:n, if v then -s else s) where (n,s) = nakaSort p r
		_ -> (p:q:r,1) 		
	commuteIn [] = actOnNakaVac o
	commuteIn (p:r) = case (o,p) of
		(P _ _, P _ _) -> if o<p then Vak [(o:p:r,1)] else Vak cI 
		_ -> Vak cI
		where
		cI = case comm of [] -> ted; _ -> sparseNub $ ted ++ comm
		ted = [(n,x*sign*sign2) | (q,x) <- unVak $ commuteIn r, let (n,sign2)=nakaSort p q]
		comm =  [ (ds,x*y) | (q,x) <- commutator o p, (ds,y) <- unVak $ nakaState (q++r) ]
		sign= if odd (degree p) && odd (degree o) then -1 else 1

-- Transforms state representations
toDel (Vak l)  = Vak $ sparseNub[ (p,x*y)|(o,x) <- l, (p,y) <- unVak$delState o] 
toNaka (Vak l) = Vak $ sparseNub[ (p,x*y)|(o,x) <- l, (p,y) <- unVak$nakaState o] 

scale a (Vak sta) = Vak $ scal a sta
add (Vak s) (Vak t) = Vak $ sparseNub $ s ++ t

multLists l stat = toNaka $ ml l stat where
	ml [] stat = stat
	ml (l:r) stat = let 
		Vak s = ml r stat
		ns = sparseNub[ (t,x*y*z) | (a,x) <- s, (o,y) <- l, (t,z)<- unVak $ delState (o++a) ]
		in Vak ns

-- Chern classes related to ChT
cT = (!!) c where
	c = [([],1::Rational)] : [if odd k then [] else cc k | k<-[1..] ]
	cc k = [ (ChT (2*i):o, x*fact(2*i)/fromIntegral(-k) ) | i<-[1..div k 2], (o,x) <- cT (k-2*i) ]
	fact n = fromIntegral $ product [1..n]

\end{lstlisting}
