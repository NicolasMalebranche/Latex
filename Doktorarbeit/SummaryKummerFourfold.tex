\section[Conclusion on cohomology of generalized Kummer fourfolds]{Conclusion on cohomology of generalized Kummer fourfolds %
\sectionmark{Cohomology of generalized Kummer fourfolds}}
\sectionmark{Cohomology of generalized Kummer fourfolds}
Let $A$ be a complex abelian surface with integral basis of $H^2(A,\Z)$ given as in Notation~\ref{TorusClasses}. Let $\theta: \kum{A}{2}\hookrightarrow A\hilb{3}$ be the embedding. 
Let us summarize our results:
\begin{theorem}\label{thetaTheorem}
The homomorphism $\theta^*:H^*(A\hilb{3},\Z)\rightarrow H^*(\kum{A}{2},\Z)$ of graded rings is surjective in every degree except $4$. Moreover, the image of $H^4(A\hilb{3},\Z)$ is the primitive overlattice of $\Sym^2(H^2(\kum{A}{2},\Z))$. 
The kernel of $\theta^*$ is the ideal generated by $H^1(A\hilb{3},\Z)$.
The image of the following integral classes under $\theta^*$ gives a basis of $\im\theta^*$:
\begin{center}
\begin{tabular}{c|l|l}
degree & preimage of class & \\
\hline
0 & $\frac{1}{6} \q_1(1)^3\vac$ & 1 \\
\hline
2 &  $\frac{1}{2}\q_1(b_i) \q_1(1)^2\vac$ for $1\leq i\leq 6$ & $j(b_i)$ \\
 & $\frac{1}{2} \q_2(1)\q_1(1)\vac $  & $e$\\
\hline
3 & $\frac{1}{2}\q_1(a^*_i)\q_1(1)^2\vac$ & \\
  & $\q_2(a_i)\q_1(1)\vac$ & \\
\hline
4 & $\q_1(b_i)\q_1(b_j)\q_1(1)\vac$ for $1\leq i\leq j\leq 6$, but $(b_i,b_j)\neq(a_1a_2,a_3a_4)$ &\\
  & $\frac{1}{2}\q_1(x)\q_1(1)^2\vac$ (instead of the missing case above)  & $Y_p$\\
  & $\frac{1}{2}\left(\q_1(b_i)^2-\q_2(b_i)\right)\q_1(1)\vac$ & \\
  & $\frac{1}{3} \q_3(1)\vac$ & $W$ \\
\hline
5 & $\q_1(a_ia_j)\q_1(a_j^*)\q_1(1) \vac$ for any choice of $j\neq i$ &\\
  & $\frac{1}{2} \q_2(a^*_i)\q_1(1)\vac $ &\\
\hline
6 & $\q_1(a_i^*)\q_1(a_j^*)\q_1(1)\vac$ for $1\leq i< j\leq 4$ & \\
  & $\q_2(x)\q_1(1)\vac$ & \\
\hline
8 & $\q_1(x)^3\vac$ &
\end{tabular}
\end{center}
\end{theorem}
\begin{proof}
The table is established by the following results:
For degree 2, see Proposition~\ref{H2Sur}. The dual classes of degree 6 can be computed using Proposition~\ref{KummerClass} and the methods from Section~\ref{Section_Hilbert}.
The odd degrees are treated by Proposition~\ref{oddcohomology}. Classes of degree 4 are studied in Section~\ref{Middle}. The classes are chosen in a way that they give a basis of $\Sym^{sat}$, which is possible by Corollary~\ref{SymSatImage}. The condition $(b_i,b_j)\neq(a_1a_2,a_3a_4)$ is more or less arbitrary, but we had to remove one class to avoid a relation of linear dependence.

The kernel of $\theta^*$ is described by the Propositions~\ref{annihilator} and~\ref{Annihideal}.
\end{proof}


\begin{thm}\label{integralbasistheorem}
Using Notation~\ref{BasisH2KA} and~\ref{TheZs}, we have:
\begin{enumerate}
\item
Let $\Sym^{sat}\!\subset \! H^4(K_2(A),\Z)$ be the primitive overlattice of the symmetric product $\Sym^2\!\left(H^2\!\left(\X,\Z\right)\right)$.
The group $\frac{\Sym^{sat}}{\Sym^2\left(H^2\left(\X,\Z\right)\right)}=(\Z/2\Z)^{\oplus 7}\oplus(\Z/3\Z)^{\oplus 8}$ is generated by the following elements:
\begin{gather*}
\frac{e \cdot y}{3},\ \frac{y^2 - \frac{1}{3} e\cdot y}{2} \text{ for } y\in\{u_1,u_2,v_1,v_2,w_1,w_2\},\\
\frac{e^2}{3} \text{ and } \frac{u_{1}\cdot u_{2}+v_{1}\cdot v_{2}+w_{1}\cdot w_{2}}{6}.
\end{gather*}
\item
Let $\Pi'$ be the lattice from Definition~\ref{defiPi} and let $\Pi'^{sat}$ be the primitive over lattice of $\Pi'$ in $H^4(K_2(A),\Z)$.
The group $\frac{\Pi'^{sat}}{\Pi'\ \ \ }=(\Z/3\Z)^{\oplus 31}$ is generated by the classes:
$$\frac{1}{3}\sum_{\tau\in\Lambda} \Big(Z_{\tau} - Z_{\tau+\tau'}\Big),
$$
with $\Lambda$ a non-isotropic group and $\tau'\in A[3]$. Moreover a basis of $\frac{\Pi'^{sat}}{\Pi'\ \ \ }$ is provided by the 31 classes described in Proposition~\ref{XXXI}. 
\item
$$
\frac{H^4(K_2(A),\Z)}{\Sym^{sat}\oplus\Pi'^{sat}}=\left(\frac{\Z}{27\Z}\right)\oplus\left(\frac{\Z}{3\Z}\right)^{\oplus 19}.
$$
Moreover, this group is generated by the class $Z_0$ and the 19 classes described in Proposition~\ref{XIX}.
\qed
\end{enumerate}
\end{thm}

%Proof: look at the ranks of H^*(A\hilb{3}):
%Rank H1*H4 = 188
%Rank H1*H5 = 239
%Rank H1*H6 = 196
%Rank H1*H7 = 102
%Rank H1*H8 = 40

