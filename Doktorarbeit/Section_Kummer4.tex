\section{Odd Cohomology of the Generalized Kummer fourfold}
Now we come to the special case $n=3$, so we study $\kum{A}{2}$, the Generalized Kummer fourfolds.
\begin{proposition}
The Betti numbers of $\kum{A}{2}$ are:
$
1,\,0,\,7,\,8,\,108,\,8,\,7,\,0,\,1.
$
\end{proposition}
\begin{proof}
This follows from G\"ottsche's formula \cite[page 49]{Gottsche}.
\end{proof}

By means of the morphism $\theta^*$, we may express part of the cohomology of $\kum{A}{2}$ in terms of Hilbert scheme cohomology. We have seen in Proposition \ref{H2Sur} that $\theta^*$ is surjective for degree $2$ and (by duality) also in degree $6$. 
The next proposition shows that this also holds true for odd degrees.
\begin{proposition}\label{oddcohomology}
A basis of $H^3(\X,\Z)$ is given by:
\begin{gather}
\label{A3_1}
\frac{1}{2}\theta^*\Big( \q_1(a^*_i)\q_1(1)^2\vac \Big), \\
\label{A3_2}
\theta^*\Big(\q_2(a_i)\q_1(1)\vac\Big).
\end{gather}
and a basis of $H^5(\X,\Z)$ is given by:
\begin{gather}
\label{A5_1}
\frac{2}{3} \theta^*\Big( \G_2(a_i) \Big), \\
\label{A5_2}
\frac{1}{2}\theta^*\Big( \q_2(a^*_i)\q_1(1)\vac \Big).
\end{gather}
\end{proposition}
\begin{proof}
The classes (\ref{A3_1}) are Poincar\'e dual to (\ref{A5_1}) and the classes (\ref{A3_2}) are Poincar\'e dual to (\ref{A5_2}) by computation, so it remains to show that all of them are integral.

It is clear from Theorem \ref{IntegralOperators} that (\ref{A3_1}) and (\ref{A3_2}) are integral. By Proposition \ref{A2Basis}, $\frac{1}{2}\q_2(a^*_i)\vac$ is integral as well. If the operator $ \q_1(1)$ is applied, we get again an integral class.

Further, $\frac{2}{3} \G_2(a_i)[\kum{A}{2}]$ is an integral class and $[\kum{A}{2}]\cdot\q_1(a^*_i)\q_1(1)^2\vac$ is non-divisible. 
Because of $\int\limits_{A\hilb{3}} [\kum{A}{2}]\cdot \G_2(a_i)  \q_1(a^*_i)\q_1(1)^2\vac = 3$, one of $\frac{1}{2}\theta^*\left( \q_1(a^*_i)\q_1(1)^2\vac \right)$ and $ 2\theta^*\left( \G_2(a_i) \right)$ is divisible by three. In view of Lemma \ref{IntegralityCheck}, it must be the latter one.
\end{proof}
It follows the following corollary which will be used in Part \ref{quotient}.
%\textcolor{green}{Corollary \ref{actionH3} to put after Proposition 3.18 \ref{} saying that it will be use in Part 2}
\begin{cor}\label{actionH3}
Let $A$ be an abelian surface and $g$ be an automorphisms on $A$. Let $g^{[[3]]}$ be the automorphisms induced by $g$ on $K_2(A)$.
We have $H^3(K_2(A),\Z)\simeq H^1(A,\Z)\oplus H^3(A,\Z)$ and the action of $g^{[[3]]}$ on $H^3(K_2(A),\Z)$ is given by the action of $g$ on $H^1(A,\Z)\oplus H^3(A,\Z)$.
\end{cor}
\begin{proof}
Let $g^{[3]}$ be the involution on $A^{[3]}$ induced by $g$.
We have $g^{[3]*}(a_{i}^{(1)})=(g^{*}a_{i})^{(1)}$ and $g^{[3]*}(a_{\overline{i}}^{(0)})=(g^{*}a_{\overline{i}})^{(0)}$.
Moreover, we have by definition, $g^{[[3]]*}\circ \theta^{*}=\theta^{*}\circ g^{[3]*}$.
The result follows from Proposition \ref{oddcohomology}.
\end{proof}

Let us summarize our results on $\theta^*$:
\begin{theorem}\label{thetaTheorem}
The homomorphism $\theta^* : H^*(A\hilb{3},\Q)\rightarrow H^*(\kum{A}{2},\Q)$ of graded rings is surjective in every degree except $4$. Moreover, the image of $H^4(A\hilb{3},\Q)$ is equal to $\Sym^2(H^2(\kum{A}{2},\Q))$. 
The kernel of $\theta^*$ is the ideal generated by $H^1(A\hilb{3},\Q)$.
%Proof: look at the ranks of H^*(A\hilb{3}):
%Rank H1*H4 = 188
%Rank H1*H5 = 239
%Rank H1*H6 = 196
%Rank H1*H7 = 102
%Rank H1*H8 = 40
\end{theorem}