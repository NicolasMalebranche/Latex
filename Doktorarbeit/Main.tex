\documentclass[letter]{article}

\usepackage[latin1]{inputenc}
\usepackage[T1]{fontenc}
\usepackage{lmodern}
\usepackage[protrusion=true,expansion=true]{microtype}
\usepackage{amsmath,amssymb,amsfonts,amscd,amsthm,mathrsfs,mathtools}
\usepackage[all]{xy}
\usepackage{appendix,
listings,
hyperref}
\usepackage{color}
\usepackage{graphicx}
%\usepackage[letterpaper]{geometry}
\usepackage{fancyvrb}
\usepackage{setspace}
\usepackage{enumitem}

\DeclareMathAlphabet{\mathpzc}{OT1}{pzc}{m}{it}

\DeclareMathOperator{\rank}{rank}
\DeclareMathOperator{\trace}{tr}
\DeclareMathOperator{\Tor}{Tor}
\DeclareMathOperator{\Ext}{Ext}
\DeclareMathOperator{\Aut}{Aut}
\DeclareMathOperator{\End}{End}
\DeclareMathOperator{\id}{id}
\DeclareMathOperator{\Hom}{Hom}
\DeclareMathOperator{\im}{Im}
\DeclareMathOperator{\Ker}{Ker}
\DeclareMathOperator{\Sym}{Sym}
\DeclareMathOperator{\SSym}{\mathbb Sym}
\DeclareMathOperator{\Hilb}{Hilb}
\DeclareMathOperator{\GL}{GL}
\DeclareMathOperator{\Sp}{Sp}
\DeclareMathOperator{\ch}{ch}
\DeclareMathOperator{\rk}{rk}
\DeclareMathOperator{\ad}{ad}
\DeclareMathOperator{\td}{td}
\DeclareMathOperator{\pr}{pr}
\DeclareMathOperator{\supp}{supp}
\DeclareMathOperator{\discr}{discr}
\DeclareMathOperator{\Vect}{Vect}
\DeclareMathOperator{\coker}{coker}
\DeclareMathOperator{\Fix}{Fix}
\DeclareMathOperator{\codim}{Codim}
\DeclareMathOperator{\Supp}{Supp}
\DeclareMathOperator{\NS}{NS}
\DeclareMathOperator{\sing}{Sing}
\DeclareMathOperator{\len}{len}
\DeclareMathOperator{\Mon}{Mon}


\newcommand{\hilb}[1]{^{[#1]}}
\newcommand{\ie}{{\it i.e. }}
\newcommand{\eg}{{\it e.g. }}
\newcommand{\loccit}{{\it loc. cit. }}
\newcommand{\vac}{|0\rangle}
\newcommand{\odd}{{\rm{odd}}}
\newcommand{\even}{{\rm{even}}}
\newcommand{\tors}{{\rm{tors}}}

\renewcommand{\d}{\mathfrak{d}}
\renewcommand{\S}{\mathbb{S}}
\newcommand{\p}{\mathfrak{p}}
\newcommand{\G}{\mathfrak{G}}
\newcommand{\q}{\mathfrak{q}}
\newcommand{\pone}{ \mathfrak{p}_{ - 1} }

\newcommand{\coloneqq}{\defIs }
\newcommand{\kum}[2]{K_{ #2 }( #1 )}
\newcommand{\X}{\kum{A}{2}}
\newcommand{\cc}{c_2(\X)}

%%%%%%%%%%%%%%%%%%%%%%%%%%%%%%%%%%%%%%%%
\newcommand{\DP}{\,{:}\,}
\newcommand{\bra}{\left<\!\!\!\:\left<}
\newcommand{\ket}{\right>\!\!\!\:\right>}

\newcommand{\nor}{\mathop{:}\nolimits\!}
\newcommand{\mal}{\!\mathop{:}\nolimits}

\newcommand{\myeq}[1]{\mathrel{\overset{\makebox[0pt]{\text{\tiny #1}}}{=}}}
\newcommand{\defeq}{\sim_{\text{def}}}
\newcommand{\pol}[2]{p_{#1}^{#2}\;\!\!}
\newcommand{\defIs}{\vcentcolon=}


%%%%%%%%%%%%%%%%%%%%%%%%%%%%%%

\renewcommand{\L}{\mathcal{L}}
\newcommand{\C}{\mathbb{C}}
\renewcommand{\H}{\mathbb{H}}
\newcommand{\R}{\mathbb{R}}
\newcommand{\Q}{\mathbb{Q}}
\newcommand{\Z}{\mathbb{Z}}
\newcommand{\F}{\mathbb{F}_{2}}
\newcommand{\IC}{\mathbb{C}}
\newcommand{\IR}{\mathbb{R}}
\newcommand{\IQ}{\mathbb{Q}}
\newcommand{\IZ}{\mathbb{Z}}
\newcommand{\One}{1}
\newcommand{\Har}{\mathpzc{H}}

%%%%%%%%%%%%%%%%%%%%%%%%%%%%%

\newcommand{\kS}{\mathfrak{S}}

\newcommand{\km}{\mathfrak{m}}
\newcommand{\kq}{\mathfrak{q}}
\newcommand{\vect}[1]{\left( \begin{smallmatrix} #1 \end{smallmatrix} \right)}
\newcommand{\plan}[2]{\left< \vect{ #1 }, \vect{ #2 } \right>}

%%%%%%%%%%%%%%%%%%%%%%%%%%%%%%

\newcommand{\lra}{\longrightarrow}
\newcommand{\ra}{\rightarrow}
\newcommand{\ou}{or}

%%%%%%%%%%%%%%%%%%%%%%%%%%%%%

\theoremstyle{plain}
\newtheorem{theorem}{Theorem}[section]
\newtheorem{thm}[theorem]{Theorem}
\newtheorem{lemma}[theorem]{Lemma}
\newtheorem{lemme}[theorem]{Lemma}
\newtheorem{proposition}[theorem]{Proposition}
\newtheorem{prop}[theorem]{Proposition}
\newtheorem{corollary}[theorem]{Corollary}
\newtheorem{cor}[theorem]{Corollary}
\newtheorem{defipro}[theorem]{Definition-Proposition}
\theoremstyle{definition}
\newtheorem{definition}[theorem]{Definition}
\newtheorem{defi}[theorem]{Definition}
\newtheorem{notation}[theorem]{Notation}
\theoremstyle{remark}
\newtheorem{remark}[theorem]{Remark}
\newtheorem{rmk}[theorem]{Remark}
\newtheorem{example}[theorem]{Example}


\setlist[enumerate]{label=(\roman*)}

\begin{document}

\title{
\bf On the cohomology of irreducible holomorphically symplectic varieties
}


\author{Simon Kapfer}
%\address{Simon Kapfer, Institut f\"ur Mathematik, Universit\"at Augsburg, D--86159 Augsburg}
%\email{simon.kapfer@math.univ-poitiers.fr}
%\email{simon.kapfer@math.uni-augsburg.de}
%\urladdr{http://www.math.uni-augsburg.de/alg/}


\date{\today}
\thispagestyle{empty}
\begin{center}
\ \\
\ \\
\ \\
\setstretch{1.9}
\textbf{ \LARGE On the cohomology of irreducible holomorphically symplectic varieties}
\\
\vspace{1cm}
\Large \textbf{Dissertation} \\
\Large 
\vspace{0.5cm}
\textnormal\large zur Erlangung des akademischen Grades \\
\vspace{5mm}
Dr.~rer.~nat. \\
\vspace{1cm}
\setstretch{1.5}
eingereicht an der \\
Mathematisch--Naturwissenschaftlich--Technischen Fakult\"at\\
der Universit\"at Augsburg
\\ \vspace{1cm}
von \\
\textbf{Simon Kapfer}\\
\vspace{8mm}
am Soundsovielten Mai 2016 \\
\vspace{15mm}
\includegraphics[height=30mm]{LogoInstitut.png}
\vspace{-10mm}
\begin{align*}
\text{Betreuer: }\ 
& \text{Prof.~Dr.~Marc Nieper-Wi\ss kirchen}\\
& \text{Prof.~Dr.~Samuel Boissi\`ere}
\end{align*}
\end{center}
\pagebreak

\pagenumbering{gobble} 
\tableofcontents

\pagebreak
\pagenumbering{arabic} 
\section{Introduction}

Irreducible holomorphically symplectic (IHS) manifolds have been introduced by Beauville \cite{Beauville} as simply-connected compact K\"ahler manifolds admitting a everywhere non-degenerate holomorohic two-form, unique up to scalar. 
Alternatively, they can be described in a differential geometric setting as compact Riemannian manifolds with holonomy group isomorphic to the symplectic group $\Sp(n,\R)$. This implies the existence of a set of complex structures, parametrized by imaginary quaternions of unit norm, such that the metric is K\"ahler with respect to all of these. 
Another name, compact Hyperk\"ahler manifolds, is therefore common to emphasize this aspect of our class of manifolds. We will use the two names interchangeably.
%BERGER classification
It can be shown that any such manifold must have even complex dimension. IHS manifolds in dimension 2 are the K3 surfaces, and the concept of an IHS manifold can be seen as a generalization of them. The two main example series are given by the deformation classes of Hilbert schemes of points on K3 surfaces and Generalized Kummer varieties. Both were identified by Beauville \cite{Beauville}. Apart from that, only two further examples due to O'Grady are known up to now.

An important structure of any IHS manifold $X$ of complex dimension $2n$ is the so-called Beauville-Bogomolov form, a non-degenerate quadratic form on $H^2(X,\Z)$ that can be described with the help of the map of the symmetric power of $H^2(X,\Z)$ to the middle cohomology group via the cup product, that relates the Beauville-Bogomolov form with the form given by Poincar\'e duality. This is called the Fujiki relation. It implies that the map $\Sym^n H^2(X,\Z)\rightarrow H^{2n}(X,\Z)$ is an embedding of lattices. We shall study the algebraic properties of that situation and give an explicit formula for the discriminant of the embedded lattice. This result depends only on the Fujiki relation and therefore holds, whenever such an equation is fulfilled. In particular, it applies also to IHS varieties with singularities.

An automorphism of an IHS manifold $X$ induces a lattice automorphism on $H^2(X,\Z)$. This obviously gives some restrictions on the set of possible automorphisms.
In recent years lattice theoretic methods have been used by
Boiss\`ere, Camere, Joumaah, Menet, Mongardi, Nieper-Wi\ss kirchen, Oguiso, Sarti, Tari, Wandel and others to give results on
automorphisms of finite order. 
An important information in this setting is given by the quotient of lattices
$$
\frac{H^{2n}(X,\Z)}{\Sym^n H^2(X,\Z)}
$$
($2n$ being the dimension of $X$) and we determine this explicitly for $X$ deformation equivalent to the Hilbert scheme of $n$ points on a K3 surface, $n=2,3$ or the Generalized Kummer in dimension four. 

Moreover, for these examples, we give a complete description of $H^*(X,\Z)$. In the Generalized Kummer case, this description, based on preliminary work of Hassett and Tschinkel \cite{Hassett}, is new. 
%I want to thank Gr\'egoire Menet here for giving me the initiation to do this and for his kind support. He also showed me how to apply these results to the construction of new IHS manifolds with singularities.
The method we use is to study in detail the cohomology of Hilbert schemes of points on surfaces. This has been started by Nakajima \cite{Nakajima} and was further developped by Ellingsrud, G\"ottsche, Lehn, Sorger \cite{EGL,LehnSorger} and Li, Qin and Wang \cite{LiQinWang,LiQinWang2,QinWang}.
We wrote two computer programs implementing their results. The first one models integral cohomology of Hilbert schemes points on K3 surfaces. The second one computes rational cohomology for Hilbert schemes of points on general surfaces. The source code is in the appendix. 

Cohomology of Generalized Kummer manifolds is more subtle. Over complex coefficients, a modification of the above mentioned model was developed by Britze \cite{Britze}. This includes representation theory and prevents the methods from applying to cohomology with rational coeffients, too, although a general roadmap is contained in \cite{Twisted}.
However, in low dimensions it can be done otherwise: while in dimension two the resulting manifolds are the well-known K3 surfaces, the four dimensional case is much less studied. It turns out that the cohomology can by described by pulling back from the surrounding Hilbert scheme of three points on a torus and this description is sufficient for all degrees except $4$, where the ideas from \cite{Hassett} complete the picture.

\subsection{Overview on the results}
This work has partly been published in \cite{Kapfer2} and \cite{Kapfer}. Accordingly, the thesis is divided into three parts:

The first part \cite{Kapfer} studies the algebraic properties of the Fujiki relation. For a compact Hyperk\"ahler manifold $X$ of dimension $2n$ this allowss to equip the symmetric power $\Sym^nH^2(X)$ with a symmetric bilinear form induced by the Beauville--Bogomolov form. We develop a formula for its discriminant and compare it to the form given by the Poincar\'e pairing.
We get:
\begin{theorem}
Denote $d+1$ the rank of $H^2(X,\Z)$ and denote $c_X$ the Fujiki constant.
The discriminant of $\Sym^n\!H^2(X,\Z)$ is given by
\begin{gather*}
\left(\discr \left(H^2(X,\Z)\right)\right)^{\binom{d+n}{d+1}}\cdot c_X^{\binom{d+n}{d}} \cdot \prod_{i=1}^n i^{\binom{n-i+d}{d}d} 
\cdot C, \\
\qquad \text{with } \ 
C=
\left\{
 \begin{array}{*2{l}p{5cm}}
 \displaystyle\prod_{\substack{i=1 \\ i\ \text{odd}\\\ }}^{2n+d-1}i^{\binom{n-i+d}{d}} &\text{if }d\!+\! 1\text{ is odd}, \\
 \displaystyle\prod_{i=1}^{n+\frac{d-1}{2}} i^{\binom{n-i+d}{d} - \binom{n-2i+d}{d}} &\text{if }d\! +\! 1\text{ is even}.
\end{array}
\right.
\end{gather*}
\end{theorem}
The construction generalizes to a definition for an induced symmetric bilinear form on the symmetric power of any free module equipped with a symmetric bilinear form, yielding Theorem \ref{maintheorem}. We point out in Section \ref{polynomialSection} how the situation is related to the theory of orthogonal polynomials in several variables.
In Definition \ref{hdef} we construct a basis of homogeneous polynomials that are orthogonal when integrated over the unit sphere $\S^d$, or equivalently, over $\R^{d+1}$ with a Gaussian kernel.

In the second part we recall the theory on cohomology of Hilbert schemes of points on surfaces, using the Nakajima operator technique, working a bit on commutator relations (Section \ref{Section_Hilbert}).
We give a description of the Hilbert scheme of two points on a torus via Nakajima operators in Proposition \ref{A2Basis}, but it is clear how to derive the generalization on general surfaces.

We proceed by describing the integral cohomology of the Generalized Kummer fourfold giving an explicit basis, using Hilbert scheme cohomology and tools developed by Hassett and Tschinkel in Theorem \ref{thetaTheorem}. It turns out that Hilbert scheme cohomology is almost sufficient:
\begin{theorem}
Let $A$ be a complex abelian surface and denote $\theta: \kum{A}{2}\hookrightarrow A^{[3]}$ the embedding of the Generalized Kummer fourfold into the Hilbert scheme of three points on $A$.
The homomorphism $\theta^*:H^*(A\hilb{3},\Z)\rightarrow H^*(\kum{A}{2},\Z)$ of graded rings is surjective in every degree except $4$. Moreover, the image of $H^4(A\hilb{3},\Z)$ is the primitive overlattice of $\Sym^2(H^2(\kum{A}{2},\Z))$. 
The kernel of $\theta^*$ is the ideal generated by $H^1(A\hilb{3},\Z)$.
\end{theorem}
As an illustration of the result, I include an application due to Gr\'egoire Menet to a IHS variety with singularities, obtained by a partial resolution of the Generalized Kummer quotiented by a symplectic involution. The Beauville--Bogomolov form of this new variety is the first example of such a form that is odd.

The last part is computational. In Section \ref{CompSection} we work out some structural results for integral cohomology of Hilbert schemes on K3 surfaces, using a computer based method. As an example, we get:
\begin{theorem}
Denote by $S\hilb{3}$ the Hilbert scheme of 3 points on a projective K3 surface (or a deformation equivalent space).
The cup product mappings have the following cokernels:
\begin{gather}
\frac{H^4(S\hilb{3},\IZ)}{\Sym^2 H^2(S\hilb{3},\IZ)}  \cong \frac{\IZ}{3\IZ} \oplus \IZ^ {\oplus 23}\\
\frac{H^6(S\hilb{3},\IZ)}{H^2(S\hilb{3},\IZ)\smile H^4(S\hilb{3},\IZ)} \cong \left(\frac{\IZ}{3\IZ}\right)^{\oplus 23}
\end{gather}
\end{theorem}

The appendix dumps the source code for computing Hilbert scheme cohomology. In Appendix \ref{IntCode} an implementation for integral cohomology of K3$\hilb{n}$ is given. 
Appendix \ref{VertexCode} implements rational cohomology of $A\hilb{n}$ for general surfaces $A$. 
We use the programming language Haskell.


\subsection{Acknowledgements}
I am most grateful to my advisors Marc Nieper-Wi\ss kirchen and Samuel Boissi\`ere for their constant support
and to my collaborator Gr\'egoire Menet who suggested to me the project on Generalized Kummer varieties.
%I thank Sven Prüfer, K\'evin Tari, 
I also want to thank my office mates Christian H\"ubschmann, Cl\'ement Chesseboeuf and Caren Schinko for many diverting discussions.
I got a one year scholarship from the DAAD, which helped me a lot in progressing on my doctorate.


%\part{Preliminaries}

\chapter{Lattices}\label{latticeSubsection}
Let us first recall some elementary facts about lattices that will be needed throughout the text.
A reference for this section is Chapter 8.2.1 of \cite{Dolgachev}. 
\begin{definition}
By a \emph{lattice} $L$ we mean a free $\Z$--module of finite rank, equipped with a non-degenerate, integer-valued symmetric bilinear form, denoted by $B$ or $\left<\ ,\;\right>$. 
By a homomorphism or \emph{embedding} $L\subset M$ of lattices we mean a map $:L\rightarrow M$ that preserves the bilinear forms on $L$ and $M$ respectively. It is automatically injective. We always have the injection of a lattice $L$ into its dual space $L^*\coloneqq \Hom(L,\Z)$, given by $x \mapsto \left<x,\ \right>$. A lattice is called \emph{unimodular}, if this injection is an isomorphism, \ie if it is surjective. By tensoring with $\Q$, we can interpret $L$ as well as $L^*$ as a discrete subset of the $\Q$--vector space $L\otimes \Q$. Note that this gives a kind of lattice structure to $L^*$, too, but the symmetric bilinear form on $L^*$ may now take rational coefficients.


If $L\subset M$ is an embedding of lattices of the same rank, then the \emph{index} $|M\DP L|$ of $L$ in $M$ is defined as the order of the finite group $M/L$.
There is a chain of embeddings $L\subset M \subset M^* \subset L^*$ with $|L^*\DP M^*| =|M\DP L| $.

The quotient $L^*/L$ is called the \emph{discriminant group} and denoted $A_L$. The index of $L$ in $L^*$ is called $\discr L$, the \emph{discriminant} of $L$.
Choosing a basis $(x_i)_i$ of $L$, we may express $\discr L$ as the absolute value of the determinant of the so-called \emph{Gram matrix} $G$ of $L$, which is defined by $G_{ij}\coloneqq \left<x_i,x_j\right>$. $L$ is unimodular, if and only if $\det G =\pm 1$.

The lattice $L$ is called \emph{odd}, if there exists a $v\in L$, such that $\left<v,v\right>$ is odd, otherwise it is called \emph{even}. 
If the map $v \mapsto \left<v,v\right>$ takes both negative and positive values on $L$, the lattice is called \emph{indefinite}. 
\end{definition}
\begin{example}
Up to isomorphism there is a unique even unimodular lattice of rank two. It is called the \emph{hyperbolic lattice} $U$. Its Gram matrix is given by 
$\left(\begin{smallmatrix} 0&1\\ 1&0 \end{smallmatrix}\right)$.
\end{example}
\begin{proposition} \label{squareDiscr}Let $M$ be a lattice. Let $L\subset M$ be a sublattice of the same rank. Then $|M\DP L|$ equals $\sqrt{\frac{\discr L}{\discr M}}$.
\end{proposition}
\begin{proof}
%Since $M$ is unimodular, $|L^*\DP M|=|L^*\DP M^*| =|M\DP L| $ and therefore $|L^*\DP L| = |L^*\DP M||M\DP L|  = |M\DP L|^2$.
We have $|L^*\DP L|=|L^*\DP M^*||M^*\DP M||M\DP L|$. Hence:
$\frac{|L^*\DP L|}{|M^*\DP M|}=|L^*\DP M^*||M\DP L|=|M\DP L|^2.$
\end{proof}
An embedding $L\subset M$ is called \emph{primitive}, if the quotient $M/L$ is free. We denote by $L^\perp$ the orthogonal complement of $L$ within $M$. Since an orthogonal complement is always primitive, the double orthogonal complement $ L^{\perp\perp}$ is a primitively embedded overlattice of $L$. It is clear that $\discr( L^{\perp\perp})$ divides $\discr L$. 
\begin{proposition}\label{TorsionQuotient} Let $L\subset M$ be an embedding of lattices. Then the order of the torsion part of $M/L$ 
divides $\discr L$.
\end{proposition}
\begin{proof}
The torsion part is the index of $M/( L^{\perp\perp})$ in $M/L$. But this is equal to $|L^{\perp\perp}\DP L| = |L^* \DP (L^{\perp\perp})^*|$ and clearly divides $|L^*\DP L|$.
\end{proof}
\begin{proposition}\label{discrOrthPrim}
Let $M$ be unimodular. Let $L\subset M$ be a primitive embedding. Then $\discr L = \discr L^\perp$.
\end{proposition}
\begin{proof}
Consider the orthogonal projection $ : M\otimes\Q \rightarrow L\otimes \Q$. Its restriction to $M$ has kernel equal to $L^\perp$ and image in $L^*$. Hence we have an embedding of lattices $M/L^\perp \subset L^*$. Quotienting by $L$, we get an injective map $: M/(L\oplus L^\perp) \rightarrow L^*/L$. 
Now by Proposition~\ref{squareDiscr}, $\sqrt{\discr(L) \discr (L^\perp)} =|M \DP (L\oplus L^\perp)| \leq |L^*\DP L| = \discr L$. So we get $\discr L^\perp \leq \discr L$. Exchanging the roles of $L=L^{\perp\perp}$ and $L^\perp$ gives the inequality in the opposite direction.
\end{proof}
\begin{corollary}\label{latticeCor}
Let $L\subset M$ be an embedding of lattices with unimodular $M$. Let $n$ be the order of the torsion part of $M/L$. Then $\discr L^\perp =\discr L^{\perp\perp} = \frac{1}{n^2}\discr L$.
\end{corollary}

\begin{example}
Given a free $\Z$-module $L$ with the structure of a commutative ring and a linear form $I:V\rightarrow \Z$, the setting $\left<v,w\right>=I(vw)$ defines a bilinear form giving $L$ the structure of a lattice if it is non-degenerate. This is the case in topology:
For a compact complex manifold $X$ of dimension $d$ Poincar\'e duality induces a non-degenerate bilinear form on $H^d(X,\Z)$: 
$$
\left<\alpha,\beta\right> = \int_X\alpha\beta.
$$ 
This unimodular lattice will be referred to as the \emph{Poincar\'e lattice}.
\end{example}

\part{Symmetric Powers of Symmetric Bilinear Forms}
\documentclass{amsart}

\usepackage{amsmath,amssymb,amsfonts}
\usepackage[all]{xy}
\usepackage{appendix,listings,hyperref}

\DeclareMathOperator{\rank}{rk}
\DeclareMathOperator{\trace}{tr}
\DeclareMathOperator{\Aut}{Aut}
\DeclareMathOperator{\End}{End}
\DeclareMathOperator{\id}{id}
\DeclareMathOperator{\Hom}{Hom}
\DeclareMathOperator{\Sym}{Sym}
\DeclareMathOperator{\Hilb}{Hilb}
\DeclareMathOperator{\len}{len}

\newcommand{\hilb}[1]{^{[#1]}}
\newcommand{\ie}{{\it i.e. }}
\newcommand{\eg}{{\it e.g. }}
\newcommand{\loccit}{{\it loc. cit. }}
\newcommand{\vac}{|0\rangle}
\newcommand{\odd}{{\rm{odd}}}
\newcommand{\even}{{\rm{even}}}
\newcommand{\tors}{{\rm{tors}}}

\newcommand{\p}[2]{p_{#1}^{#2}\;\!\!}
\renewcommand{\L}{\mathcal{L}}

\newcommand{\coloneqq}{:=}
\newcommand{\bra}{\left<\!\!\!\:\left<}
\newcommand{\ket}{\right>\!\!\!\:\right>}
\newcommand{\myeq}[1]{\mathrel{\overset{\makebox[0pt]{\text{\tiny #1}}}{=}}}


%%%%%%%%%%%%%%%%%%%%%%%%%%%%%%

\newcommand{\G}{\mathbb{G}}
\newcommand{\R}{\mathbb{R}}
\newcommand{\Q}{\mathbb{Q}}
\newcommand{\Z}{\mathbb{Z}}
\renewcommand{\S}{\mathbb{S}}
\renewcommand{\H}{\mathbb{H}}

%%%%%%%%%%%%%%%%%%%%%%%%%%%%%

\newcommand{\kS}{\mathfrak{S}}

%%%%%%%%%%%%%%%%%%%%%%%%%%%%%%

\newcommand{\lra}{\longrightarrow}
\newcommand{\ra}{\rightarrow}

%%%%%%%%%%%%%%%%%%%%%%%%%%%%%

\theoremstyle{plain}
\newtheorem{theorem}{Theorem}[section]
\newtheorem{lemma}[theorem]{Lemma}
\newtheorem{proposition}[theorem]{Proposition}
\newtheorem{corollary}[theorem]{Corollary}
\theoremstyle{definition}
\newtheorem{definition}[theorem]{Definition}
\newtheorem{notation}[theorem]{Notation}
\theoremstyle{remark}
\newtheorem{remark}[theorem]{Remark}
\newtheorem{example}[theorem]{Example}


%%%%%%%%%%%%%%%%%%%%%%%%%%%%%

\begin{document}

\title[Symmetric Powers, Hom.~Orth.~Polynomials, Hyperk\"ahlers]{Symmetric Powers of Symmetric Bilinear Forms, Homogeneous Orthogonal Polynomials on the Sphere and an Application in Compact Hyperk\"ahler Manifolds}


\author{Simon Kapfer}
\address{Simon Kapfer, Laboratoire de Math\'ematiques et Applications, UMR CNRS 6086, Universit\'e de Poitiers, T\'el\'eport 2, Boulevard Marie et Pierre Curie, F-86962 Futuroscope Chasseneuil}
\email{simon.kapfer@math.univ-poitiers.fr}
%\urladdr{http://www.math.uni-augsburg.de/alg/}


\date{\today}

%\keywords{Symmetric Bilinear Forms on Symmetric Powers, Orthogonal Polynomials in Several Variables,Homogeneous Orthogonal Polynomials, Hermite Polynomials, Hankel matrices, Hyperk\"ahler Manifolds, Beauville--Bogomolov Form, Beauville--Fujiki relation}

\begin{abstract} 
We study a construction for a symmetric bilinear form on the space $\Sym^kV$, derived from a form on $V$. We point out that it is related to integrating homogeneous polynomials over a sphere and give an orthogonal basis of such polynomials. Since the construction also applies to the Beauville--Fujiki relation from Hyperk\"ahler theory, we remark some consequences for integer cohomology of compact Hyperk\"ahler manifolds.
\end{abstract}

\maketitle

%%%%%%%%%%%%%%%%%%%%%%%%%%%%%%%%%%%%%%%%%%%%%%%%%%%%%%%%%%%%%%%%%%
%%%%%%%%%%%%%%%%%%%%%%%%%%%%%%%%%%%%%%%%%%%%%%%%%%%%%%%%%%%%%%%%%%
%%%%%%%%%%%%%%%%%%%%%%%%%%%%%%%%%%%%%%%%%%%%%%%%%%%%%%%%%%%%%%%%%%

\section{Introduction}
Our motivation originated in Hyperk\"ahler theory. The Beauville--Bogomolov--Fujiki form $q$ for a compact Hyperk\"ahler manifold $X$ is a quadratic form on the integral cohomology group $H^2$ which is defined by an equation of the structure
\begin{equation}
q(x)^k = I(x^{2k}),
\end{equation}
where $x^{2k}$ means a power in the cohomology ring, and $I$ is a linear form (in fact, a scaled integral). But we may take a different point of view and consider $q(x)^k$ rather than $q(x)$, now seen as a quadratic form on the symmetric product $\Sym^kH^2$.

Every quadratic form $q$ has an associated symmetric bilinear form $\left<\ ,\;\right>$, obtained by polarization: $2\left<x,y\right> = q(x+y)-q(x)-q(y)$. We may now ask for the relation between the associated bilinear forms for $q $ and $q^k$. 



\section{Terminology and helper formulas} \label{boring}
In this section 
we give a few standard definitions and recall some facts on elementary calculus. We also mention technical formulas needed for our proofs.
\begin{definition}\label{multiindex}
For a multi-index $\alpha=(\alpha_0,\ldots,\alpha_d)$ of length $\len(\alpha)\coloneqq d+1$ we define: $x^\alpha \coloneqq x_0^{\alpha_0}\ldots x_d^{\alpha_d}$. The degree is defined by $|\alpha |\coloneqq\sum\alpha_i$, the factorial is $\alpha! \coloneqq \prod \alpha_i!$. Further, we set
$\alpha'\coloneqq(\alpha_0,\ldots,\alpha_{d-1})$. We introduce the lexicographical ordering on multi-indices: $\alpha < \beta$ iff $\alpha_d < \beta_d$ or $(\alpha_d=\beta_d) \wedge (\alpha'<\beta')$.
\end{definition}
\begin{definition}
The binomial coefficient for nonnegative integers $k$ and arbitrary $z$ is defined as:
$\binom{z}{k} \coloneqq \frac{z(z-1)\ldots(z-k+1)}{k!}$. Thus we have $\binom{-z}{k}=(-1)^k\binom{z+k-1}{k}$. For negative $k$ we set $\binom{z}{k}\coloneqq 0$.
\end{definition}
We introduce the difference operator $\Delta f(n) \coloneqq f(n+1)-f(n)$. It has the following properties similar to the differential operator:
\begin{align}
 \sum_{i=0}^n \Delta(f) & = f\,\Big|_0^{n+1}= f(n+1)-f(0) && \text{(telescoping sum)} \\
 \Delta (fg)(n) &= f(n+1) \Delta g(n) + g(n)\Delta f(n) &&\text{(product rule)} \\
\label{sumbyparts}  \sum_{i=0}^n g(i)\Delta f(i) & = (fg)\Big|_0^{n+1} - \sum_{i=0}^n f(i+1)\Delta g(i) && \text{(summation by parts)}
\end{align}
This often applies to the binomial coefficient, since we have: 
\begin{equation} \label{binomdiff}
\textstyle \Delta \binom{n}{k}=\binom{n+1}{k}-\binom{n}{k} = \binom{n}{k-1}.
\end{equation}
%The same recurrence holds for the cardinality of the set $\{|\alpha| =k\}$ of multi-indices of length $d+1$ and degree $k$, because $\{ |\alpha| = k \} = \bigcup_{j=0}^k  \{ |\alpha'| = j \} \times \{ k-j \}  $
Because $\Sym^k\! K^{d+1} \cong \Sym^k\! K^d \oplus \,\Sym^{k-1}\! K^{d+1}\!\otimes\! K$ fulfils a similar recurrence relation, we conclude that
\begin{equation} \label{binomcount} \textstyle
\binom{k+d}{d} =\binom{k+d}{k} = \rank \left(\Sym^k K^{d+1} \right) = \text{card}\big(\{|\alpha| =k\}\big).
\end{equation}

The following identity for integers $d,k,m\geq 0$ is proven by induction over $k$:
\begin{align} \label{facprod1}
\prod_{j=0}^k (k-j)!^{\binom{j+d-1}{d-1}} &= \ \prod_{i=1}^k i^{\binom{k-i+d}{d}}
% \\ \label{facprod2}
 %\prod_{j=0}^k (j+m)!^{\binom{j+d-1}{d-1}} &= (k+m)!^{\binom{k+d}{d}}, \prod_{i=m+1}^{k+m} i^{ -\binom{i-m+d-1}{d}}.
\end{align}
We will also need the identity:
\begin{equation} \label{evensum}
\sum_{\substack{i=1\\i\text{ even}}}^{2k+d+1}\textstyle \binom{k-i+d}{d-1} = \left\{ 
 \begin{array}{*2{c}p{5cm}}0 &\text{if }d\text{ is even}, \vspace{1mm}\\
 \binom{k+d}{d} &\text{if }d\text{ is odd},
\end{array}\right.
\end{equation}
which is proven by splitting the sum into:
$$
\sum_{\substack{i=1\\i\text{ even}}}^{k+1} \textstyle\binom{k-i+d}{d-1} + \displaystyle\sum_{\substack{i=k+d+1\\i\text{ even}}}^{2k+d+1}\textstyle \binom{k-i+d}{d-1}
= \displaystyle\sum_{\substack{i=1\\k-i\text{ even}}}^{k+1}\!\! \textstyle\binom{i+d-2}{d-1} +(-1)^{d-1} \!\!\!\!
 \displaystyle\sum_{\substack{i=1\\k+d+i\text{ even}}}^{k+1} \!\!\!\! \textstyle\binom{i+d-2}{d-1} .
$$

\begin{definition}\label{doublefactorial}
We define the double factorial for $ n\geq -1$ by 
$$n!! \,\coloneqq \prod_{i=0}^{\left\lfloor\!\frac{n-1}{2}\!\right\rfloor }(n-2i)=n(n-2)(n-4)\ldots $$
Clearly, $(n-1)!!\,n!! = n!$ and $(2n)!! = 2^n n!$.
\end{definition}
\begin{proposition} \label{partitioncount}
The number of partitions of the set $\{1,\ldots,2k\}$ into pairs equals $(2k-1)!! = \frac{(2k)!}{2^kk!}$.
\end{proposition}
\begin{proof}
Given such a partition, look at the pair that contains the element $1$. There are $2k-1$ possible partners for this element; removing the pair leaves a partition of a set of cardinality $(2k-2)$ into pairs. Then proceed by induction.
\end{proof}



Denote $\Gamma(t) \coloneqq \int_{0}^{\infty}r^{t-1}e^{-r}dr$ the gamma function. It satisfies:
\begin{align}
\label{ffgamma}
n!&=\Gamma(n+1),\qquad (2n-1)!!\sqrt{\pi} =2^{n}\Gamma\left(n+\tfrac{1}{2}\right), \\
\label{doublegamma}
n!&\sqrt{\pi}=2^{n}\Gamma\left(\tfrac{n}{2}+1\right)\Gamma\left(\tfrac{n+1}{2}\right) ,\\
\int_0^\infty& r^se^{-\frac{1}{2}r^2} dr = 2^{\frac{s-1}{2}}\Gamma\left(\tfrac{s+1}{2}\right).
\end{align}
It follows, that:
\begin{align}    \label{monoint}
 \int_{\R^{d+1}}x^\alpha x^\beta & e^{-\frac{1}{2}\|x\|^2} dx = \ \ 
\displaystyle \prod_{i=0}^d \int_{-\infty}^\infty x_i^{\alpha_i+\beta_i} e^{-\frac{1}{2}x_i^2} dx_i \\
 &= \left\{
\begin{array}{*2{l}p{5cm}}
(2 \pi)^{\frac{d+1}{2}}\prod_{i=0}^d (\alpha_i+\beta_i-1)!! &\text{if all }\alpha_i+\beta_i\text{ are even}, \vspace{0.2cm} \\ 
 0 &\text{otherwise}.
\end{array}
  \right.\nonumber
\end{align}
The reader may also consult \cite{Folland} for that kind of calculus. In particular, \cite[Eq.~(4)]{Folland} yields:
\begin{lemma}\label{homosphere}
Let $f:\R^{n+1}\rightarrow\R$ be a continuous homogeneous function of degree $k$, that is $f(sx) = s^kf(x)\; \forall s\in\!\R$. Then, using polar coordinates $(r,\omega) = (\|x\|,\frac{x}{\|x\|})$:
\begin{align*}
\int_{\R^{d+1}}f(x)e^{-\frac{1}{2}\|x\|^2} dx &= \int_{\S^d}\!\int_0^\infty\! f(r\omega) r^d e^{-\frac{1}{2}r^2}dr d\omega \\
&= 2^{\frac{k+d-1}{2}}\Gamma\!\left(\tfrac{k+d+1}{2}\right)\int_{\S^d}f(\omega)d\omega .
\end{align*}
\end{lemma}



\section{Symmetric Bilinear Forms on Symmetric Powers}
Let $V$ be a vector space (or a free module) over a field (resp.~a commutative ring) $K$ of rank $d+1$ with basis $\{x_0,\ldots,x_{d}\}$, equipped with a symmetric bilinear form $\left<\,\ ,\ \right>: V\times V \rightarrow K$. We will freely identify the symmetric power $\Sym^kV$ with the space $K[x_0,\ldots,x_d]_k$ of homogeneous polynomials of degree $k$. 

There are at least two possibilities to define an induced bilinear form on $\Sym^kV$. The first one is to define on monomials:
\begin{equation}\label{Garr}
\left(\!\left( x_{n_1}\ldots x_{n_k}\,,\, x_{m_1}\ldots x_{m_k}\right)\!\right) \coloneqq \sum_\sigma  
\prod_{i=1}^k \left< x_{n_i},x_{m_{\sigma(i)}}\right>,
\end{equation}
the sum being over all permutations $\sigma$ of $\{1,\ldots,k\}$, as studied by McGarraghy in \cite{McGarr}. However, we will \emph{not} consider 
this construction. Instead, we make the following
\begin{definition} \label{formdef} On the basis $\{x_{n_1}\ldots x_{n_k}\;|\;0\leq n_1\leq\ldots\leq n_k\leq d\}$ of $\Sym^kV$, we define a symmetric bilinear form $\bra\ \,,\ \ket$ by: 
\begin{equation}
\label{mydef}
\bra x_{n_1}\ldots x_{n_k}\,,\,x_{n_{k+1}}\ldots x_{n_{2k}} \ket \coloneqq \sum_{\mathcal{P}} \prod_{\{i,j\}\in\mathcal{P}} \left<x_{n_i},x_{n_j}\right>,
\end{equation}
where the sum is over all partitions $\mathcal{P}$ of $\{1,\ldots,2k\}$ into pairs.
\end{definition}
If $U\in O(V)$ is an orthogonal transformation, then the induced diagonal action of $U^{\otimes k}$ on $\Sym^kV$ is orthogonal in both cases. This shows that the two definitions are independent of the choice of the base of $V$. 
\begin{example}
To contrast the two definitions, observe that in the case $k=2$
\begin{align}
\left(\!\left( ab,cd \right)\!\right) &= \left<a,c\right>\left<b,d\right>+\left<a,d\right>\left<b,c\right>, \\
\bra ab,cd\ket &= \left<a,c\right>\left<b,d\right>+\left<a,d\right>\left<b,c\right> + \left<a,b\right>\left<c,d\right>.
\end{align}
\end{example}
\begin{remark}
Note that (\ref{Garr}) does not require symmetry of the bilinear form $\left<\,\ ,\ \right>$ on $V$. Indeed, the definition would also be valid for an arbitrary bilinear form~$: V\times W \rightarrow K$, yielding a bilinear form $:\Sym^kV\times\Sym^kW\rightarrow K$. On the other hand, if the form on $V$ is not symmetric, then (\ref{mydef}) is not well-defined.
\end{remark}
\begin{remark}
The defining equation (\ref{mydef}) works equally well, if the two arguments have different degree. So we can easily extend our definition to a symmetric bilinear form~$\bra\ \,,\ \ket:\Sym^*V\times\Sym^*V \rightarrow K$. Then we have: $\bra a,bc\ket =\bra ab,c\ket$. Note that $\Sym^kV$ is in general not orthogonal to $\Sym^lV$ unless $k-l$ is an odd number.
\end{remark}
We wish to investigate some properties of this construction. Let $G$ be the Gram matrix of $\left< \ ,\;\right>$, \ie $G_{ij} = \left<x_i,x_j\right>$ and
let $\G$ be the Gram matrix of $\bra\ ,\;\ket$. We use multi-index notation, cf.~Definition~\ref{multiindex}.
\begin{proposition} \label{intequiv}Assume $K=\R$ and $G$ is positive definite, so its inverse $G^{-1} $ exists. Then $\bra\ ,\;\ket$ takes an analytic integral form:
\begin{equation*}
\bra x^\alpha, x^\beta \ket = \frac{1}{c}\int_{\R^{d+1}} x^\alpha x^\beta d\mu(x),
\end{equation*}
where the integration measure is $d\mu(x) = \exp\left(-\frac{1}{2}\sum_{i,j} G^{-1}_{ij}x_ix_j\right)dx$ and the normalization constant is $c=\int_{\R^{n+1}} d\mu(x)=\sqrt{(2\pi)^{d+1}\det G}$.
\end{proposition}
\begin{proof} Note that we need positive definiteness of $G$ to make the integral converge. We make use of the content in Section \ref{boring}.
First, observe that both sides of the equation are invariant under orthogonal transformations of the base space $\R^{d+1}$. We may therefore assume that $G= \text{diag}\left(a_0,\ldots,a_d\right)$ is a diagonal matrix. Then the integral splits nicely:
\begin{align*}
\frac{1}{c}\int_{\R^{d+1}}& x^\alpha x^\beta d\mu(x)= \frac{1}{c}\prod_{i=0}^d \int_{-\infty}^\infty x_i^{\alpha_i+\beta_i} e^{-\frac{1}{2a_i}x_i^2}dx_i \\=&\: 
\frac{1}{c}\prod_{i=0}^d a_i^{\frac{\alpha_i+\beta_i+1}{2}}\int_{-\infty}^\infty x^{\alpha_i+\beta_i} e^{-\frac{1}{2}x^2}dx\\
\myeq{(\ref{monoint})}\  &\left\{
\begin{array}{*2{l}p{5cm}}\displaystyle \prod_{i=0}^d a_i^{\frac{\alpha_i+\beta_i}{2}}(\alpha_i+\beta_i-1)!! &\text{if all }\alpha_i+\beta_i\text{ are even}, \vspace{0.2cm} \\ 
 0 &\text{otherwise}.
\end{array}
 \right.
\end{align*}
On the other hand, if $G$ is diagonal, then every partition into pairs in Equation (\ref{mydef}) that contains a pair of two different numbers will not contribute to the sum. But the number of partitions of the multiset $\{n|\,0\!\leq\!n\!\leq\! d,\text{ multiplicity of } n = \alpha_n+\beta_n\}$ into pairs, such that every pair consists of two equal numbers, is by Prop.~\ref{partitioncount} evidently equal to $\prod_{i=0}^d (\alpha_i+\beta_i-1)!!$ if all $\alpha_i+\beta_i$ are even and $0$ otherwise. So we obtain the same formula for $\bra x^\alpha ,x^\beta\ket$.
\end{proof}

The next theorem gives a formula for the determinant of $\G$. This is of particular interest when $K=\Z$, because in this case we are in the setting of lattice theory, and $\det \G$ is an important lattice-theoretic invariant, called the discriminant.
\begin{theorem} \label{maintheorem}
The determinant of the Gram matrix $\G$ of $\bra\ ,\;\ket$ is:
\begin{equation}
\det(\G)= \det(G)^{\binom{d+k}{d+1}}\,\theta_{d,k}
\end{equation}
where $\theta_{d,k}$ is a combinatorial factor given by:
\begin{equation} \label{thetaDef}
\theta_{d,k} = \left\{
 \begin{array}{*2{l}p{5cm}}
 \displaystyle\prod_{i=1}^k i^{\binom{k-i+d}{d}d}\prod_{\substack{i=1 \\ i\ \text{odd}\\\ }}^{2k+d-1}i^{\binom{k-i+d}{d}} &\text{if }d\text{ is even}, \\
 \displaystyle\prod_{i=1}^k i^{\binom{k-i+d}{d}d}\prod_{i=1}^{k+\frac{d-1}{2}} i^{\binom{k-i+d}{d} - \binom{k-2i+d}{d}} &\text{if }d\text{ is odd}.
\end{array}
\right.
\end{equation}
\end{theorem}
\begin{remark} If $d$ or $k$ is small, this simplifies as follows:
\begin{gather*}
\theta_{d,0}=\theta_{d,1} =1,\qquad \theta_{d,2} = 2^{d}(d+3), \\
\theta_{0,k} = (2k-1)!!, \qquad \theta_{1,k} = (k!)^{k+1}.
\end{gather*}
\end{remark}
\begin{remark}
We didn't mention the base ring $K$ in the theorem. In fact, if the formula holds for $K=\Z$, then automatically for any $K$, because $\det\G$ is defined only by sums and products that don't depend on $K$. However, for the proof we work with $\R$-valued coefficients. It is clear that the formula holds for $K=\R$ if and only if it holds for $K=\Z$. We require further that $G$ is positive definite, because we want to use Prop.~\ref{intequiv}. That this means no loss of generality, is seen by the following argument: Let $Q\subset \R^{(d+1)\times(d+1)}$ be the set of real symmetric square matrices of size $d+1$. The subset $R$ of all matrices $G\in Q$ that satisfy our formula is Zariski-closed. But on the other hand, the positive definite matrices form a nonempty subset $P\subset Q$ which is open in the analytic topology. So if $P\subset R$, then necessarily $R=Q$.
\end{remark}
\begin{proof}
As one may imagine, finding the factor $\theta_{d,k}$ is the hard part. We have to postpone this to Section \ref{hsection}. We will therefore reduce the statement to the case when $G$ is the identity matrix, which is proven in Theorem \ref{thetaCor}. Since any orthogonal transformation $U\in O(V)$ induces a  transformation $U^{\otimes k} \in O(\Sym^k V)$ and thus doesn't affect determinants, we may assume that $G$ is a diagonal matrix. So let us check, what happens if we apply a coordinate transformation $x\mapsto\tilde x$ that changes the last coordinate by $\tilde{x}_d = \gamma x_d$ and leaves the other coordinates invariant. Let $\tilde{G}$ and $\tilde{\G}$ be the Gram matrices corresponding to the new coordinates. We clearly have: $\tilde{x}^\alpha = \gamma^{\alpha_d} x^\alpha$. Extracting the factor $\gamma$ from the Leibniz determinant formula, which is of the form $\det \tilde{\G}=\sum\limits_\sigma\pm\prod\limits_{|\alpha|=k} \bra \tilde{x}^{\alpha},\tilde{x}^{\sigma(\alpha)}\ket$, we get: 
\vspace{-2mm}
$$
\frac{\det \tilde{\G}}{\det\G} = \prod_{|\alpha|=k}\gamma^{2\alpha_d} = \prod_{i=0}^k\ \prod_{|\alpha '|=k-i} \gamma^{2i} \; \myeq{(\ref{binomcount})}\;\prod_{i=0}^k\gamma^{2i\binom{k-i+d-1}{d-1}} \; \myeq{(\ref{sumbyparts})} \; \gamma^{2\binom{d+k}{d+1}}.
$$
Now, if $G= \text{diag}\left(a_0,\ldots,a_d\right)$, we apply successively coordinate transformations that map $x_i$ to $\frac{x_i}{\sqrt{a_i}}$. We get a factor $(a_0\ldots a_d)^{\binom{d+k}{d+1}} = \det G^{\binom{d+k}{d+1}}$.
\end{proof}





\section{Homogeneous Orthogonal Polynomials on the sphere}
In this section we will construct a basis for the space of homogeneous polynomials of degree $k$ in $d+1$ variables, $\R[x_0,\ldots,x_d]_k$, that is orthogonal with respect to the bilinear form given by
\begin{equation*}
 \bra f,g\ket = \int_{\R^{d+1}}f(x)g(x) d\mu(x),
\end{equation*}
where the measure is $d\mu(x) = (2\pi)^{-\frac{d+1}{2}}e^{-\frac{1}{2}\|x\|^2}dx$. In order to do this, we wish to apply the Gram-Schmidt process to the (lexicographically ordered) monomial basis $(x^\alpha)_{|\alpha |=k}$. Our result is stated in Subsection \ref{hsection}. 
\begin{remark}
Although our definition of $\bra\ ,\;\ket$ doesn't mention the sphere, in view of Lemma \ref{homosphere}, we could equivalently integrate the homogeneous polynomials over the unit sphere $\S^d$:
$$
 \bra f,g\ket = c \int_{\S^{d}}f(\omega)g(\omega) d\omega, \qquad c = 2^{\frac{k}{2}-1}\pi^{-\frac{d+1}{2}}\Gamma\!\left(\tfrac{k+d+1}{2}\right).
$$ 
This is the reason why we call our polynomials orthogonal on the sphere. However, we prefer to integrate over $\R^{d+1}$, since this avoids the unwanted constant $c$.
\end{remark}
\begin{remark}
 If the homogeneity constraint was dropped, the answer to the problem would be much simpler: A basis of $ \bra\ ,\;\ket$-orthogonal polynomials is given by products $H_{\alpha_0}\!(x_0)\ldots H_{\alpha_d}\!(x_d)$ of Hermite polynomials in one variable, see also \cite[Sect.~2.3.4]{Dunkl}.
\end{remark}



\subsection{Generalities on orthogonal polynomials in one variable}
Given a nondegenerate symmetric bilinear form on the space of polynomials $K[x]$, one may ask for a basis of polynomials $(p_n)_n$ that are mutually orthogonal with respect to that form. To find such a basis, one could start with the monomial basis $(x^n)_n$ and apply some version of the Gram--Schmidt algorithm. The result will be an infinite lower triangular matrix $T$ such that $p_n =\sum_j T_{nj} x^j$. If our bilinear form now depends only on the product of its two arguments, the procedure simplifies as follows:

Let $\L$ be a linear functional such that the induced bilinear form $(f,g)=\L(fg)$ is nondegenerate when restricted to $K[x]_{\leq n}$, the space of polynomials of bounded degree, for all $n\geq 0$. Let $(p_n)_n$ be the associated sequence of monic orthogonal polynomials, \ie the leading term of $p_n(x)$ is $x^n$ and $(p_k,p_{n})=0$ for $k\neq n$. Then we have
\begin{theorem} \cite[Thm.~4.1]{Chihara} There are constants $c_n,\: d_n$ such that
\begin{equation*}
 p_0(x) = 1,\qquad  p_{n+1}(x) = (x-c_n)p_n(x) - d_np_{n-1}(x).
\end{equation*}
\end{theorem}
But also the converse is true:
\begin{theorem}[Favard's theorem] \cite[Thm.~4.4]{Chihara} \label{favard}
Let $(p_n)_n$ be a sequence of polynomials, such that $\deg p_n =n$ and the following three-term recurrence holds:
$$p_0(x) = 1,\qquad  p_{n+1}(x) = (x-c_n)p_n(x) - d_np_{n-1}(x).
$$
Then there exists a unique linear functional $\L$ such that $\L(1)=1$ and $\L(p_kp_{n})=0$ for $k\neq n$. 
\end{theorem}
\begin{theorem}\cite[Thm.~4.2]{Chihara} \label{generalLnorm}
Under the conditions of the above theorem, we have for $n\geq 1$: $$\L(p_n^2) = d_n\L(p_{n-1}^2).$$
\end{theorem}
\begin{remark} \label{finiteFarvard}Since we shall deal with finite polynomial families, we need a little modification of Favard's theorem:
If $(p_n)_{n\leq N}$ is a finite sequence that satisfies a three-term recurrence as above, then we can always extend it to an infinite sequence by choosing arbitrary constants $c_n$, $d_n$ for $n\geq N$. But for every such extension, the resulting functional $\L$ from Favard's theorem will satisfy $\L(1) =1$ and $\L(p_n)=\L(p_np_0)=0$ for $n\geq 1$. So $\L$ will always be uniquely determined on $K[x]_{\leq N}$, the space of degree-bounded polynomials.
\end{remark}






\subsection{A polynomial family}
\begin{definition} Let $n,\ m$ be integers with $0\leq 2n\leq m+1$, a condition that we always will assume silently. We define polynomials $\p{n}{m}$ of degree $n$:
\begin{equation*}
\p{n}{m}(x) \coloneqq \sum_{\substack{j=0\\ n-j\ \text{even}}}^n (-1)^{\frac{n-j}{2}} \frac{n!\,(m-2n)!!}{j!\,(m-n-j)!!\,(n-j)!!}\:x^j.
\end{equation*}
\end{definition}
\begin{lemma} \label{trigonometric}
For $n\geq 1$, we have a trigonometric differential relation:
\begin{equation*}
\frac{d}{d\omega} \Big[\p{n-1}{m-2}\big(\tan(\omega)\big)\cos(\omega)^{m-1} \Big]= (n-m)\, \p{n}{m}\big(\tan(\omega)\big)\cos(\omega)^{m-1}.
\end{equation*} 
\end{lemma}
\begin{proof} This is straightforward.
Firstly, we calculate $ \frac{d}{d\omega}\! \left[\sin(\omega)^j \cos(\omega)^{m-j-1}\right] = j \sin(\omega)^{j-1} \cos(\omega)^{m-j}-(m\!-\!j\!-\!1)\sin(\omega)^{j+1} \cos(\omega)^{m-j-2}$, and so
\begin{align*} \textstyle
\frac{d}{d\omega} &\Big[\p{n-1}{m-2}\big(\tan(\omega)\big)\cos(\omega)^{m-1} \Big] 
\\&=\!\! \textstyle \sum\limits_{\substack{j=0\\ n-j\ \text{odd}}}^{n-1} \!\!\!\frac{(-1)^{\frac{n-j-1}{2}}  (n-1)!\,(m-2n)!!}{j!\,(m-n-j-1)!!\,(n-j-1)!!}\frac{d}{d\omega}\! \left[\sin(\omega)^j \cos(\omega)^{m-j-1}\right]
\\ &=\!\!\textstyle\sum\limits_{\substack{j=0\\ n-j\ \text{even}}}^{n-2} \!\!\!\frac{(-1)^{\frac{n-j-2}{2}}(n-1)!\,(m-2n)!!}{j!\,(m-n-j-2)!!\,(n-j-2)!!}\sin(\omega)^{j} \cos(\omega)^{m-j-1} \\[-3mm]
&\hspace{3cm}-\textstyle\sum\limits_{\substack{j=1\\ n-j\ \text{even}}}^{n}\!\!\! \frac{(-1)^{\frac{n-j}{2}}(n-1)!\,(m-2n)!!\,(m-j)}{(j-1)!\,(m-n-j)!!\,(n-j)!!}\sin(\omega)^{j} \cos(\omega)^{m-j-1} \\
&=\!\!\textstyle\sum\limits_{\substack{j=0\\ n-j\ \text{even}}}^{n} \!\!\!\frac{(-1)^{\frac{n-j}{2}}(n-1)!\,(m-2n)!!}{j!\,(m-n-j)!!\,(n-j)!!}\underbrace{\big[(j\!-\!n)(m\!-\!n\!-\!j)-j(m\!-\!j)\big] }_{=n(n-m)}\tan(\omega)^{j} \cos(\omega)^{m-1}
\\ &= \textstyle(n-m)\, \p{n}{m}\big(\tan(\omega)\big)\cos(\omega)^{m-1}.
\end{align*}
\end{proof}

\begin{proposition} \label{threeterm} For $0\leq 2n\leq m-1$, we have a three-term recurrence:
\begin{equation*}
\p{0}{m}(x) = 1,\qquad \p{1}{m}(x) = x, \qquad \p{n+1}{m}(x) = x\p{n}{m}(x) -d_n^m \p{n-1}{m}(x),
\end{equation*} 
where $d_n^m\coloneqq \frac{n(m-n+1)}{(m-2n)(m-2n+2)}$.
\end{proposition}
\begin{proof}
We start from the right: $ x\p{n}{m}(x) -d_n^m \p{n-1}{m}(x)$ gives
\begin{align*}
 &\textstyle \sum\limits_{\substack{j=1\\ n-j\ \text{odd}}}^{n+1} \!\!\!\frac{(-1)^{\frac{n-j+1}{2}} n!\,(m-2n)!!}{(j-1)!\,(m-n-j+1)!!\,(n-j+1)!!}\: x^j 
\ - \sum\limits_{\substack{j=0\\ n-j\ \text{odd}}}^{n-1} \!\!\!d_n^m\frac{(-1)^{\frac{n-j-1}{2}} (n-1)!\,(m-2n+2)!!}{j!\,(m-n-j+1)!!\,(n-j-1)!!}\: x^j 
\\=&\textstyle \sum\limits_{\substack{j=0\\ n-j\ \text{odd}}}^{n+1} \!\!\!\frac{(-1)^{\frac{n-j+1}{2}} (n+1)!\,(m-2n-2)!!}{j!\,(m-n-j-1)!!\,(n-j+1)!!}\underbrace{\textstyle
\frac{j(m-2n)+(m-n+1)(n-j+1)}{(n+1)(m-n-j+1)}}_{=1} \,x^j = \p{n+1}{m}(x) .
\end{align*}
\end{proof}

\begin{theorem} \label{pthm}We define, for $m\geq 1$, a linear functional $\L$ on the vector space of polynomials of degree less than $m$, by setting
\begin{equation}
\L: f \longmapsto \int_0^\infty\!\!\! \int_{-\infty}^\infty z^{m-1}f\left(\frac{y}{z}\right) e^{-\frac{y^2+z^2}{2}} dydz.
\end{equation}
Then the $\p{n}{m}$ form a set of orthogonal polynomials with respect to the induced bilinear form, \ie for $k\neq n,\ k\!+\! n\leq m\!-\!1$ we have $\L(\p{k}{m}\p{n}{m})=0$ and for $2n\leq m\!-\!1$:
\begin{equation} \label{Lpn2}
\L(\p{n}{m}\p{n}{m}) = 2^{\frac{3}{2}m-2n-\frac{1}{2}}  \frac{n!}{(m-n)!}
\Gamma\left(\frac{m}{2}-n\right)\Gamma\left(\frac{m}{2}-n+1\right)\Gamma\left(\frac{m+1}{2}\right).
\end{equation}
\end{theorem}
\begin{proof}
Since $\p{n}{m}$ satisfy the three-term relation in Prop.~\ref{threeterm}, by Favard's theorem and Remark \ref{finiteFarvard}, there exists a unique functional $\L'$ with $\L'(1)=1$, such that the $\p{n}{m}$ form an orthogonal basis with respect to the bilinear form induced by $\L'$. We claim that $\L$ is a scalar multiple of $\L'$.
Since $(\p{n}{m})_{n} $ is a basis of the space of polynomials, we must show that, for $n\geq 1$, $\L(\p{n}{m}) = \L(\p{n}{m}\p{0}{m}) =0$.
Using polar coordinates $(y,z) = (r\cos\omega,r\sin\omega)$ and Lemma \ref{trigonometric}, we get:
\begin{align*}
\L(\p{n}{m}) &= \int_0^\pi\!\!\int_0^\infty\p{n}{m}\!\left(\tfrac{\cos\omega}{\sin\omega}\right)\sin(\omega)^{m-1} r^m e^{-\frac{r^2}{2}}dr d\omega \\
&=\int_0^\infty r^m e^{-\frac{r^2}{2}}dr \int_{\frac{\pi}{2}}^{\frac{3\pi}{2}} (-1)^n \p{n}{m}\big(\tan(\omega)\big)\cos(\omega)^{m-1}d\omega\\
&= 2^{\frac{m-1}{2}}\Gamma\left(\tfrac{m+1}{2}\right)\Big[\tfrac{(-1)^n}{n-m}p_{n-1}^{m-2}\big(\tan(\omega)\big)\cos(\omega)^{m-1} \Big]_{\frac{\pi}{2}}^{\frac{3\pi}{2}} =0,
\end{align*}
while $\L(1) = 2^{\frac{m-1}{2}}\sqrt{\pi}\,\Gamma\!\left(\frac{m}{2}\right)=2^{\frac{3m-1}{2}}  \frac{1}{m!}
\Gamma\left(\frac{m}{2}\right)\Gamma\left(\frac{m}{2}+1\right)\Gamma\left(\frac{m+1}{2}\right)$ by (\ref{monoint}) and (\ref{doublegamma}). To verify that equation (\ref{Lpn2}) holds for $n\geq 1$, too, we must show that the right hand side satisfies the recurrence from Theorem~\ref{generalLnorm}, but this is immediate:
$$ 
\frac{2^{\frac{3}{2}m\!-\!2n\!-\!\frac{1}{2}}  \frac{n!}{(m\!-\!n)!}
\Gamma\!\left(\frac{m}{2}\!-\!n\right)\Gamma\!\left(\frac{m}{2}\!-\!n\!+\!1\right)\Gamma\!\left(\frac{m+1}{2}\right)}{
2^{\frac{3}{2}m\!-\!2n\!+\!\frac{3}{2}}  \frac{(n\!-\!1)!}{(m\!-\!n\!+\!1)!}
\Gamma\!\left(\frac{m}{2}\!-\!n\!+\!1\right)\Gamma\!\left(\frac{m}{2}\!-\!n\!+\!2\right)\Gamma\!\left(\frac{m+1}{2}\right)} = \frac{n(m\!-\!n\!+\!1)}{(m\!-\!2n)(m\!-\!2n\!+\!2)} =d_n^m.
$$
\end{proof}
\begin{corollary} \label{pcor}
 $\L(x^{k}\p{n}{m}) = 0$ for $k < n$ and $\L(x^n\p{n}{m})= \L(\p{n}{m}\p{n}{m})$.
\end{corollary}
\begin{proof} By the theorem, we have $\L(x^{0}\p{n}{m}) = 0$ for $n>0$, so the case $k=0$ holds true.
Now the three-term recurrence from Proposition \ref{threeterm} allows us to inductively conclude that $\L(x^k\p{n}{m}) = \L(x^{k-1}\p{n+1}{m})+d_n^m\L(x^{k-1}\p{n-1}{m}) =0$.
The second assertion, $\L(x^n\p{n}{m})= \L(\p{n}{m}\p{n}{m})$ is trivial in the case $n\leq 1$. For $n\geq 1$, the three-term recurrence yields now $\L(x^n\p{n}{m}) = \L(x^{n-1}\p{n+1}{m})+d_n^m\L(x^{n-1}\p{n-1}{m})=d_n^m\L(x^{n-1}\p{n-1}{m})$, so $\L(x^n\p{n}{m})$ and $\L(\p{n}{m}\p{n}{m})$ (by Theorem \ref{generalLnorm}) satisfy the same recurrence relation and therefore must be equal.
\end{proof}



\subsection{Homogeneous orthogonal polynomials} \label{hsection}
We are now ready to give the desired basis of homogeneous polynomials that are orthogonal on the sphere. Inspired by the procedure for spherical harmonics, see \cite[p.~35]{Dunkl}, where homogeneous polynomials were defined by recursion over $d$, we make the following
\begin{definition} \label{hdef}
For multi-indices $\alpha=(\alpha_0,\ldots,\alpha_d)$ we recursively define homogeneous polynomials $h_\alpha$ of degree $|\alpha |$ by $h_{(\alpha_0)}(x) \coloneqq x_0^{\alpha_0}$ and, for $d\geq 1$,
$$
h_\alpha(x) \coloneqq \p{\alpha_d}{2|\alpha |+d}\left(\frac{x_d}{r}\right) r^{\alpha_d}h_{\alpha'}(x'),
$$
where we have set $r=\sqrt{x_0^2 +\ldots+x_{d-1}^2}=\|x'\|$. Note that the definition of $\p{n}{m}$ implies that $\p{n}{m}(\frac{1}{y})y^n$ is an even polynomial, so all square roots vanish.
\end{definition}
\begin{theorem}
For all multi-indices $\alpha,\;\beta$ of length $d+1$ and degree $k$ we have:
\begin{align} \label{hth1}
\bra h_\alpha ,h_\alpha \ket &= \alpha_d!\,\frac{\left(2|\alpha '|\!+\!d\right)!!\,\left(2|\alpha |\!+\!d\!-\!1\right)!!}{\left(|\alpha '|+|\alpha |+d\right)!}\bra h_{\alpha '},h_{\alpha '}\ket ,\\
\label{hth2}
\bra h_\alpha,h_\beta \ket &= 0 \quad\text{for }\alpha\neq\beta,\\
\label{hth3}
\bra x^\alpha, h_\alpha \ket &= \bra h_\alpha ,h_\alpha \ket ,\\
\label{hth4}
\bra x^\alpha,h_\beta \ket &= 0 \quad\text{for }\alpha < \beta.
\end{align}
\end{theorem}
\begin{remark}
This means that the $h_\alpha(x),\ |\alpha|=k$ form an orthogonal basis of $\R[x_0,\ldots,x_d]_k$ that comes from a Gram--Schmidt process applied to the monomials (in lexicographic order). Indeed, equations (\ref{hth1}) and (\ref{hth2}) say that the $h_\alpha(x)$ are orthogonal, while equations (\ref{hth3}) and (\ref{hth4}) imply that the transition matrix $T^{-1}$, defined by $ x^\alpha = \sum_\beta T^{-1}_{\alpha\beta}h_\beta,\ T^{-1}_{\alpha\beta} \coloneqq \frac{\bra x^\alpha,h_\beta\ket}{\bra h_\beta,h_\beta\ket} $ is lower triangular with all diagonal elements equal to $1$.
\end{remark}
\begin{proof}
For equation (\ref{hth1}), we use polar coordinates on $\R ^d$ to compute $\bra h_\alpha ,h_\alpha \ket$:
\begin{align*}
 &\int_{\R^{d+1}}\left[\p{\alpha_d}{2|\alpha|+d}\left(\frac{x_d}{r}\right) r^{\alpha_d}h_{\alpha'}(x)\right]^2 e^{-\frac{1}{2}\|x\|^2} dx \\
 = & \int_0^\infty\!\!\int_{\R} \left[\p{\alpha_d}{2|\alpha|+d}\left(\frac{x_d}{r}\right)\right]^2 r^{2|\alpha'|+2\alpha_d+d-1} e^{-\frac{r^2+x_d^2}{2} }dx_ddr\int_{\S^{d-1}}\left[h_{\alpha'}(\omega)\right]^2 d\omega \\
\myeq{(\ref{Lpn2})}& \  \frac{\alpha_d!\, 2^{2|\alpha '|+|\alpha|+\frac{3}{2}d-\frac{1}{2}} }{\left(|\alpha|+|\alpha'|+d\right)!} { \textstyle
\Gamma\!\left( |\alpha'| \!+\!\frac{d}{2}\right)\Gamma\!\left( |\alpha'|\!+\! \frac{d}{2}\!+\! 1\right) \Gamma\!\left( |\alpha| \!+\! \frac{d+1}{2}\right) }\int_{\S^{d-1}}\!\!\left[h_{\alpha'}(\omega)\right]^2 d\omega \\
\myeq{Lemma~\ref{homosphere}}&\quad\quad \alpha_d!\,2^{|\alpha '|+|\alpha |+d+\frac{1}{2}}\,
\frac{\Gamma\!\left(|\alpha '|\!+\!\frac{d}{2}\!+\!1\right)\Gamma\!\left(|\alpha |\!+\!\tfrac{d+1}{2}\right)}{\left(|\alpha '|+|\alpha |+d\right)!} \int_{\R^d} \left[h_{\alpha'}(x')\right]^2 dx' \\
\myeq{(\ref{ffgamma})}& \quad\; 
\alpha_d!\,\frac{\left(2|\alpha '|\!+\!d\right)!!\,\left(2|\alpha |\!+\!d\!-\!1\right)!!}{\left(|\alpha '|+|\alpha |+d\right)!}\, \sqrt{2\pi}\int_{\R^d} \left[h_{\alpha'}(x')\right]^2 dx' .
\end{align*}
For the proof of (\ref{hth2}), we may assume that $\alpha_d \neq \beta_d$. Then we use the calculation above to see that Thm. \ref{pthm} now implies the vanishing of the integral. Equations (\ref{hth3}) and (\ref{hth4}) follow from Corollary \ref{pcor} in the same way.
\end{proof}
\begin{theorem}\label{thetaCor} Let $D(d,k):=\det\limits_{|\alpha|,|\beta|=k}\bra x^\alpha ,x^\beta \ket $ be the determinant of the Gram matrix of $\bra\ ,\;\ket$. Then:
$$
D(d,k) = \theta_{d,k}
$$
where $\theta_{d,k}$ is defined as in Equation (\ref{thetaDef}).
\end{theorem}
\begin{proof} We do a double induction over $k$ and $d$. First check that $D(d,0) = \theta_{d,0}=1$ and $D(0,k) = \theta_{0,k} =(2k-1)!!$.
From the above theorem it is clear that $D(d,k)=\prod_{|\alpha | =k}\bra h_\alpha ,h_\alpha \ket$. Since $\{ |\alpha| = k \} = \bigcup_{j=0}^k  \{ |\alpha'| = j \} \times \{ k-j \}  $, we have from Equation (\ref{hth1}):
$$
D(d,k) = \prod_{j=0}^k D(d\!-\!1,j) \prod_{|\alpha'| = j} (k-j)!\,\frac{(2j+d)!!\,(2k+d-1)!!}{(j+k+d)!},
$$
hence 
\begin{align*}
R(d,k) \;&\coloneqq\; \frac{D(d,k)}{\prod_{j=0}^k D(d\!-\!1,j) } \;=\; \prod_{j=0}^k \left[\frac{(2j+d)!!\,(2k+d-1)!!}{(j+k+d)!} (k-j)!\right]^{\binom{j+d-1}{d-1}} \\
&\qquad\myeq{(\ref{facprod1})}\; \prod_{j=0}^k\left[ \frac{(2j+d)!!}{(j\!+\!k\!+\!d)!!}\right]^{\binom{j+d-1}{d-1}}(2k\!+\!d\!-\!1)^{\binom{k+d}{d}} \;\prod_{i=1}^k i^{\binom{k-i+d}{d}} .
\end{align*}
% R(d,k)=\frac{\displaystyle\prod_{i=k+d+1}^{2k+d} i^{\binom{i-k-1}{d} }}{\displaystyle \prod_{\substack{i=d+1 \\ i+d\text{ even}}}^{2k+d} i^{\binom{\frac{i+d}{2}-1}{d}
We will now show the principal inductive step: $\frac{D(d,k+1)}{D(d,k)D(d-1,k+1)}=\frac{\theta_{d,k+1}}{\theta_{d,k}\theta_{d-1,k+1}}$. The left hand side clearly equals
\begin{align*}
 \frac{R(d,k+1)}{R(d,k)} &= \frac{(2k\!+\!d\!+\!2)!!^{\binom{k+d}{d-1}}\,(2k\!+\!d\!+\!1)^{\binom{k+d+1}{d}}\,(2k\!+\!d\!+\!1)!!^{\binom{k+d}{d-1}}}{(2k\!+\!d\!+\!2)!^{\binom{k+d}{d-1}} \prod\limits_{j=0}^k(j\!+\!k\!+\!d\!+\!1)^{\binom{j+d-1}{d-1}}} \prod_{i=1}^{k+1} i^{\binom{k-i+d}{d-1}}\\
&= \frac{(2k+d+1)^{\binom{k+d}{d}}}{\prod\limits_{i=k+d+1}^{2k+d+1}i^{ \binom{k-i+d}{d-1}} }\prod_{i=1}^{k+1} i^{\binom{k-i+d}{d-1}}.
\end{align*}
If we split $\theta_{d,k} = A(d,k)B(d,k)$ with $A(d,k)\coloneqq\prod_{i=1}^k i^{\binom{k-i+d}{d}d}$, we see that
\begin{align*}
\frac{A(d,k+1)}{A(d,k)A(d-1,k+1)} &=
 \prod_{i=1}^{k+1}i^{\binom{k-i+d+1}{d}d-\binom{k-i+d}{d}d -\binom{k-i+d}{d-1}(d-1)} \ 
 \myeq{(\ref{binomdiff})}\ \prod_{i=1}^{k+1} i^{\binom{k-i+d}{d-1}},
\end{align*}
while the other factor $B(d,k)$ gives, for even $d$,
\begin{gather*}
\frac{B(d,k+1)}{B(d,k)B(d-1,k+1)} = \frac{ (2k\!+\!d\!+\!1)^{\binom{-k-1}{d}} \prod\limits_{\substack{i=1 \\ i\ \text{odd} }}^{2k+d+1}i^{\binom{k-i+d+1}{d}-\binom{k-i+d}{d}}  }{\prod\limits_{i=1}^{k+\frac{d}{2}} i^{\binom{k-i+d}{d-1}} \prod\limits_{\substack{i=1 \\ i\ \text{even} }}^{2k+d} \left(\frac{i}{2}\right)^{- \binom{k-i+d}{d-1}} } \\
\myeq{\ref{binomdiff}}\ \frac{ (2k\!+\!d\!+\!1)^{\binom{k+d}{d}}  \prod\limits_{i=1}^{2k+d+1}i^{\binom{k-i+d}{d-1}}  }{ \prod\limits_{i=1}^{k+\frac{d}{2}} i^{\binom{k-i+d}{d-1}} \prod\limits_{\substack{i=1 \\ i\ \text{even} }}^{2k+d} 2^{ \binom{k-i+d}{d-1}} } 
\ \myeq{(\ref{evensum})} \ \frac{ (2k\!+\!d\!+\!1)^{\binom{k+d}{d}} }{ \prod\limits_{i=k+d+1}^{2k+d+1}i^{ \binom{k-i+d}{d-1}} },
\end{gather*}
but also for odd $d$,
\begin{gather*}
\frac{B(d,k+1)}{B(d,k)B(d-1,k+1)} = \frac{ (k+\tfrac{d+1}{2})^{-\binom{-k-1}{d}} \prod\limits_{i=1}^{k+\frac{d+1}{2}} i^{\binom{k-i+d}{d-1}-\binom{k-2i+d}{d-1}}  }{ \prod\limits_{\substack{i=1 \\ i\ \text{odd} }}^{2k+d+1}i^{\binom{k-i+d}{d-1}} } \\
= \frac{ (k+\tfrac{d+1}{2})^{\binom{k+d}{d}} \prod\limits_{i=1}^{k+\frac{d+1}{2}} i^{\binom{k-i+d}{d-1}}  \prod\limits_{\substack{i=1 \\ i\ \text{even} }}^{2k+d} 2^{ \binom{k-i+d}{d-1}}}{ \prod\limits_{i=1}^{2k+d+1}i^{\binom{k-i+d}{d-1}}} 
\ \myeq{(\ref{evensum})} \ \frac{ (2k\!+\!d\!+\!1)^{\binom{k+d}{d}} }{ \prod\limits_{i=k+d+1}^{2k+d+1}i^{ \binom{k-i+d}{d-1}} }.
\end{gather*}
\end{proof}



\section{Application in Hyperk\"ahler geometry}
Let $X$ be a compact Hyperk\"ahler manifold of complex dimension $2k$. These objects are also called Irreducible Holomorphic Symplectic manifolds. The second cohomology group $H^2(X,\Z)$ comes with an integral quadratic form, called the Beauville--Bogomolov form $q_X$, which can be computed by an integration over some cup--product power, see \cite[Subsection~2.3]{OGrady}:
\begin{equation} \label{fujiki}
\int_X \alpha ^{2k} = (2k-1)!!\,c_X q_X(\alpha)^k,\qquad \alpha\in H^2(X,\Z).
\end{equation}
This equation is referred to as the Beauville--Fujiki relation. The constant $c_X\in\Q$ is chosen such that the quadratic form $q_X$ is indivisible and its signum is such that $q_X(\sigma + \bar{\sigma}) > 0$ for a holomorphic two-form $\sigma$ with $\int_X\sigma\bar{\sigma} = 1$. There is an alternative description, as shown in \cite[Chap.~23]{Huybrechts}. Up to a scalar factor $\tilde{c}$, $q_X$ is equal to:
\begin{equation}\label{bb}
 \tilde{c}\,q_X(\alpha) = \frac{k}{2}\int_X \alpha^2 (\sigma\bar{\sigma})^{k-1} + (1-k)\left(\int_X\alpha\,\sigma^{n-1}\bar{\sigma}^{n}\right)\left(\int_X\alpha\,\sigma^{n}\bar{\sigma}^{n-1}\right).
\end{equation}
Now $q_X$, by polarisation, gives rise to a symmetric bilinear form $\left<\ ,\;\right>$ on $H^2(X,\Z)$, namely $2\left<\alpha,\beta\right> \coloneqq q_X(\alpha+\beta)-q_X(\alpha) -q_X(\beta)$. On the other hand, from (\ref{fujiki}) one deduces (again by polarisation, cf.~\cite[Eq.~3.2.4]{OGrady}) that:
\begin{equation}
 \int_X \alpha_1\wedge\ldots\wedge\alpha_{2k} = c_X \bra \alpha_1\ldots\alpha_k\,,\,\alpha_{k+1}\ldots\alpha_{2k}\ket,
\end{equation}
with the induced form $\bra\ ,\;\ket$ on $\Sym^kH^2(X,\Z)$, according to Definition \ref{formdef}. Since the Poincar\'e pairing $(\beta_1,\beta_2)_X \coloneqq \int_X\beta_1\wedge\beta_2$ gives $H^{2k}(X,\Z)$ the structure of an unimodular lattice, we have got an imbedding of lattices:
\begin{equation}
\Big( \Sym^kH^2(X,\Z),\: c_X\!\bra\ ,\;\ket\Big) \longrightarrow \Big(H^{2k}(X,\Z),\;(\ ,\;)_X\Big).
\end{equation}
From this observation and Theorem \ref{maintheorem}, we deduce some interesting corollaries:
\begin{corollary}
Let $X$ be a compact Hyperk\"ahler manifold of complex dimension $2k$. Denote $h_2$ resp.~$d_2$ the rank and the discriminant of $H^2(X,\Z)$. Then the torsion part of the quotient
$$
\frac{H^{2k}(X,\Z)}{\Sym^kH^2(X,\Z)}
$$
contains no prime factors that are bigger than $2k+h_2-2$ and don't divide neither $c_X$ nor $d_2$. \qed
\end{corollary}
For the known examples of compact Hyperk\"ahler manifolds, we can refine this a bit, using \cite[Table~1]{OGrady}:
\begin{corollary}
 The torsion part of the quotient
$$
\frac{H^{2k}(X,\Z)}{\Sym^kH^2(X,\Z)}
$$
contains no prime factors bigger than
\begin{itemize}
\item $2k+21$, if $X$ is $S^{[k]}$, the Hilbert scheme of $k$ points on a K3 surface $S$,
\item $2k+5$, if $X$ is $K^{[[k]]}$, the generalized Kummer variety of a torus,
\item $16$, if $X$ is the 10--dimensional O'Grady manifold,
\item $6$, if $X$ is the 6--dimensional O'Grady manifold.
\end{itemize} \qed
\end{corollary}
\begin{remark}
The cases $X=S^{[2]}$ and $X=S^{[3]}$ were already studied in \cite[Prop.~6.6]{BNS} and \cite[Prop.~2.4]{Kapfer}, using explicit calculations. The case $X=S^{[2]}$ is particularly nice, because $\Sym^2H^2(S^{[2]},\Z)$ and $H^4(S^{[2]},\Z)$ have the same rank. Since the rank and the discriminant of $H^2(S^{[2]},\Z)$ are $23$ and $-2$, Theorem \ref{maintheorem} implies that the cardinality of the quotient is precisely $\sqrt{2^{24}\,2^{22}(22+3)}$.
\end{remark}

\emph{Acknowledgements.} We thank Samuel Boissi\`ere, Cl\'ement Chesseboeuf and K\'evin Tari for useful conversations and the University of Poitiers for its hospitality. The author was supported by a DAAD grant.

\bibliographystyle{amsplain}
\bibliography{SymBil.tex}


\end{document}

\newpage
\part{Hilbert schemes and Generalized Kummer varieties}
\section{Super algebras} \label{SuperSection}
Let us recall some material on super algebras, which will be useful in Sections \ref{Section_Hilbert} and \ref{Section_GeneralKummer} to understand the cohomology structure of the Hilbert schemes of points on surfaces.
\begin{definition}
 A super vector space $V$ over a field $k$ is a vector space with a $\Z/2\Z$-graduation, that is a decomposition
$$
 V = V^{+} \oplus V^{-},
$$
called the even and the odd part of $V$. Elements of $V^{+}$ are called homogeneous of even degree, elements of $V^{-}$ are called homogeneous of odd degree.
The degree of a homogeneous element $v$ is denoted by $|v|\in \Z/2\Z$.
Direct sum and tensor product of two super vector spaces $V$ and $W$ yield again super vector spaces:
\begin{align*}
 (V\oplus W)^{+} &= V^{+}\oplus W^{+}, & (V\oplus W)^{-} &= V^{-}\oplus W^{-}, \\
 (V\otimes W)^{+} &= (V^{+}\otimes W^{+})\oplus (V^{-}\otimes W^{-}), & (V\otimes W)^{-} &=(V^{+}\otimes W^{-})\oplus (V^{-}\otimes W^{+}).
\end{align*}
\end{definition}
\begin{definition}
A superalgebra $R$ is a unital associative $k$-algebra which carries a super vector space structure. Define the supercommutator by setting for homogeneous elements $u,v \in R$:
\begin{align*}
[u,v] \defIs  uv - (-1)^{|u||v|} v u.
\end{align*}
$R$ is called supercommutative, if $[u,v]=0$ for all $u,v\in R$. Note that a graded commutative algebra $R = \bigoplus\limits_n R^n$ is supercommutative in a natural way, by setting $R^{+}=\bigoplus\limits_{n\text{ even}} R^n$, $R^{-}=\bigoplus\limits_{n\text{ odd}} R^n$.

For a supercommutative algebra $R$, the tensor power $R^{\otimes n}$ is again a supercommutative algebra, if we set for the product:
$$
(u_1\otimes\cdots\otimes u_n)\cdot(v_1\otimes\cdots\otimes v_n) 
=  (-1)^{\sum\limits_{i>j}|u_i||v_j|} u_1v_1\otimes\cdots\otimes u_nv_n.
%,\quad \text{where } k= {\sum_{j<i}|u_i||v_j|}.
$$
\end{definition}
\begin{definition}
Let $V$ be a super vector space over $k$ and $n\geq 0$. Then the supersymmetric power $\SSym^n(V)$ of $V$ is a super vector space, given by
\begin{align*}
\SSym^n(V) &= \bigoplus_{p+q=n} \Sym^p(V^{+}) \otimes \Lambda^q(V^{-}), \\
\SSym^n(V)^{+} &= \bigoplus_{\substack{p+q=n \\ q\text{ even} }} \Sym^p(V^{+}) \otimes \Lambda^q(V^{-}), &
\SSym^n(V)^{-} &= \bigoplus_{\substack{p+q=n \\ q\text{ odd} }} \Sym^p(V^{+}) \otimes \Lambda^q(V^{-}).
\end{align*}
The supersymmetric algebra $\SSym^*(V)\defIs  \bigoplus\limits_n \SSym^n(V)$ on $V$ is a supercommutative algebra over $k$, where the product of two elements $s\otimes e \in \Sym^p(V^{+}) \otimes \Lambda^q(V^{-})$ and $s'\otimes e' \in \Sym^{p'}(V^{+}) \otimes \Lambda^{q'}(V^{-})$ is given by 
$$
(s\otimes e)\diamond (s'\otimes e') = (s s')\otimes (e\wedge e') \ \  \in \Sym^{p+p'}(V^{+}) \otimes \Lambda^{q+q'}(V^{-}).
$$
\end{definition}
\begin{remark}
The supersymmetric power $\SSym^n (V)$ can be realized as a quotient of $V^{\otimes n}$ by an action of the symmetric group $\mathfrak S_n$. This action can be described as follows: If $\tau\in \mathfrak S_n$ is a transposition that exchanges two numbers $i<j$, then $\tau$ permutes the corresponding tensor factors in $v_1\otimes  \cdots\otimes v_n$ introducing a sign
$(-1)^{|v_i||v_j|+(|v_i|+|v_j|)\sum_{i<k<j} |v_k|}$.
\end{remark}


Now let $U$ be a vector space over a field $k$ of characteristic $0$ and look at the exterior algebra $H\defIs  \Lambda^* U$. 
Since $H$ is a super vector space, we can construct the supersymmetric power $S^n\defIs  \SSym^n( H)$.
We may identify $S^n$ with the space of $\mathfrak S_n$-invariants in $H^{\otimes n}$ by means of the linear projection operator
$$
\pr : H^{\otimes n} \longrightarrow S^n , \quad \pr = \frac{1}{n!}\sum_{\pi \in\mathfrak S_n} \pi.
$$
The multiplication in $H^{\otimes n}$ induces a multiplication on the subspace of invariants, which makes $S^n$ a supercommutative algebra. Of course, it is different from the product $\diamond$ discussed above.

Since $H$ is generated as an algebra by $U=\Lambda^1(U)\subset H$, we may define a homomorphism of algebras:
$$ s : H \longrightarrow S^n, \quad s(u) = \pr( u \otimes 1\otimes\cdots\otimes 1)\text{ for }u\in U, $$
so $S^n$ becomes an algebra over $H$.
\begin{lemma}
\label{SuperFree}
The morphism $s$ turns $S^n$ into a free module over $H$, for $n\geq 1$.
\end{lemma}
\begin{proof}
We start with the tensor power $H^{\otimes n}$ and the ring homomorphism 
$$
\iota : H \longrightarrow H^{\otimes n},\quad h\longmapsto h\otimes 1\otimes\cdots\otimes 1
$$
that makes $H^{\otimes n}$ a free $H$-module. Note that $\pr \iota \neq s$, since $\pr$ is not a ring homomorphism.
(For example, $\pr\iota(h)\neq s(h)$ for any nonzero $h\in\Lambda^2(U)$.)
We therefore modify the $H$-module structure of $H^{\otimes n}$:

For some $u\in U$, denote $u^{(i)} \defIs  1^{\otimes i-1}\otimes u\otimes 1^{\otimes n-i+1} \in H^{\otimes n}$. Then $H^{\otimes n}$ is generated as a $k$-algebra by the elements $\{u^{(i)}\,,\,u\in U\}$. Now consider the ring automorphism
$$
\sigma : H^{\otimes n} \longrightarrow H^{\otimes n}, \quad u^{(1)} \longmapsto u^{(1)} +u^{(2)} + \ldots + u^{(n)}, \quad
u^{(i)} \longmapsto u^{(i)} \text{ for } i>1.
$$
Then we have $\sigma\iota = s$ on $S^n$. On the other hand, if $\{b_i\}$ is a $k$-basis of $V$, then $\{b_i^{(j)},\,j>1\} $ is a $\iota$-basis for $H^{\otimes n}$, and $\{\sigma(b_i^{(j)})\}$ is a $\sigma\iota$-basis for $H^{\otimes n}$.
Now if we project the basis elements, we get a set $\{\pr(\sigma(b_i^{(j)}))\}$ that spans $S^n$. Eliminating linear dependent vectors (this is possible over the rationals), we get a $s$-basis of $S^n$.
\end{proof}


\section{Actions of the symplectic group over finite fields}\label{Section_Symplectic}
The aim of this section is to provide some special computations used in Section \ref{Middle}.

Let $V$ be a symplectic vector space of dimension $n\in 2\mathbb{N}$ over a field $k$ with a nondegenerate symplectic form $\omega : \Lambda^2 V \rightarrow k$. A line is a one-dimensional subspace of $V$ through the origin, a plane is a two-dimensional subspace of $V$. A plane $P\subset V$ is called isotropic, if $\omega (x,y)=0$ for any $x,y\in P$, otherwise non-isotropic.  The symplectic group $\Sp V$ is the set of all linear maps $\phi : V\rightarrow V$ with the property $\omega(\phi(x),\phi(y)) = \omega(x,y)$ for all $ x,y\in V$.
\begin{proposition}\label{transitively}
The symplectic group $\Sp V$ acts transitively on the set of non-isotropic planes as well as on the set of isotropic planes.
\end{proposition}
\begin{proof}
Given two planes $P_1$ and $P_2$, we may choose vectors $v_1,v_2,w_1,w_2$ such that $v_1,v_2$ span $P_1$, $w_1,w_2$ span $P_2$ and $\omega(u_1,u_2) =\omega(w_1,w_2)$. We complete $\{v_1,v_2\}$ as well as $\{w_1,w_2\}$ to a symplectic basis of $V$.
Then define $\phi(v_1)=w_1$ and $\phi(v_2)=w_2$. 
It is now easy to see that the definition of $\phi$ can be extended to the remaining basis elements to give a symplectic morphism.
\end{proof}
\begin{remark} \label{simplePlanes}
The set of planes in $V$ can be identified with the simple tensors in $\Lambda^2V$ up to multiples. Indeed, given a simple tensor $v\wedge w \in \Lambda^2 V$, the span of $v$ and $w$ yields the corresponding plane. Conversely, any two spanning vectors $v$ and $w$ of a plane give the same element $v\wedge w$ (up to multiples).
\end{remark}
%\begin{proposition}
%If $\phi\in\Sp V$ acts through multiplication of a scalar, $\phi(v) = \alpha v$, then $\alpha = \pm 1$ (this is immediate from the definition). Moreover, if $\phi(v)\wedge \phi(w) = \alpha v\wedge w$, then $\alpha=1$.
%\end{proposition}
%\begin{proof}
%We may assume that $V$ is two-dimensional, generated by $v$ and $w$. Our condition on $\phi$ reads then $\det\phi = \alpha$. But the condition on %$\phi$ being symplectic is $\det\phi = 1$, because on a two-dimensional vector space there is only one symplectic form up to scalar multiple. 
%\end{proof}
\begin{remark} \label{PlaneTriple}
 If $k$ is the field with two elements, then the set of planes in $V$ can be identified with the set $\{\{x,y,z\}\;|\;x,y,z\in V\backslash\{0\},\,x+y+z=0\}$. Observe that for such a $\{x,y,z\}$, $\omega(x,y)=\omega(x,y)=\omega(y,x)$ and this value gives the criterion for isotropy.
\end{remark}
From now on, we assume that $k$ is finite of cardinality $q$.
\begin{proposition}\label{OrbitesSp}
\begin{align}
&\text{The number of lines in $V$ is }\frac{q^n-1}{q-1}, \\
&\text{the number of planes in $V$ is }\frac{(q^n-1)(q^{n-1}-1)}{(q^2-1)(q-1)}, \\
&\text{the number of isotropic planes in $V$ is }\frac{(q^n-1)(q^{n-2}-1)}{(q^2-1)(q-1)}, \\
&\text{the number of non-isotropic planes in $V$ is }\frac{q^{n-2}(q^n-1)}{q^2-1}.
\end{align}
\end{proposition}
\begin{proof}
A line $\ell$ in $V$ is determined by a nonzero vector. There are $q^n - 1$ nonzero vectors in $V$ and $q-1$ nonzero vectors in $\ell$. A plane $P$ is determined by a line $\ell_1 \subset V$ and a unique second line $\ell_2\in V/\ell_1$. We have $\frac{q^2-1}{q-1}$ choices for $\ell_1$ in $P$. The number of planes is therefore
$$
\frac{ \frac{q^n-1}{q-1} \cdot\frac{q^{n-1}-1}{q-1}}{\frac{q^2-1}{q-1} } = \frac{(q^n-1)(q^{n-1}-1)}{(q^2-1)(q-1)}.
$$
For an isotropic plane we have to choose the second line from $\ell_1^\perp/\ell_1$. This is a space of dimension $n-2$, hence the formula. The number of non-isotropic planes is the difference of the two previous numbers.
\end{proof}

We want to study the free $k$-module $k[V]$ with basis $\{X_i \,|\, i\in V\}$. It carries a natural $k$-algebra structure, given by
$X_i\cdot X_j \defIs  X_{i+j}$ with unit $1=X_0$. This algebra is local with maximal ideal $\mathfrak m$ generated by all elements of the form $(X_i-1)$.

We introduce an action of $\Sp (4,k)$ on $k[V]$ by setting $\phi(X_i) = X_{\phi(i)}$. Furthermore, the underlying additive group of $V$ acts on $k[V]$ by $v( X_i) = X_{i+v} =X_iX_v$. 
\begin{definition}
For a line $\ell\subset V$ define $S_\ell \defIs  \sum_{i\in\ell} X_i$. For a vector $0\neq v\in \ell$ we set $S_v\defIs S_\ell$.
\end{definition}
\begin{lemma}\label{SympLemma}
Let $P\subset V$ be a plane and $\ell_1,\ell_2\subset P$ two different lines spanning $P$. Then we have
$$
S_{\ell_1}S_{\ell_2}=\sum_{i\in P}X_i = \sum_{\ell\subset P}S_\ell.
$$
\end{lemma}
\begin{proof}
The first equality is clear. For the second equality observe that every point $i\in P$ is contained in one line, if we count modulo $q$.
\end{proof}

\begin{definition} \label{SymplecticIdeal}
We define two subsets of $k[V]$:
\begin{gather*}
M \defIs  \left\{\sum_{i\in P}X_i \,|\, P\subset V \text{ plane}\right\}, \\
N  \defIs  \left\{\sum_{i\in P}X_i \,|\, P\subset V \text{ non-isotropic plane}\right\}.
\end{gather*}
Let $(M)$ and $(N)$ be the ideals generated by $M$ and $N$, respectively. 
Further, let $D$ be the linear span of $\{v(b) - b \,|\, b\in N, v\in V \}$. Then $D$ is in fact an ideal, namely the product of ideals $\mathfrak m\cdot (N)$.
\end{definition}
\begin{proposition}
We have $(M)=(N)$.
\end{proposition}
\begin{proof}
We have to show that $\sum_{i\in P}X_i \in (N)$ for an isotropic plane $P$. Let $v,w$ be two spanning vectors of $P$ and $u$ a vector with $\omega(u,v)\neq 0$. Denote $P'$ the non-isotropic plane spanned by $u $ and $v$. By Lemma~\ref{SympLemma}, we have
$$
S_uS_vS_w = \sum_{\ell\subset P'} S_{\ell}S_w= \left(S_v+\sum_{\lambda\in k}S_{u + \lambda v}\right)S_w.
$$
Now $w$ spans a non-isotropic plane with every line in $P'$, except one, namely the line that contains $v$. So it follows that
$$
\sum_{i\in P}X_i = S_vS_w = S_uS_vS_w  - \sum_{\lambda\in k} S_{u + \lambda v}S_w, 
$$
and we see that the right hand side is an element of $(N)$.
\end{proof}
For the rest of this section, we assume $\dim_k V=4$. 
\begin{proposition} \label{SymplecticIdealsDimension}
The following table illustrates the dimensions of $(N)$ and $D$ for some $k$.
\vspace{2mm}
\begin{center}
\begin{tabular}{c||c|c|c}
 $k$ & $\dim_k(N)$ & $\dim_k D$ \\
\hline
$\mathbb F_2$   & 11 &  5  \\
$\mathbb F_3$  & 50 & 31  \\
$\mathbb F_5$  &355 &270
\end{tabular}
\end{center}
\end{proposition}
Since this is computed numerically using a naive approach, we do not give a formal proof.
\begin{rmk}\label{c2}
We remark that $X\defIs \sum_{i\in V}X_i\in D$. Indeed, let $P$, $P'$ be two non-isotropic planes with $P \cap P' = 0$. Then $X = \left( \sum_{i\in P}X_i\right)\left(  \sum_{i\in P'}X_i \right)$ and both factors are contained in $(N)\subset \mathfrak m$, so $X\in \mathfrak m \cdot (N) = D$.
\end{rmk}
Let us now consider some special orthogonal sums.
Set $S\defIs \Sym^2 (\Lambda^2V)$. Take two vectors $v,w\in V$ with $\omega(v,w)=1$ and set $x\defIs  (v\wedge w)^2\in S$. Denote $P$ the plane spanned by $v$ and $w$ and set $y\defIs  \sum_{i\in P}X_i\in  k[V]$.

We consider now the action of $\Sp V$ on $S\oplus k[V]$. 
Denote $O$ the vector space spanned by the elements $\phi(x)\oplus \phi(z),$ for $\phi \in \Sp V,$ $z \in (y)$
and $U$ the vector space spanned by the elements $\phi(x),$ for $\phi \in \Sp V$.
\begin{proposition}\label{CombinedSymplectic}
Then we have by numerical computation:
\vspace{2mm}
\begin{center}
\begin{tabular}{c||c|c}
 $k$ & $\dim_k O$  & $\dim_k U$\\
\hline
$\mathbb F_2$ & 11 & 6 \\
$\mathbb F_3$ & 51  & 20 \\
$\mathbb F_5$ & 375  & 20 
\end{tabular}
\end{center}
\end{proposition}
Now we prove the following lemma that we will need for a divisibility argument in Section \ref{Middle}.
\begin{lemme}\label{cleffinclassesdiv}
We assume that $k=\mathbb F_3$. Let $\pr_1: S\oplus k[V]\rightarrow S$ and $\pr_2: S\oplus k[V]\rightarrow k[V]$ the projection. 
We have: 
\begin{itemize}
\item[(i)]
$\dim \ker\pr_{2|O}=1$.
\item[(ii)]
$\dim \ker\pr_{1|O}=31$.
%and $\ker\pr_{1|O} = D$.
\end{itemize}
\end{lemme}
\begin{proof}
We have $\pr_{1}(O)=U$ and $\pr_{2}(O)=(N)$. Using the dimension tables from Propositions \ref{CombinedSymplectic} and \ref{cleffinclassesdiv}, we get
\begin{gather*}
\dim \ker\pr_{1|O} = \dim O - \dim U  = 31,\\
\dim \ker\pr_{2|O} = \dim O - \dim (N) = 1.
\qedhere
\end{gather*}
\end{proof}
\section{Complex abelian surfaces}\label{AbelianSection}
Denote $A$ a complex abelian surface (a torus of dimension $2$). As such, it always can be written as a quotient
$$
A = \C^2/\Lambda,
$$
where $\Lambda\subset \C^2$ is a lattice of rank $4$, embedded in $\C^2$. 
Depending on the imbedding, we get different complex manifolds, projective or not. Of course, all of them are equivalent by deformation.
\subsection{Morphisms and special cases}
\begin{definition}
An isogeny between abelian surfaces $A=\C^2/\Lambda\rightarrow A'=\C^2/\Lambda'$ means a surjective holomorphic map that preserves the group structure. It is given by a complex linear map, that maps $\Lambda$ to a sublattice of $\Lambda'$. 
\end{definition}
\begin{example}
For a number $n\neq 0$, the multiplication map $n: A\rightarrow A$, $x\mapsto n\cdot x$ is an isogeny.
\end{example}


By an automorphism of $A$ we mean a biholomorphism preserving the group structure. It can be represented by a $\C$-linear map $ M:\C^2 \rightarrow \C^2$ with $M\Lambda =\Lambda$. Have a look in \cite{Fujiki} or the appendix of \cite{Ghys} for some reference.
Let us now come to the very special case that $A=E\times E$ can be written as the square of an elliptic curve. Note that $A$ is projective, because every elliptic curve is. 
Now write $E$ as $E=\C/\Lambda_0$. We may assume that $\Lambda_0 $ is spanned by $1$ and a vector $\tau\in\C\backslash\R$. The automorphism group, up to isogeny, is given by (\cite{Ghys})
$\GL(2,\End(\Lambda_0))$, where
$\End(\Lambda_0)$ is the set $\{z\in\C \;|\; z\Lambda_0\subset \Lambda_0\}$.
\begin{proposition} \label{EndLambda}
There are two possibilities for $\End(\Lambda_0)$, depending on $\tau$:
\begin{enumerate}
 \item Both the real part and the square norm of $\tau$ are rational numbers, say $2\Re(\tau) = \frac{p}{r}$ and $\|\tau\|^2 = \frac{q}{r}$ with $r>0$ as small as possible. Then $\End(\Lambda_0)= \Z+ r\tau\Z$.
 \item At least one of $\Re(\tau), \|\tau\|^2$ is irrational. Then $\End(\Lambda_0)=\Z$.
\end{enumerate}
\end{proposition}
\begin{proof}
Given $z\in\End(\Lambda_0)$, we have $$z\cdot 1 = a + b\tau\text{ and }z\cdot \tau = c+ d\tau\text{ with }a,b,c,d\in \Z.$$ 
We get the condition
$$
(a+b\tau)\tau = c+d\tau\quad \Leftrightarrow \quad b\tau^2 + (a-d)\tau -c =0.
$$
Up to scalar multiples, there is a unique real quadratic polynomial that annihilates $\tau$, namely $ (x -\tau )(x-\bar{\tau})=x^2 - 2\Re(\tau)x+ \|\tau\|^2$. 
If all coefficients of that polynomial are rational numbers, then $z=a+b\tau$ gives a solution for arbitrary $a\in\Z$, $b\in r\Z$. Otherwise, the condition must be the zero polynomial, so $b=0$.
\end{proof}
Now we study the action of automorphisms on torsion points in a very special case. This will be needed in the technical proof of Theorem \ref{fin}.
\begin{definition}\label{elliptic6}
Denote $\xi\in\C$ a primitive sixth root of unity and $E_\xi$ the elliptic curve given by the choice $\Lambda_0 = \left<1,\xi\right>$, so by Proposition \ref{EndLambda}, $\End(\Lambda_0)=\Lambda_0$ is the ring of Eisenstein integers. Define a group $G_\xi$ of automorphisms of $E_\xi\times E_\xi$ by the following generators in $\GL(2,\End(\Lambda_0))$:
\begin{align*}
g_1 &= \left( {\begin{array}{cc}
   \xi & 0 \\       0 & 1      
   \end{array} } \right),
 &
g_2 &= \left( {\begin{array}{cc}
   0 & 1 \\       1 & 0      
   \end{array} } \right),
 &
g_3 &= \left( {\begin{array}{cc}
   1 & 1 \\       0 & 1     
   \end{array} } \right).
\end{align*}
\end{definition}
For $A=E_\xi\times E_\xi$, let $V =A[2]$ be the (fourdimensional) $\mathbb F_2$-vector space of $2$-torsion points on $A$ and let $\mathfrak T$ be the set of planes in $V$. Note that by Remark \ref{PlaneTriple} a plane in $V$ can be identified with an unordered triple $\{x,y,z\}$ with $0\neq x,y,z\in V$ and $x+y+z=0$. The action of $G_\xi$ on $A$ induces actions of $G_\xi$ on $A[2]$ and $\mathfrak T$. 
\begin{lemma}\label{orbitesG}
There are two orbits of $G_\xi$ on $\mathfrak T$, of cardinalities $5$ and $30$.
\end{lemma}
\begin{proof}
Note that the generators $g_2$ and $g_3$ exist because $A$ is of the form $E\times E$, while $g_1$ exists only in the special case $E=E_\xi$. Indeed, multiplication with $\xi$ induces a cyclic permutation on $E_\xi[2]$. 
The orbits can be explicitely determined by a suitable computer program. For verification, we give one of the orbits explicitely.
Denote $x_1,x_2,x_3$ the non-zero points in $E_\xi[2]$. The orbit of cardinality five is then given by
\begin{align*}
\{(0,x_1),(0,x_2),(0,x_3)\} &,& \{(x_1,0),(x_2,0),(x_3,0)\} &,& \\
\{(x_1,x_1),(x_2,x_2),(x_3,x_3)\} &,& \{(x_1,x_2),(x_2,x_3),(x_3,x_1)\} &,& \{(x_1,x_3),(x_2,x_1),(x_3,x_2)\}.
\end{align*}
\end{proof}

\subsection{Homology and Cohomology}\label{monodromyexplication}
The fundamental group $\pi_1(A,\Z) = H_1(A,\Z)$ is a free $\Z$-module of rank $4$, which is canonically identified with the lattice $\Lambda$. Indeed, the projection of every path in $\C^2$ from $0$ to $v\in \Lambda$ gives a unique element of $\pi_1(A,\Z)$. Conversely, any closed path in $A$ with basepoint $0$ lifts to a unique path in $\C^2$ from $0$ to some $v\in\Lambda$.
So the first cohomology $H^1(A,\Z)$ is freely generated by four elements, too. Moreover, by \cite[Sect.~I.1]{Mumford}, the cohomology ring is isomorphic to the exterior algebra
$$
H^*(A,\Z) = \Lambda^*(H^1(A,\Z)).
$$
\begin{notation} \label{TorusClasses}
We denote the generators of $H^1(A,\Z)$ by $a_i$, $1\leq i\leq 4$ and their respective duals by $a_i^*\in H^3(A,\Z)$. 
If $A=E\times E$ is the product of two elliptic curves, we choose the $a_i$ in a way such that $\{a_1,a_2\}$ and $\{a_3,a_4\}$ give bases of $H^1(E,\Z)$ in the decomposition $H^1(A) = H^1(E)\oplus H^1(E)$.
We denote the generator of the top cohomology $H^4(A,\Z)$ by $x := a_1 a_2 a_3 a_4$.
\end{notation}



Let $A$ be a abelian surface. We recall that a \emph{principal polarization} of $A$ is a polarization $L$ such that there exists a basis of $H_1(A,\Z)$, with respect to which the symplectic bilinear form on $H_1(A,\Z)$ induced by $c_1(L)$:
\begin{equation}
\omega_L(x,y)=x\cdot c_1(L)\cdot y,
\label{symplecticprinc}
\end{equation}
is given by the matrix:
$$\left( {\begin{array}{cccc}
   0 & 0 & 1 & 0 \\    0 &  0 & 0 & 1\\ -1 & 0 & 0 & 0\\ 0 & -1 & 0 & 0     
   \end{array} } \right).$$
%We remark that a principal polarization $L$ provides a symplectic bilinear form $\omega_L$ on $H_1(A,\Z)$ as follows:

%for all $x$ and $y$ in $H_1(A,\Z)$.
%TODO: Principal polarization/Jacobians \\
%Let $p$ be a prime number, $n\in\mathbb{N}^{*}$, $\mu_p^n$ the group of $p^n$-th roots of the unity and $\Z_p(1):=\underleftarrow{\lim}\mu_p^n$. 
%We recall that the \emph{Weil pairing} $e_p^L$ can be defined as follows.
%$$e_p^L(x,y)=\varsigma^{-x\cdot c_1(L)\cdot y},$$
%for all $x$, $y$ in $H_1(X,\Z)$ and $\varsigma=(...,e^{\frac{2i\pi}{p^n}},...)$.
We recall the following result. 
\begin{prop}
Let $(A,L)$ be a principally polarized abelian surface. The group $H_1(A,\Z)$ is endowed with the symplectic from $\omega_L$ defined in (\ref{symplecticprinc}). Let $\Mon (H_1(A,\Z))$ be the image of monodromy representations on $H_1(A,\Z)$.
Then $\Mon (H_1(A,\Z))\supset\Sp(H_1(A,\Z))$.
%TODO: Weil pairing for tori
\end{prop}
\begin{proof}
It can be seen as follows.
Let $\mathcal{M}_2$ be the moduli space of curves of genus $2$ and $\mathcal{A}_2$ be the moduli space of principally polarized abelian surfaces.
By the Torelli theorem (see for instance \cite[Theorem 12.1]{Milne}), we have an injection $J:\mathcal{M}_2\hookrightarrow \mathcal{A}_2$ given by taking the Jacobian of the curve endowed with its canonical polarization. Moreover, the moduli spaces $\mathcal{M}_2$ and $\mathcal{A}_2$ are both of dimension 3. 
%$J(\mathcal{M}_2)=\mathcal{A}_2\smallminus \mathcal{A}_1\times\mathcal{A}_1$.

Now if $\mathscr{C}_2$ is a curve of genus 2, we have by Theorem 6.4 of \cite{Farb}: 
$$\Mon (H_1(\mathscr{C}_2,\Z))\supset \Sp(H_1(\mathscr{C}_2,\Z)),$$
where the symplectic form on $H_1(\mathscr{C}_2,\Z)$ is given by the cup product. 
Then the result follows from the fact that the lattices $H_1(\mathscr{C}_2,\Z)$ and $H_1(J(\mathscr{C}_2),\Z)$ are isometric.
%, where $H_1(\mathscr{C}_2,\Z)$ is endowed with the cup product and $H_1(J(\mathscr{C}_2),\Z)$ with $\omega_$ with $L$ the principal polarization.
\end{proof}
\begin{rmk}\label{SPA2}
Let $(A,L)$ be a principally polarized abelian surface and $p$ a prime number. The group $H_1(A,\Z)$ tensorized by $\mathbb{F}_p$ can be seen as the group $A[p]$ of points of $p$-torsion on $A$ and the form $\omega_L\otimes\mathbb{F}_p$ provides a symplectic form on $A[p]$. Then $\Mon (A[p])$, the image of the monodromy representation on $A[p]$ contains the group $\Sp(A[p])$. 
\end{rmk}
Now, we are ready to recall Proposition 5.2 of \cite{Hassett} on the monodromy of the generalized Kummer fourfold.
\begin{prop}\label{Hassettmonodromy}
Let $A$ be an abelian surface and $K_2(A)$ the associated generalized Kummer fourfold. 
The image of the monodromy representation on $\Pi=\left\langle\left. Z_\tau\right|\ \tau\in A[3]\right\rangle$ contains the semidirect product
$\Sp(A[3])\ltimes A[3]$ which acts as follows:
$$f\cdot Z_\tau= Z_{f(\tau)}\ \text{and}\ \tau'\cdot Z_\tau= Z_{\tau+\tau'},$$
for all $f\in \Sp(A[3])$ and $\tau'\in A[3]$.
\end{prop}
\section{Recall on the theory of integral cohomology of quotients}\label{IntegralTools}
\sectionmark{Integral cohomology of quotients}
The main references of this section are \cite{Lol} and \cite{BNS}.

Let $G=\left\langle \iota\right\rangle$ be the group generated by an involution $\iota$ on a complex manifold $X$.
As denoted in \cite[Section 5]{BNS}, let $\mathcal{O}_{K}$ be the ring $\Z$ with the following $G$-module structure:
$\iota\cdot x=-x$ for $x\in \mathcal{O}_{K}$. For $a\in \Z$, we also denote by $(\mathcal{O}_{K},a)$ the module $\Z\oplus\Z$ whose $G$-module structure is defined by $\iota\cdot(x,k)=(-x+ka,k)$. We also denote by $N_{2}$ the $\mathbb{F}_{2}[G]$-module $(\mathcal{O}_{K},a)\otimes\mathbb{F}_{2}$.
We recall Definition-Proposition 2.2.2 of \cite{Lol}.
\begin{defipro}\label{defiprop}
Assume that $H^{*}(X,\Z)$ is torsion-free. Then for all $0\leq k \leq 2\dim X$, we have an isomorphism of $\Z[G]$-modules:
$$H^{k}(X,\Z)\simeq \bigoplus_{i=1}^{r}(\mathcal{O}_{K},a_{i})\oplus \mathcal{O}_{K}^{\oplus s}\oplus\Z^{\oplus t},$$
for some $a_{i}\notin 2\Z$ and $(r,s,t)\in\mathbb{N}^3$.
We get the following isomorphism of $\mathbb{F}_{2}[G]$-modules:
$$H^{k}(X,\mathbb{F}_{2})\simeq N_{2}^{\oplus r}\oplus\mathbb{F}_{2}^{\oplus (s+t)}.$$
We denote $l_{2}^k(X)\defIs r$, $l_{1,-}^k(X)\defIs s$, $l_{1,+}^k(X)\defIs t$, $\mathcal{N}_{2}\defIs N_{2}^{\oplus r}$ and $\mathcal{N}_{1}\defIs \mathbb{F}_{2}^{\oplus s+t}$.
\end{defipro}
\begin{rmk}
These invariants are uniquely determined by $G$, $X$ and $k$.
\end{rmk}
We recall an adaptation of Proposition 5.1 and Corollary 5.8 of \cite{BNS} that can be found in Section 2.2 of \cite{Lol}.
\begin{prop}\label{sarti}
Let $X$ be a compact complex manifold of dimension $n$ and $\iota$ an involution. Assume that $H^{*}(X,\Z)$ is torsion free.
We have:
\begin{itemize}
\item[(i)]
$\rk H^{k}(X,\Z)^{\iota}=l_{2}^k(X)+l_{1,+}^k(X).$
\item[(ii)]
We denote $\sigma\defIs \id+\iota^*$ and $S^k_\iota\defIs  \Ker \sigma \cap H^{k}(X,\Z)$. 
We have $H^{k}(X,\Z)^{\iota}\cap S^k_\iota=0$ and
$$\frac{H^{k}(X,\Z)}{H^{k}(X,\Z)^{\iota}\oplus S^k_\iota}=(\Z/2\Z)^{\oplus l_2^k(X)}.$$
\end{itemize}
\end{prop}
\begin{rmk}\label{x+ix}
Note that the elements of $(\mathcal{O}_{K},a_{i})^{\iota}$ are of the form $x+\iota^{*}(x)$ with $x\in (\mathcal{O}_{K},a_{i})$.
\end{rmk}
Let $\pi: X\rightarrow X/G$ be the quotient map. 
We denote by $\pi^{*}$ and $\pi_{*}$ respectively the pull-back and the push-forward along $\pi$. We recall that 
\begin{equation}
\pi_*\circ \pi^*=2\id \text{ and } \pi^*\circ\pi_* =\id+\iota^*.
\label{pietpi}
\end{equation}
Assuming that $H^{k}(X,\Z)$ is torsion free, we obtain the following exact sequence (Proposition 3.3.3 of \cite{Lol}), which will be useful in the next section:
\begin{equation}
\xymatrix{ 0\ar[r]&\pi_{*}(H^{k}(X,\Z))\ar[r] & H^{k}(X/G,\Z)/\tors\ar[r] & (\Z/2\Z)^{\oplus\alpha_{k}}\ar[r]& 0,}
\label{classicexact}
\end{equation}
with $\alpha_{k}\in \mathbb{N}$.
We also recall the commutativity behaviour of $\pi_*$ with the cup product.
\begin{prop}\cite[Lemma 3.3.7]{Lol}\label{commut}
Let $X$ be a compact complex manifold of dimension $n$ and $\iota$ an involution. Assume that $H^{*}(X,\Z)$ is torsion free.
Let $0\leq k \leq 2n$, $m$ be integers such that $km\leq 2n$, and let $(x_{i})_{1\leq i \leq m}$ be elements of $H^{k}(X,\Z)^{\iota}$.
Then $$\pi_{*}(x_{1})\cdot...\cdot \pi_{*}(x_{m})=2^{m-1}\pi_{*}(x_{1}\cdot...\cdot x_{m}).$$
\end{prop}
We also recall Definition 3.3.4 of \cite{Lol}. 
\begin{defi}
Let $X$ be a compact complex manifold and $\iota$ be an involution. 
Let $0\leq k\leq 2n$, and assume that $H^{k}(X,\Z)$ is torsion free. 
Then if the map $\pi_{*}:H^{k}(X,\Z)\rightarrow H^{k}(X/G,\Z)/\tors$ is surjective, we say that $(X,\iota)$ is \emph{$H^{k}$-normal}.
\end{defi}
\begin{rmk}\label{Hnormal}
The $H^{k}$-normality is equivalent to the following property:

For $x\in H^{k}(X,\Z)^{\iota}$, $\pi_{*}(x)$ is divisible by 2 if and only if there exists a $y\in H^{k}(X,\Z)$ such that 
$x=y+\iota^{*}(y)$.
\end{rmk}
We also need to recall Definition 3.5.1 of \cite{Lol} about fixed loci.
\begin{defi}\label{negligible}
Let $X$ be a compact complex manifold of dimension $n$ and $G$ an automorphism group of prime order $p$. 
\begin{enumerate}
\item
We will say that $\Fix G$ is negligible if the following conditions are verified:
\begin{itemize}
\item[$\bullet$]
$H^{*}(\Fix G,\Z)$ is torsion-free.
\item[$\bullet$]
$\codim \Fix G\geq \frac{n}{2}+1$.
\end{itemize}
\item
We will say that $\Fix G$ is almost negligible if the following conditions are verified:
\begin{itemize}
\item[$\bullet$]
$H^{*}(\Fix G,\Z)$ is torsion-free.
\item[$\bullet$]
$n$ is even and $n\geq 4$.
\item[$\bullet$]
$\codim \Fix G =\frac{n}{2}$, and the purely $\frac{n}{2}$-dimensional part of $\Fix G$ is connected and simply connected. We denote the $\frac{n}{2}$-dimensional component by $Z$.
\item[$\bullet$]
The cocycle $\left[Z\right]$ associated to $Z$ is primitive in $H^{n}(X,\Z)$.
\end{itemize}
\end{enumerate}
\end{defi}
Now we are ready to provide Theorem 2.65 of \cite{Lol}.
\begin{thm}\label{utile'}
Let $G=\left\langle \varphi\right\rangle$ be a group of prime order $p=2$ acting by automorphisms on a K�hler manifold $X$ of dimension $2n$. 
We assume:
\begin{enumerate}
\item
$H^{*}(X,\Z)$ is torsion-free,
\item
$\Fix G$ is negligible or almost negligible,
\item
$l_{1,-}^{2k}(X)=0$ for all $1\leq k \leq n$, and
\item
$l_{1,+}^{2k+1}(X)=0$ for all $0\leq k \leq n-1$, when $n>1$.
\item

$l_{1,+}^{2n}(X)+2\left[\sum_{i=0}^{n-1}l_{1,-}^{2i+1}(X)+\sum_{i=0}^{n-1}l_{1,+}^{2i}(X)\right]
= \sum_{k=0}^{\dim \Fix G}h^{2k}(\Fix G,\Z).$
\end{enumerate}
Then $(X,G)$ is $H^{2n}$-normal.
\end{thm}
We will also need a proposition from Section 7 of \cite{BNS} about Smith theory. Let $X$ be a topological space and let $G=\left\langle \iota\right\rangle$ be an involution acting on $X$. 
Let $\sigma\defIs 1+\iota\in \mathbb{F}_{2}[G]$. We consider the chain complex $C_{*}(X)$ of $X$ with coefficients in $\mathbb{F}_{2}$ and its subcomplex $\sigma C_{*}(X)$. We denote by $X^{G}$ the fixed locus of the action of $G$ on $X$. 
\begin{prop}\label{SmithProp}
\begin{enumerate}
\item (\cite{Bredon}, Theorem 3.1). There is an exact sequence of complexes:
$$\xymatrix@C=20pt{0\ar[r] &\sigma C_{*}(X)\oplus C_{*}(X^{G})\ar[r]^{\ \ \ \ \ \ f}&C_{*}(X) \ar[r]^{\sigma}&\sigma C_{*}(X) \ar[r]&0
},$$ where $f$ denotes the sum of the inclusions.
\item (\cite{Bredon}, (3.4) p.124). There is an isomorphism of complexes:
$$\sigma C_{*}(X)\simeq C_{*}(X/G,X^{G}),$$
where $X^{G}$ is identified with its image in $X/G$.
\end{enumerate}
\end{prop}

\section{Odd cohomology of the Hilbert scheme of two points}
Let $A$ be a complex torus \textbf{TODO: Does it work if $A$ is a general (projective) surface with torsion-free cohomology?} of dimension $2$ and $A\hilb{2}$ the Hilbert scheme of 2 points. 
It can be constructed as follows: Consider the direct product $A\times A$. Denote 
$$b: \widetilde{A\times A} \rightarrow A\times A $$ 
the blow-up along the diagonal $\Delta \cong A$ with exceptional divisor $E$, so we have $i: E\rightarrow \widetilde{A\times A}  $. Since the normal bundle of $\Delta$ in $A\times A$ is trivial, we have:
$$
E \cong  \Delta\times \mathbb{P}^1.
$$
The action of $\mathfrak{S}_2$ on $A\times A$ lifts to an action on $\widetilde{A\times A}$. 
We have the pushforward $i_*:H^*(E,\Z)\rightarrow H^*(\widetilde{A\times A} ,\Z) $.

The quotient by the action of $\mathfrak{S}_2$ is 
$$ \pi:\widetilde{A\times A} \rightarrow A\hilb{2}.$$ 
Now, $A\hilb{2}$ is a compact complex manifold with torsion-free cohomology, \cite[Theorem~2.2]{Totaro}.
By \cite[Proposition~0.1]{Menet}, there is an exact sequence
$$
0 \rightarrow \pi_*(H^k(\widetilde{A\times A,\Z})) \rightarrow H^k(A\hilb{2},\Z) \rightarrow \left(\frac{\Z}{2{\Z}}\right)^{\alpha_k}\rightarrow 0
$$
with $k\in \left\{1,...,8\right\}$.
\begin{proposition} \label{Alpha35}
We have:
$$\alpha_3=0\ \text{ and }\ \alpha_5=4.$$
\end{proposition}
\subsection{Preliminary Lemmas}
%Fist we need to calculate the following invariant:
We denote $V=\widetilde{A\times A}\smallsetminus E$ and $U=V/\mathfrak S_{2}$, where $\mathfrak{S}_{2}=\left\langle \sigma_{2}\right\rangle$. 
\begin{lemma}\label{1}
We have: $H^{k}(A\times A,\Z)=H^{k}(V,\Z)$
for all $k\leq 3$.
\end{lemma}
\begin{proof}
We have $V=A\times A\smallsetminus \Delta$.
We have the following natural exact sequence:
$$\xymatrix{ \cdots\ar[r]&H^{k}(A\times A,V,\Z)\ar[r] & H^{k}(A\times A,\Z)\ar[r] & H^{k}(V,\Z)\ar[r]& \cdots}$$
Moreover, by Thom isomorphism $H^{k}(A\times A,V,\Z)=H^{k-4}(\Delta,\Z)=H^{k-4}(A,\Z)$.
Hence $H^{k}(A\times A,V,\Z)=0$ for all $k\leq 3$.
Hence $H^{k}(A\times A,\Z)=H^{k}(V,\Z)$
for all $k\leq 2$. It remains to consider the following exact sequence:
$$\xymatrix{ 0\ar[r]&H^{3}(A\times A,\Z)\ar[r]&H^{3}(V,\Z)\ar[r] & H^{4}(A\times A,V,\Z)\ar[r]^{\rho} & H^{4}(A\times A,\Z)}.$$
The map $\rho$ is given by $\Z \left[\Delta\right] \rightarrow H^{4}(A\times A,\Z)$.
The class $\left\{x\right\}\times A$ is also in $H^{4}(A\times A,\Z)$ and intersects $\Delta$ in one point.
Hence the class of $\Delta$ in $H^{4}(A\times A,\Z)$ is not trivial and the map $\rho$ is injective.
%Moreover, we know by K�nneth formula that:
%$$H^{4}(A\times A,\Z)=H^{0}(A,\
It follows $$H^{3}(A\times A,\Z)=H^{3}(V,\Z).$$
\end{proof}
Now we will calculate the invariant $l_{1,-}^{2}(A\times A)$ and $l_{1,+}^{1}(A\times A)$ defined in Section 1.2 of \cite{Menet} 

\textbf{TODO : recall the definition in the redaction of the application.}
\begin{lemma}\label{2}
We have: $l_{1,-}^{2}(A\times A)=l_{1,+}^{1}(A\times A)=0$.
\end{lemma}
\begin{proof}
By K�nneth formula we have:
$$H^{1}(A\times A,\Z)=H^{0}(A,\Z)\otimes H^{1}(A,\Z)\oplus H^{1}(A,\Z)\otimes H^{0}(A,\Z).$$
The elements of $H^{0}(A,\Z)\otimes H^{1}(A,\Z)$ and $H^{1}(A,\Z)\otimes H^{0}(A,\Z)$ are exchanged under the action of $\sigma_2$. It follows that $l_{2}^{1}(A\times A)=4$ and necessary $l_{1,-}^{1}(A\times A)=l_{1,+}^{1}(A\times A)=0$.

By K�nneth formula we also have:
\begin{align*}
H^{2}(A\times A,\Z)&=H^{0}(A,\Z)\otimes H^{2}(A,\Z)\oplus H^{1}(A,\Z)\otimes H^{1}(A,\Z)\\
&\oplus H^{2}(A,\Z)\otimes H^{0}(A,\Z).
\end{align*}
As before every elements $x\otimes y\in H^{2}(A\times A,\Z)$ are sent to $y\otimes x$ by the action of $\sigma_2$. A such element is fixed by the action of $\sigma_2$ if $x=y$. It follows:
$$l_{2}^{2}(A\times A)=6+6=12,$$
$$l_{1,+}^{2}(A\times A)=4,$$
and necessary:
$$l_{1,-}^{2}(A\times A)=0.$$
\end{proof}
\begin{lemma}\label{3}
The group $H^{3}(U,\Z)$ is torsion free.
\end{lemma}
\begin{proof}
Using the spectral sequence of equivariant cohomology, it follows from Proposition 2.6 of \cite{Menet}, Lemma \ref{1} and \ref{2}.
\end{proof}
\subsection{In degree 3}\label{=}
%Now, we prove that $\alpha_3=0$.
By Theorem 7.31 of \cite{Voisin}, we have:
\begin{equation}
H^{3}(\widetilde{A\times A},\Z)=H^{3}(A\times A,\Z)\oplus H^{1}(\Delta,\Z).
\label{voisin1}
\end{equation}
It follows that $$H^{3}(A^{[2]},\Z)\supset \pi_{*}(H^{3}(A\times A,\Z))\oplus \pi_{*}(H^{1}(\Delta,\Z)).$$
Moreover, by K�nneth formula, we have:
\begin{align*}
H^{3}(A\times A,\Z)&=H^{0}(A,\Z)\otimes H^{3}(A,\Z)\oplus H^{1}(A,\Z)\otimes H^{2}(A,\Z)\\
&\oplus H^{2}(A,\Z)\otimes H^{1}(A\Z)\oplus H^{3}(A,\Z)\otimes H^{0}(A,\Z).
\end{align*}
Hence all elements in $H^{3}(A\times A,\Z)^{\mathfrak{S}_2}$ are written $x+\sigma_{2}^{*}(x)$ with $x\in H^{3}(A\times A,\Z)$.
Since $\frac{1}{2}\pi_{*}(x+\sigma_{2}^{*}(x))=\pi_{*}(x)$, it follows that $\pi_{*}(H^{3}(A\times A,\Z))$ is primitive in $H^{3}(A^{[2]},\Z)$.
Moreover by (\ref{voisin1}):
\begin{equation}
l_2^3(\widetilde{A\times A})=\rk H^{3}(A\times A,\Z)^{\mathfrak{S}_2}=28.
\label{l1}
\end{equation}
and
\begin{equation}
l_{1,+}^3(\widetilde{A\times A})=\rk H^{1}(\Delta,\Z)^{\mathfrak{S}_2}=4,\ \text{ and }\ l_{1,-}^3(\widetilde{A\times A})=0.
\label{l4}
\end{equation}
It remains to prove the following lemma.
\begin{lemma}
The group $\pi_{*}(H^{1}(\Delta,\Z))$ is primitive in $H^{3}(A^{[2]},\Z)$.
\end{lemma}
\begin{proof}
We consider the following commutative diagram:
\begin{equation}
\xymatrix@C=10pt{ \ar[d]^{d\widetilde{\pi}^{*}}H^{3}(\mathscr{N}_{A^{[2]}/E},\mathscr{N}_{A^{[2]}/E}-0,\Z)=H^{3}(A^{[2]},U,\Z)\ar[r]^{\ \ \ \ \ \ \ \ \ \ \ \ \ \ \ \ \ \ \ \ g}&\ar[d]_{\pi^{*}}H^{3}(A^{[2]},\Z)\\
H^{3}(\mathscr{N}_{\widetilde{A\times A}/E},\mathscr{N}_{\widetilde{A\times A}/E}-0,\Z)=H^{3}(\widetilde{A\times A},V,\Z)\ar[r]^{\ \ \ \ \ \ \ \ \ \ \ \ \ \ \ \ \ \ \ \ h}&H^{3}(\widetilde{A\times A},\Z),
}
\label{ThomII}
\end{equation}
By proof of Theorem 7.31 of \cite{Voisin}, the map $h$ is injective and its image in $H^{3}(\widetilde{A\times A},\Z)$ is $H^{1}(\Delta,\Z)$. Hence Diagram (\ref{ThomII}) shows that $g$ is also injective and has image $\pi_{*}(H^{1}(\Delta,\Z))$ in $H^{3}(A^{[2]},\Z)$.
It follows the exact sequence:
$$\xymatrix{ 0\ar[r]&H^{3}(A^{[2]},U,\Z)\ar[r]^g&H^{3}(A^{[2]},\Z)\ar[r]&H^{3}(U,\Z)}.$$
However, by Lemma \ref{3}, $H^{3}(U,\Z)$ is torsion free; it follows that $\pi_{*}(H^{1}(\Delta,\Z))$ is primitive in $H^{3}(A^{[2]},\Z)$.
\end{proof}
\subsection{In degree 5}
By Theorem 7.31 of Voisin, we have:
\begin{equation}
H^{5}(\widetilde{A\times A},\Z)=H^{5}(A\times A,\Z)\oplus H^{3}(\Delta,\Z).
\label{voisin2}
\end{equation}
It follows that $$H^{5}(A^{[2]},\Z)\supset \pi_{*}(H^{5}(A\times A,\Z))\oplus \pi_{*}(H^{3}(\Delta,\Z)).$$
Moreover, by K�nneth formula, we have:
\begin{align*}
H^{5}(A\times A,\Z)&=H^{1}(A,\Z)\otimes H^{4}(A,\Z)\oplus H^{2}(A,\Z)\otimes H^{3}(A,\Z)\\
&\oplus H^{3}(A,\Z)\otimes H^{2}(A,\Z)\oplus H^{4}(A,\Z)\otimes H^{1}(A,\Z).
\end{align*}
As before, $\pi_{*}(H^{5}(A\times A,\Z))$ is primitive in $H^{5}(A^{[2]},\Z)$.
Moreover by (\ref{voisin2}):
\begin{equation}
l_2^5(\widetilde{A\times A})=\rk H^{5}(A\times A,\Z)^{\mathfrak{S}_2}=28,
\label{l2}
\end{equation}
and 
\begin{equation}
l_{1,+}^5(\widetilde{A\times A})=\rk H^{3}(\Delta,\Z)^{\mathfrak{S}_2}=4,\ \text{ and }\ l_{1,-}^5(\widetilde{A\times A})=0.
\label{l3}
\end{equation}
\begin{lemma}
The lattice $\pi_{*}(H^{3}(\widetilde{A\times A},\Z)\oplus H^{5}(\widetilde{A\times A},\Z))$ has discriminant $2^8$.
\end{lemma}
\begin{proof}
By Definition-Proposition 1.7 2) and 3) of \cite{Menet}, (\ref{l1}) and (\ref{l2}):

\footnotesize
$$\frac{H^{3}(\widetilde{A\times A},\Z)\oplus H^{5}(\widetilde{A\times A},\Z)}{H^{3}(\widetilde{A\times A},\Z)^{\mathfrak{S}_2}\oplus H^{5}(\widetilde{A\times A},\Z)^{\mathfrak{S}_2}\oplus \left(H^{3}(\widetilde{A\times A},\Z)^{\mathfrak{S}_2}\oplus H^{5}(\widetilde{A\times A},\Z)^{\mathfrak{S}_2}\right)^\bot}=\left(\Z/2\Z\right)^{l_2^3(\widetilde{A\times A})+l_2^5(\widetilde{A\times A})}.$$
\normalsize
Since $H^{3}(\widetilde{A\times A},\Z)\oplus H^{5}(\widetilde{A\times A},\Z)$ is an unimodular lattice, 
it follows that 
$$\discr H^{3}(\widetilde{A\times A},\Z)^{\mathfrak{S}_2}\oplus H^{5}(\widetilde{A\times A},\Z)^{\mathfrak{S}_2}=2^{l_2^3(\widetilde{A\times A})+l_2^5(\widetilde{A\times A})}.$$
Then by Lemma 2.18 3) of \cite{Menet},
$$\discr \pi_{*}(H^{3}(\widetilde{A\times A},\Z)^{\mathfrak{S}_2}\oplus H^{5}(\widetilde{A\times A},\Z)^{\mathfrak{S}_2})=2^{l_2^3(\widetilde{A\times A})+l_2^5(\widetilde{A\times A})+\rk \left[H^{3}(\widetilde{A\times A},\Z)^{\mathfrak{S}_2}\oplus H^{5}(\widetilde{A\times A},\Z)^{\mathfrak{S}_2}\right]}.$$
Then by Proposition 1.6 of \cite{Menet}:
$$\discr \pi_{*}(H^{3}(\widetilde{A\times A},\Z)^{\mathfrak{S}_2}\oplus H^{5}(\widetilde{A\times A},\Z)^{\mathfrak{S}_2})=2^{2\left( l_2^3(\widetilde{A\times A})+ l_2^5(\widetilde{A\times A})\right)+l_{1,+}^3(\widetilde{A\times A})+l_{1,+}^5(\widetilde{A\times A})}.$$
Then by Lemma 2.17 and 2.3 of \cite{Menet}, 
$$\discr \pi_{*}(H^{3}(\widetilde{A\times A},\Z)\oplus H^{5}(\widetilde{A\times A},\Z))=2^{l_{1,+}^3(\widetilde{A\times A})+l_{1,+}^5(\widetilde{A\times A})}=2^8.$$
\end{proof}
The lattice $H^{3}(A^{[2]},\Z)\oplus H^{5}(A^{[2]},\Z)$ is unimodular. Hence:
$$\frac{H^{3}(A^{[2]},\Z)\oplus H^{5}(A^{[2]},\Z)}{\pi_{*}(H^{3}(\widetilde{A\times A},\Z)\oplus H^{5}(\widetilde{A\times A},\Z))}=(\Z/2\Z)^{4}.$$
However, by Lemma \ref{1}, we know that $\pi_{*}(H^{3}(\widetilde{A\times A},\Z))=H^{3}(A^{[2]},\Z)$.
It follows that
$$\frac{H^{5}(A^{[2]},\Z)}{\pi_{*}(H^{5}(\widetilde{A\times A},\Z))}=(\Z/2\Z)^{4}.$$

\section[Nakajima operators for Hilbert schemes of points on surfaces]{Nakajima operators for Hilbert schemes of points on surfaces%
\sectionmark{Nakajima operators}}
\sectionmark{Nakajima operators}
\label{Section_Hilbert}
Let $A$ be a smooth projective complex surface. 
%Set $H\defIs H^*(A,\Q)$. 
Let $A\hilb{n}$ the Hilbert scheme of $n$ points on the surface, \ie the moduli space of finite subschemes of $A$ of length $n$.
%$$\Hilb^ n : S \mapsto \left\{ Z\subset S\times A\,|\,Z\text{ closed subscheme, } p_1 : Z\rightarrow S \text{ flat and finite of degree }n\right\}.$$
$A\hilb{n}$ is again smooth and projective of dimension $2n$, cf.~\cite{Fogarty}. 
Their rational cohomology can be described in terms of Nakajima's~\cite{Nakajima} operators. First consider the direct sum
$$
\H \defIs  \bigoplus_{n=0}^{\infty} H^*(A\hilb{n},\Q).
$$
This space is bigraded by cohomological \emph{degree} and the \emph{weight}, which is given by the number of points $n$. The unit element in $H^0(A\hilb{0},\Q) \cong \Q$ is denoted by $\vac$, called the \emph{vacuum}.
\begin{defipro}
There are linear operators $\q_m(a)$, for each $m\geq 1$ and $a \in H^*(A,\Q)$, acting on $\H,$ which have the following properties: They depend linearly on $a$, and if $a\in H^k(A,\Q)$ is homogeneous, the operator $\q_{m}(a)$ is bihomogeneous of degree $k+2(m-1)$ and weight $m$:
$$
\q_{m}(a) : H^l(A\hilb{n}) \rightarrow H^{l+k+2(m-1)}(A\hilb{n+m})
$$
To construct them, first define incidence varieties $\mathcal Z_m\subset A\hilb{n}\times A\times A\hilb{n+m}$ by
$$
\mathcal Z_m \defIs  \left\{(\xi,x,\xi')\, |\, \xi\subset\xi',\, \supp(\xi') -\supp(\xi) = mx \right\}.
$$
Then $\q_m(a)(\beta) $ is defined as the Poincar\'e dual of 
$$
\pr_{3*}\left( \left(\pr_2^*(\alpha)\cdot \pr_3^*(\beta)\right) \cap [\mathcal Z_m] \right).
$$
\end{defipro}
Consider now the superalgebra generated by the $\q_m(a)$. 
Every element in $\H$ can be decomposed uniquely as a linear combination of products of operators $\q_{m}(a)$, acting on the vacuum. 
In other words, the $\q_m(a)$ generate $\H$ and there are no algebraic relations between them (except the linearity in $a$ and the super-commutativity).
\begin{example}
The unit $1_{A\hilb{n}} \in H^0({A\hilb{n}},\Q)$ is given by $\frac{1}{n!}\q_1(1)^n\vac$. The sum of all $1_{A\hilb{n}}$ in the formal completion of $\H$ is sometimes denoted by
$
\left|1\right> \defIs  \exp(\q_1(1))\vac.
$
\end{example}

\begin{definition}
To give the cup product structure of $\H$, define operators $\G(a)$ for $a \in H^*(A)$. Let $\Xi_n \subset A\hilb{n}\times A$ be the universal subscheme. Then the action of $\G(a)$ on $H^*(A\hilb{n})$ is multiplication with the class
$$
\pr_{1*}\left( \ch(\mathcal{O}_{\Xi_n})\cdot \pr_2^*(\td(A)\cdot a) \right) \in H^*(A\hilb{n}).
$$
For $a \in H^k(A)$, we define $\G_i(a)$ as the component of $\G(a)$ of cohomological degree $k+2i$. A differential operator $\mathfrak{d}$ is given by $\G_1(1)$. It means multiplication with the first Chern class of the tautological sheaf $\pr_{1*}\left( \mathcal{O}_{\Xi_n}\right)$.
\end{definition}
\begin{notation} 
We abbreviate $\q \defIs  \q_1(1)$ and for its derivative $\q'\defIs  [\mathfrak d, \q]$. For any operator $X$ we write $X^{(k)}$ for the $k$-fold derivative: $X^{(k)} \defIs \ad^k(\d) (X)$.
\end{notation}
In~\cite{LehnSorger} and~\cite{LiQinWang} we find various commutation relations between these operators, that allow to determine all multiplications in the cohomology of the Hilbert scheme. First of all, if $X$ and $Y$ are operators of degree $d$ and $d'$, their commutator is defined in the super sense: 
$$
[X,Y] \defIs  XY - (-1)^{dd'}YX.
$$
The integral on $A\hilb{n}$ induces a non-degenerate bilinar form on $\H$: for classes $\alpha,\,\beta\in H^*(A\hilb{n})$ it is given by
$$
(\alpha,\beta)_{A\hilb{n}} \defIs   \int_{A\hilb{n}}\alpha\cdot\beta.
$$
If $X$ is a homogeneous linear operator of degree $d$ and weight $m$, acting on $\H$, define its adjont $X^\dagger$ by
$$
(X(\alpha),\beta)_{A\hilb{n+m}}  = (-1)^{d|\alpha|}( \alpha , X^\dagger (\beta))_{A\hilb{n}}.
$$
We put $\q_0(a) \defIs 0$ and for $m<0$, $\q_m(a) \defIs  (-1)^m \q_{-m}(a)^\dagger$. Note that, for all $m\in\Z$, the bidegree of $\q_m(a)$ is $(m,|a| + 2(|m|-1))$. If $m$ is positive, $\q_m$ is called a creation operator, otherwise it is called annihiliation operator. Now define
$$
\mathfrak{L}_m(a) \defIs  \left\{ 
\begin{array}{ll}
 \tfrac{1}{2}\sum\limits_{k\in\Z}\sum\limits_{i}\q_k( a_{(1)})\q_{m-k}( a_{(2)}), & \text{ if } m\neq 0, \vspace{4mm}\\
 \sum\limits_{k>0}\sum\limits_{i}\q_k( a_{(1)})\q_{-k}( a_{(2)}), & \text{ if } m= 0. \\
\end{array}
\right.
$$
where $\sum_i a_{(1)}\otimes  a_{(2)}$ is the push-forward of $a$ along the diagonal $\tau_2 :A \rightarrow A\times A$ (in Sweedler notation).
\begin{remark}
This can be expressed more elegantly using normal ordering: the operator $\nor\q_m\q_n\mal(a\otimes b)$ is defined in a way such that the annihilation operator act first. Then we may write $\mathfrak{L}_m(a) = \sum_k
\nor \q_k \q_{m-k} \mal (\tau_{2*}(a))$.
\end{remark}
\begin{remark}
In a similar manner as above, we can use the integral over $A$ to define a bilinear form on $H^*(A,\Q)$. The adjoint of the multiplication map gives a coassiocative comultiplication
$$
\Delta : H^*(A,\Q) \longrightarrow H^*(A,\Q)\otimes H^*(A,\Q)
$$
that corresponds to $\tau_{2*}$. The sign convention in~\cite{LehnSorger} is such that $-\Delta =  \tau_{2*}$. We denote by $\Delta^k$ the $k$-fold composition of $\Delta$.
\end{remark}
\begin{lemma}\cite[Thm.~2.16]{LiQinWang} Denote $K_A\in H^2(A,\Q)$ the class of the canonical divisor. We have:
\label{commutators}
\begin{align}
[\q_m(a), \q_n(b)] &= m\cdot \delta_{m+n} \cdot \int_A ab \\
\label{qLcommute}
[\mathfrak{L}_m(a),\q_n(b)] &= -n\cdot \q_{m+n}(ab) \\
\label{DiffNaka}
[\mathfrak{d},\q_m(a)] &= m \cdot \mathfrak{L}_m(a) + \tfrac{m(|m|-1)}{2} \q_m(K_A a) \\
[\mathfrak{L}_m(a),\mathfrak{L}_n(b)] &= (m-n) \mathfrak{L}_{m+n}(ab) - \frac{m^3-m}{12} \delta_{m+n}\int_A abe \\
[\G(a),\q_1(b)] &= \exp(\ad(\mathfrak{d}))(\q_1(a b) ) \\
[\G_i(a),\q_1(b)] &= \tfrac{1}{k!} \ad(\mathfrak{d})^k(\q_1(a b) ) 
\end{align}
\end{lemma}
\begin{remark}\label{HRep}
Note (cf.~\cite[Thm.~3.8]{LehnSorger}) that (\ref{qLcommute}) together with (\ref{DiffNaka}) imply that 
\begin{equation}\label{NakaDel}
\q_{m+1}(a) = \tfrac{(-1)^m}{m!}(\ad\q')^m\left(\q_1(a)\right),
\end{equation}
so there are two ways of writing an element of $\H$: As a linear combination of products of creation operators $\q_m(a)$ or as a linear combination of products of the operators $\mathfrak{d}$ and $\q_1(a)$. This second representation is more suitable for computing cup-products, but not faithful. 
Equations (\ref{DiffNaka}) and (\ref{NakaDel}) permit now to switch between the two representations, using that
\begin{gather}
\mathfrak{d} \vac = 0 ,  \\
\mathfrak{L}_m(a) \vac = \left\{ 
\begin{array}{cl}
 \tfrac{1}{2}\sum\limits_{k=1}^{m-1}\sum\limits_{i}\q_k(a_{(1)})\q_{m-k}(a_{(2)})\vac, & \text{ if } m>1, \vspace{4mm}\\
 0, & \text{ if } m\leq 1. \\
\end{array}
\right.\\
\end{gather}
\end{remark}
Next we give some formulas involving higher derivatives of Nakajima operators that can be of use in formal computations.
\begin{proposition}
Suppose $K_Aa=0$. 
Denote $\mathbf e\defIs -\chi(A)x$ the Euler class of $A$.
Note that if $A$ is a torus, $\mathbf e=0$. 
For all $k,m$, the following formulas hold:
\begin{align}
\label{adq}
\ad \q  \frac{\q_m^{(k+1)}(a)}{m^{k+1}} & = (k+1)  \frac{\q_{m+1}^{(k)}(a)}{(m+1)^{k}} + \frac{k^3-k}{24}\frac{\q_{m+1}^{(k-2)}(a\mathbf e)}{(m+1)^{k-2}} , \\
\label{adq'}
\ad \q' \frac{\q_m^{(k)}(a)}{m^k} & = (k-m) \frac{\q_{m+1}^{(k)}(a)}{(m+1)^k}+ \frac{k(k-1)(k-3m-2)}{24}\frac{\q_{m+1}^{(k-2)}(a\mathbf e)}{(m+1)^{k-2}} .
\end{align}
%For all $s,t,k\geq 0$ the following formula holds:
\end{proposition}
\begin{proof}
Let us start with (\ref{adq}). This is a consequence of Theorem 4.2 of~\cite{LiQinWang2} which states that 
\begin{align*}
\frac{\q_m^{(k)}(a)}{m^k}
& = \frac{1}{k+1}\sum_{i_0+\ldots+i_k=m} \nor \q_{i_0}\cdots \q_{i_k} \mal  (\tau_*(a)) \\
& +  k \sum_{j_0+\ldots+j_{k-2}=m} \frac{j_0^2+\ldots+j_{k-2}^2-1}{24}\nor \q_{j_0}\cdots \q_{j_{k-2}} \mal  (\tau_*(a\mathbf e)) .
\end{align*}
Using that 
$[\q,\nor \q_{i_0}\cdots \q_{i_k} \mal  (\Delta^k(a)) ] = \sum_{r=0}^k \delta_{i_r+1} \nor\q_{i_0}\cdots \widehat{\q}_{i_r} \cdots \q_{i_k} \mal (\tau_*(a))$, 
we calculate:
{\allowdisplaybreaks
\begin{align*}
\ad\q \frac{\q_m^{(k+1)}(a)}{m^{k+1}} 
  =\, &\frac{1}{k+2}\sum_{i_0+\ldots+i_{k+1}=m}\!\! \left[\q,\nor \q_{i_0}\cdots \q_{i_{k+1}} \mal (\tau_*(a))\right] \\
 & + (k+1)\!\! \sum_{j_0+\ldots+j_{k-1}=m}\!\! \frac{j_0^2+\ldots+j_{k-1}^2-1}{24}\left[\q,\nor \q_{j_0}\cdots \q_{j_{k-1}} \mal (\tau_*(a\mathbf e)) \right]\\
  =\, & \sum_{i_0+\ldots+i_{k}=m+1} \!\!\nor \q_{i_0}\cdots \q_{i_{k}} \mal  (\tau_*(a)) \\
 & + k(k+1)\!\! \sum_{j_0+\ldots+j_{k-2}=m+1}\!\! \frac{j_0^2+\ldots+j_{k-2}^2}{24}\nor \q_{j_0}\cdots \q_{j_{k-2}} \mal (\tau_*(a\mathbf e)) \\
  =\, & (k+1)  \frac{\q_{m+1}^{(k)}(a)}{(m+1)^{k}} + \frac{k^3-k}{24}\frac{\q_{m+1}^{(k-2)}(a\mathbf e)}{(m+1)^{k-2}}.
\end{align*}
}
Equation (\ref{adq'}) follows from (\ref{adq}) using the Jacobi identity: $\ad \q' = \ad [\mathfrak d,\q] = \ad \mathfrak d  \ad \q - \ad \q\ad \mathfrak d  $.
\end{proof}
\begin{corollary} \label{adqCorollary}
Suppose $K_Aa=0$.
Iterated application of the above proposition gives
 \begin{equation}
 \ad(\q)^s \frac{\q_{m}^{(k+s)}(a)}{m^{k+s}(k+s)!} =\frac{\q_{m+s}^{(k)}(a)}{(m+s)^kk!} + \frac{s}{24}\frac{\q_{m+s}^{(k-2)}(a\mathbf e)}{(m+s)^{k-2}(k-2)!}.
\end{equation}
\end{corollary}
\begin{proposition}


Suppose $K_Aa=0$.
In the formal completion of $\H$ we have:
$$
\left[ \G(a),\exp(\q)\right] =
\exp(\q) \sum_{\substack{s\geq 1\\k\geq 0}}\frac{(-1)^{s-1}}{s!}\left(\frac{\q_s^{(k)}(a)}{s^{k}k!} + \frac{s-1}{24}\frac{\q_s^{(k)}(a\mathbf e)}{s^{k}k!}\right).
$$
%More explicitely, we have for the class $\G_r(a)1$:
%$$
%\G_r(a)\frac{1}{n!}\q^n\vac = 
%\sum_{\substack{s+k=r+1\\s>k}} \frac{(-1)^{s+1}}{k!\, s^k}\frac{1}{s!(n-s)!} \q_1(1)^{n-s} \mathfrak d^k \q_s(a)\vac.
%$$
\end{proposition}
\begin{proof}
Equation (4.6) of~\cite{LehnSorger} evaluates 
\begin{align*}
\left[ \G(a),\exp(\q)\right] &= \exp(\q) \sum_{\substack{s\geq 1\\k\geq 0}} \frac{(-\ad \q)^{s-1}}{s!}\left(\frac{(\ad \mathfrak d)^k}{k!} (\q_1(a)) \right) \\
 &\stackrel{\text{Cor~\ref{adqCorollary}}}{=} \exp(\q)  \sum_{s\geq 1}\frac{(-1)^{s-1}}{s!}\left(\sum_{k\geq s-1}\frac{\q_s^{(k-s+1)}(a)}{s^{k-s+1}(k-s+1)!}
  \right.\\&\hspace{140pt} \left. + \sum_{k\geq s+1}\frac{s-1}{24}\frac{\q_s^{(k-s-1)}(a)}{s^{k-s-1}(k-s-1)!} \right)\\
  &=\exp(\q) \sum_{\substack{s\geq 1\\k\geq 0}}\frac{(-1)^{s-1}}{s!}\left(\frac{\q_s^{(k)}(a)}{s^{k}k!} + \frac{s-1}{24}\frac{\q_s^{(k)}(a\mathbf e)}{s^{k}k!}\right).
\qedhere
\end{align*}
%\begin{gather*}
%\sum_{n,k}\G_k(a)\frac{\q_1(1)^n}{n!}\vac = \G(a)\exp(\q_1(1))\vac \\
%\stackrel{!}{=} \exp(\q_1(1)) \sum_{i,j}(-1)^{i+1} \frac{\mathfrak d^j}{j!}\frac{\q_i(1)}{i!i^j}\vac
%\end{gather*}
\end{proof}
\clearpage
\begin{example} 
\begin{align}
 \G_0(a)\q^n\vac & = n\cdot \q^{n-1}\q_1(a)\vac, \\
 \G_1(a)\q^n\vac & = -\binom{n}{2} \q^{n-2}\q_2(a)\vac, \\
 \G_2(a)\q^n\vac & = \binom{n}{3} \q^{n-3}\q_3(a)\vac - \binom{n}{2} \q^{n-2}\mathfrak{L}_2(a)\vac.
\end{align}
\end{example}

\begin{remark}
We adopted the notation from~\cite{LiQinWang}, which differs from the conventions in~\cite{LehnSorger}. Here is part of a dictionary:
\begin{center}
\begin{tabular}{c|c} 
Notation from~\cite{LiQinWang} & Notation from~\cite{LehnSorger} \\\hline
operator of weight $w$ and degree $d$ & operator of weight $w$ and degree $d-2w$\\
$\q_m(a) $ & $\p_{-m}(a)$ \\
$ \mathfrak{L}_m(a) $ & $ - L_{-m}(a)$ \\
$\mathfrak{G}(a)$ & $a\hilb{\bullet}$\\
$ \mathfrak{d} $ & $ \partial $ \\
$\tau_{2*}(a)$& $-\Delta(a)$
\end{tabular}
\end{center}
\end{remark}

By sending a subscheme in $A$ to its support, we define a morphism
\begin{equation}\label{HilbertChow}
\rho : A\hilb{n} \longrightarrow \Sym^n(A),
\end{equation}
called the Hilbert--Chow morphism. The cohomology of $\Sym^n(A)$ is given by elements of the $n$-fold tensor power of $H^*(A)$ that are invariant under the action of the group of permutations $\mathfrak{S}_n$. A class in $H^*(A\hilb{n},\Q)$ which can be written using only the operators $\q_1(a)$ of weight $1$ comes from a pullback along $\rho$:
\begin{equation}
\label{qSym}
\q_1(b_1)\cdots \q_1(b_n)\vac = \rho^*\left( \sum_{\pi\in\mathfrak{S}_n } \pm b_{\pi(1)}\otimes\ldots\otimes b_{\pi(n)} \right), \quad b_i\in H^*(A,\Q),
\end{equation}
where signs arise from permuting factors of odd degrees. In particular,
\begin{gather} \label{q0primitive}
\frac{1}{n!}\q_1(b)^n \vac = \rho^*\big( b\otimes \ldots \otimes b\big),
\\ \label{q1primitive}
\frac{1}{(n-1)!}\q_1(b)\q^{n-1}\vac =\rho^*\Big( b\otimes\! 1\!\otimes\ldots\otimes\! 1\; + \;\ldots\; +\; 1\!\otimes\ldots\otimes\! 1\! \otimes b\Big) .
\end{gather}
\begin{remark}
With the notation from Section~\ref{SuperSection}, we have that
$$
H^*(\Sym^n(A),\Q) \cong \SSym^n(H^*(A,\Q)).
$$
Under this isomorphism the ring structure of $\SSym^n(H^*(A,\Q))$ corresponds to the cup product and the action of the operator $\q_1(a)$ corresponds to the operation $a\,\diamond$.
%Note that these description implies that the unit in $H^*(A\hilb{n},\Q)$ is given by
%$$
%1_{A\hilb{n}} = \frac{1}{n!} \q_1(1)^n\vac.
%$$
\end{remark}

%\begin{lemma}\label{maxPower}
%Assume $b\in H^2(A,\Q)$ and set $B_n\defIs \frac{1}{(n-1)!}\q_1(b)\q^{n-1}\vac =\G_0(b)1$. With the notion of double factorial $(2n-1)!! \defIs  \frac{(2n)!}{2^n n!}$ we have
%$$
%\left(B_n\right)^{2n} = (2n-1)!! \q_1(b^2)^n\vac.
%$$
%\end{lemma}
%\begin{proof}
%Multiplication with the class $\frac{1}{(n-1)!}\q_1(b)\q^{n-1}\vac$ is given by the operator $\G_0(b)$. Set $\alpha_{i,j} \defIs  \q^i\q_1(b)^j\q_1(b^2)^{n-i-j}\vac$. Then we have with Lemma~\ref{commutators}:
%\begin{equation}
%\G_0(b) \alpha_{i,j} = i\cdot\alpha_{i-1,j+1} + j\cdot \alpha_{i,j-1}.
%\end{equation}
%By an inductive argument one shows now that
%$$
%\G_0(b)^r \alpha_{n,0} = n! \sum_{\substack{i+j+k=n\\j+2k=r}}   Factor \cdot \alpha_{i,j}  
%$$
%Looking at (\ref{q0primitive}) and (\ref{q1primitive}), observe that 
%$$
%(B_n)^k = \rho^*\left(\sum_{i_1+\ldots+i_n= k}b^{i_1}\otimes\cdots\otimes b^{i_n}\right).
%$$
%Since the case $k=2n$ implies that all the $i_r$ are equal to $2$, the number of non-zero summands equals the number of partitions of a set of cardinality $2n$ into pairs. By~\cite[Prop.~2,4]{Kapfer}, this is $(2n-1)!!$.
%\end{proof}


\section{On integral cohomology of Hilbert schemes}\label{basisHilb2}

For the study of integral cohomology, first note that if $a \in H^*(A,\Z)$ is an integral class, then $\q_{m}(a) $ maps integral classes to integral classes. Such operators are called integral. Qin and Wang studied them in~\cite{QinWang}. We need the following results:

\begin{lemma} (\cite{QinWang}\label{IntegralOperators}, see also Thm.~\ref{QinWangTheorem}).
The operators $\frac{1}{n!}\q_1(1)^n$ and $\frac{1}{2}\q_2(1) $ are integral.
Let $b\in H^2(A,\Z)$ be monodromy equivalent to a divisor. Then the operator $\frac{1}{2}\q_1(b)^2 - \frac{1}{2}\q_2(b)$ is integral. 
\end{lemma}
\begin{remark}
Qin and Wang~\cite[Thm 1.1 et seq.]{QinWang} conjecture that their theory works even without the restriction on $b\in H^2(A,\Z)$. 
\end{remark}

\begin{corollary} \label{IntegralOperatorsTorus}
If $A$ is a torus, the operator $\frac{1}{2}\q_1(b)^2 - \frac{1}{2}\q_2(b)$ is integral for all $b\in H^2(A,\Z)$. 
\end{corollary}
\begin{proof}
The Nakajima operators are preserved under deformations of $A$. 
Moreover, by %\cite[Thm.~II]{Shioda},
\cite{Borcea}, 
the image of the monodromy representation on $H^2(A,\Z)$ is given by $O^{+,+}(H^2(A,\Z))$, the group of isometry on $H^2(A,\Z)$ which preserve the orientation of the negative and positive definite part of $H^2(A,\R)$.
%Since the lattice is even and contains two copies of the hyperbolic lattice, a theorem of Eichler~\cite[Prop.~3.7.3]{Scattone} states that the automorphism group of $H^2(A,\Z)$ acts transitively on classes of the same norm. 

%%The monodromy operators induce the entire automorphism group $SL(4,\Z)$ on $H^1(A,\Z)$.
%%Since $H^2(A,\Z)= \Lambda^2H^1(A,\Z)$, we see with a similar argument as in Remark~\ref{simplePlanes} that the monodromy operators act transitively on the simple tensors of $\Lambda^2H^1(A,\Z)$ with fixed norm.
%%Suppose that the N\'eron-Severi group $\NS(A)$ contains a basis element of $H^2(A,\Z)$ which can be written as a simple tensor (such $A$ exist).
%%Then, by action of monodromy operators, we see that the claim holds for all simple tensors.
%%An arbitrary element in $H^2(A,\Z)$ is linear combination of simple tensors. Since the integrality of our operator is preserved under linear combinations, the corollary is proved for that particular torus $A$.

Suppose now that the N\'eron-Severi group $\NS(A)$ contains a copy of the hyperbolic lattice $U$ (such $A$ exist).
%Since $U\subset \NS(A)$, there are divisors of arbitrary even norm,
%so every class can be mapped to a divisor by the action of a monodromy 
Let us denote $H^2(A,\Z)=U_1\oplus U_2\oplus U_3$ with $\NS(A)=U_1$.
We consider two isometries in $O^{+,+}(H^2(A,\Z))$, $\varphi_2$ and $\varphi_3$, defined in the following way:
$\varphi_2$ exchanges $U_1$ and $U_2$ and acts as $-\id$ on $U_3$ and $\varphi_3$ exchanges $U_1$ and $U_3$ and acts as $-\id$ on $U_2$.
Using these two isometries, all elements of $U_2$ and $U_3$ are monodromy equivalent to a divisor.
Then Lemma~\ref{IntegralOperators} establishes the corollary for that particular $A$. 
Now, since all tori are equivalent by deformation, a general torus can always be deformed to our special $A$. Since the integrality of an operator is a topological invariant, $\frac{1}{2}\q_1(b)^2 - \frac{1}{2}\q_2(b)$ remains integral for all $b\in H^2(A,\Z)$.
\end{proof}

\begin{proposition} Assume that $H^*(A,\Z)$ is free of torsion.
Let $(a_i) \subset H^1(A,\Z)$ and $(b_i)\subset H^2(A,\Z)$ be bases of integral cohomology as in Notation~\ref{TorusClasses}. Denote $a_i^*\in H^3(A,\Z)$ resp.~$b_i^*\in H^2(A,\Z)$ the elements of the dual bases. Let $x$ be the generator of $H^4(A,\Z)$. Modulo torsion, the following classes form a basis of $H^2(A\hilb{n},\Z)$:
\begin{itemize}
 \item $\frac{1}{(n-1)!}\q_{1}(b_{i})\q_{1}(1)^{n-1}\vac = \G_0(b_i) 1$,
 \item $ \frac{1}{(n-2)!}\q_{1}(a_{i})\q_{1}(a_{j})\q_{1}(1)^{n-2}\vac = \G_0(a_i) \G_0(a_j)1,\  i < j$, 
 \item $ \frac{1}{2(n-2)!}\q_{2}(1) \q_{1}(1)^{n-2}\vac$. We denote this class by $\delta = \d 1$.
\end{itemize}
Their respective duals in $H^{2n-2}(A\hilb{n},\Z)$ are given by
\begin{itemize}
 \item $\q_{1}(b_{i}^*)\q_{1}(x)^{n-1}\vac$,
 \item $\q_{1}(a_{j}^*)\q_{1}(a_{i}^*)\q_{1}(x)^{n-2}\vac,\  i < j$,
 \item $\q_2(x)\q_{1}(x)^{n-2} \vac$.
\end{itemize}
\end{proposition}
\begin{proof} It is clear from the above lemma that these classes are all integral.
G\"ottsche's formula~\cite[p.~35]{Gottsche} gives the Betti numbers of $A\hilb{n}$ in terms of the Betti numbers of $A$: 
$h^1(A\hilb{n}) = h^1(A)$, and $h^2(A\hilb{n}) = h^2(A)+ \frac{h^1(A)(h^1(A)-1)}{2} + 1$. It follows that the given classes span a lattice of full rank.

Next we have to show that the intersection matrix between these classes is in fact the identity matrix. Most of the entries can be computed easily using the simplification from (\ref{qSym}). For products involving $\delta$ (this is the action of $\mathfrak{d}$) or its dual, first observe that $\mathfrak{d}\q_1(x)^m\vac = 0 $ and $ \mathfrak{L}_1(a)\q_1(x)^m\vac =0$ for every class $a$ of degree at least 1. Then compute:
\begin{gather*}
\delta \cdot\q_2(x)\q_{1}(x)^{n-2} \vac = \mathfrak{d}\q_2(x)\q_{1}(x)^{n-2} \vac = 2 \mathfrak{L}_2(x) \q_{1}(x)^{n-2} \vac = \q_{1}(x)^{n}\vac,
\\
\mathfrak{d}\q_{1}(b_{i}^*)\q_{1}(x)^{n-1}\vac =  \mathfrak{L}_1(b_i^*) \q_{1}(x)^{n-1} \vac = 0,
\\
\mathfrak{d}\q_{1}(a_{j}^*)\q_{1}(a_{i}^*)\q_{1}(x)^{n-2}\vac = \left(\mathfrak{L}_1(a_j^*) +\q_{1}(a_{j}^*)\mathfrak{d}\right)\q_{1}(a_{i}^*)\q_{1}(x)^{n-2}\vac = 
  \\ =\left(-\q_1(a_i^*)\mathfrak{L}_1(a_j^*) + \q_{1}(a_{j}^*)\mathfrak{L}_1(a_i^*)\right)\q_{1}(x)^{n-2}\vac  = 0,
\\
\G_0(b_i)\q_2(x)\q_{1}(x)^{n-2} \vac = 0, 
\\
\G_0(a_i)\G_0(a_j)\q_2(x)\q_{1}(x)^{n-2} \vac = 0.
\qedhere
\end{gather*}
\end{proof}

\begin{remark}
If $A$ is a complex torus, a theorem of Markman~\cite{Markman} ensures that $H^*(A\hilb{n},\Z)$ is free of torsion.
\end{remark} 

\begin{proposition} \label{A2Basis}
Let $A$ be a complex abelian surface. Using Notation~\ref{TorusClasses}, a basis of $H^*(A\hilb{2},\Z)$ is given by the following classes.
\begin{center}
\begin{tabular}{c|c|l|l}
 degree & Betti number & class & multiplication with class \\\hline
 0 & 1 & $\frac{1}{2}\q_1(1)^2\vac$ & $\id$ \\ \hline
 1 & 4 &  $\q_1(1)\q_1(a_i)\vac$ & $\G_0(a_i)$ \\ \hline
 2 & 13 & $\frac{1}{2}\q_2(1)\vac$ & $\d$ \\ 
   &  & $\q_1(a_i)\q_1(a_j)\vac$ for $i<j$ & $\G_0(a_i)\G_0(a_j)$ \\
   &  & $\q_1(1)\q_1(b_i)\vac$ & $\G_0(b_i)$ \\\hline
 3 & 32 & $\q_2(a_i)\vac$  & $-2\G_1(a_i) $ \\
   &  & $\q_1(a_i)\q_1(b_j)\vac$ & $\G_0(a_i)\G_0(b_j)$ \\ 
   &  & $\q_1(1)\q_1(a^*_i)\vac$ & $\G_0(a^*_i)$ \\\hline
 4 & 44 & $\left(\frac{1}{2}\q_1(b_i)^2-\frac{1}{2}\q_2(b_i)\right)\vac$ & $\frac{1}{2} \G_0(b_i)^2 + \G_1(b_i) $ \\
   &  & $\q_1(a_i)\q_1(a^*_j)\vac$ & $\G_0(a_i)\G_0(a^*_j)$ \\
   &  & $ \q_1(b_i)\q_1(b_j)\vac$ for $i\leq j$ &  $\G_0(b_i)\G_0(b_j)$ \\\hline
 5 & 32 & $\frac{1}{2}\q_2(a^*_i)\vac$ & $-\G_1(a^*_i)$ \\
   &  & $\q_1(a^*_i)\q_1(b_j)\vac$ & $ \G_0(a^*_i)\G_0(b_j)$ \\
   &  & $\q_1(a_i)\q_1(x)\vac$ & $\G_0(a_i)\G_0(x)$ \\\hline
 6 & 13 & $\q_2(x)\vac$ & $-2\G_1(x)$ \\
   &  & $\q_1(a^*_i)\q_1(a^*_j)\vac$ for $i<j$ & $\G_0(a^*_i)\G_0(a^*_j)$ \\
   &  & $\q_1(b_i)\q_1(x)\vac$ & $ \G_0(b_i)\G_0(x)$ \\\hline
 7 & 4 & $\q_1(a^*_i)\q_1(x)\vac$ & $\G_0(a^*_i)\G_0(x) $ \\\hline
 8 & 1 & $\q_1(x)^2\vac$ & $\G_0(x)^2$ 
\end{tabular}
\end{center}
\begin{proof}
The Betti numbers come from G\"ottsche's formula~\cite{Gottsche}.
One computes the intersection matrix of all classes under the Poincar\'e duality pairing and finds that it is unimodular. 
So it remains to show that all these classes are integral. By Lemma~\ref{IntegralOperators} this is clear for all classes except 
those of the form $\frac{1}{2}\q_2(a^*_i)\vac \in H^5(A\hilb{2},\Z)$.

Evaluating the Poincar\'e duality pairing between degrees 3 and 5 gives:
\begin{gather*}
 \q_2(a_i)\vac \cdot \q_2(a^*_i)\vac = 2, \\
 \q_1(a_i)\q_1(b_j)\vac \cdot  \q_1(a^*_i)\q_1(b^*_j)\vac = 1, \\
 \q_1(1)\q_1(a^*_i)\vac \cdot \q_1(x)\q_1(a_i)\vac = 1,
\end{gather*}
while the other pairings vanish. Therefore, one of $\q_2(a_i)\vac$ and $\q_2(a^*_i)\vac$ must be divisible by $2$. 
With the considerations from Section~\ref{OddHilb2} in mind, we can interpret $\q_2(a_i)\vac\in H^3(A\hilb{2},\Z)$ and $\q_2(a^*_i)\vac\in H^5(A\hilb{2},\Z)$ as classes concentrated on the exceptional divisor, that is, as elements of $\pi_* j_*H^*(E,\Z)$. Indeed,
the pushforward of a class $a\otimes 1 \in H^{k}(E,\Z)$ is given by 
$$
\pi_* j_*(a\otimes 1) = \q_2(a)\vac \in H^{k+2}(A\hilb{n},\Z).
$$
When pushing forward to the Hilbert scheme, the only possibility to get a factor $2$ is in degree $5$, by Proposition~\ref{Alpha35}. 
\end{proof}

\end{proposition}

\label{integralcohomology}
\section{Generalized Kummer varieties and the morphism to the Hilbert scheme}
\label{Section_GeneralKummer}
\begin{definition}
Let $A$ be a complex projective torus of dimension $2$ and $A\hilb{n}$, $n\geq 1$, the corresponding Hilbert scheme of points. Denote $\Sigma : A\hilb{n} \rightarrow A$ the summation morphism, a smooth submersion that factorizes via (\ref{HilbertChow}) the Hilbert--Chow morphism $: A\hilb{n}\stackrel{\rho}{\rightarrow}\Sym^n(A)\stackrel{\sigma}{\rightarrow} A$. Then the generalized Kummer variety $\kum{A}{n-1}$ is defined as the fiber over $0$:
\begin{equation}\label{square}
\begin{CD}
\kum{A}{n-1} @>\theta >> A\hilb{n}\\
@VVV @VV\Sigma V\\
\{0\} @> >> A
\end{CD}
\end{equation}
\end{definition}
\begin{theorem} \cite[Theorem 2]{Spanier}\label{torsion}
The cohomology of the generalized Kummer, $H^*(\kum{A}{n-1},\Z)$, is torsion free. 
\end{theorem}
Our first objective is to collect some information about this pullback diagram. 
We use Notation \ref{TorusClasses}.

\begin{proposition}\label{KummerClass}
Let $\alpha_i := \frac{1}{(n-1)!}\kq_{1}(1)^{n-1}\kq_1(a_i)\vac = \G_0(a_i)1$. The class of %the Poincar\'e dual of 
$\kum{A}{n-1}$ in $H^4(A\hilb{n},\Z)$ is given by
$$
%\prod_{i=1}^4 \left(\tfrac{1}{2}\pone(1)^2\pone(\alpha_i)\vac\right).
[\kum{A}{n-1}]=\alpha_1\cdot\alpha_2\cdot\alpha_3\cdot\alpha_4.
$$ 
\end{proposition}
\begin{proof}
Since the generalized Kummer variety is the fiber over a point, its 
%Poincar\'e dual 
class must be the pullback of $x\in H^4(A)$ under $\Sigma$. But $\Sigma^* (x) = \Sigma^*(a_1)\cdot \Sigma^*(a_2)\cdot \Sigma^*(a_3)\cdot \Sigma^*(a_4)$, so we have to verify that $\Sigma^* (a_i) = \alpha_i$. To do this, we want to use the decomposition $\Sigma = \sigma\rho$.
The pullback along $\sigma$ of a class $a\in H^1(A,\Q)$ on $H^1(\Sym^n(A),\Q)$ 
%$\cong H^1(A^n,\Q)^{\mathfrak{S}_n}$ 
is given by $a\otimes 1\otimes \cdots\otimes 1 + \ldots + 1\otimes \cdots\otimes 1\otimes a$. It follows from (\ref{q1primitive}) that $\Sigma^* (a_i) = \frac{1}{(n-1)!}\kq_{1}(1)^{n-1}\kq_1(a_i)\vac $.
\end{proof}
The morphism $\theta$ induces a homomorphism of graded rings
\begin{equation}
\theta^* :H^*(A\hilb{n})\longrightarrow H^*(\kum{A}{n-1})
\end{equation}
and by the projection formula, we have
\begin{equation}
\theta_*\theta^*(\alpha)  = [\kum{A}{n-1}]\cdot\alpha.
\end{equation}

\begin{lemma}\label{petitlemmeenplus}
 Let $\beta\in H^*(K_{n-1}(A),\Q)$. Then there is a class $B\in H^{*}(A\hilb{n},\Q)$ such that 
 $$\theta_*(\beta)=\frac{1}{n^4}B\cdot [\kum{A}{n-1}].$$
\end{lemma}
\begin{proof}
For a point $a\in A$, we denote by $t_a$ the morphism on $A\hilb{n}$ induced by the translation by $a$.
Then we consider the morphism $\Theta :\kum{A}{n-1}\times A \longrightarrow A\hilb{n}$ defined by $\Theta(\xi,a)=t_a(\theta(\xi))$. It fits in a pullback diagram
\begin{equation}
\begin{CD}
\kum{A}{n-1}\times A @>\Theta >> A\hilb{n}\\
@VV\pr_2V @VV\Sigma V\\
A @> n\cdot >> A
\end{CD}
\end{equation}
that realizes $\kum{A}{n-1}\times A$ as a $n^4$-fold covering of $A\hilb{n}$ over $A$.
Now, for $\beta\in H^*(K_{n-1}(A),\Q)$ set
$$
B:=\Theta_*(\beta\otimes 1).
$$
Then the projection formula gives
\begin{align*}
B\cdot [K_{n-1}(A)]&= \Theta_*\left(\beta\otimes 1\cdot \Theta^*[\kum{A}{n-1}]\right) \\
&=n^4 \Theta_*\left((\beta\otimes 1)\cdot  (1\otimes x)\right)\\
&=n^4 \Theta_*(\beta \otimes x)\\
&=n^4\theta_*(\beta).
\end{align*}
\end{proof}

\begin{proposition}\label{annihilator}
The kernel of $\theta^*$ is equal to the annihilator of $[\kum{A}{n-1}]$.
\end{proposition}
\begin{proof}
Assume $\alpha\in \ker(\theta^*)$. Then we have
$
[\kum{A}{n-1}]\cdot \alpha = \theta_*\theta^*(\alpha) = 0
$. 
Consersely, if $\alpha\notin \ker(\theta^*)$,
let $\beta\in H^*(\kum{A}{n-1},\Q)$ be the Poincar\'e dual of $\theta^*(\alpha)$, so $\beta\cdot \theta^*(\alpha)\neq 0$.
Then by projection formula:
$
\theta_*(\beta)\cdot \alpha\neq 0.
$
By Lemma \ref{petitlemmeenplus}, there exists $B\in H^*(A\hilb{n},\Q)$ such that 
$B\cdot [\kum{A}{n-1}]\cdot \alpha\neq 0$. It follows that $ [\kum{A}{n-1}]\cdot \alpha\neq 0$.
\end{proof}

\begin{corollary} \label{KummerEquality}
$\theta^*(\alpha) = \theta^*(\beta)$ if and only if $[\kum{A}{n-1}]\cdot \alpha = [\kum{A}{n-1}]\cdot \beta$. 
\qed
\end{corollary}

\begin{proposition}\label{Annihideal}
The annihilator of $[\kum{A}{n-1}]$ in $H^*(A\hilb{n},\Q)$ is the ideal generated by $H^1(A\hilb{n})$. 
\end{proposition}
\begin{proof}
Set $H=H^*(A,\Q)$ and consider the exact sequence of $H$-modules
$$
0 \longrightarrow 
%H^{\geq 1}(A,\Q)  
J
\longrightarrow H \stackrel{x\cdot}{\longrightarrow} H.
$$
It is clear that $J$ is the ideal in $H$ generated by $H^{1}(A,\Q)$. 
Now denote $J^{(n)}$ the ideal generated by $H^1(\Sym^n(A),\Q)$ in $H^*(\Sym^n(A),\Q)\cong\SSym^n(H)$.
By the freeness result of Lemma \ref{SuperFree}, tensoring with $\SSym^n(H)$ yields another exact sequence of $H$-modules
$$
0 \longrightarrow {J}^{(n)} \longrightarrow \SSym^n(H) \xrightarrow{\sigma(x)\cdot} \SSym^n(H).
$$
Now let $\mathfrak{H}$ be the operator algebra spanned by products of $\mathfrak d$ and $\q_1(a)$ for $a\in H^*(A)$. Let $\mathfrak C$ be the graded commutative subalgebra of $\mathfrak H$ generated by $\q_1(a)$ for $a\in H^*(A)$. The action of $\mathfrak H$ on $\vac$ gives $\H$ and the action of $\mathfrak C$ on $\vac$ gives $\rho^*(H^*(\Sym^n(A),\Q))\cong \SSym^n(H)$.
By sending $\mathfrak d$ to the identity, we define a linear map $c : \mathfrak H \rightarrow \mathfrak C$. 
Denote $J\hilb{n}$ the ideal generated by $H^1(A\hilb{n},\Q)$ in $H^*(A \hilb n,\Q)$. We claim that for every $\mathfrak y\in \mathfrak H$:
$$
\mathfrak y\vac \in J\hilb{n} \Leftrightarrow c(\mathfrak y)\vac \in J\hilb{n}.
$$
To see this, we remark that $H^1(A \hilb n,\Q) \cong H^1(A ,\Q)  $ and the multiplication with a class in $H^1(A \hilb n,\Q) $ is given by the operator $\mathfrak G_0(a)$ for some $a\in H^1(A ,\Q)$. Due to the fact that $\mathfrak d$ is also a multiplication operator (of degree 2), $\mathfrak G_0(a)$ commutes with $\mathfrak d$. It follows that for $\mathfrak y =\mathfrak G_0(a) \mathfrak r$ we have $c(\mathfrak y) = \mathfrak G_0(a) c(\mathfrak r)$.

Now denote $\mathfrak k$ the multiplication operator with the class $[\kum{A}{n-1}]$. We have:
$
[\mathfrak k, \mathfrak d] = 0.
$
Now let $y\in H^*(A\hilb{n},\Q)$ be a class in the annihilator of $[\kum{A}{n-1}]$. We can write $y= \mathfrak y\vac$ for a $\mathfrak y\in\mathfrak H$. Choose $\tilde y \in \SSym^n (H)$ in a way that $\rho^*(\tilde y) = c(\mathfrak y) \vac$. Then we have:
$$
0=[\kum{A}{n-1}]\cdot y = \mathfrak k\, \mathfrak y \vac =  \mathfrak k \,c(\mathfrak y)\vac = \rho^*(\sigma^*(x) \cdot \tilde y).
$$
Since $\rho^*$ is injective, $\tilde y$ is in the annihilator of $\sigma^*(x)$, so $\tilde y \in J^{(n)}$. It follows that $c(\mathfrak y)\vac$ and $y$ are in the ideal generated by $H^1(A\hilb{n},\Q)$.
\end{proof}

\begin{theorem}\cite[Th\'eor\`eme 4]{Beauville}
$\kum{A}{n-1}$ is a irreducible holomorphically symplectic manifold. In particular, it is simply connected and the canonical bundle is trivial.
\end{theorem}
This implies that $H^2(\kum{A}{n-1},\Z)$ admits an integer-valued nondegenerated symmetric bilinear form (called Beauville--Bogomolov form) $B$ which gives $H^2(\kum{A}{n-1},\Z)$ the structure of a lattice. Looking for instance, the useful table in the introduction of \cite{Rapagnetta}, we know that this lattice is
isomorphic to $U^{\oplus 3}\oplus \left< -2n \right>$, for $n\geq 3$. 
We have the Fujiki formula for $\alpha\in H^2(\kum{A}{n-1},\Z)$:
\begin{equation} \label{fujiki}
%\int_{\kum{A}{n-1}} \alpha^{2n-2} = n\frac{(2n-2)!}{2^{n-1}(n\! - \! 1)!} q(a)^{n-1}
\int_{\kum{A}{n-1}} \alpha^{2n-2} = n\cdot(2n-3)!!\cdot B(\alpha,\alpha)^{n-1}
\end{equation}

\begin{proposition}\label{H2Sur} Assume $n\geq 3$. Then
$\theta^*$ is surjective on $H^2(A\hilb{n},\Z)$.
\end{proposition}
\begin{proof}
By \cite[Sect.~7]{Beauville}, $\theta^{\ast} : H^2(A\hilb{n},\C) \rightarrow H^2(\kum{A}{n-1},\C)$ is surjective. 
But by Proposition 1 of \cite{Britze}, the lattice structure of $\im \theta^*$ is the same as of $H^2(\kum{A}{n-1})$, so the image of $H^2(A\hilb{n},\Z)$ must be primitive. The result follows.
%We use two formulas in \cite[pp.~8--11]{Britze}. Let $b\in H^2(A,\Z)$ and set $\alpha = \frac{1}{(n-1)!}\kq_{1}(1)^{n-1}\kq_1(b)\vac\in H^2(A\hilb{n},\Z)$. Then  
%\begin{equation} 
%\int_{A\hilb{n}}\alpha^{2n} = \binom{2n}{2} \frac{\int_A b^2}{n^2} \int_{\kum{A}{n-1}} \theta^* \alpha^{2n-2}
%\int_{A\hilb{3}}j(a)^6 = \frac{5}{3} \int_A a^2 \int_{\kum{A}{2}} \theta^* j(a)^4
%\end{equation}
%By Lemma \ref{maxPower}, the left hand side of this equation equals $(2n-1)!!\cdot \left(\int_a b^2\right)^n$. By (\ref{fujiki}), the right hand side gives $(2n-1)!! \cdot \left(\int_A b^2\right) \cdot B(\alpha,\alpha)^{n-1}$. So
%we get
%$\int_A b^2 = B(\alpha,\alpha)$, giving the set of all $\alpha$ a lattice structure isomorphic to $H^2(A,\Z)$. 
%Secondly, we must show that for $e:=\theta^*\delta$ we get: $B(e,e) = -2n$. But this follows now from Proposition 1 in \cite{Britze}.
%Remark: $\theta^*\delta$ seems to be indivisible (because of (\ref{fujiki})), but every product with $\theta^*\delta$ is divisible by 3. Indeed, the value of (\ref{fujiki}) for $\alpha=\theta^*\delta$ is 324.
\end{proof}
\begin{notation}\label{BasisH2KA}
 We have seen that, for $n\geq 3$,
 $$
 H^2(\kum{A}{n-1},\Z) \cong H^2(A,\Z) \oplus\left<\theta^*(\delta)\right>.
 $$
We denote the injection $ : H^2(A,\Z) \rightarrow H^2(\kum{A}{n-1},\Z)$ by $j$. It can be described by 
$$
j : a \longmapsto \frac{1}{(n-1)!}\theta^*\left(\q_1(a)\q_1(1)^{n-1}\vac\right).
$$ 
Further, we set $e:=\theta^*(\delta)$. Using Notation \ref{TorusClasses}, we give the following names for classes in $H^2(\kum{A}{n-1},\Z)$:
\begin{align*}
u_1 &:= j(a_1 a_2), & v_1 &:= j(a_1 a_3), & w_1 &:= j(a_1 a_4), \\ 
u_2 &:= j(a_3 a_4), & v_2 &:= j(a_4 a_2), & w_2 &:= j(a_2 a_3),
\end{align*}
These elements form a basis of $H^2(\kum{A}{n-1},\Z)$ with the following intersection relations under the Beauville-Bogomolov form:
\begin{align*}
B(u_1,u_2) &= 1, & B(v_1,v_2) &= 1, & B(w_1,w_2) &= 1,  &
B(e,e)&= -2n,
\end{align*}
and all other pairs of basis elements are orthogonal.
\end{notation}


\section{Odd Cohomology of the Generalized Kummer fourfold}
Now we come to the special case $n=3$, so we study $\kum{A}{2}$, the Generalized Kummer fourfolds.
\begin{proposition}
The Betti numbers of $\kum{A}{2}$ are:
$
1,\,0,\,7,\,8,\,108,\,8,\,7,\,0,\,1.
$
\end{proposition}
\begin{proof}
This follows from G\"ottsche's formula \cite[page 49]{Gottsche}.
\end{proof}

By means of the morphism $\theta^*$, we may express part of the cohomology of $\kum{A}{2}$ in terms of Hilbert scheme cohomology. We have seen in Proposition \ref{H2Sur} that $\theta^*$ is surjective for degree $2$ and (by duality) also in degree $6$. 
The next proposition shows that this also holds true for odd degrees.
\begin{proposition}\label{oddcohomology}
A basis of $H^3(\X,\Z)$ is given by:
\begin{gather}
\label{A3_1}
\frac{1}{2}\theta^*\Big( \q_1(a^*_i)\q_1(1)^2\vac \Big), \\
\label{A3_2}
\theta^*\Big(\q_2(a_i)\q_1(1)\vac\Big).
\end{gather}
and a basis of $H^5(\X,\Z)$ is given by:
\begin{gather}
\label{A5_1}
\frac{2}{3} \theta^*\Big( \G_2(a_i) \Big), \\
\label{A5_2}
\frac{1}{2}\theta^*\Big( \q_2(a^*_i)\q_1(1)\vac \Big).
\end{gather}
\end{proposition}
\begin{proof}
The classes (\ref{A3_1}) are Poincar\'e dual to (\ref{A5_1}) and the classes (\ref{A3_2}) are Poincar\'e dual to (\ref{A5_2}) by computation, so it remains to show that all of them are integral.

It is clear from Theorem \ref{IntegralOperators} that (\ref{A3_1}) and (\ref{A3_2}) are integral. By Proposition \ref{A2Basis}, $\frac{1}{2}\q_2(a^*_i)\vac$ is integral as well. If the operator $ \q_1(1)$ is applied, we get again an integral class.

Further, $\frac{2}{3} \G_2(a_i)[\kum{A}{2}]$ is an integral class and $[\kum{A}{2}]\cdot\q_1(a^*_i)\q_1(1)^2\vac$ is non-divisible. 
Because of $\int\limits_{A\hilb{3}} [\kum{A}{2}]\cdot \G_2(a_i)  \q_1(a^*_i)\q_1(1)^2\vac = 3$, one of $\frac{1}{2}\theta^*\left( \q_1(a^*_i)\q_1(1)^2\vac \right)$ and $ 2\theta^*\left( \G_2(a_i) \right)$ is divisible by three. In view of Lemma \ref{IntegralityCheck}, it must be the latter one.
\end{proof}
It follows the following corollary which will be used in Part \ref{quotient}.
%\textcolor{green}{Corollary \ref{actionH3} to put after Proposition 3.18 \ref{} saying that it will be use in Part 2}
\begin{cor}\label{actionH3}
Let $A$ be an abelian surface and $g$ be an automorphisms on $A$. Let $g^{[[3]]}$ be the automorphisms induced by $g$ on $K_2(A)$.
We have $H^3(K_2(A),\Z)\simeq H^1(A,\Z)\oplus H^3(A,\Z)$ and the action of $g^{[[3]]}$ on $H^3(K_2(A),\Z)$ is given by the action of $g$ on $H^1(A,\Z)\oplus H^3(A,\Z)$.
\end{cor}
\begin{proof}
Let $g^{[3]}$ be the involution on $A^{[3]}$ induced by $g$.
We have $g^{[3]*}(a_{i}^{(1)})=(g^{*}a_{i})^{(1)}$ and $g^{[3]*}(a_{\overline{i}}^{(0)})=(g^{*}a_{\overline{i}})^{(0)}$.
Moreover, we have by definition, $g^{[[3]]*}\circ \theta^{*}=\theta^{*}\circ g^{[3]*}$.
The result follows from Proposition \ref{oddcohomology}.
\end{proof}

Let us summarize our results on $\theta^*$:
\begin{theorem}\label{thetaTheorem}
The homomorphism $\theta^* : H^*(A\hilb{3},\Q)\rightarrow H^*(\kum{A}{2},\Q)$ of graded rings is surjective in every degree except $4$. Moreover, the image of $H^4(A\hilb{3},\Q)$ is equal to $\Sym^2(H^2(\kum{A}{2},\Q))$. 
The kernel of $\theta^*$ is the ideal generated by $H^1(A\hilb{3},\Q)$.
%Proof: look at the ranks of H^*(A\hilb{3}):
%Rank H1*H4 = 188
%Rank H1*H5 = 239
%Rank H1*H6 = 196
%Rank H1*H7 = 102
%Rank H1*H8 = 40
\end{theorem}

 \section{Middle cohomology}\label{Middle}
The middle cohomology $H^4(\X,\Z)$ has been studied by Hassett and Tschinkel in \cite{Hassett}. We first recall some of their results,
then we proceed by using $\theta^*$ to give a partial description of $H^4(\X,\Z)$ in terms of the well-understood cohomology of $A\hilb{3}$. 
Finally, we find a basis of $H^4(\X,\Z)$ using the action of the monodromy group.
\begin{notation}
For each $\tau \in A$, denote $W_\tau$ the Brian\c con subscheme of $A\hilb{3}$ supported enitrely at the point $\tau$. If $\tau\in A[3]$ is a point of three-torsion, $W_\tau$ is actually a subscheme of $\X$. We will also use the symbol $W_\tau$ for the corresponding class in $H^4(\X,\Z)$. Further, set 
$$
W := \sum_{\tau\in A[3]} W_\tau.
$$
For $p\in A$, denote $Y_p$ the locus of all $\{x,y,p\}$ in $\X$. The corresponding class $Y_p \in H^4(\X,\Z)$ is independent of the choice of the point $p$. Then set $Z_\tau := Y_p - W_\tau$ and denote $\Pi$ the lattice generated by all $Z_\tau$, $\tau \in A[3]$.
\end{notation}
\begin{proposition}
Denote by $\Sym := \Sym^2\left(H^2\left(\X,\Z\right)\right) \subset H^4\left(\X,\Z\right) $ the span of products of integral classes in degree $2$.
Then 
$$
\Sym+\Pi \ \subset\  H^4\left(\X,\Z\right)
$$
is a sublattice  of full rank.  
\end{proposition}
\begin{proof}
This follows from \cite[Proposition 4.3]{Hassett}.
\end{proof}

In Section 4 of \cite{Hassett}, one finds the following formula:
\begin{equation}
Z_{\tau}\cdot D_{1}\cdot D_{2}=2\cdot B(D_{1},D_{2}),
\label{ZT}
\end{equation}
for all $D_{1}$, $D_{2}$ in $H^{2}(\X,\Z)$, $\tau\in A[3]$ and $B$ the Beauville-Bogomolov form on $\X$.

\begin{definition}\label{defiPi}
We define $\Pi' := \Pi \cap \Sym^{\perp}$. It follows from (\ref{ZT}) that $\Pi'$ can be described as the span of all classes of the form $Z_\tau -Z_0$ or alternatively as the set of all
$
\sum_\tau \alpha_\tau Z_\tau $, such that $ \sum_\tau \alpha_\tau =0$.
Note that in \cite{Hassett} the symbol $\Pi'$ denotes something different.
\end{definition}
\begin{remark}
Since $\rk \Sym = 28$ and $\rk \Pi' = 80$, the lattice $\Sym\oplus\Pi'\subset H^4\left(\X,\Z\right)$ has full rank.
\end{remark}

\begin{proposition}
The class $W$ can be written with the help of the square of half the diagonal as
\begin{align} 
W &= \theta^*\Big( \q_3(1)\vac \Big) \\
\label{WFormula}
&= 9 Y_p + e^2.
\end{align}
The second Chern class is non-divisible and given by 
\begin{align}
\label{sumZ}
\cc &= \frac{1}{3}\sum_{\tau\in A[3] } Z_\tau \\
\label{ChernY}
&= \frac{1}{3} \Big(72Y_p-e^2 \Big). 
\end{align}
\end{proposition}
\begin{proof} 
In Section 4 of \cite{Hassett} one finds the equations
\begin{gather}
W = \frac{3}{8}\Big(\cc + 3e^2\Big),\label{WW} \\
Y_p = \frac{1}{72}\Big(3\cc + e^2\Big),
\end{gather}
from which we deduce (\ref{WFormula}) and (\ref{ChernY}).
Equation (\ref{sumZ}) and the non-divisibility are from \cite[Proposition 5.1]{Hassett}.
\end{proof}

\begin{proposition}\label{ImSym}
The image of $H^4(A\hilb{3},\Q)$ under the morphism $\theta^*$ is equal to $\Sym^2H^2(\X,\Q)$.
\end{proposition}
\begin{proof}
We start by giving a set of generators of $H^4(A\hilb{n},\Q)$. Theorem 5.30 of \cite{LiQinWang} ensures that it is possible to do this in terms of multiplication operators. To enumerate elements of $H^*(A,\Q)$, we follow Notation \ref{TorusClasses}. Bais elements of $H^2(A,\Q)$ will be denoted by $b_i$ for $1\leq i\leq 6$. Then our set of generators is given by:
\begin{center}
\begin{tabular}{l|c}
multiplication operator & number of classes \\ \hline
$\G_0(a_1)\G_0(a_2)\G_0(a_3)\G_0(a_4) $  & $1$ \\
$\G_0(a_i)\G_0(a_j)\G_0(b_k)$ for $i<j$  & $\binom{4}{2}\cdot 6 = 36$ \\
$\G_0(a_i)\G_0(a_j^*)$ & $4\cdot 4 = 16$ \\
$\G_0(b_i)\G_0(b_j)$ for $i\leq j$& $\binom{6+1}{2}= 21$ \\
$\G_0(x)$ &  $1$  \\ \hline
$\G_0(a_i)\G_0(a_j)\G_1(1)$ for $i<j$ & $\binom{4}{2} = 6$ \\
$\G_0(a_i)\G_1(a_j)$ & $4\cdot 4=16$ \\
$\G_0(b_i)\G_1(1)$ & $6$ \\
$\G_1(b_i)$ & $6$ \\
$\G_1(1)^2 $ & $1$ \\ \hline
$\G_2(1) $ &$1$
\end{tabular}
\end{center}
Any multiplication operator of degree $4$ can be written as a linear combination of these $111$ classes. Likewise, the dimension of $H^4(A\hilb{n},\Q)$ is $111$ for all $n\geq 4$, according to G\"ottsche's formula \cite{Gottsche}. However, for $n=3$, the $8$ classes $\G_0(x)$, $\G_1(b_i)$ and $\G_2(1) $ can be expressed as linear combinations of the others, so we are left with $103$ linearly independent classes that form a basis of $H^4(A\hilb{3},\Q)$. Multiplication with the class $[\X]$ is given by the operator $\G_0(a_1)\G_0(a_2)\G_0(a_3)\G_0(a_4) $ and annihilates every class that contains an operator of the form $\G_0(a_i)$. There are $75$ such classes, so by Proposition \ref{annihilator}, $\ker \theta^*\subset H^4(A\hilb{3},\Q)$ has dimension at least $75$ and $\im\theta^*$ has dimension at most $103-75=28$. However, since the image of $\theta^*$ must contain $\Sym^2H^2(\X,\Q)$, which is $28$-dimensional, equality follows.
\end{proof}


\begin{proposition} \label{ChernSym}
We have:
\begin{equation}
\cc= 4 u_1u_2 + 4v_1v_2 + 4 w_1 w_2 - \frac{1}{3} e^2. 
\end{equation}
In particular, $\cc\in \Sym \otimes\Q $.
\end{proposition}
We shall give two different proofs. The first one uses Nakajima operators, the second one is based on results of \cite{Hassett}.
\begin{proof}[Proof 1]
First note that the defining diagram (\ref{square}) of the Kummer manifold is the pullback of the inclusion of a point, so the normal bundle of $\X$ in $A\hilb{3}$ is trivial. The Chern class of the tangent bundle of $\X$ is therefore given by the pullback from $A\hilb{3}$: $c(\X) = \theta^*\left(c(A\hilb{3})\right)$. Proposition \ref{ImSym} allows now to conclude that $\cc\in \Sym \otimes \Q$.

To obtain the precise formula, we use a result of Boissi\`ere, \cite[Lemma 3.12]{Boissiere}, giving a commutation relation for the Chern character multiplication operator on the Hilbert scheme. We get:
\begin{align*}
c_2(A\hilb{3}) & = 3\q_1(1)\mathfrak L_2(1)\vac - \frac{1}{3} \q_3(1)\vac \\
 & =\frac{8}{3}\q_1(1)\mathfrak L_2(1)\vac - \frac{1}{3} \delta^2 .
\end{align*}
With Corollary \ref{KummerEquality} one shows now, that $\cc$ is given as stated.
\end{proof}
\begin{proof}[Proof 2]
It follows from (\ref{sumZ}) that $\cc\in \Pi'^\perp$, so $\cc \in \Sym\otimes\Q$. Moreover, together with (\ref{ZT}) we get that
\begin{equation*}
\cc\cdot D_{1}\cdot D_{2}=54\cdot B(D_{1},D_{2})
\end{equation*}
for all $D_{1}$, $D_{2}$ in $H^{2}(\X,\Z)$. Using the non-degeneracy of the Poincar\'e pairing and our knowledge about the Beauville--Bogomolov form on $\X$, we can calculate that
$$
\cc= 4 u_1u_2 + 4v_1v_2 + 4 w_1 w_2 - \frac{1}{3} e^2.  \qedhere
$$
\end{proof}


\begin{corollary}\label{Pi'}
The intersection $\Sym\cap \Pi$ is one-dimensional and spanned by $3\cc$. 
\end{corollary}
\begin{proof}
By Proposition \ref{ChernSym} and (\ref{sumZ}), $3\cc\in \Sym\cap \Pi$. Since the ranks of $\Sym$, $\Pi$ and $H^4(\X,\Z)$ are $28$, $81$ and $108$, respectively, the intersection cannot contain more.
\end{proof}

\begin{corollary}\label{Classuvw}
\begin{equation} \label{YSym}
Y_p =  \frac{1}{6}\Big(u_1u_2 + v_1v_2 +  w_1 w_2 \Big).
\end{equation}
\end{corollary}
\begin{remark}
Using Nakajima operators, we may write
\begin{gather}
Y_p = \frac{1}{9} \theta^*\Big( \q_1(1)\mathfrak L_2(1) \vac \Big) =  \frac{1}{2}\theta^*\Big(\q_1(x)\q_1(1)^2\vac\Big).
\end{gather}
\end{remark}

From the intersection properties $Z_\tau \cdot Z_{\tau'} = 1$ for $\tau\neq \tau'$ and $Z_\tau^2 = 4$ from Section 4 of \cite{Hassett}, we compute
\begin{equation}
 \discr \Pi' = 3^{84}.
 \label{discrPi}
\end{equation}
On the other hand, a formula developed in \cite{Kapfer} evaluates
\begin{equation} \label{DiscrSym}
\discr \Sym \ = 2^{14}\cdot 3^{38},
\end{equation}
so the lattices $\Sym$ and $\Pi'$ cannot be primitive. Denote $\Sym^{sat}$ and $\Pi'^{sat}$ the respective primitive overlattices. $\Sym \oplus\Pi'$ is a sublattice of $H^4(\X,\Z)$ of index $2^{7}\cdot 3^{61}$ and we claim that $\Sym^{sat}\oplus \Pi'^{sat}$ has index $3^{22}$. To obtain a basis of $H^4(\X,\Z)$, we are now going to find
\begin{itemize}
 \item $7$ classes in $\Sym$ divisible by $2$,
 \item $8$ classes in $\Sym$ divisible by $3$,
 \item $31$ classes in $\Pi'$ divisible by $3$ and
 \item $20$ classes in $\Sym^{sat}\oplus \Pi'^{sat}$, one divisible by $3^3$ and $19$ divisible by $3$.
\end{itemize}

\begin{proposition}\label{classedivisibleSym}
For $y\in\{u_1,u_2,v_1,v_2,w_1,w_2\}$, the class
$
e \cdot y
$
is divisible by $3$ and 
$
 y^2 - \frac{1}{3} e\cdot y
$
is divisible by $2$.
\end{proposition}
\begin{proof}
We have $y= \theta^*\left(\q_1(a)\q_1(1)^2\vac\right)$ for some $a\in H^2(A,\Z)$. A computation yields:
$$
e\cdot y = 3\cdot \theta^*\Big( \q_2(a)\q_1(1)\vac \Big)
\quad \text{and}\quad
y^2 = \theta^*\Big(\q_1(a)^2\q_1(1)\vac\Big)
$$
so $e\cdot y$ is divisible by $3$. Furthermore, by Corollary \ref{IntegralOperatorsTorus}, the class 
$
\frac{1}{2} \q_1(a)^2\q_1(1)\vac - \frac{1}{2} \q_2(a)\q_1(1)\vac 
$
is contained in $H^4(A\hilb{3},\Z)$, so its pullback
$
 \frac{1}{2}y^2 - \frac{1}{6} e\cdot y
$
is an integral class, too.
\end{proof}
From Proposition \ref{ChernSym} we see that
$e^2$ is divisible by $3$ and by Corollary \ref{Classuvw}
the class $u_1u_2 + v_1v_2+w_1w_2$ is divisible by 6.
\begin{corollary}\label{SymSatImage}
The image of $H^4(A\hilb{3},\Z)$ under $\theta^*$ is equal to $\Sym^{sat}$. \qed
\end{corollary}
Now we come to $\Pi'$. 
The first thing to note is that $\Pi'$ is defined topologically for all deformations of $\X$ and the primitive overlattice of $\Pi'$ is a topological invariant, too.  
By applying a suitable deformation, we may therefore assume without loss of generality that $A$ is the product of two elliptic curves $A=E_1\times E_2$. After \cite[Eq.~(12)]{Hassett}, for a non-isotropic plane $\Lambda \subset A[3]$ and any $\tau_0\in A[3]$, the classes 
\begin{equation}
 \frac{1}{3}\sum_{\tau\in\Lambda} \Big(Z_{\tau} - Z_{\tau+\tau_0}\Big)
\end{equation}
are integral. The monodromy representation acts on the $Z_\tau$ via the symplectic group $\Sp(4,\mathbb F_3)$. Modulo $\Pi'$, the orbit of these classes is a $\mathbb F_3$-vector space naturally isomorphic to $D$ as introduced in Definition \ref{SymplecticIdeal}, so by Proposition \ref{SymplecticIdealsDimension}, we get a subspace of $\Pi'$ of rank $31$ of classes divisible by $3$.

The class $Z_0$ is not contained in $\Sym$ nor in $\Pi'$.
It can be written as follows:
$$
Z_0=\frac{\sum_{\tau\in A[3]}Z_\tau-\sum_{\tau\in A[3]}(Z_\tau-Z_0)}{81}
\stackrel{(\ref{sumZ})}{=} \frac{c_2(K_2(A))-\frac{1}{3}\sum_{\tau\in A[3]}(Z_\tau-Z_0)}{27},
$$
This is the class in $\Sym^{sat}\oplus \Pi'^{sat}$ divisible by $27$. Let us now find the remaining $19$ classes divisible by $3$.


Hassett and Tschinkel in Proposition 7.1 of \cite{Hassett} provide the class of a Lagrangian plane $P\subset K_2(A)$, which can be expressed as follows:
$$\left[P\right]=\frac{1}{216}(6u_1-3e)^2+\frac{1}{8}\cc-\frac{1}{3}\sum_{\tau\in \Lambda'} Z_{\tau},$$
where $\Lambda'=E_1[3]\times 0\subset A[3]$. We rearrange this expression a bit using (\ref{WW}):
\begin{align*}
\left[P\right]&=\frac{1}{216}(6u_1-3e)^2+\frac{1}{8}\cc-\frac{1}{3}\sum_{\tau\in \Lambda'} Z_{\tau}\\
&=\frac{36u_1^2+9e^2-36u_1\cdot e}{216} +\frac{W}{3}-\frac{3}{8}e^2-\frac{1}{3}\sum_{\tau\in \Lambda'} Z_{\tau}\\
%&=\frac{36u_1^2-72e^2-36u_1\cdot e}{216} +\frac{W}{3}-\frac{1}{3}\sum_{\tau\in \Lambda'} Z_{\tau}\\
&=\frac{u_1^2-2e^2-u_1\cdot e}{6} +\frac{W}{3}-\frac{1}{3}\sum_{\tau\in \Lambda'} Z_{\tau}.
\end{align*}
The classes $e^2$ and $u_1\cdot e$ are both divisible by 3 and by (\ref{WW}), $W$ is divisible by 3, so the following class is integral:
$$
\mathfrak{N}:=\frac{u_1^2+\sum_{\tau\in \Lambda'} (Z_{\tau}-Z_0)}{3}
.
$$
Now we will conclude using the action of the monodromy group $\Sp(A[3])\ltimes A[3]$ on the element $\mathfrak{N}$ and the considerations from Section \ref{Section_Symplectic}. 
%Before, we have to introduce some notation.
%Let denote by $O_1$ the $\Ft$ vector space generated by the elements of the orbit of $u_1^2+\sum_{\tau\in \Lambda'} Z_{\tau}$ under the action of $\Sp(A[3])\ltimes A[3]$. We denote by $O_2$ the $\Ft$ vector space generated by the orbit, under the action of $\Sp(A[3])$, of the elements $u_1^2+\sum_{\tau\in\Lambda'} (Z_{\tau} - Z_{\tau+\tau_0})$, with $\tau_0\in A[3]$.
%We denote by $\Sym^{o}=\Sym\cap c_2(K_2(A))^{\bot}\otimes $ and the projection $\Pr_1:\Sym^{o} $
Proposition \ref{CombinedSymplectic} states now that
the orbit of $\mathfrak{N}$ under the action of $\Sp(A[3])\ltimes A[3]$ is spanning a space of rank $51$ modulo $\Sym\oplus\Pi'$. However, by Lemma \ref{cleffinclassesdiv}, the intersection of that space with $\Sym^{sat}$ is one-dimensional and the intersection with $\Pi'^{sat}$ has dimension $31$, so we are left with $19$ linearly independent elements of the form:
$\frac{x+y}{3}$ with $x\in \Sym\smallsetminus \left\{0\right\}$, and $y\in \Pi'\smallsetminus \left\{0\right\}$. These are the 19 classes which were missing. 

\section{Conclusion on cohomology of Generalized Kummer fourfolds}
Let $A$ be a complex abelian surface and $(b_i)\subset H^2(A,\Z)$ an integral basis. We also use Notation \ref{TorusClasses}. Let $\theta: \kum{A}{2}\hookrightarrow A\hilb{3}$ be the embedding. 
Let us summarize our results on $\theta^*$:
\begin{theorem}\label{thetaTheorem}
The homomorphism $\theta^*:H^*(A\hilb{3},\Z)\rightarrow H^*(\kum{A}{2},\Z)$ of graded rings is surjective in every degree except $4$. Moreover, the image of $H^4(A\hilb{3},\Z)$ is the primitive overlattice of $\Sym^2(H^2(\kum{A}{2},\Z))$. 
The kernel of $\theta^*$ is the ideal generated by $H^1(A\hilb{3},\Z)$.
The image by $\theta^*$ of the following integral classes gives a basis of $\im\theta^*$:
\begin{center}
\begin{tabular}{c|l|l}
degree & preimage of class & alternative name  \\
\hline
0 & $\frac{1}{6} \q_1(1)^3\vac$ & 1 \\
\hline
2 &  $\frac{1}{2}\q_1(b_i) \q_1(1)^2\vac$ for $1\leq i\leq 6$ & $j(b_i)$ \\
 & $\frac{1}{2} \q_2(1)\q_1(1)\vac $  & $e$\\
\hline
3 & $\frac{1}{2}\q_1(a^*_i)\q_1(1)^2\vac$ & \\
  & $\q_2(a_i)\q_1(1)\vac$ & \\
\hline
4 & $\q_1(b_i)\q_1(b_j)\q_1(1)\vac$ for $1\leq i\leq j\leq 6$, but $(b_i,b_j)\neq(a_1a_2,a_3a_4)$ &\\
  & $\frac{1}{2}\q_1(x)\q_1(1)^2\vac$ (instead of the missing case above)  & $Y_p$\\
  & $\frac{1}{2}\left(\q_1(b_i)^2-\q_2(b_i)\right)\q_1(1)\vac$ & \\
  & $\frac{1}{3} \q_3(1)\vac$ & $W$ \\
\hline
5 & $\q_1(a_ia_j)\q_1(a_j^*)\q_1(1) \vac$ for any choice of $j\neq i$ &\\
  & $\frac{1}{2} \q_2(a^*_i)\q_1(1)\vac $ &\\
\hline
6 & $\q_1(a_i^*)\q_1(a_j^*)\q_1(1)\vac$ for $1\leq i< j\leq 4$ & \\
  & $\q_2(x)\q_1(1)\vac$ & \\
\hline
8 & $\q_1(x)^3\vac$ & top class
\end{tabular}
\end{center}
\end{theorem}
\begin{proof}
The table is established by the following results:
For degree 2, see Proposition \ref{H2Sur}. The dual classes of degree 6 can be computed using Proposition \ref{KummerClass} and the methods from Section \ref{Section_Hilbert}.
The odd degrees are treated by Proposition \ref{oddcohomology}. Classes of degree 4 are studied in Section \ref{Middle}. The classes are chosen in a way that they give a basis of $\Sym^{sat}$, which is possible by Corollary \ref{SymSatImage}. The condition $(b_i,b_j)\neq(a_1a_2,a_3a_4)$ is more or less arbitrary, but we had to remove one class to avoid a relation of linear dependence.

The kernel of $\theta^*$ is described by the Propositions \ref{annihilator} and \ref{Annihideal}.
\end{proof}


\begin{thm}\label{integralbasistheorem}
Using Notation \ref{BasisH2KA} and \ref{TheZs}, we have:
\begin{enumerate}
\item
Let $\Sym^{sat}$ be the primitive overlattice of $\Sym^2\left(H^2\left(\X,\Z\right)\right)$ in $H^4(K_2(A),\Z)$.
The group $\frac{\Sym^{sat}}{\Sym^2\left(H^2\left(\X,\Z\right)\right)}=(\Z/2\Z)^{\oplus 7}\oplus(\Z/3\Z)^{\oplus 8}$ is generated by the elements:
$$\frac{e \cdot y}{3},\ \frac{y^2 - \frac{1}{3} e\cdot y}{2} \text{ for } y\in\{u_1,u_2,v_1,v_2,w_1,w_2\},\ 
\frac{e^2}{3} \text{ and } \frac{u_{1}\cdot u_{2}+v_{1}\cdot v_{2}+w_{1}\cdot w_{2}}{6}.$$
\item
Let $\Pi'$ be the lattice from Definition \ref{defiPi} and let $\Pi'^{sat}$ be the primitive over lattice of $\Pi'$ in $H^4(K_2(A),\Z)$.
The group $\frac{\Pi'^{sat}}{\Pi'\ \ \ }=(\Z/3\Z)^{\oplus 31}$ is generated by the classes:
$$\frac{1}{3}\sum_{\tau\in\Lambda} \Big(Z_{\tau} - Z_{\tau+\tau'}\Big),
$$
with $\Lambda$ a non-isotropic group and $\tau'\in A[3]$. Moreover a basis of $\frac{\Pi'^{sat}}{\Pi'\ \ \ }$ is provided by the 31 classes described in Proposition \ref{XXXI}. 
\item
$$
\frac{H^4(K_2(A),\Z)}{\Sym^{sat}\oplus\Pi'^{sat}}=\left(\frac{\Z}{27\Z}\right)\oplus\left(\frac{\Z}{3\Z}\right)^{\oplus 19}.
$$
Moreover, this group is generated by the class $Z_0$ and the 19 classes described in Proposition \ref{XIX}.
\end{enumerate}
\end{thm}

%Proof: look at the ranks of H^*(A\hilb{3}):
%Rank H1*H4 = 188
%Rank H1*H5 = 239
%Rank H1*H6 = 196
%Rank H1*H7 = 102
%Rank H1*H8 = 40


\section{Symplectic involution on \texorpdfstring{$K_{2}(A)$}{the Kummer fourfold}}\label{Involution} 
This section and the following one is contributed by Gr\'egoire Menet.

Let $X$ be an irreducible symplectic manifold. Denote $$\nu: \Aut (X)\rightarrow \Aut H^{2}(X,\Z)$$
the natural morphism. Hassett and Tschinkel (Theorem 2.1 in~\cite{Hassett}) have shown that $\Ker \nu$ is a deformation invariant. 
Let $X$ be an irreducible symplectic fourfold of Kummer type. Then Oguiso in~\cite{Oguiso} has shown that $\Ker \nu =(\Z/3\Z)^3\rtimes\Z/2\Z$.

Let $A$ be an abelian variety and $g$ an automorphism of $A$. Let us denote by $T_{A[3]}$ the group of translations of $A$ by elements of $A[3]$. 
If $g\in  T_{A[3]}\rtimes\Aut_{\Z} (A)$, then $g$ induces a natural automorphism on $K_2(A)$. 
We denote the induced automorphism by $g^{[[3]]}$. If there is no ambiguity, we also denote the \emph{induced automorphism} by the same letter $g$
to avoid too complicated formulas.

When $X=K_2(A)$, we have more precisely, by Corollary 3.3 of~\cite{BNS2}, $$\Ker \nu = T_{A[3]}\rtimes(-\id_A)^{[[3]]}.$$ 
\subsection{Uniqueness and fixed locus}
\begin{thm}\label{SymplecticInvo}
Let $X$ be an irreducible symplectic fourfold of Kummer type and $\iota$ a symplectic involution on $X$. Then:
\begin{enumerate}
\item
We have $\iota\in \Ker \nu$.
\item
Let $A$ be an abelian surface. Then 
the couple $(X,\iota)$ is deformation equivalent to $(K_2(A),t_\tau \circ (-\id_A)^{[[3]]})$,
where $t_\tau$ is the morphism induced on $K_2(A)$ by the translation by $\tau\in A[3]$.
\item
The fixed locus of $\iota$ is given by a K3 surface and 36 isolated points.
\end{enumerate}
\end{thm}
\begin{proof}
\begin{enumerate}
\item
If $\iota\notin \Ker \nu$, by the classification of Section 5 of~\cite{MongWanTari}, the unique possible action of $\iota$ on $H^{2}(X,\Z)$ is given by $H^{2}(X,\Z)^{\iota}=U\oplus (2)^2\oplus(-6)$. We will show that it is impossible. Let us assume that $H^{2}(X,\Z)^{\iota}=U\oplus (2)^2\oplus(-6)$, we will find a contradiction.

As done in Section 3 of~\cite{Mongardi}, consider a local universal deformation space of $X$:
$$\Phi:\mathcal{X}\rightarrow \Delta,$$
where $\Delta$ is a small polydisk and $\mathcal{X}_0=X$. By restricting $\Delta$, we can assume that $\iota$ extends to an automorphism
$M$ on $\mathcal{X}$ and $\mu$ on $\Delta$, such that we have the following commutative diagram: 
$$\xymatrix{
 \mathcal{X}\ar[d]\ar[r]^{M} & \mathcal{X}\ar[d]\\
  \Delta \ar[r]^{\mu} & \Delta
   }$$
Moreover, the differential of $\mu$ at 0 is given by the action of $\iota$ on $H^1(T_X)$ which is the same as the action on $H^{1,1}(X)$, since the symplectic holomorphic 2-form induces an isomorphism between the two and the symplectic holomorphic 2-form is preserved by the action of $\iota$. We may assume that $\mu$ is a linear map. So $\Delta^\mu$ is smooth and $\dim \Delta^\mu=\rk H^{2}(X,\Z)^{\iota}-2=3$.  
Moreover, by~\cite{Markmanou} we can find $x\in \Delta^\mu$ such that %$\NS (\mathcal{X}_x)=A_1(-1)^2\oplus(-6)$. 
$\mathcal{X}_x$ is bimeromorphic to a Kummer fourfold $K_2(A)$.
%with $\NS(A)=A_1(-1)^2$ and 
Since $H^{2}(X,\Z)^{\iota}=U\oplus (2)^2\oplus(-6)$, $\iota_x\defIs  M_{\mathcal{X}_x}$ induces a bimeromorphic involution $i$ on $K_2(A)$ with $H^{2}(K_2(A),\Z)^{i}=U\oplus (2)^2\oplus(-6)$.

Since $i$ preserves the holomorphic 2-from, $\NS(K_2(A))\supset \left[H^{2}(K_2(A),\Z)^{i}\right]^{\bot}=(-2)^2$. The involution $i$ also induces a trivial involution on $A_{H^{2}(X,\Z)}$, so the half class of the diagonal $e$ is contained in $H^{2}(K_2(A),\Z)^{i}\cap\NS(K_2(A))$. It follows that $\NS(K_2(A))\supset(-2)^2\oplus\Z e$. Moreover, the morphism $j$ defined in Notation~\ref{BasisH2KA} respects the Hodge structure so $\NS(K_2(A))=j(\NS(A))\oplus\Z e$. It follows that $\NS(A)\supset (-2)^2$.
Now we construct an involution $g$ on $H^{2}(A,\Z)$ given by $-\id$ on $(-2)^2$ and $\id$ on $((-2)^2)^{\bot}$ and extended to an involution on $H^{2}(A,\Z)$ by Corollary 1.5.2 of~\cite{Lattice}. Then by Theorem 1 of~\cite{Shioda}, $g$ provides a symplectic automorphism on $A$ with: $H^{2}(A,\Z)^{g}=((-2)^2)^\bot=U\oplus (2)^2$.
It follows from the classification of Section 4 of~\cite{MongWanTari0}, that $A=\C/\Lambda$ with $\Lambda=\left\langle (1,0),(0,1),(x,-y),(y,x)\right\rangle$, $(x,y)\in \C^2\smallsetminus \R^2$ and $g=\left(
\begin{array}{cc}
0 & -1\\
1 & 0 
\end{array} \right)$. 

Let also denote by $g$ the automorphism on $K_2(A)$ induced by $g$. By construction, $g\circ i$ acts trivially on $H^{2}(K_2(A),\Z)$. Hence by Corollary 3.3 and Lemma 3.4 of~\cite{FujikiK}, $g\circ \iota$ extends to an automorphism of $K_2(A)$. In particular, $i$ extends to a symplectic involution on $K_2(A)$. Then $g\circ i\in \Ker \nu$.

By Corollary~\ref{actionH3}, $t_\tau$ acts trivially on $H^{3}(K_2(A),\Z)$. Hence by Corollary 3.3 of~\cite{BNS2}, we have necessarily:
\begin{gather*}
g^{*}_{|H^{3}(K_2(A),\Z)}=i^{*}_{|H^{3}(K_2(A),\Z)}\circ (-\id_A)^{*}_{|H^{3}(K_2(A),\Z)}\ \\
\ou\ \ g^{*}_{|H^{3}(K_2(A),\Z)}=i^{*}_{|H^{3}(K_2(A),\Z)}.
\end{gather*}
By Corollary~\ref{actionH3}, $g^{*}_{|H^{3}(K_2(A),\Z)}$ has order 4. But on the other hand, both $i^{*}_{|H^{3}(K_2(A),\Z)}\circ (-\id_A)^{*}_{|H^{3}(K_2(A),\Z)}$ and $i^{*}_{|H^{3}(K_2(A),\Z)}$ have order 2, which is a contradiction.
\item
Let $X$ be a irreducible symplectic fourfold of Kummer type and $\iota$ a symplectic involution on $X$. By (1) of the above theorem, we have $\iota\in\Ker \nu$. Then by Theorem 2.1 of~\cite{Hassett}, the couple $(X,\iota)$ deform to a couple $(K_2(A),\iota')$ with $A$ an abelian surface and $\iota'\in\Ker \nu$ a symplectic involution on $ K_2(A)$. Then we conclude with Corollary 3.3 of~\cite{BNS2}.
\item
Let $A$ be an abelian surface. By Section 1.2.1 of~\cite{Tari}, the fixed locus of $t_\tau \circ (-\id_A)^{[[3]]}$ on $K_2(A)$ is given by a K3 surface an 36 isolated points. 
Now let $X$ be a irreducible symplectic fourfold of Kummer type and $\iota$ a symplectic involution on $X$. 
By (2) of the above theorem, $\Fix\iota$ deforms to the disjoint union of a K3 surface and 36 isolated points. Moreover, $\iota$ is a symplectic involution, so the holomorphic 2-form of $X$ restricts to a non-degenerated holomorphic 2-form on $\Fix\iota$. Then necessarily, $\Fix\iota$ consists of a K3 surface and 36 isolated points.
\qedhere
\end{enumerate}
\end{proof}
\begin{rmk}\label{RemarkSymplecticInv}
\begin{enumerate}
\item
We also remark that the K3 surface fixed by $(t_\tau \circ (-\id_A))$ is given by the sub-manifold $$Z_{-\tau}=\overline{\left\{\left.(a_1,a_2,a_3)\ \right|\ a_1=-\tau,\ a_2=-a_3+\tau,\ a_2\neq -\tau\right\}}$$ defined in Section 4 of~\cite{Hassett}. 
\item
Considering the involution $-\id_A$,
the set $$\mathcal{P}\defIs \left\{\left.\xi\in K_{2}(A)\right|\ \Supp \xi= \left\{a_{1},a_{2},a_{3}\right\},\ a_{i}\in A[2]\smallsetminus \left\{0\right\}, 1\leq i\leq 3 \right\}$$ provides 35 fixed points and the vertex of $$W_{0}\defIs \left\{\left.\xi\in K_{2}(A)\right|\ \Supp \xi=\left\{0\right\}\right\}$$ supplies the 36th point. We denote by $p_1,...,p_{35}$ the points of $\mathcal{P}$ and by $p_{36}$ the vertex of $W_{0}$.
\end{enumerate}
\end{rmk}
\subsection{Action on the cohomology}\label{actioncoh}
From Theorem~\ref{SymplecticInvo}, we can assume that $X=K_2(A)$ and $\iota=-\id_A$. To consider $t_\tau \circ (-\id_A)$ instead of  $-\id_A$ only has the effect of exchanging the role of $[Z_0]$ and $[Z_{-\tau}]$.
Hence we do not lose any generality assuming that $\iota=-\id_A$. Now, we calculate the invariants $l_i^j(K_{2}(A))$ defined in Definition-Proposition~\ref{defiprop}. It will be used in Section~\ref{BeauvilleForm}.

From Theorem~\ref{SymplecticInvo} (1), the involution $\iota$ acts trivially on $H^{2}(K_{2}(A),\Z)$.
It follows 
\begin{equation}
l_{2}^2(K_{2}(A))=l_{1,-}^2(K_{2}(A))=0 \text{ and } l_{1,+}^2(K_{2}(A))=7.
\label{l22}
\end{equation}
From Corollary~\ref{actionH3}, the involution $\iota$ acts as $-\id$ on $H^{3}(K_{2}(A),\Z)$.
%We have $\rk H^{3}(K_{2}(A),\Z)^{\iota}= 2$.
It follows 
\begin{equation}
l_{2}^3(K_{2}(A))=l_{1,+}^3(K_{2}(A))=0 \text{ and } l_{1,-}^3(K_{2}(A))=8.
\label{l3}
\end{equation}
By Definition~\ref{defiPi}, we have:
$$H^{4}(K_{2}(A),\Q)=\Sym^{2} H^{2}(K_{2}(A),\Q)\oplus^{\bot} \Pi'\otimes\Q,$$
where $\Pi'=\left\langle Z_{\tau}-Z_{0},\ \tau\in A[3]\smallsetminus \left\{0\right\}\right\rangle$. The involution
$\iota^*$ fixes $\Sym^{2} H^{2}(K_{2}(A),\Z)$ and $\iota^*(Z_{\tau}-Z_{0})=Z_{-\tau}-Z_{0}$. It provides the following proposition.
\begin{prop}\label{invariants}
We have $l_{1,-}^4(K_{2}(A))=0$, $l_{1,+}^4(K_{2}(A))=28$ and $l_{2}^4(K_{2}(A))=40$.
%We have $H^{4}(K_{2}(A),\Z)^{\iota}=\Sym^{2} H^{2}(K_{2}(A),\Q)\oplus \Pi'^{\iota}\otimes\Q$, with
%$\rk \Pi'^{\iota}=40$.
\end{prop}
\begin{proof}
%We denote $\alpha=\tr \iota^{*}_{H^{1,1}(K_{2}(A))}$ and $\beta=\tr \iota^{*}_{H^{2,1}(K_{2}(A))}$.
%We apply Holomorphic Lefschetz-Riemann-Roch formula Theorem 1 of~\cite{Camere} to the bundle $\Omega_{K_{2}(A)}$ and we find the following formula:
%$$\beta-2\alpha=-\frac{N}{4}-12K+\frac{1}{4}\sum_{\iota(S_{i})=S_{i}} c_{2}(K_{2}(A))\cdot S_{i},$$
%where $N$ is the number of isolated fixed points of $\iota$, $K$ is the number of K3 surfaces fixed by $\iota$ and $S_{i}$ are fixed surfaces by $\iota$.
%Hence, it follows from Proposition~\ref{fix}:
%$$\beta-2\alpha=-\frac{36}{4}-12+\frac{1}{4}c_{2}(K_{2}(A))\cdot Z_{0}.$$
%Moreover by Proposition 4.3 of~\cite{Kummer}, we have:
%$c_{2}(K_{2}(A))\cdot Z_{0}=28$, hence:
%$$\beta-2\alpha=-14.$$
%By (1), we know that $\alpha=5$. 
%It follow that $\beta=-4$.
%Since $\dim H^{2,1}(K_{2}(A))=4$, it follows that $\iota$ acts as $-\id$ on $H^{3}(K_{2}(A),\Q)$.
%Let $\iota_{0}$ be the involution on $A^{[3]}$ induced by $-\id$.
%We have $\iota_{0}^{*}(a_{i}^{(1)})=-a_{i}^{(1)}$ and $\iota_{0}^{*}(a_{\overline{i}}^{(0)})=-a_{\overline{i}}^{(0)}$.
%Moreover, we have by definition, $\iota^{*}\circ \theta^{*}=\theta^{*}\circ \iota_{0}^{*}$.
%It follows by Proposition 3.18~\ref{} that the involution $\iota$ acts as $-\id$ on $H^{3}(K_{2}(A),\Q)$.

Let $\mathcal{S}$ be the over-lattice of $\Sym^{2} H^{2}(K_{2}(A),\Z)$ where we add all the classes divisible by 2 in $H^{4}(K_{2}(A),\Z)$.
%By (\ref{discrPi}), the discriminant of $\Pi'$ is not divisible by 2. 
%Since $H^{4}(K_{2}(A),\Z)$ is unimodular, it follows that the discriminant of $\mathcal{S}$ is also not divisible by 2.
From Section~\ref{Middle}, we know that the discriminant of $\mathcal{S}$ is not divisible by 2.
Hence, we have:
$$H^{4}(K_{2}(A),\F)=\mathcal{S}\otimes\F\oplus \Pi'\otimes\F.$$
Moreover, we have: $$\iota^{*}(Z_{\tau}-Z_{0})=Z_{-\tau}-Z_{0},$$
for all $\tau\in A[3]\smallsetminus \left\{0\right\}$.
Hence $\Vect_{\F}(Z_{\tau}-Z_{0},Z_{-\tau}-Z_{0})$ is isomorphic to $N_{2}$ as a $\F[G]$-module  (see the notation in Definition-Proposition~\ref{defiprop}).
Moreover $H^{2}(K_{2}(A),\Z)$ is invariant by the action of $\iota$, hence $\Sym^{2} H^{2}(K_{2}(A),\Z)$ and $\mathcal{S}$ is also invariant by the action of $\iota$. 
It follows that $\mathcal{S}\otimes\F=\mathcal{N}_{1}$ and $\Pi'\otimes\F=\mathcal{N}_{2}$.
Since $\rk \mathcal{S}=28$, we have $l_{1,+}^{4}+l_{1,-}^{4}=28$.
However, $\mathcal{S}$ is invariant by the action of $\iota$, it follows that $l_{1,-}^{4}=0$ and $l_{1,+}^{4}=28$.
On the other hand $\rk \Pi'=80$, it follows that $l_{2}^{4}=40$.
%Moreover, we have:
%$$\iota^{*}(Z_{\tau})=Z_{-\tau}.$$
%So 
%\begin{equation}
%\Pi'^{\iota}=\left\langle Z_{\tau}+Z_{-\tau}-2Z_{0},\ \tau\in A[3]\smallsetminus \left\{0\right\}\right\rangle.
%\label{Pi'i}
%\end{equation}
%Hence, we have $\rk \Pi'^{\iota}=\frac{\# A[3] -1}{2}=40$.
\end{proof}

\section[Application to singular irreducible symplectic varieties]{Application to singular irreducible symplectic varieties%
\sectionmark{Singular irreducible symplectic varieties}}\label{BeauvilleForm}
\sectionmark{Singular irreducible symplectic varieties}
\subsection{Statement of the main theorem}\label{statement}

In~\cite{Nanikawa}, Namikawa proposes a definition of the Beauville-Bogomolov form for some singular irreducible symplectic varieties. He assumes that the singularities are only $\mathbb{Q}$-factorial with a singular locus of codimension $\geq 4$. Under these assumptions, he proves a local Torelli theorem. This result was completed by a generalization of the Fujiki formula by Matsushita in~\cite{Mat}. 
\begin{thm}\label{Form}
Let $X$ be a projective irreducible symplectic variety of dimension $2n$ with only $\mathbb{Q}$-factorial singularities, and $\codim\sing X\geq 4$.
There exists a unique indivisible integral symmetric non-degenerated bilinear form $B_{X}$ on $H^2(X,\Z)$ and a unique positive constant $c_{X}\in \mathbb{Q}$, such that for any $\alpha\in H^2(X,\mathbb{C})$,
\begin{equation}
\alpha^{2n}=c_{X}B_{X}(\alpha,\alpha)^n
\label{Fujiki}
\end{equation}
and such that for $0\neq \omega\in H^{0}(\Omega_{U}^{2})$ a holomorphic 2-from on the smooth locus $U$ of $X$:
\begin{equation}
B_{X}(\omega+\overline{\omega},\omega+\overline{\omega})>0.
\label{Fujikiposi}
\end{equation}
Moreover, the signature of $B_{X}$ is $(3,h^{2}(X,\mathbb{C})-3)$.

The form $B_{X}$ is called the Beauville--Bogomolov form of $X$.
\end{thm}
\begin{proof}
The statement of the theorem in~\cite{Mat} does not say that the form is integral. However, let $X_{s}$ be a fiber of the Kuranishi family of $X$, with the same idea as Matsushita's proof, we can see that $B_{X}$ and $B_{X_{s}}$ are proportional. Then, it follows using the proof of Theorem 5 a), c) of~\cite{Beauville}.
\end{proof}
We can also consider its polarized form.
\begin{prop}\label{beauville}
Let $X$ be a projective irreducible symplectic variety of dimension $2n$ with $\codim\sing X\geq 4$.
The equality (\ref{Fujiki}) of Theorem~\ref{Form} implies that
$$\alpha_{1}\cdot...\cdot\alpha_{2n}=\frac{c_{X}}{(2n)!}\sum_{\sigma\in S_{2n}}B_{X}(\alpha_{\sigma(1)},\alpha_{\sigma(2)})...B_{X}(\alpha_{\sigma(2n-1)},\alpha_{\sigma(2n)}).$$
for all $\alpha_{i}\in H^{2}(X,\Z)$.
\end{prop}
These results were then generalized by Kirschner for symplectic complex spaces in~\cite{Tim}. 
In~\cite[Theorem 2.5]{Lol2} was appeared the first concrete example of Beauville-Bogomolov lattice for a singular irreducible symplectic variety. The variety studied in~\cite{Lol2} is a partial resolution of an irreducible symplectic manifold of $K3^{[2]}$-type quotiented by a symplectic involution. The objective of this section is to provide a new example of a Beauville-Bogomolov lattice replacing the manifold of $K3^{[2]}$-type by a fourfold of Kummer type. Knowing the integral basis of the cohomology group of the generalized Kummer provided in Part~\ref{integralcohomology}, this calculation becomes possible. Moreover the calculation will be much simpler as in~\cite{Lol2} because of the general techniques for calculating integral cohomology of quotients developed in~\cite{Lol} and the new technique using monodromy developed in Lemma~\ref{Ddelta}. 
The other techniques developed in~\cite{Lol2} are also in~\cite{Lol}, so to simplify the reading, we will only cite~\cite{Lol2} in the rest of the section.



Concretely, let $X$ be an irreducible symplectic fourfold of Kummer type and $\iota$ a symplectic involution on $X$. By Theorem~\ref{SymplecticInvo} the fixed locus of $\iota$ is the union of 36 points and a K3 surface $Z_0$. Then the singular locus of $K\defIs X/\iota$ is the union of a K3 surface and 36 points. The singular locus is not of codimension four. We will lift to a partial resolution of singularities,
$K'$ of $K$, obtained by blowing up the image of $Z_0$. By Section 2.3 and Lemma 1.2 of~\cite{Fujiki2}, the variety $K'$ is an irreducible symplectic V-manifold which has singular locus of codimension four.


All Section~\ref{BeauvilleForm} is devoted to prove the following theorem.
\begin{thm}\label{theorem}
Let $X$ be an irreducible symplectic fourfold of Kummer type and $\iota$ a symplectic involution on $X$.
Let $Z_0$ be the K3 surface which is in the fixed locus of $\iota$.
We denote $K=X/\iota$ and $K'$ the partial resolution of singularities of $K$ obtained by blowing up the image of $Z_0$.
Then the Beauville--Bogomolov lattice $H^2(K',\Z)$ is isomorphic to $U(3)^{3}\oplus\left(
\begin{array}{cc}
-5 & -4\\
-4 & -5 
\end{array} \right)$, and the Fujiki constant $c_{K'}$ is equal to $8$.
\end{thm}
The Beauville-Bogomolov form is a topological invariant, hence from Theorem~\ref{SymplecticInvo} we can assume that $X$ is a generalized Kummer fourfold and $\iota=-\id_A$. As it will be useful to prove Lemma~\ref{Ddelta}, we can assume even more. All generalized Kummer fourfolds are deformation equivalent, hence we can assume that $A=E_\xi\times E_\xi$, where $E_\xi$ is the elliptic curve provided in Definition~\ref{elliptic6}: $$E_\xi\defIs \frac{\C}{\left\langle 1,e^{\frac{2i\pi}{6}}\right\rangle}.$$
\subsection{Overview on the proof of Theorem~\ref{theorem}}
We first provide all the notation that we will need during the proof in Section~\ref{nota}.
Then the proof is divided into the following steps:
\begin{enumerate}
\item
First (\ref{l22}), (\ref{l3}), Proposition~\ref{invariants} and Corollary~\ref{utile'} will prove the $H^4$-normality of $(K_2(A),\iota)$ in Section~\ref{H4}.
\item
The knowledge of the elements divisible by 2 in $\Sym^{2} H^{2}(K_2(A),\Z)$ from Section~\ref{Middle} and the $H^4$-normality allow us to prove the $H^2$-normality of $(K_2(A),\iota)$ in Section~\ref{H2}. 
\item
An adaptation of the $H^2$-normality (Lemma~\ref{primitive2}) and several lemmas in Section~\ref{basisinteK'} will provide an integral basis of $H^{2}(K',\Z)$ (Theorem~\ref{fin}).
\item
Knowing an integral basis of $H^{2}(K',\Z)$, we end the calculation of the Beauville--Bogomolov form in Section~\ref{beauK'} using intersection theory and the generalized Fujiki formula (Theorem~\ref{Form}).
\end{enumerate}
\subsection{Notation}\label{nota}
Let $K_{2}(A)$ be a generalized Kummer fourfold endowed with the symplectic involution $\iota$ induced by $-\id_A$.
We denote by $\pi$ the quotient map $K_{2}(A)\rightarrow K_{2}(A)/\iota$.
From Theorem~\ref{SymplecticInvo}, we know that the singular locus of the quotient $K_{2}(A)/\iota$ is the K3 surface, image by $\pi$ of $Z_{0}$, and 36 isolated points. We denote $\overline{Z_{0}}\defIs \pi(Z_{0})$. 
We consider $r':K'\rightarrow K_{2}(A)/\iota$ the blow-up of $K_{2}(A)/\iota$ in $\overline{Z_{0}}$ and we denote by $\overline{Z_{0}}'$ the exceptional divisor.
% symplectic variety (see Lemma of~\cite{}) with $\Sing K_{2}(A)/\iota= Z_{0} \cup \left\{36\ pts\right\}$.
%By Section 2.3 and Lemma 1.2 of~\cite{Fujiki}, $K'$ is a irreducible symplectic variety with only isolated singular points.
%The goal of this section is to calculate the Beauville--Bogomolov form of $K'$.
We also denote by $s_{1}:N_{1}\rightarrow K_{2}(A)$ the blowup of $K_{2}(A)$ in $Z_{0}$; and denote by $Z_{0}'$ the exceptional divisor in $N_{1}$. Denote by $\iota_{1}$ the involution on $N_{1}$ induced by $\iota$. We have $K'\simeq N_{1}/\iota_{1}$, and we denote $\pi_{1}:N_{1}\rightarrow K'$ the quotient map.

Consider the blowup $s_{2}:N_{2}\rightarrow N_{1}$ of $N_{1}$ in the 36 points $p_1,...,p_{36}$ fixed by $\iota_{1}$ and the blowup $\widetilde{r}:\widetilde{K}\rightarrow K'$ of $K'$ in its 36 singulars points. We denote the exceptional divisors by $E_{1},...,E_{36}$ and $D_{1},...,D_{36}$ respectively. We also denote $\widetilde{\overline{Z_{0}}}=\widetilde{r}^{*}(\overline{Z_{0}}')$ and $\widetilde{Z_{0}}=s_{2}^{*}(Z_{0}')$.
Denote $\iota_{2}$ the involution induced by $\iota$ on $N_{2}$ and $\pi_{2}:N_{2}\rightarrow N_{2}/\iota_{2}$ the quotient map. 
We have $N_{2}/\iota_{2}\simeq \widetilde{K}$. To finish, we denote $V=K_{2}(A)\smallsetminus \Fix \iota$ and $U=V/\iota$. We collect this notation in a commutative diagram
\begin{equation}
\xymatrix{
 \widetilde{K}\ar[r]^{\widetilde{r}} & K' \ar[r]^{r'}&  K_{2}(A)/\iota&U\ar@{_{(}->}[l]\\
  N_{2}\ar@(dl,dr)[]_{\iota_{2}} \ar[r]^{s_{2}} \ar[u]^{\pi_{2}}& N_{1}\ar@(dl,dr)[]_{\iota_{1}} \ar[r]^{s_{1}} \ar[u]^{\pi_{1}} & K_{2}(A)\ar@(dl,dr)[]_{\iota}\ar[u]^{\pi}&V\ar[u]\ar@{_{(}->}[l]
   }
   \label{commutativediagram}
\end{equation} 
Also, we set $s=s_2\circ s_1$ and $r=\widetilde{r}\circ r'$. We denote also $e$ the half of the class of the diagonal in $H^{2}(K_2(A),\Z)$ as states in Notation~\ref{BasisH2KA}.

\begin{rmk}\label{commut2}
We can commute the push-forward maps and the blow-up maps as proved in Lemma 3.3.21 of~\cite{Lol}.
Let $x\in H^{2}(N_1,\Z)$, $y\in H^{2}(K_2(A),\Z)$, we have:
$$\pi_{2*}(s_2^{*}(x))=\widetilde{r}^{*}(\pi_{1*}(x)),$$
$$\pi_{1*}(s_1^{*}(y))=r'^{*}(\pi_{*}(y)),$$
\end{rmk}
Moreover, we will also use the notation provided in Notation~\ref{BasisH2KA} and in Section~\ref{Middle}.
\subsection{\texorpdfstring{The couple $(K_{2}(A),\iota)$ is $H^{4}$-normal}{H4 normality}}\label{H4}
\begin{prop}\label{H4norm}
The couple $(K_{2}(A),\iota)$ is $H^{4}$-normal.
\end{prop}
\begin{proof}
We apply Theorem~\ref{utile'}.
\begin{enumerate}
\item
By Theorem~\ref{torsion}, $H^{*}(K_{2}(A),\Z)$ is torsion-free. 
\item
From Remark~\ref{RemarkSymplecticInv} (1), we know that the connected component of dimension 2 of $\Fix \iota$  is given by $Z_{0}$ which is a K3 surface, hence is simply connected. 
Moreover by Proposition 4.3 of~\cite{Hassett} $Z_{0}\cdot Z_{\tau}=1$ for all $\tau\in A[3]\smallsetminus \left\{0\right\}$. Hence the class of $Z_{0}$ in $H^{4}(K_{2}(A),\Z)$ is primitive. It follows that $\Fix \iota$ is almost negligible (Definition~\ref{negligible}). 
\item
By (\ref{l22}) and Proposition~\ref{invariants}, we have $l_{1,-}^{2}(K_{2}(A))=l_{1,-}^{4}(K_{2}(A))=0.$
\item
By (\ref{l3}) and Proposition~\ref{invariants}, we have $l_{1,+}^{3}(K_{2}(A))=0.$ Moreover, we have $H^{1}(K_{2}(A))=0$, so $l_{1,+}^{1}(K_{2}(A))=0.$
\item
We have to check the following equality:
\begin{align*}
&l_{1,+}^{4}(K_{2}(A))+2\left[l_{1,-}^{1}(X)+l_{1,-}^{3}(X)+l_{1,+}^{0}(X)+l_{1,+}^{2}(X)\right]\\
&= 36h^{0}(pt)+h^{0}(Z_{0})+h^{2}(Z_{0})+h^{4}(Z_{0}).
\end{align*}
By (\ref{l22}), (\ref{l3}) and Proposition~\ref{invariants}:
\begin{align*}
l_{1,+}^{4}(K_{2}(A))&+2\left[l_{1,-}^{1}(X)+l_{1,-}^{3}(X)+l_{1,+}^{0}(X)+l_{1,+}^{2}(X)\right]\\
&=28+2(8+1+7)=60.
\end{align*}
Moreover, since $Z_{0}$ is a K3 surface, we have:
$$36h^{0}(pt)+h^{0}(Z_{0})+h^{2}(Z_{0})+h^{4}(Z_{0})=36+1+22+1=60.$$
\end{enumerate}
It follows from Corollary~\ref{utile'} that $(K_{2}(A),\iota)$ is $H^4$-normal.
\end{proof} 
\begin{rmk}\label{primitive1}
As explained in Proposition 3.5.20 of~\cite{Lol}, the proof of Theorem~\ref{utile'} provide first that $\pi_{2*}(s^*(H^4(K_2(A),\Z)))$ is primitive in $H^4(\widetilde{K},\Z)$ and then the $H^4$ normality. 
So, the lattice $\pi_{2*}(s^*(H^4(K_2(A),\Z)))$ is primitive in $H^4(\widetilde{K},\Z)$.
\end{rmk}
%Moreover by Corollary 3.5.19 of~\cite{Lol}, we also have:
%\begin{rmk}
%The group $H^4(\widetilde{K},\Z)$ is torsion-free.
%\end{rmk}
\subsection{\texorpdfstring{The couple $(K_{2}(A),\iota)$ is $H^{2}$-normal}{H2 normality}}\label{H2}
\begin{prop}
The couple $(K_{2}(A),\iota)$ is $H^{2}$-normal.
\end{prop}
\begin{proof}
We prove that the pushforward  
$\pi_{*}:H^{2}(K_{2}(A),\Z)\rightarrow H^{2}(K_{2}(A)/\iota,\Z)/\tors$ is surjective. 
By Remark~\ref{Hnormal}, it is equivalent to prove that for all $x\in H^{2}(K_{2}(A),\Z)^{\iota}$,
$\pi_{*}(x)$ is divisible by 2 if and only if there exists $y\in H^{2}(K_{2}(A),\Z)$ such that $x=y+\iota^{*}(y)$. 

Let $x\in H^{2}(K_{2}(A),\Z)^{\iota}=H^{2}(K_{2}(A),\Z)$ such that $\pi_{*}(x)$ is divisible by 2, we will show that there exists $y\in H^{2}(K_{2}(A),\Z)$ such that $x=y+\iota^{*}(y)$.
By Proposition~\ref{commut}, $\pi_{*}(x^2)$ is divisible by 2.
However, $x^2\in H^{4}(K_{2}(A),\Z)^{\iota}$; since $(K_{2}(A),\iota)$ is $H^{4}$-normal by Proposition~\ref{H4norm}, it means that there is $z\in H^{4}(K_{2}(A),\Z)$ such that
$x^2=z+\iota^{*}(z)$.

Let $\mathcal{S}$ be, as before, the over-lattice of $\Sym^{2} H^{2}(K_{2}(A),\Z)$ where we add all the classes divisible by 2 in $H^{4}(K_{2}(A),\Z)$.
By Definition~\ref{defiPi} and (\ref{discrPi}), there exist $z_{s}\in \mathcal{S}$, $z_{p}\in \Pi'$ and $\alpha\in \mathbb{N}$ such that:
$3^\alpha\cdot z= z_{s}+z_{p}$. 
Hence, we have:
$$3^\alpha\cdot x^2=2z_{s}+z_{p}+\iota^{*}(z_{p}).$$
Since $x^2\in \Sym$, by Corollary~\ref{Pi'}, $z_{p}+\iota^{*}(z_{p})=0$.
It follows:
\begin{equation}
3^\alpha\cdot x^2=2z_{s}.
\label{haha}
\end{equation}
%$$x^2=\frac{2z_{s}}{5\cdot3^\alpha}.$$
%Moreover, since $H^{4}(K_{2}(A),\Z)$ is torsion-free, $z'\defIs \frac{z_{s}}{5\cdot3^\alpha}\in H^{4}(K_{2}(A),\Z)$. 
let $(u_{1},u_{2},v_{1},v_{2},w_{1},w_{2},e)$ be the integral basis of $H^{2}(K_{2}(A),\Z)$ introduced in Notation\ref{BasisH2KA}.
We can write:
$$x=\alpha_{1}u_{1}+\alpha_{2}u_{2}+\beta_{1}v_{1}+\beta_{2}v_{2}+\gamma_{1}w_{1}+\gamma_{2}w_{2}+de.$$
Then $$3^\alpha\cdot x^{2}=\alpha_{1}^{2}u_{1}^{2}+\alpha_{2}^{2}u_{2}^{2}+\beta_{1}^{2}v_{1}^{2}+\beta_{2}^{2}v_{2}^{2}+\gamma_{1}^{2}w_{1}^{2}+\gamma_{2}^{2}w_{2}^{2}+d^{2}e^{2}\mod 2H^{4}(K_{2}(A),\Z).$$
%We also have:
%$$5\cdot3^\alpha\cdot x^{2}=\alpha_{1}^{2}u_{1}^{2}+\alpha_{2}^{2}u_{2}^{2}+\beta_{1}^{2}v_{1}^{2}+\beta_{2}^{2}v_{2}^{2}+\gamma_{1}^{2}w_{1}^{2}+\gamma_{2}^{2}w_{2}^{2}+d^{2}e^{2}\mod 2H^{4}(K_{2}(A),\Z).$$
It follows by (\ref{haha}) that 
$\alpha_{1}^{2}u_{1}^{2}+\alpha_{2}^{2}u_{2}^{2}+\beta_{1}^{2}v_{1}^{2}+\beta_{2}^{2}v_{2}^{2}+\gamma_{1}^{2}w_{1}^{2}+\gamma_{2}^{2}w_{2}^{2}+d^{2}e^{2}$ is divisible by 2.
However by Corollary~\ref{Classuvw} and Proposition~\ref{classedivisibleSym}, we have:
\begin{multline}
\mathcal{S}=\Sym^2 H^{2}(K_{2}(A),\Z)\ \  + \\
\left\langle \frac{u_{1}\cdot u_{2}+v_{1}\cdot v_{2}+w_{1}\cdot w_{2}}{2}; 
\frac{u_{i}^{2}-\frac{1}{3}u_{i}\cdot e}{2};\frac{v_{i}^{2}-\frac{1}{3}v_{i}\cdot e}{2};\frac{w_{i}^{2}-\frac{1}{3}w_{i}\cdot e}{2},i\in\left\{1,2\right\}\right\rangle.
\label{generatorsS}
\end{multline}
The $\frac{1}{2}(\alpha_{1}^{2}u_{1}^{2}+\alpha_{2}^{2}u_{2}^{2}+\beta_{1}^{2}v_{1}^{2}+\beta_{2}^{2}v_{2}^{2}+\gamma_{1}^{2}w_{1}^{2}+\gamma_{2}^{2}w_{2}^{2}+d^{2}e^{2})$ is in $\mathcal{S}$ and so can be expressed as a linear combination of the generators of $\mathcal{S}$.
Then, it follows from (\ref{generatorsS}) that all the coefficients of $\alpha_{1}^{2}u_{1}^{2}+\alpha_{2}^{2}u_{2}^{2}+\beta_{1}^{2}v_{1}^{2}+\beta_{2}^{2}v_{2}^{2}+\gamma_{1}^{2}w_{1}^{2}+\gamma_{2}^{2}w_{2}^{2}+d^{2}e^{2}$ are divisible by 2.
%Maybe use \otimes\F to a better presentation. 
It means that $x$ is divisible by 2. This is what we wanted to prove.
\end{proof}
With exactly the same proof working in $H^4(\widetilde{K},\Z)$ and using Remark~\ref{primitive1}, we provide the following lemma.
\begin{lemme}\label{primitive2}
The lattice $\pi_{2*}(s^*(H^2(K_2(A),\Z)))$ is primitive in $H^2(\widetilde{K},\Z)$.
\end{lemme}
\subsection{Calculation of \texorpdfstring{$H^{2}(K',\Z)$}{H2 of K'}}\label{basisinteK'}
This section is devoted to prove the following theorem.
\begin{thm}\label{fin}
Let $K'$, $\pi_1$, $s_1$ and $\overline{Z_0}'$ be respectively the variety, the maps and the class defined in Section~\ref{nota}. 
We have $$H^{2}(K',\Z)=\pi_{1*}s_{1}^{*}H^2(K_2(A),\Z)\oplus\Z\left(\frac{\pi_{1*}(s_{1}^{*}(e))+\overline{Z_{0}}'}{2}\right)\oplus\Z\left(\frac{\pi_{1*}(s_{1}^{*}(e))-\overline{Z_{0}}'}{2}\right).$$
%where $e$ is half the class of the diagonal in $H^2(K_2(A),\Z)$.
\end{thm}

First we need to calculate some intersections.
\begin{lemme}\label{Fulton}
\begin{enumerate}
\item
We have $E_{l}\cdot E_{k}=0$ if $l\neq k$, $E_{l}^{4}=-1$ and $E_{l}\cdot z=0$ for all $(l,k)\in \left\{1,...,28\right\}^{2}$
and for all $z\in s^{*}(H^{2}(K_2(A),\Z))$.
\item
We have $e^4=324$.
\end{enumerate}
We already have some properties of primitivity:

\begin{enumerate}
\item
$\pi_{1*}(s_1^{*}(H^{2}(K_{2}(A),\Z)))$ is primitive in $H^{2}(K',\Z)$,
\item
The group $\widetilde{\mathcal{D}}=\left\langle \widetilde{\overline{Z_0}},D_{1},...,D_{36},\frac{\widetilde{\overline{Z_0}}+D_{1}+...+D_{36}}{2}\right\rangle$ is primitive in $H^{2}(\widetilde{K},\Z)$.
\item
$\overline{Z_{0}}'$ is primitive in $H^{2}(K',\Z)$,
%\item[3)]
%$\pi_{*}(e^{2})+\overline{Z_{0}}$ is divisible by 2.
\end{enumerate}
\end{lemme}
\begin{proof}
\begin{enumerate}
\item
It is proven using adjunction formula. 
It is the same statement as Proposition 4.6.16 1) of~\cite{Lol}.
\item
It follows directly from the Fujiki formula (\ref{fujiki}).
\item
By Lemma~\ref{primitive2}, $\pi_{2*}(s^*(H^2(K_2(A),\Z)))$ is primitive in $H^{2}(\widetilde{K},\Z)$. Then by Remark~\ref{commut2}, $r'^*(\pi_{*}(H^2(K_2(A),\Z)))$ is primitive in $H^{2}(K',\Z)$. Using again Remark~\ref{commut2}, we get the result.
\item[]
The proof of the last two points is the same as Lemma 4.6.14 of~\cite{Lol} and will be omitted. 
%%If $\overline{Z_{0}}'$ is divisible by 2, by Lemma~\ref{Fulton}, $r^{*}(\overline{Z_{0}})$ is divisible by 2. 
%%Since $r^{*}(\pi_{*}(H^{4}(K_{2}(A),\Z)))$ primitive in $H^{4}(K',\Z)$, it means that $Z_{0}=a+\iota^{*}(a)$ with $a\in H^{4}(K_{2}(A),\Z)$. 
%%But this is impossible since $Z_{0}\cdot \frac{u_{1}\cdot u_{2}+v_{1}\cdot v_{2}+w_{1}\cdot w_{2}}{2}=3$ is odd.
%%Hence $\overline{Z_{0}}'$ is primitive in $H^{2}(K',\Z)$.

%%Can also be proven as in proof of Lemma 2.33 of~\cite{Lol2}.
%%\item[(3)]
%Now assume that there is $b\in H^{2}(K_{2}(A),\Z)$ such that $r^{*}(\pi_{*}(b))+\overline{Z_{0}}'$ is divisible by 2.
%It follows by Lemma~\ref{Fulton} and Lemma 2.18 of~\cite{Lol} that $r^{*}(\pi_{*}(b^2-Z_{0}))$ is divisible by 2.
%Since $r^{*}(\pi_{*}(H^{4}(K_{2}(A),\Z)))$ primitive in $H^{4}(K',\Z)$, it means that $b^2-Z_{0}=a+\iota^{*}(a)$ with $a\in H^{4}(K_{2}(A),\Z)$. 
%%Moreover by Proposition 5.1 of~\cite{Kummer}, we can write $Z_{0}$ as follows:
%%$$Z_{0}=3c_{2}(K_{2}(A))-\sum_{\tau\in A[3]\smallsetminus \left\{0\right\}}Z_{\tau}.$$
%%Hence by Proposition 3.14~\ref{}:
%%$$Z_{0}=12u_{1}\cdot u_{2}+12v_{1}\cdot v_{2}+12w_{1}\cdot w_{2}-e^2-\sum_{\tau\in A[3]\smallsetminus \left\{0\right\}}Z_{\tau}.$$
%%It follows:
%%$$e^{2}+Z_{0}=12u_{1}\cdot u_{2}+12v_{1}\cdot v_{2}+12w_{1}\cdot w_{2}-\sum_{\tau\in A[3]\smallsetminus \left\{0\right\}}Z_{\tau}.$$
%It follows that:
%$$81 b^{2}-12u_{1}\cdot u_{2}+12v_{1}\cdot v_{2}+12w_{1}\cdot w_{2}+e^2=a'+\iota^{*}(a'),$$
%with $a'\in H^{4}(K_{2}(A),\Z)$.
%Hence we can write:
%$$b^{2}+e^2=a''+\iota^{*}(a''),$$
%with $a'\in H^{4}(K_{2}(A),\Z)$.
%Using the same method as in Section~\ref{H2}, it proves that 
%$$b=e \mod 2 H^{2}(K_{2}(A),\Z).$$
%%Hence $\pi_{*}(e^{2}+Z_{0})$ is divisible by 2.
\end{enumerate}
\end{proof}
With Lemma~\ref{Fulton} (iii) and (v), it only remains to prove that $\pi_{1*}(s_{1}^{*}(e))+\overline{Z_{0}}'$ is divisible by 2 which will be done in Lemma~\ref{dernierlemme}. To prove this lemma, we first prove that $\pi_{2*}(s^{*}(e))+\widetilde{\overline{Z_{0}}}$ is divisible by 2. Knowing that $\widetilde{\overline{Z_0}}+D_{1}+...+D_{36}$ is divisible by 2, we only have to show that $\pi_{2*}(s^{*}(e))+D_{1}+...+D_{36}$ is divisible by 2 which is done by Lemma~\ref{exist} and~\ref{Ddelta}.
%We have the following exact sequence:
%$$\xymatrix@C=15pt{H^{1}(U,\Z)\ar[r] &H^{2}(\widetilde{M},U,\Z)\ar[r]&H^{2}(\widetilde{M},\Z) \ar[r]&H^{2}(U,\Z) \ar[r]&H^{3}(\widetilde{M},U,\Z)
%}.$$
%By Proposition ?, we have $H^{1}(U,\Z)=0$ and $H^{2}(U,\Z)=\Z/2\Z$.
%Moreover, by Thom's isomorphism, we have $H^{3}(\widetilde{M},U,\Z)$ $\simeq H^{1}(\stackrel{2}{Z_{0}},\Z)\oplus(\oplus_{i=1}^{36}H^{1}(D_{i},\Z))=0$.
%We have also $H^{2}(\widetilde{M},U,\Z)$ $\simeq H^{0}(\stackrel{2}{Z_{0}},\Z)\oplus(\oplus_{i=1}^{36}H^{0}(D_{i},\Z))$.
%Then the exact sequence gives:
%$$H^{2}(U,\Z)\simeq H^{2}(\widetilde{M},\Z)/\left\langle \widetilde{\Sigma},D_{1},...,D_{28}\right\rangle.$$
%Let $\widetilde{\mathcal{D}}$ be the primitive overgroup of $\left\langle \stackrel{2}{Z_{0}},D_{1},...,D_{28}\right\rangle$ in $H^{2}(\widetilde{M},\Z)$. Since $H^{2}(U,\Z)=\Z/2\Z$, $\widetilde{\mathcal{D}}/\left\langle \stackrel{2}{Z_{0}},D_{1},...,D_{28}\right\rangle=\Z/2\Z$.
%But, we still know that $\stackrel{2}{Z_{0}}+D_{1}+...+D_{36}$ is divisible by $2$ by Proposition ?. So $\widetilde{\mathcal{D}}=\left\langle \widetilde{\Sigma},D_{1},...,D_{36},\frac{\stackrel{2}{Z_{0}}+D_{1}+...+D_{36}}{2}\right\rangle$.

First we need to know the group $H^{3}(\widetilde{K},\Z)$.
\begin{lemme}\label{H3}
We have $H^{3}(\widetilde{K},\Z)=0.$
\end{lemme}
\begin{proof}
We have the following exact sequence: 
$$
\xymatrix@C=10pt@R=0pt{H^{3}(K_2(A),V,\Z)\ar[r] &H^{3}(K_2(A),\Z)\ar[r]^{f} &H^{3}(V,\Z)\ar[r]& 
H^{4}(K_2(A),V,\Z)\ar[r]^{\rho} &H^{4}(K_2(A),\Z).}$$
By Thom isomorphism, $H^{3}(K_2(A),V,\Z)=0$ and $H^{4}(K_2(A),V,\Z)=H^{0}(Z_0,\Z)$.
Moreover $\rho$ is injective, so $H^{3}(V,\Z)=H^{3}(K_2(A),\Z)$. 

Hence by (\ref{l22}), (\ref{l3}) and Proposition 3.2.8 of~\cite{Lol}, we find that $H^{3}(U,\Z)=0$.
Since $H^{3}(K_2(A),\Z)^\iota=0$, $H^{3}(\widetilde{K},\Z)$ is a torsion group. 
Hence the result follows from the exact sequence
$$\xymatrix@C=10pt@R=0pt{H^{3}(\widetilde{K},U,\Z)\ar[r] &H^{3}(\widetilde{K},\Z)\ar[r] &H^{3}(U,\Z)}$$
and from the fact that $H^{3}(\widetilde{K},U,\Z)=0$ by Thom isomorphism.
\end{proof}
\begin{lemme}\label{exist}
There exists $D_e$ which is a linear combination of the $D_i$ with coefficient 0 or 1 such that $\pi_{2*}(s^{*}(e))+D_e$ is divisible by 2.
\end{lemme}
\begin{proof}
First, we have to use Smith theory as in Section 4.6.4 of~\cite{Lol}.

Look at the following exact sequence:
$$\xymatrix@C=10pt@R0pt{0\ar[r] &H^{2}(\widetilde{K},\widetilde{\overline{Z_{0}}}\cup(\cup_{k=1}^{36}D_{k}),\mathbb{F}_{2}))\ar[r]&H^{2}(\widetilde{K},\mathbb{F}_{2}) \ar[r]& H^{2}(\widetilde{\overline{Z_{0}}}\cup(\cup_{k=1}^{36}D_{k}),\mathbb{F}_{2}))\\
\ar[r]&H^{3}(\widetilde{K},\widetilde{\overline{Z_{0}}}\cup(\cup_{k=1}^{36}D_{k}),\mathbb{F}_{2})\ar[r] &0.\ \ \ \ \ \ \ \ \ &
}$$
First, we will calculate the dimension of the vector spaces $H^{2}(\widetilde{K},\widetilde{\overline{Z_{0}}}\cup(\cup_{k=1}^{36}D_{k}),\mathbb{F}_{2})$ and $H^{3}(\widetilde{K},$ $\widetilde{\overline{Z_{0}}}\cup(\cup_{k=1}^{36}D_{k}),\mathbb{F}_{2})$.
By (2) of Proposition~\ref{SmithProp}, we have 
$$H^{*}(\widetilde{K},\widetilde{Z_{0}}\cup(\cup_{k=1}^{36}D_{k}),\mathbb{F}_{2})\simeq H^{*}_{\sigma}(N_{2}).$$
%where $H^{*}_{\sigma}(N_{2})$ is the cohomology group of the complex $\sigma C_{*}(N_{2})$.

The previous exact sequence gives us the following equation:

$$h^{2}_{\sigma}(N_{2})-h^{2}(\widetilde{K},\mathbb{F}_{2})+h^{2}(\widetilde{Z_{0}}\cup(\cup_{k=1}^{36}D_{k}),\mathbb{F}_{2})-h^{3}_{\sigma}(N_{2})=0.$$
As $h^{2}(\widetilde{K},\mathbb{F}_{2})=8+36=44$ and $h^{2}(\widetilde{Z_{0}}\cup(\cup_{k=1}^{36}D_{k}),\mathbb{F}_{2})=23+36=59$, we obtain:
%h^{2}_{\sigma}(N_{2})-(16+28)+(23+28)-h^{3}_{\sigma}(N_{2})=0
$$h^{2}_{\sigma}(N_{2})-h^{3}_{\sigma}(N_{2})=-15.$$

%where $h^{*}_{\sigma}(\widetilde{X})$ denote the dimension over $\mathbb{F}_{2}$ of $H^{*}_{\sigma}(\widetilde{X})$.
Moreover by 2) of Proposition~\ref{SmithProp}, we have the exact sequence
$$\xymatrix@C=10pt@R0pt{0\ar[r] &H^{1}_{\sigma}(N_{2})\ar[r]&H^{2}_{\sigma}(N_{2}) \ar[r]&H^{2}(N_{2},\mathbb{F}_{2}) \ar[r]&H^{2}_{\sigma}(N_{2})\oplus H^{2}(\widetilde{Z_{0}}\cup(\cup_{k=1}^{36}E_{k}),\mathbb{F}_{2})\\
\ar[r]&H^{3}_{\sigma}(N_{2})\ar[r]&\coker\ar[r] &0.\ \ \ \ \ \ \ \ \ \ \  &
}$$
By Lemma 7.4 of~\cite{BNS}, $h^{1}_{\sigma}(N_{2})=h^{0}(\widetilde{Z_{0}}\cup(\cup_{k=1}^{36}E_{k}),\mathbb{F}_{2})-1$.
Then we get the equation
\begin{align*}
&h^{0}(\widetilde{Z_{0}}\cup(\cup_{k=1}^{36}E_{k}),\mathbb{F}_{2})-1-h^{2}_{\sigma}(N_{2})+h^{2}(N_{2},\mathbb{F}_{2})\\
&-h^{2}_{\sigma}(N_{2})-h^{2}(\widetilde{Z_{0}}\cup(\cup_{k=1}^{36}E_{k}),\mathbb{F}_{2})+h^{3}_{\sigma}(N_{2})-\alpha=0,\\
\end{align*}
where $\alpha=\dim \coker$.
%&28-h^{2}_{\sigma}(\widetilde{X})+(23+29)-h^{2}_{\sigma}(\widetilde{X})-(23+28)+h^{3}_{\sigma}(\widetilde{X})=0\\
So
$$21-\alpha-2h^{2}_{\sigma}(N_2)+h^{3}_{\sigma}(N_2)=0.$$
From the two equations, we deduce that
$$h^{2}_{\sigma}(N_{2})=36-\alpha,\ \ \ \ \ h^{3}_{\sigma}(N_{2})=51-\alpha.$$

Come back to the exact sequence
$$\xymatrix@C=15pt{0\ar[r] &H^{2}(\widetilde{K},\widetilde{\overline{Z_{0}}}\cup(\cup_{k=1}^{36}D_{k}),\mathbb{F}_{2})\ar[r]&\ar[r]^{\varsigma^{*}\ \ \ \ \ \ \  }H^{2}(\widetilde{K},\mathbb{F}_{2}) & H^{2}(\widetilde{\overline{Z_{0}}}\cup(\cup_{k=1}^{36}D_{k}),\mathbb{F}_{2}),
}$$
where $\varsigma:\widetilde{\overline{Z_{0}}}\cup(\cup_{k=1}^{36}D_{k})\hookrightarrow \widetilde{K}$ is the inclusion.
Since $h^{2}(\widetilde{K},\widetilde{\overline{Z_{0}}}\cup(\cup_{k=1}^{36}D_{k}),\mathbb{F}_{2})=h^{2}_{\sigma}(N_{2})=36-\alpha$, we have $\dim_{\mathbb{F}_{2}} \varsigma^{*}(H^{2}(\widetilde{K},\mathbb{F}_{2}))=(8+36)-36+\alpha=8+\alpha$.
We can interpret this as follows.
Consider the homomorphism
\begin{align*}
\varsigma^{*}_{\Z}:H^{2}(\widetilde{K},\Z)&\rightarrow H^{2}(\widetilde{\overline{Z_{0}}},\Z)\oplus (\oplus_{k=1}^{36} H^{2}(D_{k},\Z))\\
 u&\rightarrow (u\cdot\widetilde{\overline{Z_{0}}},u\cdot D_{1},...,u\cdot D_{36}).
\end{align*}
Since this is a map of torsion free $\Z$-modules (by Lemma~\ref{H3} and universal coefficient formula), we can tensor by $\mathbb{F}_{2}$,
$$\varsigma^{*}=\varsigma^{*}_{\Z}\otimes \id_{\mathbb{F}_{2}}: H^{2}(\widetilde{K},\Z)\otimes\mathbb{F}_{2}\rightarrow H^{2}(\widetilde{\overline{Z_{0}}},\Z)\oplus (\oplus_{k=1}^{36} H^{2}(D_{k},\Z))\otimes\mathbb{F}_{2},$$
and we have $8+\alpha$ independent elements such that the intersection with the $D_{k}$ $k\in\left\{1,...,36\right\}$ and $\widetilde{\overline{Z_{0}}}$ are not all zero.
But, $\varsigma^{*}(\pi_{2*}(H^{2}(N_2,\Z)))=0$ and $\varsigma^{*}(\widetilde{\overline{Z_0}},\left\langle D_1,...,D_{36}\right\rangle)$, (it follows from Proposition~\ref{commut}). By Lemma~\ref{Fulton} (iv), the element $\widetilde{\overline{Z_0}}+D_{1}+...+D_{36}$ is divisible by 2. Hence necessary, it remains $7+\alpha$ independent elements in $H^2(\widetilde{K},\Z)$ of the form  $\frac{u+d}{2}$ with $u\in \pi_{2*}(s^{*}(H^{2}(K_{2}(A),\Z)))$ and $d\in\left\langle D_1,...,D_{36}\right\rangle$. 

Let denote by $u_1,...,u_{7+\alpha}$ the $7+\alpha$ elements in $\pi_{2*}(s^{*}(H^{2}(K_{2}(A),\Z)))$ provided above.  
By Lemma~\ref{Fulton} (iv) $\left\langle D_1,...,D_{36}\right\rangle$ is primitive in $H^2(\widetilde{K},\Z)$. Hence necessary, the element $u_1,...,u_{7+\alpha}$ view as element in $\pi_{2*}(s^{*}(H^{2}(K_{2}(A),\mathbb{F}_{2})))$ are linearly independent. Since $\dim_{\mathbb{F}_{2}} \pi_{2*}(s^{*}(H^{2}(K_{2}(A),\mathbb{F}_{2})))=7$, it follows that $\alpha=0$ and $\Vect_{\mathbb{F}_{2}}(u_1,...,u_{7})=\pi_{2*}(s^{*}(H^{2}(K_{2}(A),\mathbb{F}_{2})))$.
Hence there exists $D_e$ which is a linear combination of the $D_i$ with coefficient 0 or 1 such that $\pi_{2*}(s^{*}(e))+D_e$ is divisible by 2.
\end{proof}
\begin{lemme}\label{Ddelta}
We have:
$$D_e=D_1+...+D_{36}.$$
\end{lemme}
\begin{proof}
We know from Remark~\ref{SPA2} that the image of the monodromy representation on $A[2]$ contains the symplectic group $\Sp A[2]$. 
We recall from Remark~\ref{RemarkSymplecticInv} (2), that the $D_1,...,D_{35}$ are given by $\pi_2(s^{-1}(\mathcal{P})$.
It follows that the image of the monodromy representation on $H^2(\widetilde{K},\Z)$ contains the isometries which act on $D_1,...,D_{35}$
as the elements $f$ of $A[2]$:
$$f\cdot \pi_2(s^{-1}(\{a_1,a_2,a_3\})=\pi_2(s^{-1}(\{f(a_1),f(a_2),f(a_3)\}),$$
and act trivially on $D_{36}$ and $\pi_{2*}(s^{*}(e))$. 
As explained in Remark~\ref{PlaneTriple} the 2 orbits of the action of $\Sp A[2]$ on the set $\mathfrak{D}\defIs \left\{D_1,...,D_{35}\right\}$ correspond to the two sets of isotropic and non-isotropic planes in $A[2]$. Hence by Proposition~\ref{OrbitesSp} (3), (4)  the action of $\Sp A[2]$ on the set $\mathfrak{D}$ has 2 orbits: one of 15 elements and another of 20 elements. 

On the other hand, as we mentioned in the end of Section~\ref{statement}, we can assume that $A=E_\xi\times E_\xi$ where $E_\xi$ is the elliptic curve introduced in Definition~\ref{elliptic6}. Hence there is the following automorphism group acting on $A$:
$$G\defIs \left\langle \left(
\begin{array}{cc}
\rho & 0\\
0 & 1 
\end{array} \right), 
\left(
\begin{array}{cc}
0 & 1\\
1 & 0 
\end{array} \right),
\left(
\begin{array}{cc}
1 & 1\\
0 & 1 
\end{array} \right)
 \right\rangle,$$
 where $\rho=e^{\frac{2i\pi}{6}}$.
The group $G$ extends naturally to an automorphism group of $N_2$ which we denote also $G$.  Moreover, the action of $G$ restricts to the set $\mathfrak{D}$. Then by Lemma~\ref{orbitesG} the action of $G$ on $\mathfrak{D}$ has 2 orbits: one of 5 elements and one of 30 elements. Also the group $G$ acts trivially on $D_{36}$ and on $\pi_{2*}(s^{*}(e))$. 

Hence the combined action of $G$ and $\Sp A[2]$ acts transitively on $\mathfrak{D}$.
Since $\pi_{2*}(s^{*}(e))$ is fixed by the action of $G$ and $\Sp A[2]$, $D_e$ has also to be fixed by the action of $G$ and $\Sp A[2]$ else it will contradict Lemma~\ref{Fulton} (iv).
It follows that there are only 3 possibilities for $D_e$: 
\begin{enumerate}
\item
$D_e=D_{36}$,
\item
$D_e=D_1+...+D_{35}$,
\item
or $D_e=D_1+...+D_{36}$.
\end{enumerate}
Let $d$ be the number of $D_i$ with coefficient equal to 1 in the linear decomposition of $D_e$. The number $d$ can be 1, 35 or 36.
%Let denote by $E_e$ be a linear combination of the $E_i$ with coefficient 0 or 1 such that $\pi_{2*}(E_e)=D_e$. Let $d$ be the number of $E_i$ with coefficient equal to 1. We know that $d$ can be 1, 35 or 36.

Then from Lemma~\ref{Fulton} (i), (ii) and Proposition~\ref{commut}
$$\left(\frac{\pi_{2*}(s^{*}(e))+D_e}{2}\right)^4=\frac{324-d}{2}.$$ 
Hence $d$ has to be divisible by 2.
It follows that $D_e=D_1+...+D_{36}$.
\end{proof} 
\begin{lemme}\label{dernierlemme}
The class $\pi_{1*}(s_1^{*}(e))+\overline{Z_{0}}'$ is divisible by 2.
\end{lemme}
\begin{proof}
We know that $\pi_{2,*}(s^{*}(e))+\widetilde{\overline{Z_{0}}}$ is divisible by 2.
Indeed by Lemma~\ref{Fulton} (iv), $\widetilde{\overline{Z_{0}}}+D_1+...+D_{36}$ is divisible by 2 and by Lemma~\ref{exist} and~\ref{Ddelta},
$\pi_{2,*}(s^{*}(e))+D_1+...+D_{36}$ is divisible by 2. 

We can find a Cartier divisor on $\widetilde{K}$ which corresponds to $\frac{\pi_{2*}(s^{*}(e))+\widetilde{\overline{Z_0}}}{2}$ and which does not meet 
$\cup_{k=1}^{36} D_k$.
Then this Cartier divisor induces a Cartier divisor on $K'$ which necessarily corresponds to half the cocycle $\pi_{1*}(s_{1}^{*}(e))$ $+\overline{Z_0}'$.
\end{proof}
\subsection{Calculation of \texorpdfstring{$B_{K'}$}{BB form of K'}}\label{beauK'}
We finish the proof of Theorem~\ref{theorem}, calculating $B_{K'}$. We continue using the notation provided in Section~\ref{nota}. 
\begin{lemme}\label{Zinter}
We have $$\overline{Z_{0}}'^{2}=-2r^{*}(\overline{Z_{0}}).$$
%We have $$\hat{Z_{0}}^{2}=-s_{1}^{*}(Z_{0}).$$
\end{lemme}
\begin{proof}
%Since $K_{2}(A)$ is hyperk�hler and $Z_{0}$ is a K3 surface, we have $c_{1}(\mathscr{N}_{Z_{0}/K_{2}(A)})=0$. Then it is the same proof as 
We use the same technique as in Lemma 4.6.12 of~\cite{Lol}.
Consider the following diagram:
$$\xymatrix{
Z_{0}' \ar[d]^{g}\ar@{^{(}->}[r]^{l_{1}}& N_{1} \ar[d]^{s_{1}}\\
   Z_{0} \ar@{^{(}->}[r]^{l_{0}}  & K_{2}(A),
   }$$
where $l_{0}$ and $l_{1}$ are the inclusions and $g\defIs s_{1|Z_0'}$.
By Proposition 6.7 of~\cite{Fulton}, we have:
$$s_{1}^{*}l_{0*}(Z_{0})=l_{1*}(c_{1}(E)),$$
where $E\defIs g^{*}(\mathscr{N}_{Z_{0}/K_{2}(A)})/\mathscr{N}_{Z_{0}'/N_{1}}$.
Hence $$s_{1}^{*}l_{0*}(Z_{0})=c_{1}(g^{*}(\mathscr{N}_{Z_{0}/K_{2}(A)}))-Z_{0}'^{2}.$$
Since $K_{2}(A)$ is hyperk�hler and $Z_{0}$ is a K3 surface, we have $c_{1}(\mathscr{N}_{Z_{0}/K_{2}(A)})=0$.
So
$$Z_{0}'^{2}=-s_{1}^{*}l_{0*}(Z_{0}).$$
Then the result follows from Proposition~\ref{commut}.
\end{proof}
\begin{prop}\label{passage}
We have the formula $$B_{K'}(\pi_{1*}(s_{1}^{*}(\alpha),\pi_{1*}(s_{1}^{*}(\beta)))=6\sqrt{\frac{2}{c_{K'}}}B_{K_2(A)}(\alpha,\beta),$$
where $c_{K'}$ is the Fujiki constant of $K'$ and $\alpha$, $\beta$ are in $H^{2}(K_2(A),\Z)^{\iota}$ and $B_{K_2(A)}$ is the Beauville--Bogomolov form of $K_2(A)$.
\end{prop} 
\begin{proof}
The ingredient for the proof is the Fujiki formula.

By (\ref{Fujiki}) of Theorem~\ref{Form}, we have $$(\pi_{1*}(s_{1}^{*}(\alpha)))^{4}=c_{K'}B_{K'}(\pi_{1*}(s_{1}^{*}(\alpha),\pi_{1*}(s_{1}^{*}(\alpha)))^{2}.$$
$$\alpha^{4}=9B_{K_2(A)}(\alpha,\alpha)^{2}.$$
Moreover, by Proposition~\ref{commut}, $$(\pi_{1*}(s^{*}(\alpha)))^{4}=8s^{*}(\alpha)^{4}=8\alpha^{4}.$$
By statement (\ref{Fujikiposi}) of Theorem~\ref{Form}, we get the result.
\end{proof}
In particular, it follows:
\begin{equation}
B_{K'}(\pi_{1*}(s_{1}^{*}(e),\pi_{1*}(s_{1}^{*}(e)))=-36\sqrt{\frac{2}{c_{K'}}}
\label{deltahaha}
\end{equation}
\begin{lemme}\label{ortho}
$$B_{K'}(\pi_{1*}(s_{1}^{*}(\alpha)),\overline{Z_0}')=0,$$
for all $\alpha\in H^{2}(S^{[2]},\Z)^{\iota}$.
\end{lemme}
\begin{proof}
We have $\pi_{1*}(s_{1}^{*}(\alpha))^{3}\cdot\overline{Z_0}'=8s_{1}^{*}(\alpha)^{3}\cdot\Sigma_{1}$ by Proposition~\ref{commut},
and $s_{1*}(s_{1}^{*}(\alpha^{3})\cdot Z_0')=\alpha^{3}\cdot s_{1*}(Z_0')=0$ by the projection formula.
We conclude by Proposition~\ref{beauville}.
\end{proof}
\begin{lemme}\label{Z2}
We have:
$$B_{K'}(\overline{Z_0}',\overline{Z_0}')=-4\sqrt{\frac{2}{c_{K'}}}.$$
\end{lemme}
\begin{proof}
We have:
\begin{align*}
\overline{Z_0}'^{2}\cdot\pi_{1*}(s_{1}^{*}(e))^{2}
&=\frac{c_{K'}}{3}B_{M'}(\overline{Z_0}',\overline{Z_0}')\times B_{K'}(\pi_{1*}(s_{1}^{*}(e)),\pi_{1*}(s_{1}^{*}(e)))\\
&=\frac{c_{K'}}{3}B_{K'}(\overline{Z_0}',\overline{Z_0}')\times\left(-36\sqrt{\frac{2}{c_{K'}}}\right)\\
\end{align*}
\vspace{-1cm}
\begin{equation}
=-12\sqrt{2c_{K'}}B_{K'}(\overline{Z_0}',\overline{Z_0}')\ \ \ \ \ \ 
\label{jenesaispas1}
\end{equation}
By Proposition~\ref{commut}, we have 
\begin{equation}
\overline{Z_0}'^{2}\cdot\pi_{1*}(s_{1}^{*}(e))^{2}=8Z_0'^{2}\cdot (s_{1}^{*}(e))^{2}.
\label{jenesaispas2}
\end{equation}
By the projection formula, 
$Z_0'^{2}\cdot (s_{1}^{*}(e))^{2}=s_{1*}(Z_0'^{2})\cdot e^{2}$.
Moreover by lemma~\ref{Zinter}, $s_{1*}(Z_0'^{2})=-Z_0$. 
Hence
\begin{equation}
Z_0'^{2}\cdot (s_{1}^{*}(e))^{2}=-Z_0\cdot e^{2}.
\label{jenesaispas3}
\end{equation}
It follows from (\ref{jenesaispas1}), (\ref{jenesaispas2}) and (\ref{jenesaispas3}) that
%Indeed, we can write
%$$\omega_{\Sigma_{1}}=\omega_{N_{1}}\otimes\mathscr{N}_{\Sigma_{1}/N_{1}}.$$
%The map $s_{|\Sigma_{1}}:\Sigma_{1}\rightarrow\Sigma$ is a $\mathbb{P}^{1}$-bundle, so $\omega_{\Sigma_{1}}=\mathcal{O}_{\Sigma_{1}}(-2u(\Sigma))$,
%where $u:\Sigma\hookrightarrow\Sigma_{1}$ is the embedding.
%We obtain:
%\begin{align*}
%\mathcal{O}_{\Sigma_{1}}(-2u(\Sigma))&=\mathcal{O}_{N_{1}}(\Sigma_{1})\otimes\mathcal{O}_{N_{1}}(\Sigma_{1})\otimes\mathcal{O}_{\Sigma_{1}}\\
%&=\mathcal{O}_{N_{1}}(2\Sigma_{1})\otimes\mathcal{O}_{\Sigma_{1}}.
%\end{align*}
%So $\Sigma_{1}^{2}=-s_{1}^{*}(\Sigma)$.
%Indead, we know that $\Sigma_{1}\rightarrow \Sigma$ is a $\mathbb{P}^{1}$-bundle, then the canonical bundle of $\Sigma_{1}$ is $K_{\Sigma_{1}}=-2\Sigma$ and the canonical bundle of $N_{1}$ is $K_{N_{1}}=\Sigma_{1}$.
\begin{equation}
-8Z_0\cdot e^{2}=-12\sqrt{2c_{K'}}B_{K'}(\overline{Z_0}',\overline{Z_0}').
\label{jenesaispas4}
\end{equation}
Moreover from Section 4 of~\cite{Hassett}, we have:
\begin{equation}
Z_0\cdot e^{2}=-12.
\label{jenesaispas5}
\end{equation}
So by (\ref{jenesaispas4}) and (\ref{jenesaispas5}):
\begin{equation*}
B_{K'}(\overline{Z_0}',\overline{Z_0}')=-8\sqrt{\frac{1}{2c_{K'}}}.
\qedhere\end{equation*}       
\end{proof}
Now we are able to finish the calculation of the Beauville--Bogomolov form on $H^{2}(K',\Z)$.
By (\ref{deltahaha}), Propositions~\ref{passage}, Lemma~\ref{ortho},~\ref{Z2} and Theorem~\ref{fin},
the Beauville--Bogomolov form on $H^{2}(K',\Z)$ gives the lattice:
$$ U^{3}\left(6\sqrt{\frac{2}{c_{K'}}}\right) \oplus -\frac{1}{4}\sqrt{\frac{2}{c_{K'}}}\left(
\begin{array}{cc}
40 & 32\\
32 & 40 
\end{array} \right)$$
$$=U^{3}\left(6\sqrt{\frac{2}{c_{K'}}}\right) \oplus -\sqrt{\frac{2}{c_{K'}}}\left(
\begin{array}{cc}
10 & 8\\
8 & 10 
\end{array} \right)$$
Then it follows from the integrality and the indivisibility of the Beauville--Bogomolov form that $c_{K'}=8$, and we get Theorem~\ref{theorem}.
\subsection{Betti numbers and Euler characteristic of \texorpdfstring{$K'$}{K'}}
\begin{prop}\label{b}
We have:
\begin{itemize}
\item $b_2(K')=8$,
\item $b_{3}(K')=0,$
\item $b_{4}(K')=90,$
\item $\chi(K')=108.$
\end{itemize}
\end{prop}
\begin{proof}
It is the same proof as Proposition 4.7.2 of~\cite{Lol}.
From Theorem 7.31 of~\cite{Voisin}, (\ref{l22}), (\ref{l3}) and Proposition~\ref{invariants}, we get the betti numbers.
Then $\chi(K')=1-0+8-0+90-0+8-0+1=108$.
\end{proof}
\newpage
\part{Computing Cup-Products in integral cohomology of Hilbert schemes of points on K3 surfaces}
\documentclass{amsart}

\usepackage{amsmath,amssymb,amsfonts}
\usepackage[all]{xy}

\DeclareMathOperator{\rank}{rank}
\DeclareMathOperator{\trace}{tr}
\DeclareMathOperator{\Tor}{Tor}
\DeclareMathOperator{\Ext}{Ext}
\DeclareMathOperator{\Aut}{Aut}
\DeclareMathOperator{\End}{End}
\DeclareMathOperator{\id}{id}
\DeclareMathOperator{\Hom}{Hom}
\DeclareMathOperator{\im}{Im}
\DeclareMathOperator{\Ker}{Ker}
\DeclareMathOperator{\Sym}{Sym}
\DeclareMathOperator{\Hilb}{Hilb}

\newcommand{\hilb}[1]{^{[#1]}}
\newcommand{\ie}{{\it i.e. }}
\newcommand{\eg}{{\it e.g. }}
\newcommand{\loccit}{{\it loc. cit. }}
\newcommand{\ii}{{\rm i}}
\newcommand{\dual}[1]{{#1}\spcheck}
\newcommand{\abs}[1]{|{#1}|}
\newcommand{\Kummer}[2]{{#2}^{\llbracket#1\rrbracket}}
\newcommand{\vac}{|0\rangle}
\newcommand{\odd}{{\rm{odd}}}
\newcommand{\even}{{\rm{even}}}
\newcommand{\tors}{{\rm{tors}}}

\newcommand{\aG}{{\rm{a}}_G}
\newcommand{\mG}{{\rm{m}}_G}

% for spectral sequences

\newcommand{\coloneqq}{:=}

\makeatletter
\newcommand{\rmnum}[1]{\romannumeral #1}
\newcommand{\Rmnum}[1]{\expandafter\@slowromancap\romannumeral #1@}
\makeatother


\newcommand{\FI}{F_{\text{\Rmnum{1}}}}
\newcommand{\FII}{F_{\text{\Rmnum{2}}}}
\newcommand{\EI}{\text{}^{\text{\Rmnum{1}}}\!E}
\newcommand{\EII}{\text{}^{\text{\Rmnum{2}}}\!E}
\newcommand{\dI}{\text{}^{\text{\Rmnum{1}}}\!d}
\newcommand{\dII}{\text{}^{\text{\Rmnum{2}}}\!d}

% for total derived functors and hypercohomology

\newcommand{\HH}{\mathbf{H}}
\newcommand{\LL}{\mathbf{L}}
\newcommand{\RR}{\mathbf{R}}

% for equivariant cohomology

\newcommand{\BG}{BG}
\newcommand{\EG}{EG}


%%%%%%%%%%%%%%%%%%%%%%%%%%%%%%

\newcommand{\IC}{\mathbb{C}}
\newcommand{\IR}{\mathbb{R}}
\newcommand{\IQ}{\mathbb{Q}}
\newcommand{\IZ}{\mathbb{Z}}


%%%%%%%%%%%%%%%%%%%%%%%%%%%%%

\newcommand{\kS}{\mathfrak{S}}

\newcommand{\km}{\mathfrak{m}}
\newcommand{\kq}{\mathfrak{q}}

%%%%%%%%%%%%%%%%%%%%%%%%%%%%%%

\newcommand{\lra}{\longrightarrow}
\newcommand{\ra}{\rightarrow}

%%%%%%%%%%%%%%%%%%%%%%%%%%%%%

\theoremstyle{plain}
\newtheorem{theorem}{Theorem}[section]
\newtheorem{lemma}[theorem]{Lemma}
\newtheorem{proposition}[theorem]{Proposition}
\newtheorem{corollary}[theorem]{Corollary}
\theoremstyle{definition}
\newtheorem{definition}[theorem]{Definition}
\newtheorem{notation}[theorem]{Notation}
\theoremstyle{remark}
\newtheorem{remark}[theorem]{Remark}
\newtheorem{example}[theorem]{Example}




%%%%%%%%%%%%%%%%%%%%%%%%%%%%%

\begin{document}

\title[Products in $H^\ast(\Hilb^n(K3), \IZ)$]{Computing Cup-Products in integer cohomology of Hilbert schemes of points on K3 surfaces}


\author{Simon Kapfer}
\address{Simon Kapfer, Lehrstuhl f\"ur Algebra und Zahlentheorie, Universit\"ats\-stra{\ss}e~14, D-86159 Augsburg}
\email{simon.kapfer@math.uni-augsburg.de}
%\urladdr{http://www.math.uni-augsburg.de/alg/}


\date{\today}

%\subjclass{Primary 14J50, Secondary 14C50, 55T10}

%\keywords{}

\begin{abstract} 
We study the images of cup 
products in integer cohomology of the Hilbert scheme of $n$ points on a K3 surface. 
\end{abstract}

\maketitle


%%%%%%%%%%%%%%%%%%%%%%%%%%%%%%%%%%%%%%%%%%%%%%%%%%%%%%%%%%%%%%%%%%
%%%%%%%%%%%%%%%%%%%%%%%%%%%%%%%%%%%%%%%%%%%%%%%%%%%%%%%%%%%%%%%%%%
%%%%%%%%%%%%%%%%%%%%%%%%%%%%%%%%%%%%%%%%%%%%%%%%%%%%%%%%%%%%%%%%%%
\section{Preliminaries}
\begin{definition}
Let $n$ be a natural number. A partition of $n$ is a sequence $\lambda = (\lambda_1\geq\ldots\geq\lambda_k>0)$ of natural numbers such that $\sum_i \lambda_i =n$. It is convenient to write $\lambda = (1^{m_1},2^{m_2},\ldots)$ as a sequence of multiplicities. We define the weight $\|\lambda\| :=\sum m_i i =n$ and the length $|\lambda| := \sum_i m_i =k$. We also define $z_\lambda \coloneqq\prod_i i^{m_i} m_i!$. 
\end{definition}
\begin{definition} \label{SymFun}
Let $\Lambda$ be the ring of symmetric functions. Let $m_\lambda$ and $p_\lambda$ denote the monomial and the power sum symmetric functions. They are both indexed by partitions and form a basis of $\Lambda$. They are linearly related by $p_\lambda = \sum_{\mu} \psi_{\lambda\mu}m_\mu$, the sum being over partitions with the same weight as $\lambda$, hence finite. The base change matrix $(\psi_{\lambda\mu})$ has integral entries, but its inverse $(\psi_{\lambda\mu}^{-1)}$ has not. For example, $p_{(2,1,1)} = 2m_{(2,1,1)} + 2m_{(2,2)}+2m_{(3,1)}+m_{(4)} $ but $ m_{(2,1,1)} = \frac{1}{2}p_{(2,1,1)} -\frac{1}{2}m_{(2,2)} -p_{(3,1)}+p_{(4)}$. A method to determine the coefficients $(\psi_{\lambda\mu})$ is given in \cite[Sect. 3.7]{Lascoux}.
\end{definition}
\begin{definition}
Let $S$ be a projective K3 surface. We fix integral bases $1$ of $H^0(S,\IZ)$, $x$ of $H^4(X,\IZ)$ and $\alpha_1,\ldots ,\alpha_{22}$ of $H^2(S,\IZ)$. The cup product induces a symmetric bilinear form $B_{H^2}$ on $H^2(X,\IZ)$ and thus the structure of a lattive isomorphic to $U^{\oplus 3}\oplus E_8(-1)^{\oplus 2}$, \ie three times the hyperbolic lattice and two times the negative $E_8$ lattice.
We may extend $B_{H^2}$ to a symmetric non-degenerate bilinear form on $H^\ast(S,\IZ)$ by setting $ B(1,1) = 0,\ B(1,\alpha_i) = 0,\ B(1,x) = 1, \ B(x,x) = 0$.
\end{definition}
\begin{definition}
$B$ induces a form $B\otimes B$ on $\Sym^2H^\ast(S,\IZ)$. So the cup-product 
\begin{equation*}
\mu : \Sym ^2H^{*}(S,\IZ) \longrightarrow H^\ast(S,\IZ) 
\end{equation*}
has an adjoint comultiplication $\Delta$ that is coassociative, given by:
\begin{equation*}
\Delta : H^\ast(S,\IZ) \longrightarrow \Sym^2H^\ast(S,\IZ),\quad \Delta = (B\otimes B)^{-1}\mu^TB
\end{equation*}
The image of 1 under the composite map $\mu(\Delta(1)) = B(\Delta(1),\Delta(1)) = 24 x$, denoted by $e$ is called the Euler Class.
\end{definition}
We denote by $S\hilb{n}$ the Hilbert scheme of $n$ points on $S$, \ie the classifying space of all zero-dimensional closed subschemes of length $n$, which is smooth. 
A classical result by Nakajima gives an explicit description of $H^\ast(S\hilb{n},\IQ)$ in terms of creation operators
$\kq_l(\beta)$, 
$\beta\in H^\ast(S,\IQ)$, acting on the direct sum 
$\bigoplus_n H^\ast(S\hilb{n},\IQ)$. 
An integral basis for $H^\ast(S\hilb{n},\IZ)$ in terms of Nakajima's operators was given by Qin--Wang:
\begin{theorem} \cite[Thm. 5.4.]{QinWang} Let $\km_{\nu,\alpha} := \sum_\rho \psi_{\nu\rho}^{-1}\,\kq(\alpha)$, with coefficients as in Definition \ref{SymFun}. The classes
$$ \frac{1}{z_\lambda} \kq_\lambda(1)\kq_\mu(x)\km_{\nu^1,\alpha_1}\ldots\km_{\nu^{22},\alpha_{22}}\vac,\quad \|\lambda\| +\|\mu\| + \sum_{i=1}^{22}\|\nu^i\| = n
$$ 
form an integral basis for $H^\ast(S\hilb{n},\IZ)$. Here,
$\lambda,\; \mu,\; \nu^i$ are partitions.
\end{theorem}
\begin{notation}
To enumerate the basis of $H^\ast(S\hilb{n},\IZ)$, we introduce the following abbreviation:
$$ 
1^\lambda \alpha_1^{\nu^1}\ldots\alpha_{22}^{\nu^{22}}x^\mu :=
\frac{1}{z_{\tilde{\lambda}} }
\kq_{\tilde{\lambda}}(1)\kq_\mu(x)\km_{\nu^1,\alpha_1}\ldots\km_{\nu^{22},\alpha_{22}}\vac
$$
where the partition $\tilde{\lambda}$ is built from $\lambda$ by appending sufficiently many Ones, such that $\|\tilde{\lambda}\| +\|\mu\| + \sum\|\nu^i\| = n $. If $\|\lambda\| +\|\mu\| + \sum\|\nu^i\| > n, $ we put $1^\lambda \alpha_1^{\nu^1}\ldots\alpha_{22}^{\nu^{22}}x^\mu =0$.
\end{notation}
The ring structure of $H^\ast(S\hilb{n}, \IQ)$ has been studied in \cite{LehnSorger}, where an explicit algebraic model is constructed, which we recall briefly:
\begin{definition} \cite[Sect. 2]{LehnSorger}
Let $\pi$ be a permutation of $n$ letters, written as a sum of disjoint cycles. To each cycle we may associate an element of $A:=H^\ast(S,\IQ)$. This defines an element in $A^{\otimes m}$, $m$ being the number of cycles. So these mappings span a vector space over $\IQ$. The space obtained by taking the direct sum over all $\pi \in S_n$ will be denoted by $A\{S_n\}$. 

To define a ring structure, take two permutations $\pi,\tau$, together with mappings. The result of the multiplication will be the permutation $\pi\tau$, together with a mapping of cycles. To construct the mappings to $A$, look first at the orbit space of the group of permutations $\left<\pi,\tau\right>$, generated by $\pi$ and $\tau$. For each cycle of $\pi, \tau$ contained in one orbit $B$ of $\left<\pi,\tau\right>$, multiply with the associated element of $A$. Also multiply with a certain power of the Euler class $e^g$. 
Afterwards, apply the comultiplication $\Delta$ repeatedly on the product to get a mapping from the cycles of $\pi\tau$ contained in $B$ to $A$. 
Here the "graph defect" $g$ is defined as follows: Let $u,v,w$ be the number of cycles contained in $B$ of $\pi,\;\tau,\;\pi\tau$, respectively. Then $g:=\frac{1}{2}\left(|B| + 2- u-v-w\right)$. Now follow this procedure for each orbit $B$. 

\end{definition}
The symmetric group $S_n$ acts on $A\{S_n\}$ by conjugation. This action preserves the ring structure. Therefore the space of invariants $A\hilb{n} := \left(A\{S_n\} \right)^{S_n}$ becomes a subring. The main theorem of Lehn and Sorger can now be stated:
\begin{theorem} \cite[Thm. 3.2.]{LehnSorger}
The following map is an isomorphism of rings:
\begin{align*}
H^\ast(S\hilb{n},\IQ) & \longrightarrow A\hilb{n} \\
\kq_{n_1}(\beta_1)\ldots \kq_{n_k}(\beta_k) \vac &\longmapsto \sum_{\sigma\in S_n} \sigma a \sigma^{{-}1} 
\end{align*}
with $n_1+\ldots + n_k =n$ and $a\in A\{S_n\} $ corresponds to an arbitrary permutation with $k$ cycles of lengths $n_1,\ldots,n_k$ that are associated to the classes $\beta_1,\ldots,\beta_k \in H^\ast(S,\IQ)$, respectively.
\end{theorem}

Since $H^\text{odd}(S\hilb{n},\IZ) = 0$ and $H^\text{even}(S\hilb{n},\IZ)$ is torsion-free by \cite{Markman}, we can apply these results to $H^\ast(S\hilb{n}, \IZ)$ to determine the multiplicative structure of cohomology with integer coefficients.

\section{Computational results} 
We now give some results in low degrees, obtained by computing multiplication matrices with respect to the integral basis and a then reducing to Smith normal form (both done by a computer).
\begin{remark}
Denote $h^k(S\hilb{n})$ the rank of $H^k(S\hilb{n},\IZ)$. We have:
\begin{itemize}
\item $h^2(S\hilb{n}) = 23 $ for $n\geq 2$.
\item $h^4(S\hilb{n}) = 276,\; 299,\; 300$ for $n=2,3, \geq 4$ resp.
\item $h^6(S\hilb{n}) = 23,\; 2554,\; 2852,\; 2875,\; 2876$ for $n=2,3,4,5,\geq6$ resp.
\end{itemize}
\end{remark}

The algebra generated by classes of degree 2 is an interesting object to study. For cohomology with complex coefficients, Verbitsky has proven in \cite{Verbitsky} that the cup product mapping from $\Sym^k H^2(S\hilb{n},\IC)$ to $H^{2k}(S\hilb{n},\IC)$ is injective for $k\leq n$. One concludes that this also holds for integral coefficients.
\begin{proposition} Studying the image of $\Sym^2H^2$ in $H^4$, we obtain: 
$$
\frac{H^4(S\hilb{2},\IZ)}{\Sym^2 H^2(S\hilb{2},\IZ)}  \cong \left(\frac{\IZ}{2\IZ}\right)^{\oplus 23} \oplus \frac{\IZ}{5\IZ}
$$
This was already known to Boissi\`{e}re, Nieper-Wi\ss kirchen and Sarti, \cite[Prop. 3]{BNS}.
$$
\frac{H^4(S\hilb{3},\IZ)}{\Sym^2 H^2(S\hilb{3},\IZ)} \cong \frac{\IZ}{3\IZ} \oplus \IZ^ {\oplus 23} 
$$
The torsion part of the quotient is generated by the integral class $1^{(3)}$.
$$
\frac{H^4(S\hilb{n},\IZ)}{\Sym^2 H^2(S\hilb{n},\IZ)} \cong  \IZ^ {\oplus 24}, \quad \text{for }n\geq 4.
$$
This was already proven by Markman, \cite[Thm. 1.10]{Markman2}.
\end{proposition}
\begin{proposition} Studying triple products of $H^2(S\hilb{n},\IZ)$, we get:
$$
\frac{H^6(S\hilb{2},\IZ)}{\Sym^3 H^2(S\hilb{2},\IZ)} \cong 
\frac{\IZ}{2\IZ}
$$
The quotient is generated by the integral class $1^{(2)}$.
$$
\frac{H^6(S\hilb{3},\IZ)}{\Sym^3 H^2(S\hilb{3},\IZ)} \cong  \left(\frac{\IZ}{2\IZ}\right)^{\oplus 230}\oplus \left(\frac{\IZ}{36\IZ}\right)^{\oplus 22}\oplus \frac{\IZ}{72\IZ} \oplus \IZ^{\oplus 507}
$$
$$
\frac{H^6(S\hilb{4},\IZ)}{\Sym^3 H^2(S\hilb{4},\IZ)} \cong  \frac{\IZ}{2\IZ} \oplus \IZ^{\oplus 552}
$$
For $n\geq 5$, the quotient is free.
\end{proposition}
We study now cup products between classes of degree 2 and 4. The case of $S\hilb{n}$ is of particular interest.
\begin{proposition} Comparing $H^2(S\hilb{n},\IZ) \cup H^4(S\hilb{n},\IZ) $ with $H^6(S\hilb{n},\IZ) $, we obtain:
\begin{align} 
\frac{H^6(S\hilb{2},\IZ) }{H^2(S\hilb{2},\IZ)\cup H^4(S\hilb{2},\IZ)} &= 0 
\\
\frac{H^6(S\hilb{3},\IZ)}{H^2(S\hilb{3},\IZ)\cup H^4(S\hilb{3},\IZ)} &\cong \left(\frac{\IZ}{3\IZ}\right)^{\oplus 22} \oplus \frac{\IZ}{3\IZ}
\\
\frac{H^6(S\hilb{4},\IZ)}{H^2(S\hilb{4},\IZ)\cup H^4(S\hilb{4},\IZ)} &\cong  \left(\frac{\IZ}{6\IZ}\right)^{\oplus 22}\oplus\frac{\IZ}{108\IZ} \oplus\frac{\IZ}{2\IZ} 
\\
\frac{H^6(S\hilb{5},\IZ)}{H^2(S\hilb{5},\IZ)\cup H^4(S\hilb{5},\IZ)} &\cong 
 \IZ^{\oplus 22} \oplus \IZ
\\
\frac{H^6(S\hilb{n},\IZ)}{H^2(S\hilb{n},\IZ)\cup H^4(S\hilb{n},\IZ)} &\cong 
 \IZ^{\oplus 22} \oplus \IZ\oplus\IZ, \ n\geq 6.
\end{align}
In each case, the first 22 parts of the quotient are generated by the integral classes 
 $$
\alpha_i^{(1,1,1)} -3\cdot \alpha_i^{(2,1)} + 3\cdot \alpha_i^{(3)}+ 3 \cdot 1^{(2)}\alpha_i^{(1,1)} -6\cdot 1^{(2)}\alpha_i^{(2)}+6\cdot 1^{(2,2)}\alpha_i^{(1)}-3\cdot 1^{(3)}\alpha_i^{(1)},
$$ 
for $ i=1\ldots 22$. Now define an integral class
\begin{align*}
K:=&\;\sum_{i\neq j} B(\alpha_i,\alpha_j)\left[\alpha_i^{(1,1)}\alpha_j^{(1)} - 2\cdot\alpha_i^{(2)}\alpha_j^{(1)}+\frac{3}{2}\cdot 1^{(2)}\alpha_i^{(1)}\alpha_j^{(1)} \right] +\\
+&\;\sum_{i}B(\alpha_i,\alpha_i)\left[\alpha_i^{(1,1,1)} - 2\cdot\alpha_i^{(2,1)} + \frac{3}{2}\cdot 1^{(2)}\alpha_i^{(1,1)} \right]+  x^{(2)}-1^{(2)}x^{(1)}.
\end{align*} 
In the case $n=3$, the last part of the quotient is generated by $K$. 
\\In the case $n=4$, the class $ 1^{(4)}$ generates the 2-torsion part and $K+38\cdot1^{(4)}$ generates the 108-torsion part.
\\In the case $n=5$, the last part of the quotient is generated by $K + 16\cdot 1^{(4)} - 21\cdot 1^{(3,2)}$.\\
If $n\geq 6$, the two last parts of the quotient are generated by some multiples of $K -\frac{4}{3}(45-n)1^{(2,2,2)} + (48-n)1^{(3,2)}$ and $K-\frac{1}{2}(40-n)1^{(2,2,2)}+ \frac{1}{4}(48-n)1^{(4)}$.
\end{proposition}


\bibliographystyle{amsplain}
\bibliography{BiblioAutIHS}


\end{document}


\lstset{
  language={Haskell},
  basicstyle=\tiny,
  tabsize=2,
  basewidth=0.53em
}

\appendix

\section{Source Code for the combinatorial model of Lehn and Sorger}\label{IntCode}
We give the source code for our tool implementing the integral cohomology of Hilbert schemes of points on K3 surfaces. It is available online under \url{https://github.com/s--kapfer/HilbK3}. We used the language Haskell, compiled with the \textsc{GHC} software, version 7.6.3. We make use of two external packages: \textsc{permutation} and \textsc{MemoTrie}. The project is divided into 4 modules. 

\subsection{How to use the code}
The main module is in the file \verb|HilbK3.hs|, which can be opened by \textsc{GHCI} for interactive use. It provides an implementation of the ring structure of $A\hilb{n}=H^*(S\hilb{n},\IQ)$, for all $n\in\mathbb{N}$. 
%There are multiplication methods for two kinds of classes: the classes of the form $\kq_{n_1}(\beta_1)\ldots\kq_{n_k}(\beta_k)\vac$ and those of the form given in Theorem \ref{QinWangTheorem}.
It computes cup--products in reasonable time up to $n=8$.
A product of Nakajima operators is represented by a pair consisting of a partition of length $k$ and a list of the same length, filled with indices for the basis elements of $H^*(S)$. For example, the class
$$
\kq_{3}(\alpha_6)\kq_{3}(\alpha_7)\kq_2(x)\kq_{1}(\alpha_2)\kq_{1}(1)^2 \vac 
$$
in $H^{20}(S\hilb{11})$ is written as
\begin{verbatim}
*HilbK3> (PartLambda [3,3,2,1,1,1], [6,7,23,2,0,0]) :: AnBase 
\end{verbatim}
Note that the classes $1\in H^0(S)$ and $x\in H^4(S)$ have indices \verb|0| and \verb|23| in the code.
The multiplication in $A\hilb{n}$ is implemented by the method \verb|multAn|. 
%For instance,
%\begin{verbatim}
%*HilbK3> let a = (PartLambda [2,1], [5,0]) :: AnBase 
%*HilbK3> let b = (PartLambda [2,1], [6,0]) :: AnBase  
%*HilbK3> multAn a b 
%[(([1-1-1],[23,23,0]),-2),(([3],[23]),4)]
%\end{verbatim}

The classes from Theorem \ref{QinWangTheorem} are represented in the same format, as shown in the following example. The multiplication in $H^*(S\hilb{n},\IZ)$ of such classes is implemented by the method \verb|cupInt|.
\begin{example}\label{exampleSource} We want to compute the results from Example \ref{example}. We only do one particular instance for every example, since the others are similar. By Corollary \ref{stabCor}, it suffices to know the values for finitely many $n$ to deduce the general case.
\begin{enumerate}
\item We do the case $n=6,\ i=1$. 
\begin{Verbatim}[fontsize=\small]
*HilbK3> let i = 1 :: Int
*HilbK3> let x = (PartLambda [2,2,1,1], [0,0,0,0]) :: AnBase
*HilbK3> let y = (PartLambda [2,1,1,1,1], [i,0,0,0,0]) :: AnBase
*HilbK3> cupInt x y
[(([2-1-1-1-1],[0,23,1,0,0]),-2),(([2-2-2],[1,0,0]),1),
(([3-2-1],[1,0,0]),2),(([4-1-1],[1,0,0]),1)]
\end{Verbatim}
\item We do the case $n=4,\ i=j=1$. 
\begin{Verbatim}[fontsize=\small]
*HilbK3> let i = 1 :: Int; let j = 1 :: Int
*HilbK3> let x = (PartLambda [2,1,1], [i,0,0]) :: AnBase
*HilbK3> let y = (PartLambda [1,1,1,1], [j,0,0,0]) :: AnBase
*HilbK3> cupInt x y
[(([2-1-1],[1,1,0]),1),(([3-1],[1,0]),1)]
\end{Verbatim}
\item We do the case $n=4$. 
\begin{Verbatim}[fontsize=\small]
*HilbK3> let d = (PartLambda [2,1,1], [0,0,0]) :: AnBase
*HilbK3> let y = (PartLambda [1,1,1,1], [23,0,0,0]) :: AnBase
*HilbK3> [ t | t <- cupInt d d, fst t == y]
[(([1-1-1-1],[23,0,0,0]),-3)]
\end{Verbatim}
\item We do the case $n=5$. 
\begin{Verbatim}[fontsize=\small]
*HilbK3> let x = (PartLambda [2,2,1], [0,0,0]) :: AnBase
*HilbK3> let y = (PartLambda [1,1,1,1,1], [23,23,0,0,0]) :: AnBase
*HilbK3> [ t | t <- cupInt x x, fst t == y]
[(([1-1-1-1-1],[23,23,0,0,0]),3)]
\end{Verbatim}
%\item We set $i=1,\ j= 2,\ k=2, \ n=6$.
%\begin{Verbatim}[fontsize=\small]
%*HilbK3> let x = (PartLambda [1,1,1,1,1,1], [1,2,23,0,0,0]) :: AnBase
%*HilbK3> let y = (PartLambda [1,1,1,1,1,1], [23,23,23,23,0,0]) :: AnBase
%*HilbK3> [ t | t <- cupInt x x, fst t == y]
%[(([1-1-1-1-1-1],[23,23,2,1,0,0]),1),(([1-1-1-1-1-1],[23,23,23,0,0,0]),1)]
%\end{Verbatim}
\end{enumerate}
\end{example}

\subsection{What the code does}
The goal is to multiply two elements in $H^*(S\hilb{n},\IZ)$. To do this, one has to execute the following steps:
\begin{enumerate}
 \item Compute the base change matrices $\psi_{\rho\nu}$ and $\psi_{\nu\rho}^{-1}$ between monomial and power sum symmetric functions.
 \item Provide a basis and the ring structure of $A=H^*(S,\IZ)$.
 \item Create a data structure for elements in $A\hilb{n}$ and $A\{S_n\}$.
 \item Implement the multiplication in $A\{S_n\}$, \ie the map $m_{\pi,\tau}$ from Definition \ref{model}.
 \item Implement the symmetrisation $A\hilb{n} = A\{S_n\}^{S_n}$.
 \item Use the isomorphism from Theorem \ref{LSThm} to get the ring structure of $A\hilb{n}$.
 \item Write an element in $H^*(S\hilb{n},\IZ)$ as a linear combination of products of creation operators acting on the vacuum, using Theorem \ref{QinWangTheorem}.
\end{enumerate}
We now describe, where to find these steps in the code.
\begin{enumerate}
 \item The $\psi_{\rho\nu}$ are computed by the function \verb|monomialPower| in the module \verb|SymmetricFunctions.hs|, using the theory from \cite[Sect.~3.7]{Lascoux}. The idea is to use the scalar product on the space of symmetric functions, so that the power sums become orthogonal: $(p_\lambda,p_\mu) = z_\lambda \delta_{\lambda\mu}$. The values for $(p_\lambda,m_\mu)$ are given by \cite[Lemma~3.7.1]{Lascoux}, so we know how to get the matrix $\psi_{\nu\rho}^{-1}$. Since it is triangular with respect to some ordering of partitions, matrix inversion is easy.
 \item The ring structure of $H^*(S,\IZ)$ is stored in the module \verb|K3.hs|. The only nontrivial multiplications are the products of two elements in $H^2(S,\IZ)$, where the intersection matrix is composed by the matrices for the hyperbolic and the $E_8$ lattice. The cup product and the adjoint comultiplication from Definition~\ref{comult} are implemented by the methods \verb|cup| and \verb|cupAd|.
 \item The data structures for basis elements of $A\hilb{n}$ and $A\{S_n\}$ are given by \verb|AnBase| and \verb|SnBase| in the module \verb|HilbK3.hs|. Linear combinations of basis elements are always stored as lists of pairs, each pair consisting of a basis element and a scalar factor.
 \item The function $m_{\pi,\tau}$ from Definition~\ref{model} is computed by the method \verb|multSn|. It contains the following substeps: First, the orbits of $\left<\pi,\tau\right>$ are computed recursively by glueing together the orbits of $\pi$ if they have both non-emtpty intersection with an orbit of $\tau$. Second, the composition $\pi\tau$ is computed using a method from the external library \verb|Data.Permute|. Third, the functions $f^{\pi,\left<\pi,\tau\right>}$ and $f_{\left<\pi,\tau\right>,\pi\tau}$ using the (co--)products from \verb|K3.hs|.
 \item The symmetrisation morphism is implemented by \verb|toSn|. We don't konw a better way to do this than the naive approach which is summation over all elements in $S_n$.
 \item The multiplication in $A\hilb{n}$ is carried out by the method \verb|multAn|.
 \item The base change matrices between the canonical base of $A\hilb{n}$ and the base of $H^*(S\hilb{n},\IZ)$ are given by \verb|creaInt| and \verb|intCrea|. By composing \verb|multAn| with these matrices, one gets the desired multiplication in $H^*(S\hilb{n},\IZ)$, called \verb|cupInt|.
\end{enumerate}

\subsection{Module for cup product structure of K3 surfaces} 
Here the hyperbolic and the $E_8$ lattice and the bilinear form on the cohomology of a K3 surface are defined. Furthermore, cup products and their adjoints are implemented.
\input{Sexy/K3.tex}
\subsection{Module for handling partitions} 
This module defines the data structures and elementary methods to handle partitions. We define both partitions written as descending sequences of integers ($\lambda$-notation) and as sequences of multiplicities ($\alpha$-notation).
\input{Sexy/Partitions.tex}
\subsection{Module for coefficients on Symmetric Functions} 
This module provides nothing but the base change matrices $\psi_{\lambda\mu}$ and $\psi^{-1}_{\mu\lambda}$ from Definition \ref{SymFun}.
\input{Sexy/SymmetricFunctions.tex} 
\subsection{Module implementing cup products for Hilbert schemes} This is our main module. We implement the algebraic model developed by Lehn and Sorger and the change of base due to Qin and Wang. The cup product on the Hilbert scheme is computed by the function \texttt{cupInt}.
\input{Sexy/HilbK3.tex}
 


\section{Source Code for the operator model}\label{VertexCode}
We give the source code for our tool implementing the rational cohomology of Hilbert schemes of points on surfaces using Nakajima operators. 
We follow the notational conventions of~\cite{LehnSorger}. The description given there in Section 3 allows to deduce an algorithm for operator actions on $\H$. The Chern classes of  tangent bundles of Hilbert schemes is computed with the help of the description from~\cite[Section 3]{Boissiere}.
%It is available online under \url{https://github.com/s--kapfer/HilbK3}. We used the language Haskell, compiled with the \textsc{GHC} software, version 7.6.3.

In contrast to the code in the previous section, we are not restricted to K3 surfaces. Indeed, the surface may have cohomology in odd degree as well as a non-vanishing canonical class. On the other hand, the implemention of the cup product in general is slower than the model for K3 surfaces. 

\subsection{How to use the code}
The main module is \verb|LS_Operators.hs|, which can be opened with ghci.
We implemented the actions of the following operators:
\begin{itemize}
 \item The Nakajima operators $\mathfrak p_n(a)$ and $L_n(a)$ from~\cite{LehnSorger} are given by \verb|P n a| and \verb|L n a|. 
 \item The differential operator $\partial$ is given by \verb|Del|.
 \item The multiplication operators $\G_k(a)$ from~\cite{LiQinWang} related to Chern characters correspond to \verb|Ch k a|.
 \item The Chern character $\mathfrak{ch}T$ of the tangent bundle from~\cite{Boissiere} in degree $k$ corresponds to \verb|ChT k|.
\end{itemize}
\begin{example}
To evaluate the action of a single operator product on the vacuum, use the command \verb|nakaState| to show the result in terms of Nakajima operators:
\begin{verbatim}
*LS_Operators>  let a = P2 0 in nakaState [P (-4) a, P (-2) a]
1 % 1 *         p_4(P2 0) p_2(P2 0) |0>
*LS_Operators>  let a = P2 0 in nakaState [Del,L(-3) a]
6 % 1 *         p_3(P2 2) |0> +
3 % 1 *         p_2(P2 1) p_1(P2 2) |0> +
3 % 1 *         p_2(P2 2) p_1(P2 1) |0> +
(-3) % 1 *      p_1(P2 0) p_1(P2 2) p_1(P2 2) |0> +
(-3) % 1 *      p_1(P2 1) p_1(P2 1) p_1(P2 2) |0>
*LS_Operators>  let a = P2 0 in 
                       nakaState [ChT 2, P(-1) a, P(-1) a, P(-1) a]
2 % 1 *         p_3(P2 0) |0> +
18 % 1 *        p_2(P2 1) p_1(P2 0) |0> +
(-9) % 1 *      p_1(P2 0) p_1(P2 1) p_1(P2 1) |0>
\end{verbatim}
\end{example}

\subsection{What the code does}
An important observation is that we do not need to know explicitly the commutator of $\G_k(a)$ with $\mathfrak p_n(b)$ to compute the action of $\G_k(a)$ on $\H$. 
Indeed, every element of $\H$ can be written as the action on the vacuum of either a polynomial in Nakajima operators $\mathfrak p_n(a)$ or of a polynomial in the operators $\partial$ and $\mathfrak p_{-1}(a)$. We call the two representations \verb|nakaState| and \verb|delState|, respectively. For the action of a Nakajima operator, the first one is more appropriate, while a multiplication operator acts better on the second one (multiplication commutes with $\partial$ and the commutators with $\mathfrak p_n(a)$ are known). 
In addition, the necessary commutation relations to switch between the two representations are known. This is the guiding philosophy for the algorithms contained in \verb|LS_Operators.hs|.

The other module, \verb|LS_Frobenius.hs| contains nothing but the definition of a graded Frobenius algebra according to~\cite[Section 2.1]{LehnSorger} and some instances, namely the cohomologies of K3 surfaces, complex tori and projective space $\C\mathbb P^2$.

The datatype that models $\H$ is called \verb|State|. It consists of linear combinations of ordered operator products, implemented as lists of pairs, containing the product (as a list) and the scalar. 

\subsection{Module for graded Frobenius algebras} 
\input{Sexy/LS_Frobenius.tex}
\subsection{Module for the commutator algebra} 
\input{Sexy/LS_Operators.tex}

\section{Divisible classes in \texorpdfstring{$H^4(\X,\Z)$}{middle cohomology of the generalized Kummer fourfold}}\label{SpecialClasses}
Here we list the divisible classes in from Section~\ref{Middle}. The results are obtained by using a computer based calculation.
\begin{prop}\label{XXXI}
The 31 following classes of $\Pi'$ are divisible by 3 in $H^{4}(K_2(A),\Z)$ and their thirds span a $\mathbb F_3$-vector space of dimension 31 in $\frac{\Pi'^{sat}}{\Pi'}$.
$$\sum_{\tau\in\Lambda} \Big(Z_{\tau} - Z_{\tau+\tau'}\Big), \text{ with }$$
\begin{enumerate}
\item
$\Lambda=\plan{1\\0\\0\\0}{0\\1\\0\\0} \text{ and } 0\neq \tau'\in P^\perp = \plan{0\\0\\1\\0}{0\\0\\0\\1} $,

\item
$\Lambda=\plan{0\\0\\1\\0}{0\\0\\0\\1}  \text{ and } 0\neq \tau' \in P^\perp = \plan{1\\0\\0\\0}{0\\1\\0\\0} \setminus \vect{1\\0\\0\\0}$,

\item
$\Lambda=\plan{1\\0\\0\\1}{0\\1\\2\\1} \text{ and } \tau' \in \left\{ \vect{0\\1\\1\\2},\vect{1\\0\\0\\2},\vect{1\\1\\1\\1},\vect{2\\2\\2\\2} \right\}$,

\item 
$\Lambda=\plan{1\\0\\0\\0}{0\\1\\0\\1} \text{ and } \tau' \in \left\{ \vect{0\\0\\0\\1},\vect{2\\0\\1\\2},\vect{1\\0\\2\\0},\vect{1\\0\\2\\1} \right\}$,
\item
$\Lambda=\plan{1\\0\\0\\0}{0\\1\\1\\1} \text{ and } \tau' \in \left\{ \vect{0\\0\\1\\1},\vect{1\\0\\0\\1} \right\}$,

\item
$\Lambda=\plan{1\\0\\1\\1}{0\\1\\0\\1} \text{ and } \tau' \in \left\{ \vect{0\\1\\0\\2},\vect{1\\0\\2\\2} \right\}$,

\item
$\Lambda=\plan{1\\0\\1\\0}{0\\1\\0\\1} \text{ and } \tau' \in \left\{ \vect{0\\1\\0\\2},\vect{1\\0\\2\\0} \right\}$,

\item
$\Lambda=\plan{1\\0\\0\\0}{0\\1\\0\\2} \text{ and } \tau' = \vect{1\\0\\1\\0}$,

\item
$\Lambda=\plan{1\\0\\1\\1}{0\\1\\2\\2} \text{ and } \tau' = \vect{1\\1\\0\\2}$.
\end{enumerate}
\end{prop}

\begin{prop}\label{XIX}
We use Notation~\ref{BasisH2KA}.
The 19 following classes are divisible by 3 in $H^{4}(K_2(A),\Z)$ and their thirds span a sub-vector space of dimension 19 of $\frac{H^4(K_2(A),\Z)}{\Sym^{sat}\oplus\Pi'^{sat}}$.
\begin{enumerate}
\item
$u_2^2+\sum_{\tau\in \Lambda} Z_\tau-Z_0$, for $\Lambda = \plan{0\\0\\0\\1}{0\\0\\1\\0}$,
\item
$v_2^2+v_2u_2+u_2^2+\sum_{\tau\in \Lambda} Z_\tau-Z_0$, for $\Lambda= \plan{0\\0\\0\\1}{0\\1\\1\\0}$,
\item
$w_2^2+w_2u_2+u_2^2+\sum_{\tau\in \Lambda} Z_\tau-Z_0$, for $\Lambda= \plan{0\\0\\1\\0}{0\\1\\0\\1}$,
\item
$w_2^2-w_2u_2+u_2^2+\sum_{\tau\in \Lambda} Z_\tau-Z_0$, for  $\Lambda= \plan{0\\0\\1\\0}{0\\1\\0\\2}$,
\item
$w_2^2-w_2v_2+w_2u_2+v_2^2+v_2u_2+u_2^2+\sum_{\tau\in \Lambda} Z_\tau-Z_0$, for $\Lambda= \plan{0\\0\\1\\2}{0\\1\\0\\1}$,
\item
$w_1^2+w_1u_2+u_2^2+\sum_{\tau\in \Lambda} Z_\tau-Z_0$, for  $\Lambda= \plan{0\\0\\0\\1}{1\\0\\2\\0}$,
\item
$w_1^2-w_1u_2+u_2^2+\sum_{\tau\in \Lambda} Z_\tau-Z_0$, for $\Lambda= \plan{0\\0\\0\\1}{1\\0\\1\\0}$,
\item
$v_1^2+v_1u_2+u_2^2+\sum_{\tau\in \Lambda} Z_\tau-Z_0$, for $\Lambda= \plan{0\\0\\1\\0}{1\\0\\0\\1}$,
\item
$v_1^2-v_1u_2+u_2^2+\sum_{\tau\in \Lambda} Z_\tau-Z_0$, for $\Lambda= \plan{0\\0\\1\\0}{1\\0\\0\\2}$,
\item
$v_1^2+v_1w_1-v_1u_2+w_1^2+w_1u_2+u_2^2+\sum_{\tau\in \Lambda} Z_\tau-Z_0$, for $\Lambda = \plan{0\\0\\1\\2}{1\\0\\0\\2}$,
\item
$v_1^2+v_1w_1-v_1w_2-v_1v_2+v_1u_2+w_1^2+w_1w_2+w_1v_2-w_1u_2+w_2^2-w_2v_2+w_2u_2+v_2^2+v_2u_2+u_2^2+\sum_{\tau\in \Lambda} Z_\tau-Z_0$, for $\Lambda = \plan{0\\0\\1\\2}{1\\1\\0\\1}$,
\item
$v_1^2-v_1w_1+v_1w_2-v_1v_2+v_1u_2+w_1^2+w_1w_2-w_1v_2+w_1u_2+w_2^2+w_2v_2-w_2u_2+v_2^2+v_2u_2+u_2^2+\sum_{\tau\in \Lambda} Z_\tau-Z_0$, for $\Lambda = \plan{0\\0\\1\\1}{1\\2\\0\\1}$,
\item
$u_1^2+\sum_{\tau\in \Lambda} Z_\tau-Z_0$, for  $\Lambda = \plan{0\\1\\0\\0}{1\\0\\0\\0}$,
\item
$u_1^2-u_1v_2+v_2^2+ \sum_{\tau\in \Lambda} Z_\tau-Z_0$, for  $\Lambda = \plan{0\\1\\0\\0}{1\\0\\0\\1}$,
\item
$u_1^2+u_1v_2+v_2^2+\sum_{\tau\in \Lambda} Z_\tau-Z_0$, for $\Lambda= \plan{0\\1\\0\\0}{1\\0\\0\\2}$,
\item
$u_1^2+u_1w_1+w_1^2+\sum_{\tau\in \Lambda} Z_\tau-Z_0$, for $\Lambda = \plan{0\\1\\0\\2}{1\\0\\0\\0}$,
\item
$u_1^2+u_1w_1-u_1v_2+w_1^2+w_1v_2+v_2^2+\sum_{\tau\in \Lambda} Z_\tau-Z_0$, for $\Lambda = \plan{0\\1\\0\\2}{1\\0\\0\\1}$,
\item
$u_1^2-u_1w_1+u_1w_2-u_1u_2+w_1^2+w_1w_2-w_1u_2+w_2^2+w_2u_2+u_2^2+\sum_{\tau\in \Lambda} Z_\tau-Z_0$, for  $\Lambda = \plan{0\\1\\0\\1}{1\\0\\1\\0}$,
\item
$u_1^2+u_1v_1-u_1w_1+v_1^2+v_1w_1+w_1^2+\sum_{\tau\in \Lambda} Z_\tau-Z_0$, for $\Lambda = \plan{0\\1\\2\\1}{1\\0\\0\\0}$.
\end{enumerate}
\end{prop}
\bibliographystyle{amsplain}
\begin{thebibliography}{10}

\bibitem{Beauville}
A.~Beauville, \emph{Vari\'et\'es k\"ahleriennes dont la premi\`ere classe de Chern est nulle}, 
  J. Differential geometry 18 (1983) 755--782.

\bibitem{Boissiere}
S.~Boissi\`ere, \emph{Chern classes of the tangent bundle on the Hilbert scheme of points
  on the affine plane}, J.~Algebraic Geom. \textbf{14} (2005), no.~14, 761--787.

\bibitem{Generating}
S.~Boissi\`ere and M.~Nieper-Wi{\ss}kirchen, \emph{Generating series in the cohomology 
  of Hilbert schemes of points on surfaces}, LMS J.~of Computation and Math.~10 (2007), 254--270 .

\bibitem{BNS2}
S.~Boissi�re, M. Nieper-Wisskirchen and A. Sarti, 
\emph{Higher dimensional Enriques varieties and automorphisms of generalized Kummer varieties},
Journal de Math�matiques Pures et Appliqu�es,
vol. 95 (2011), no. 5 553--563.

\bibitem{BNS}
S.~Boissi\`ere, M.~Nieper-Wi{\ss}kirchen, and A.~Sarti, \emph{Smith theory and 
  Irreducible Holomorphic Symplectic Manifolds}, Journal of Topology 6 (2013), no.~2, 361--390.

\bibitem{Bredon}
G.E.~Bredon,
\newblock \emph{Intoduction to compact transformation groups},
Academic Press, New York,
(1972), Pure and Applied Mathematics, Vol. 46.

\bibitem{Britze} 
M.~Britze, \emph{On the cohomology of generalized Kummer varieties}, (2003) 

\bibitem{Chihara}
Theodore~S.~Chihara, \emph{An introduction to orthogonal polynomials},
  Mathematics and its Applications 13, Gordon and Breach Science Publishers (1978)

\bibitem{Dai}
Feng~Dai and Yuan~Xu, \emph{Spherical Harmonics}, eprint arXiv:1304.2585 (2013)

\bibitem{Denes}
J.~D\'enes, \emph{The representation of a permutation as the product of a minimal number of
  transpositions, and its connection with the theory of graphs}, Publ. Math. Institute Hung. Acad. Sci.
  \textbf{4} (1959), 63--71.

\bibitem{Dolgachev}
Igor V.~Dolgachev, \emph{Classical Algebraic Geometry}, 
  Cambridge University Press (2012)

\bibitem{Dunkl}
Charles~F.~Dunkl and Yuan~Xu, \emph{Orthogonal Polynomials of Several Variables},
  Encyclopedia of Mathematics and its Applications, Vol.~81, Cambridge University Press (2001)

\bibitem{EGL}
G.~Ellingsrud, L.~G\"ottsche and M.~Lehn, \emph{On the Cobordism Class of the Hilbert 
  Scheme of a Surface}, Journal of Algebraic Geometry \textbf{10} (2001), 81--100. 

\bibitem{Fogarty}
J.~Fogarty, \emph{Algebraic Families on an Algebraic Surface},
  Am.~J.~Math.~\textbf{10} (1968), 511--521.

\bibitem{Folland}
Gerald~B.~Folland, \emph{How to Integrate a Polynomial over a Sphere}, The American
  Mathematical Monthly, Vol.~108, no.~5, (May,~2001)

\bibitem{FujikiK}
A.~Fujiki
\emph{A theorem on bimeromorphic maps of K�hler manifolds and its applications},
Publ. Res. Inst. Math. Sci., 
17(2): 735--754, 1981.

\bibitem{Fujiki}
A.~Fujiki, \emph{Finite automorphism groups of complex tori of dimension two}, Publ. RIMS,
  Kyoto Univ., \textbf{24} (1988), 1-97.

\bibitem{Fujiki2}
A.~Fujiki
\emph{On Primitively Symplectic Compact K�hler V-manifolds of Dimension Four},
Classification of algebraic and analytic manifolds, 
(Katata, 1982), 71-250.

\bibitem{Fulton}
W.~Fulton
\emph{Intersection theory},
Second edition, Springer.

\bibitem{Ghys}
E.~Ghys and A.~Verjovsky, \emph{Locally free holomorphic actions of the complex affine group},
  Geometric Study of Foliations, World Scientific (1994), 201--217.

\bibitem{Gottsche}
L.~G\"ottsche, \emph{Hilbert Schemes of Zero-Dimensional Subschemes of Smooth Varieties},
  Lecture Notes in Mathematics 1572, Springer (1994).

\bibitem{Hassett}
B.~Hassett and Y.~Tschinkel, \emph{ Hodge theory and Lagrangian planes on 
  generalized Kummer fourfolds}, Moscow Math. Journal, 13, no. 1, 33--56, (2013) 

\bibitem{Huybrechts}
Mark~Gross, Daniel~Huybrechts, Dominic~Joyce, \emph{Calabi-Yau Manifolds and Related Geometries},
  Universitext, Springer (2003)

\bibitem{Tim}
T.~Kirschner, 
\emph{Irreducible symplectic complex spaces},
PhD thesis, arXiv:1210.4197.
  
\bibitem{Kapfer}
S.~Kapfer, \emph{Symmetric Powers of Symmetric Bilinear Forms, Homogeneous Orthogonal Polynomials 
  on the Sphere and an Application to Compact Hyperk\"ahler Manifolds} (2015).

\bibitem{Lascoux}
A.~Lascoux, \emph{Symmetric functions}, Notes of the course given at Nankai University (2001),
  \href{http://www.mat.univie.ac.at/~slc/wpapers/s68vortrag/ALCoursSf2.pdf}{http://www.mat.univie.ac.at/\~{}slc/wpapers/s68vortrag/ALCoursSf2.pdf} .

\bibitem{LehnSorger}
M.~Lehn and C.~Sorger, \emph{The cup product of Hilbert schemes for {$K3$}
  surfaces}, Invent. Math. \textbf{152} (2003), no.~2, 305--329.

\bibitem{LiQinWang2}
W.~Li, Z.~Qin and W.~Wang, \emph{Hilbert schemes and W algebras} Int.~Math.~res.~Not. (2002),
  no. 27, 1427--1456.

\bibitem{LiQinWang}
---, \emph{Vertex algebras and the cohomology ring structure of 
  Hilbert schemes of points on surfaces} (2002)  
  
\bibitem{Markman}
E.~Markman, \emph{Integral generators for the cohomology ring of moduli spaces of
  sheaves over Poisson surfaces}, Adv. Math. \textbf{208} (2007), no.~2,
  622--646.

\bibitem{Markman2}
E.~Markman, \emph{Integral constraints on the monodromy group of the
  hyper{K}\"ahler resolution of a symmetric product of a {$K3$} surface},
  Internat. J. Math. \textbf{21} (2010), no.~2, 169--223.

\bibitem{Markmanou}
E.~Markman and S.~Mehrtra
\emph{Hilbert schemes of K3 surfaces are dense in moduli},
arXiv:1201.0031v1.

\bibitem{Mat}
D.~Matsushita,
\emph{On base manifolds of Lagrangian fibrations},
Science China Mathematics (2014), 531-542.

\bibitem{Lol2}
G.~Menet,
\emph{Beauville--Bogomolov lattice for a singular symplectic variety of dimension 4},
Journal of pure and apply algebra (2014), 1455-1495.

\bibitem{Lol}
G.~Menet
\emph{Cohomologie enti�re et fibration lagrangiennes sur certaines vari�t�s holomorphiquement symplectiques singuli�res},
PhD thesis of Lille 1 University, (2014).

\bibitem{McGarr}
Se\'an McGarraghy, \emph{Symmetric Powers of Symmetric Bilinear Forms}, 
  Algebra Colloquium 12:1 (2005) 41-57

\bibitem{Milne}
J.~S.~Milne, \emph{Abelian Varieties (v2.00)}, Available at www.jmilne.org/math/ (2008)  

\bibitem{Milnor}
J.~Milnor, D.~Husem\"oller, \emph{Symmetric bilinear forms}, Ergebnisse der Mathematik
  und ihrer Grenzgebiete 73, Springer (1973).

\bibitem{Mongardi}
G.~Mongardi
\emph{Symplectic involutions on deformations of $K3^{[2]}$},
Cent. Eur. J. Math. 
10 (2012), no.4, 1472-1485.

\bibitem{MongWanTari}
G.~Mongardi, K�vin Tari and Malte Wandel,
\emph{Automorphisms of generalised Kummer fourfolds},
arXiv:1512.00225v2. 

\bibitem{MongWanTari0}
G.~Mongardi, K�vin Tari and Malte Wandel
\emph{Prime order automorphisms of abelian surfaces: a lattice-theoretic point of view},
arXiv:1506.05679v1. 

\bibitem{Mumford}
D.~Mumford, \emph{Abelian varieties}, Tata Institute of Fundamental Research Studies in Mathematics, 
  No.~5 (1970).

\bibitem{Nakajima}
H.~Nakajima, \emph{Heisenberg algebra and Hilbert schemes of points on
  projective surfaces}, Ann. of Math. (2) \textbf{145} (1997), no.~2, 379--388.

\bibitem{Nanikawa}
Y.~Namikawa,
\emph{Extension of 2-forms and symplectic varieties},
J. Reine Angew. Math. 
539 (2001), 123-147.

\bibitem{Lattice}
V.V.~Nikulin,
\emph{Integral symmetric bilinear forms and some of their applications},
Math. USSR Izv.
14 (1980), 103-167.

\bibitem{OGrady}
Kieran~G.~O'Grady, \emph{Compact Hyperk\"ahler manifolds: general theory}, lecture notes (2014)
  \url{www.mimuw.edu.pl/~gael/Document/hk-theory.pdf}

\bibitem{Oguiso}
K.~Oguiso
\emph{No cohomologically trivial non-trivial automorphism of generalized Kummer manifolds},
arXiv:1208.3750.

\bibitem{QinWang}
Z.~Qin and W.~Wang, \emph{Integral operators and integral cohomology classes of
  {H}ilbert schemes}, Math. Ann. \textbf{331} (2005), no.~3, 669--692.

\bibitem{Rapagnetta}
A.~Rapagnetta
\emph{On the Beauville form of the known irreducible symplectic varieties},
Math. Ann.
321 (2008), 77-95.

\bibitem{Scattone}
F.~Scattone, \emph{On the compactification of moduli spaces for algebraic K3 surfaces},
  Memoirs of the American Mathematical Society (1987), Volume 70, Number 374.

\bibitem{Shioda}
T.~Shioda, \emph{The period map of abelian surfaces}, J. Fac. S. Univ.Tokyo 25 (1978), 47--59.

\bibitem{Spanier}
E.~Spanier, 
\emph{The homology of Kummer manifolds},
Proc. Amer. Math. Soc.
7, (1956), 155-160.

\bibitem{Tari}
K.~Tari, 
\emph{Automorphismes des vari�t�s de Kummer g�n�ralis�es},
PhD Thesis, Universit� de Poitiers (2015).

\bibitem{Totaro}
B.~Totaro, \emph{The integral cohomology of the Hilbert scheme of two points} (2015).

\bibitem{Verbitsky}
M.~Verbitsky, \emph{Cohomology of compact hyperk\"ahler manifolds and its
  applications}, Geom. Funct. Anal. \textbf{6} (1996), no.~4, 601--611.

\bibitem{Voisin}
C.~Voisin, \emph{Hodge Theory and Complex Algebraic Geometry, I},
  Cambridge studies in advanced mathematics 76, Cambridge University Press (2002).

\end{thebibliography}


\end{document}