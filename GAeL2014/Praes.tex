\documentclass[12pt,utf8,notheorems,compress,handout]{beamer}
\usepackage{Kurier}
\usepackage{nicefrac}
\usepackage[english]{babel} % german
\usepackage{amsmath,amssymb}
\usepackage[protrusion=true,expansion=false]{microtype}
%\usepackage[framed,amsmath,thmmarks,hyperref]{ntheorem}

\usepackage{booktabs}
\usepackage{tabto}
\usepackage{tikz}
\usepackage{array}
\usepackage{textpos}
%\usepackage[all]{xy}

%\usepackage[natbib=true,style=numeric]{biblatex}
%\usepackage[babel]{csquotes}
%\bibliography{lit}

%\usepackage{hyperref}

\setlength\parskip{\medskipamount}
\setlength\parindent{0pt}

%\theoremseparator{:}
\theoremstyle{plain}  %nonumberplain
%\newtheorem{beh}{Behauptung}
\newtheorem{proposition}{Proposition}
\newtheorem{corollary}{Corollary}
\newtheorem{question}{Question}
\newtheorem{theorem}{Theorem}
\theoremstyle{definition}
\newtheorem{definition}{Definition}
\newtheorem{example}{Example}
%\newtheorem{kor}{Korollar}
%\newtheorem{satz}{Satz}
%\newtheorem{lemma}{Lemma}
%\newtheorem{hilfsaussage}{Hilfsaussage}
%\theorembodyfont{\normalfont}
\newtheorem{observation}{Observation}
%\newtheorem{defnprop}{Definition/Proposition}
%\newtheorem{bem}{Bemerkung}
%\newtheorem{bsp}{Beispiel}
%\theoremsymbol{\ensuremath{\openbox}}
%\newtheorem{proof}{Beweis}
%\newtheorem{defn}{Definition}

\newcommand{\lra}{\longrightarrow}
\newcommand{\lhra}{\ensuremath{\lhook\joinrel\relbar\joinrel\rightarrow}}
\newcommand{\thlra}{\relbar\joinrel\twoheadrightarrow}

\newcommand{\Z}{\mathbb{Z}}
\renewcommand{\C}{\mathbb{C}}
\newcommand{\N}{\mathbb{N}}
\newcommand{\R}{\mathbb{R}}
\newcommand{\Q}{\mathbb{Q}}
\newcommand{\Hom}{\mathrm{Hom}}
\newcommand{\id}{\mathrm{id}}
\newcommand{\Aut}[1]{\operatorname{Aut}(#1)}
\newcommand{\GL}[1]{\operatorname{GL}(#1)}
\newcommand{\freist}{\_{}\_{}}
\newcommand{\Set}{\mathrm{Set}}
\newcommand{\Grp}{\mathrm{Grp}}
\newcommand{\Vect}{\mathrm{Vect}}
\newcommand{\e}{\'{e}}

\def\longleadsto{\mathrel{-}\joinrel\leadsto}
\DeclareMathOperator{\ggT}{ggT}
\DeclareMathOperator{\Ob}{Ob}
\newcommand{\op}{\mathrm{op}}

\title{Integer cohomology of compact Hyperkähler manifolds}
\author[Simon Kapfer,\ \ University of Augsburg]{Simon Kapfer\vspace{-0.5cm}}
\institute{University of Augsburg}
\date{Géométrie Algébrique en Liberté\\Trieste, 27 June 2014\vspace{0.6cm}\\
\includegraphics[scale=0.07]{LogoInstitut}}

%\usetheme{Warsaw}  %Warsaw, Berkeley?
\usetheme{Darmstadt}
\useoutertheme{split}
\usecolortheme{whale}
%\usecolortheme[named=Peach]{structure}
\usefonttheme{serif}
%\usepackage{palatino}
\useinnertheme{rectangles}
%\usepackage{bookman}
%\setbeamercovered{transparent}

\setbeamertemplate{navigation symbols}{}
%\setbeamertemplate{footline}{}
%\setbeamertemplate{headline}{}

%\beamertemplateboldcenterframetitle
%\setbeamerfont{frametitle}{size={\Large}}

\newcommand*\oldmacro{}%
\let\oldmacro\insertshorttitle%


\newenvironment{changemargin}[2]{%
  \begin{list}{}{%
    \setlength{\topsep}{0pt}%
    \setlength{\leftmargin}{#1}%
    \setlength{\rightmargin}{#2}%
    \setlength{\listparindent}{\parindent}%
    \setlength{\itemindent}{\parindent}%
    \setlength{\parsep}{\parskip}%
  }%
  \item[]}{\end{list}}

\newcommand{\slogan}[1]{%
  \begin{center}%
    \setlength{\fboxrule}{2pt}%
    \setlength{\fboxsep}{-3pt}%
    {\usebeamercolor[fg]{item}\fbox{\usebeamercolor[fg]{normal
    text}\parbox{0.9\textwidth}{\begin{center}#1\end{center}}}}%
  \end{center}%
}

\makeatletter
    \newenvironment{withoutheadline}{
        \setbeamertemplate{headline}[default]
        \def\beamer@entrycode{\vspace*{-\headheight}}
    }{}
\makeatother

\newcommand{\hil}[1]{{\usebeamercolor[fg]{item}{#1}}}

\begin{document}

\setbeameroption{show notes}
\setbeamertemplate{note page}[plain]
\begin{withoutheadline}
\frame{\vspace{4mm}
\titlepage
}
\end{withoutheadline}
\addtobeamertemplate{frametitle}{}{%
\begin{textblock*}{100mm}(0.88\textwidth,-1.52cm)
\includegraphics[scale=0.18]{UniLogoNeg}
\end{textblock*}}
\addtocounter{framenumber}{-1}
\renewcommand*\insertshorttitle{%
  \oldmacro\hfill\insertframenumber\,/\,\inserttotalframenumber\hfill}
\section{General IHSM}

\frame[t]{\frametitle{Motivation}
Let $X$ be a compact Hyperkähler manifold of complex dimension $2m$ or, equivalently, an IHS manifold. Why should we be interested in $ H^*(X,\Z) $?
\pause
\begin{itemize}
\item It feels more geometric.
\pause
\item Comparing $ H^*(X,\Z) $ with $ H^*(X,\C) $ gives us information about $X$, e.g.\ on projectivity.
\pause
\item We obtain restrictions to possible automorphisms of our manifold $X$.
\item \dots
\end{itemize}
\pause
\begin{question}
Which constructions in $H^*(X,\R/\C)$ carry over to $H^*(X,\Z)$?
\end{question}
}
\frame[t]{\frametitle{Beauville--Bogomolov form}
As an example, consider the quadratic Beauville--Bogomolov form $q_X : H^*(X,\R) \rightarrow \R$
\pause
\begin{theorem}[Fujiki]
$q_X(\alpha)^m = c \int_X \alpha^{2m}$ for some $c\in\R$.
\end{theorem}
\pause
\begin{corollary}
$q_X$ can be renormalized to yield a primitive integral quadratic form: $H^*(X,\Z) \rightarrow \Z $.
\end{corollary}
}


\frame[t]{\frametitle{Hodge numbers for K3 surfaces}
\pause
For $S$ a compact Hyperkähler manifold of complex dimension two, i.e.\ a K3 surface, we have 
$$
\begin{array}{c@{\hspace{3cm}}c}
\ h^{p,q}(S) & h^k(S,\Z)\vspace{0.3cm}\\
\setlength{\arraycolsep}{0.13cm}
\begin{array}{ccccc}
  &  &1 &  & \\
  & 0&  &0 &  \\
 1&  &20&  &1  \\
  & 0&  &0 &   \\
  &  & 1&  & 
\end{array} & 
\begin{array}{c}
1\\0\\22\\0\\1
\end{array}
\end{array}
$$
and the intersection pairing on $H^2$ is isomorphic to $U^{\oplus3}\oplus E_8(-1)^{\oplus 2}$, where $U$ and $E_8$ are the bilinear forms corresponding to the hyperbolic resp. $E_8$ lattice.
}

\section{Hilbert schemes of K3 surfaces}

\frame[t,shrink=11]{\frametitle{Hodge numbers for Hilbert schemes}
Denote $S^{[n]}$ the Hilbert scheme of $n$ points on the K3 surface $S$.\\
Then, the Hodge decomposition for $S^{[2]}$ is given by: 
$$
\begin{array}{c@{\hspace{3cm}}c}
\ h^{p,q}(S^{[2]}) & h^k(S^{[2]},\Z)\vspace{0.3cm}\\
\setlength{\arraycolsep}{0.13cm}
\begin{array}{ccccccccc}
 & & & & 1 & & &  &\\
 & & &0&   &0& &  &\\
 & &1& & 21& &1& & \\
 &0& &0&   &0& &0& \\
1&&21& &\hspace{-1mm}232\hspace{-1mm}& &21&&1\\
 &0& &0&   &0& &0& \\
 & &1& & 21& &1& & \\
 & & &0&   &0& &  &\\
 & & & & 1 & & &  &  
\end{array} & 
\begin{array}{c}
1\\0\\23\\0\\276\\0\\23\\0\\1
\end{array}
\end{array}
$$
and there are formulae for all $S^{[n]}$ due to Göttsche.
}

\frame[t]{\frametitle{Betti numbers}
\begin{center}

\begin{tabular}{r|rrrrrr}
%\toprule
k & $S^{[1]}$ & $S^{[2]}$& $S^{[3]}$& $S^{[4]}$ & $S^{[5]}$ & $S^{[6]}$ \\ \midrule
0 &	1 &	1&	1&	1&	1&	1 \\
2 & 22&	23&	23&	23&	23&	23 \\
4 & 1& 276&299&300&300&300\\
6 & & 23 & 2554 & 2852 & 2875 & 2876\\
8 & & 1 &299 & 19298&22127& 22426 \\
10& & & 23 &2852&125604&147431\\
12 & & & 1 & 300 &22127 &727606
%\\\bottomrule
\end{tabular}
\end{center}
For $k$ fixed, these numbers stabilize for $n$ big enough.

}

\frame[t]{\frametitle{Cohomology of Hilbert schemes}
\begin{theorem}[Nakajima]
For each $m\geq 1$ and each $\alpha \in H^j(S,\,\Q)$, there is an operator
$$ \mathfrak{a}_{-m}(\alpha) : H^i(S^{[n]},\,\Q) \longrightarrow H^{i+j+2m-2}(S^{[n+m]},\,\Q)$$
and these operators, applied to $H^*(S^{[0]},\,\Q)\cong\Q$, span the entire cohomology of all Hilbert schemes $S^{[n]}$.
\end{theorem}
\pause 
Construction of the operators: Use the incidence scheme 
$$ \mathcal{I} \ \subset \ S^{[n]}\times S \times S^{[n+m]} $$
and define, using Poincaré duality,
 $$\mathfrak{a}_{-m}(\alpha) \beta \;\stackrel{\text{P.D.}}{=}\; p_{3\ast}\left( (p_1^*\beta\cup p_2^*\alpha)\cap \left[\mathcal{I}\right] \right).$$
}
\frame[t]{\frametitle{Integral basis for $H^*(S^{[n]},\Z)$}
This also works in integer cohomology!
\begin{theorem}[Qin, Wang]
Let $1\in H^0(S,\,\Z)$ be the canonical generator. The operators 
$$
\frac{1}{z_\lambda}\mathfrak{a}_{-\lambda}(1) :  H^i(S^{[n]},\,\Z) \longrightarrow H^{i+2m-2k}(S^{[n+m]},\,\Z)  $$
$$\quad\mathfrak{a}_{-m}(\alpha) : H^i(S^{[n]},\,\Z) \longrightarrow H^{i+j+2m-2k}(S^{[n+m]},\,\Z)$$
span the integer cohomologies $H^*(S^{[n]},\,\Z)$. Here, the composition of several operators $\mathfrak{a}_{-m_i}(1)$ is denoted via a partition $\lambda$ and $z_\lambda$ denotes some constant depending on $\lambda$.
\end{theorem}
}
\frame[t]{\frametitle{Algebra generated by $H^2(S^{[n]},\C)$}
\begin{theorem}[Verbitsky]
The subalgebra generated by $H^2(S^{[n]}, \, \C)$ in $H^*(S^{[n]}, \, \C)$ is equal to $$\frac{\operatorname{Sym}^*H^2(S^{[n]}, \, \C) }{ \left< \alpha^{n+1}\ |\ q(\alpha)=0\right>},$$ where $q$ is the Beauville-Bogomolov Form. In fact, this holds for any compact Hyperkähler manifold of complex dimension $2n$.
\end{theorem}
\pause 
What about cohomology with integral coefficients?\vspace{2mm}\\
\pause
Lehn, Sorger and Vasserot developed formulae for the cup product on $H^*(S^{[n]},\,\Q)$. We can use them also for computations in $H^*(S^{[n]},\,\Z)$.
}
\frame[t, shrink=10]{\frametitle{Ring structure of $H^*(S^{[n]},\,\Q)$}
Rough idea of the algebraic model:
\begin{itemize}
\item To any composition of operators $\mathfrak{a}_{-m_1}(\alpha_1)\ldots \mathfrak{a}_{-m_k}(\alpha_k)$ one associates a conjugacy class of the symmetric group $\S_n$, given by the partition $\lambda = (m_1,\ldots,m_k)$
\pause
\item The cup product is built with the product in $\S_n$:
\pause
\item E.g.\ $(1\,2\,3)(4\,5)\cdot(1\,4) = (1\,2\,3\,4\,5)$, but $(1\,2\,3)(4\,5)\cdot (1\,2) = (1)(2\,3)(4\,5)$.
\pause
\item If two cycles $\nu_i, \nu_j$ are joined together by a transposition, multiply the corresponding classes, e.g. $(1\,2\,3)_{\alpha_i}(4\,5)_{\alpha_j}\cdot(1\,4)_{\alpha_l} = (1\,2\,3\,4\,5)_{\alpha_i\cdot\alpha_j\cdot\alpha_l}$
\pause
\item If a cycle is split in two by a transposition, use a map adjoint to the multiplication in $H^*(S)$.
\pause
\item Sum up all such possibilities and put the constants in the right way.

\end{itemize}
}
\frame[t]{\frametitle{Algebra generated by $H^2(S^{[n]},\Z)$}
For integer cohomology, we get torsion for small $n$, e.g.\ 
\begin{itemize}
\item $H^4(S^{[2]},\Z)\;=\;\operatorname{Sym}^2H^2(S^{[2]}, \Z)  \oplus \Z_{ 2^{46}\cdot 5^2}$
\item $H^4(S^{[3]},\Z)\;=\;\operatorname{Sym}^2H^2(S^{[3]}, \Z) \oplus \Z^{23} \oplus \Z_{3}$
\item $H^4(S^{[n]},\Z)\;=\;\operatorname{Sym}^2H^2(S^{[n]}, \Z) \oplus \Z^{24}, \qquad\qquad n\geq 4.$
\end{itemize}
\begin{question}
Is there a geometric interpretation, e.g.\ for $n=3$?
\end{question}
\pause
\vspace{0.7cm}
\slogan{Thank you for your attention!}
}
\appendix
  \addtocounter{framenumber}{-1}
%\backupstart
\renewcommand*\insertshorttitle{%
  \oldmacro\hfill}

\frame[t,shrink=8]{\frametitle{References}
%\cite{qinintegral2004}\cite{lehnsorger}\cite{gross2003calabi}\cite{boissiereniepersarti}
\bibliographystyle{plain}
\bibliography{bibl}
}
\end{document}
