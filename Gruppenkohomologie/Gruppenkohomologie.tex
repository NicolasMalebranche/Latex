\documentclass[11pt]{article}
\usepackage[T1]{fontenc}
\usepackage[latin1]{inputenc}
\usepackage[german]{babel}
\usepackage{fourier}  % Use the Adobe Utopia font for the document
\usepackage{amsmath,amsthm,amssymb,amscd,color,graphicx}

% Geschwungene Kleinbuchstaben im Mathemodus (benutze \mathpzc)
\DeclareFontFamily{OT1}{pzc}{}
\DeclareFontShape{OT1}{pzc}{m}{it}{<-> s * [1.10] pzcmi7t}{}
\DeclareMathAlphabet{\mathpzc}{OT1}{pzc}{m}{it}


%Struktur
\newcommand{\point}{\vspace{3mm}\par \noindent \refstepcounter{subsection}{\bf \thesubsection.} }
	\numberwithin{equation}{subsection}
\newcommand{\tpoint}[1]{\vspace{3mm}\par \noindent \refstepcounter{subsection}{\bf \thesubsection.} 
  \numberwithin{equation}{subsection} {\em #1. ---} }
\newcommand{\epoint}[1]{\vspace{3mm}\par \noindent \refstepcounter{subsection}{\bf \thesubsection.} 
  \numberwithin{equation}{subsection} {\em #1.} }
\newcommand{\bpoint}[1]{\vspace{3mm}\par \noindent \refstepcounter{subsection}{\bf \thesubsection.} 
  \numberwithin{equation}{subsection} {\bf\em #1.} }
  
%Abk�rzungen
\newcommand{\N}{\mathbb{N}}
\newcommand{\Z}{\mathbb{Z}}
\newcommand{\Q}{\mathbb{Q}}
\newcommand{\C}{\mathbb{C}}
\newcommand{\R}{\mathbb{R}}
\newcommand{\Cinf}{C^\infty}
\newcommand{\id}{\text{id}}
\renewcommand{\O}{\mathcal{O}}
\renewcommand{\S}{\mathfrak{S}}
\newcommand{\T}{\mathpzc{T}}
\newcommand{\E}{\mathpzc{E}}
\newcommand{\Hom}{\mathpzc{Hom}}
\newcommand{\End}{\mathpzc{End}}
\newcommand{\Tor}{\mathrm{Tor}}
\newcommand{\Ext}{\mathrm{Ext}}
\newcommand{\del}{\partial}
\newcommand{\delbar}{\overline{\partial}}
\newcommand{\dzbar}{d\overline{z}}

\newcommand{\diff}[1]{\frac{\partial}{\partial #1}}
\newcommand{\vac}{\left|0\right>}

\binoppenalty=7000
\relpenalty=5000

% Title Page
\title{Memo: Gruppenkohomologie}
\author{Simon Kapfer}

\begin{document}
\maketitle
\begin{abstract}
Merkzettel zu \cite{brownGroupCoh}.
\end{abstract}
\section{Komplexe}
\bpoint{Inneres Hom} \cite[S. 5, 9]{brownGroupCoh}. Seien $C$ und $C'$ Kettenkomplexe. Dann ist $\Hom(C,C')_n := \prod_q \mathrm{Hom}(C_q, C'_{q+n})$. Der Randoperator ist $D_n(f) := d'f -(-1)^nfd$. Das ist am besten in der Form $d'\left<f,u\right> = \left<Df,u\right> + (-1)^n\left<f,du\right>$ zu merken. Kettenabbildungen zwischen Komplexen sind dann Elemente von $\ker D_0$, nullhomotope Kettenabbildungen sind exakt. Die Homologie des Hom--Komplexes im Grad 0 sind die Homotopieklassen von Kettenabbildungen. \\Eine Konstruktion mit Kokettenkomplexen geht analog.
\bpoint{Gruppenmoduln} Eine Wirkung von $G$ auf $M$ ist eine $\Z G$--Modulstruktur.  
$M^G = \ker(g-1) = \mathrm{Hom}_{\Z G}(\Z,M)$ sind die Invarianten, $M_G = M/\left< g - 1\right> = \Z\otimes_{\Z G}M$ sind die Koinvarianten. Der Invariantenfunktor ist linksexakt, der Koinvariantenfunktor ist rechtsexakt.
In Charakteristik 0 gibt es f�r endliches $G$ einen Isomorphismus $M_G \longrightarrow M^G$. In diesem Fall haben wir Exaktheit.
Wenn $M$ ein $G$--Links-- und $N$ ein $G$--Rechtsmodul ist, so ist $M\otimes_G N = M\otimes_{\Z G}N := (M\otimes N)_G$ Einen Linksmodul $M'$ kann man durch $g\mapsto g^{-1}$ k�nstlich zu einem Rechtsmodul machen. $\mathrm{Hom}_G(M,M') = (\mathrm{Hom}(M,M'))^G$. 
\bpoint{(Ko-)Skalarerweiterungen}\cite[III.3]{brownGroupCoh} Gegeben ein Ringhomomorphismus $\iota: R\rightarrow S$ und ein $R$-Modul $M$.
Skalarerweiterung ist der Funktor $M\mapsto S\otimes_R M$, Koskalarerweiterung ist $M\mapsto \mathrm{Hom}_R(S,M)$. Diese Funktoren sind links-- bzw. rechtsadjungiert zur Skalareinschr�nkung:
$\mathrm{Hom}_S(S\otimes_R M, N) \cong \mathrm{Hom}_R(M, \iota^*N) $ und $\mathrm{Hom}_R(\iota^*N, M) \cong\mathrm{Hom}_S(N, \mathrm{Hom}_R(S,M))$. F�r $R=\Z G$ und $S=\Z$ ist Erweiterung gleich Koinvariantenbildung und Koerweiterung gleich Invariantenbildung.
\section{Kohomologie}
\bpoint{Definition} $A$ und $B$ seien Komplexe mit einer $G$--Wirkung. $P$ sei eine $\Z[G]$--projektive Aufl�sung von $A$. (Projektiv impliziert flach, d. h. $P\otimes_G\_$ ist exakt.) $\Tor^G_*(A,B):=H_*(P\otimes_G B)$ und $\Ext_G^*(A,B) := H^*(\Hom_G(P,B))$. Gruppenhomologie mit Werten in einem Modul $M$ (interpretiert als Komplex im Grad 0) ist definiert als $H_*(G;M) :=\Tor^G_*(\Z,M)$. Gruppenkohomologie entsprechend als $H^*(G;M):=\Ext_G^*(\Z,M)$.
\bpoint{Abbildungskegel}

\bibliographystyle{alpha}
\bibliography{../bibl}
\end{document}          
