\documentclass{amsart}

\usepackage{amsmath,amssymb,amsfonts,amscd}
\usepackage[all]{xy}
\usepackage{appendix,listings,hyperref}

\DeclareMathOperator{\rank}{rank}
\DeclareMathOperator{\trace}{tr}
\DeclareMathOperator{\Tor}{Tor}
\DeclareMathOperator{\Ext}{Ext}
\DeclareMathOperator{\Aut}{Aut}
\DeclareMathOperator{\End}{End}
\DeclareMathOperator{\id}{id}
\DeclareMathOperator{\Hom}{Hom}
\DeclareMathOperator{\im}{Im}
\DeclareMathOperator{\Ker}{Ker}
\DeclareMathOperator{\Sym}{Sym}
\DeclareMathOperator{\Hilb}{Hilb}

\newcommand{\hilb}[1]{^{[#1]}}
\newcommand{\ie}{{\it i.e. }}
\newcommand{\eg}{{\it e.g. }}
\newcommand{\loccit}{{\it loc. cit. }}
\newcommand{\vac}{|0\rangle}
\newcommand{\odd}{{\rm{odd}}}
\newcommand{\even}{{\rm{even}}}
\newcommand{\tors}{{\rm{tors}}}

\newcommand{\p}{\mathfrak{p}}

\newcommand{\coloneqq}{:=}


%%%%%%%%%%%%%%%%%%%%%%%%%%%%%%

\newcommand{\IC}{\mathbb{C}}
\newcommand{\IR}{\mathbb{R}}
\newcommand{\IQ}{\mathbb{Q}}
\newcommand{\IZ}{\mathbb{Z}}


%%%%%%%%%%%%%%%%%%%%%%%%%%%%%

\newcommand{\kS}{\mathfrak{S}}

\newcommand{\km}{\mathfrak{m}}
\newcommand{\kq}{\mathfrak{q}}

%%%%%%%%%%%%%%%%%%%%%%%%%%%%%%

\newcommand{\lra}{\longrightarrow}
\newcommand{\ra}{\rightarrow}

%%%%%%%%%%%%%%%%%%%%%%%%%%%%%

\theoremstyle{plain}
\newtheorem{theorem}{Theorem}[section]
\newtheorem{lemma}[theorem]{Lemma}
\newtheorem{proposition}[theorem]{Proposition}
\newtheorem{corollary}[theorem]{Corollary}
\theoremstyle{definition}
\newtheorem{definition}[theorem]{Definition}
\newtheorem{notation}[theorem]{Notation}
\theoremstyle{remark}
\newtheorem{remark}[theorem]{Remark}
\newtheorem{example}[theorem]{Example}


%%%%%%%%%%%%%%%%%%%%%%%%%%%%%

\begin{document}

\title{Integral cohomology of $K^2(A)$}


\author{Simon Kapfer}
\address{Simon Kapfer, Laboratoire de Math\'ematiques et Applications, UMR CNRS 6086, Universit\'e de Poitiers, T\'el\'eport 2, Boulevard Marie et Pierre Curie, F-86962 Futuroscope Chasseneuil}
\email{simon.kapfer@math.univ-poitiers.fr}
%\urladdr{http://www.math.uni-augsburg.de/alg/}


\date{\today}

%\keywords{}

\begin{abstract} 
What we know already
\end{abstract}

\maketitle

\begin{definition}
Let $A$ be a complex projective torus of dimesion $2$ and $A\hilb{n}$ the corresponding Hilbert scheme of points. Denote $\Sigma : A\hilb{n} \rightarrow A$ the summation morphism. Then the generalized Kummer $K^{n-1} A $ is defined as the fiber over $0$:
\begin{equation*}
\begin{CD}
K^{n-1}A @>\theta >> A\hilb{n}\\
@VVV @VV\Sigma V\\
\{0\} @> >> A
\end{CD}
\end{equation*}
\end{definition}
By \cite{Beauville}, $\theta^{\ast} : H^2(A\hilb{n}) \rightarrow H^2(K^{n-1}A)$ is surjective. We have injections $j : H^2(A)\rightarrow H^2(A\hilb{n})$ and $i = \theta^* j$. The cohomology $H^*(A\hilb{n})$ is described in terms of vertex operators in \cite{LehnSorger} and \cite{LiQinWang}.

We describe now the image of $\theta^*$ in the case $n=3$:
\begin{itemize}
\item We know $j(a)=\frac{1}{2}\p_{-1}(a)\p_{-1}(1)^2\vac$, because the two are must be linearly dependent and
$$ \int_{A\hilb{3}}j(a)^6 = 15 q(a)^3, \quad \left(\frac{1}{2}\p_{-1}(a)\p_{-1}(1)^2\vac\right)^3 = 15 q(a)^3\p_{-1}(x)^3\vac.
$$
\item By \cite[p. 8]{Britze}, we have for $\alpha = j(a)=\frac{1}{2}\p_{-1}(a)\p_{-1}(1)^2\vac$: 
$$ \int_{A\hilb{3}}\alpha^6 = \frac{5}{3} q(a) \int_{K^2} \theta^* \alpha^4
$$
On the other hand, 
$$\alpha^4 = 3 q(a)^2\p_{-1}(x)^2\p_{-1}(0)\vac + 3q(a) \p_{-1}(x) \p_{-1}(a)^2\vac,$$
so if the image of both summands under $\int\theta^*$ is positive, then 
$$ \int\theta^*\p_{-1}(x)^2\p_{-1}(0)\vac = \int\theta^* \tfrac{1}{2}\p_{-1}(x) \p_{-1}(a)^2\vac=1 .$$ 

\end{itemize}


\bibliographystyle{amsplain}
\bibliography{Kummer2Bib.tex}
\end{document}
