\documentclass[11pt]{article}
\usepackage[T1]{fontenc}
\usepackage[latin1]{inputenc}
\usepackage[german]{babel}
\usepackage{fourier}  % Use the Adobe Utopia font for the document
\usepackage{amsmath,amsthm,amssymb,amscd,color,graphicx}

% Geschwungene Kleinbuchstaben im Mathemodus (benutze \mathpzc)
\DeclareFontFamily{OT1}{pzc}{}
\DeclareFontShape{OT1}{pzc}{m}{it}{<-> s * [1.10] pzcmi7t}{}
\DeclareMathAlphabet{\mathpzc}{OT1}{pzc}{m}{it}


%Struktur
\newcommand{\point}{\vspace{3mm}\par \noindent \refstepcounter{subsection}{\bf \thesubsection.} }
	\numberwithin{equation}{subsection}
\newcommand{\tpoint}[1]{\vspace{3mm}\par \noindent \refstepcounter{subsection}{\bf \thesubsection.} 
  \numberwithin{equation}{subsection} {\em #1. ---} }
\newcommand{\epoint}[1]{\vspace{3mm}\par \noindent \refstepcounter{subsection}{\bf \thesubsection.} 
  \numberwithin{equation}{subsection} {\em #1} }
\newcommand{\bpoint}[1]{\vspace{3mm}\par \noindent \refstepcounter{subsection}{\bf \thesubsection.} 
  \numberwithin{equation}{subsection} {\bf\em #1} }
  
%Abk�rzungen
\newcommand{\N}{\mathbb{N}}
\newcommand{\Z}{\mathbb{Z}}
\newcommand{\Q}{\mathbb{Q}}
\newcommand{\C}{\mathbb{C}}
\newcommand{\R}{\mathbb{R}}
\newcommand{\Cinf}{C^\infty}
\newcommand{\id}{\text{id}}
\renewcommand{\O}{\mathcal{O}}
\renewcommand{\S}{\mathfrak{S}}
\newcommand{\T}{\mathpzc{T}}
\newcommand{\E}{\mathpzc{E}}
\newcommand{\Hom}{\mathpzc{Hom}}
\newcommand{\End}{\mathpzc{End}}
\newcommand{\del}{\partial}
\newcommand{\delbar}{\overline{\partial}}
\newcommand{\dzbar}{d\overline{z}}

\newcommand{\diff}[1]{\frac{\partial}{\partial #1}}
\newcommand{\vac}{\left|0\right>}

\binoppenalty=7000
\relpenalty=5000

% Title Page
\title{Memo: Interessante Sachen}
\author{Simon Kapfer}

\begin{document}
\maketitle
\begin{abstract}
Merkzettel zu diversen Sachen und Mottos, die nicht in einen anderen Kontext eingebettet sind.
\end{abstract}
\section{Kohomologisches}
\bpoint{Gruppenkohomologie} einer Gruppe $G$ soll die (singul�re) Kohomologie eines Raumes sein, dessen Fundamentalgruppe gleich $G$ ist und dessen andere Homotopiegruppen trivial sind. Den Raum kann man konstruieren: er hei�t 'Eilenberg--MacLane Raum'.
\bpoint{Tensorprodukt �ber Gruppenringen} $M,\ N$ seien Links-$G$-Moduln. $M\otimes_G N$ ist so definiert, da� $m g\otimes g n = m \otimes n$. Dann ist $M\otimes_G N \cong (M\otimes N)_G$.
\bpoint{Welche Kohomologieklassen k�nnen als Chernklassen realisiert werden?} F�r $X$ eine projektive Kurve kann jede Klasse aus $H^2(X,\Z)$ als erste Chernklasse eines Vektorb�ndels geschrieben werden. \\F�r $X$ eine komplexe Fl�che geht das auch f�r beliebiges $c_1\in H^{1,1}(X)\cap H^2(X,\Z)$ und $c_2 \in H^4(X,\Z)\cong\Z$ (Satz von Schwarzenberger).
\bpoint{�quivariante Kohomologie} Wenn eine kompakte Liegruppe $G$ auf einem Raum $X$ wirkt, so wird die �quivariante Kohomologie $H^*_G(X;\R) := H^*(\frac{X\times EG}{G}; \R)$ �ber den folgenden (Totalkomplex des Doppel-)Komplex berechnet:
$$\Omega_G^i(X) \ = \ \bigoplus\left(S^j(\mathfrak{g}^*) \otimes \Omega^{i-2j}(X)\right)^G $$
\bpoint{Charakteristische Klassen}
Insbesondere hat man f�r $X = \{\text{pt}\} $ und $G=U(n)$ 
$$ H^*_G(X) = H^*(BG) = S^*(\mathfrak{g}^*)^G = \R[c_1,\ldots,c_n], \qquad \det(\lambda - A) = \sum_i(-1)^ic_i(A)\lambda^{n-i}$$
wobei die $c_i$ die Chernklassen sind. F�r $G = O(n)$ erh�lt man Pontrjagin--Klassen. F�r $V$ ein Vektorb�ndel �ber einem beliebigen $X$ hat man durch Wahl von lokalen Rahmen die Struktur eines $G$--Hauptfaserb�ndels und damit eine Abbildung $X\rightarrow BG$. Die charakteristischen Klassen des B�ndels ergeben sich dann durch R�ckzug von $BG$.
\bpoint{Picardgruppe} einer glatten Variet�t sind die Isomorphismenklassen von Geradenb�ndeln mit Tensorprodukt als Gruppenoperation. �quivalent Cartierdivisoren modulo lineare �quvalenz. �quivalent $H^1(X,\O_X^\times )$.
\section{Algebraisches}
\bpoint{Lieableitung und Intuition.} Seien $f(x),\ g(x)$ parameterabh�ngige, lineare Operatoren (z. B. einfach Multiplikation mit Zahlen: $f(x)\in \R$). Differentialoperatoren wie $\frac{d}{dx}$ fallen auch in diese Kategorie. Es gilt 
\[ \frac{d}{dx} f g = \left(\diff{x}f\right) g + f \frac{d}{dx} g \] 
Daher macht es Sinn, den abgeleiteten Operator $f' := \left(\diff{x}f\right)$ zu definieren als:
\[f'\quad =\quad \frac{d}{dx} f - f \frac{d}{dx} \quad=\quad \left[\frac{d}{dx},f \right]\]
Hier also eine M�glichkeit, die Lieklammer zu verstehen. Die Jacobi-Identit�t wird dann zu einem Spezialfall der Leibnizregel: 
\[  {\color{blue}[ } x,{ \color{black} [} y,z { \color{black} ]}{\color{blue}] } =
{ \color{black} [}{\color{blue}[ } x,y {\color{blue}] }, z{ \color{black} ]} +{ \color{black} [} y, {\color{blue}[ } x,z {\color{blue}] } { \color{black} ]}
 \]
Die blauen Klammern stehen jeweils f�r Ableitung nach $x$, die schwarzen Klammern sind einfach nur eine Bilinearform, die hier zuf�llig gleich der Lieklammer ist.
\bpoint{�ber $\ \mathfrak{sl}_2(\C)$.} Erzeuger: $H,\;X,\;Y$ mit $[X,Y] = H,\ [H,X] = 2X,\ [H,Y] = -2Y$. Vorstellung: $H,\ X,\ Y$ haben Grad $0,\ 2,\ -2$ und $H$ z�hlt den Grad. Jede irreduzible Darstellung von $\mathfrak{sl}_2(\C)$ sieht aus wie $V_{-n}\oplus\ldots\oplus V_n$ mit $V_k\cong\C$ Eigenraum von $H$ zum Eigenwert $k$. Ein Modell daf�r ist $\text{Sym}^n\C^2$ mit Koordinaten $x$ vom Grad $1$ und $y$ vom Grad $-1$.  
\bpoint{} Es gibt keinen K�rper, der als $\Z$--Modul frei ist.

\section{Exakte Sequenzen}
\bpoint{Eulersequenz} Auf $\mathbb{P}^n$ hat man
\begin{align*} 0 \rightarrow \O \rightarrow \O(1)^{\oplus n+1} \rightarrow \T \rightarrow 0 & \quad \text{bzw.}\\
0 \rightarrow \Omega \rightarrow \O(-1)^{\oplus n+1} \rightarrow \O \rightarrow 0 &
\end{align*}
\bpoint{Exponentialsequenz} Auf komplexen R�umen liefert die exp--Funktion
$$ 0\rightarrow 2\pi i\Z \rightarrow \O \rightarrow \O^\times \rightarrow 0 $$
Dazu die lange exakte Sequenz in Kohomologie:
$$ \rightarrow H^1(X, \O^\times) \rightarrow H^2(X,2\pi i\Z) \rightarrow H^2(X,\O) \rightarrow 
$$ 
$H^1(X, \O^\times)$ ist die Picardgruppe. Der erste Pfeil ist die erste Chernklasse. Deren Bild (oder der Kern des n�chsten Pfeils) ist die N�ron-Severi Gruppe.

\bibliographystyle{alpha}
\bibliography{bibl}
\end{document}          
