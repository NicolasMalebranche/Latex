\documentclass[12pt,notheorems,compress,handout]{beamer}
\usepackage{kurier}
\usepackage[T1]{fontenc}
\usepackage{nicefrac}
\usepackage[english]{babel} % german
\usepackage{amsmath,amssymb}
\usepackage[protrusion=true,expansion=false]{microtype}
%\usepackage[framed,amsmath,thmmarks,hyperref]{ntheorem}

\usepackage{booktabs}
\usepackage{tabto}
\usepackage{tikz-cd}
%\usetikzlibrary{matrix,arrows,decorations.pathmorphing}
\usepackage{array}
\usepackage{textpos}


%\usepackage[natbib=true,style=numeric]{biblatex}
%\usepackage[babel]{csquotes}
%\bibliography{lit}

%\usepackage{hyperref}

\setlength\parskip{\medskipamount}
\setlength\parindent{0pt}

%\theoremseparator{:}
\theoremstyle{plain}  %nonumberplain
%\newtheorem{beh}{Behauptung}
\newtheorem{proposition}{Proposition}
\newtheorem{corollary}{Corollary}
\newtheorem{question}{Question}
\newtheorem{theorem}{Theorem}
\theoremstyle{definition}
\newtheorem{definition}{Definition}
\newtheorem{example}{Example}
%\theorembodyfont{\normalfont}
\newtheorem{observation}{Observation}
%\theoremsymbol{\ensuremath{\openbox}}

\newcommand{\lra}{\longrightarrow}
\newcommand{\lhra}{\ensuremath{\lhook\joinrel\relbar\joinrel\rightarrow}}
\newcommand{\thlra}{\relbar\joinrel\twoheadrightarrow}

\newcommand{\Z}{\mathbb{Z}}
\newcommand{\C}{\mathbb{C}}
\newcommand{\N}{\mathbb{N}}
\newcommand{\R}{\mathbb{R}}
\newcommand{\Q}{\mathbb{Q}}
\newcommand{\Hom}{\mathrm{Hom}}
\newcommand{\id}{\mathrm{id}}
\newcommand{\Aut}[1]{\operatorname{Aut}(#1)}
\newcommand{\GL}[1]{\operatorname{GL}(#1)}
\newcommand{\freist}{\_{}\_{}}
\newcommand{\Set}{\mathrm{Set}}
\newcommand{\Grp}{\mathrm{Grp}}
\newcommand{\Vect}{\mathrm{Vect}}
\newcommand{\e}{\'{e}}
\newcommand{\hilb}[1]{^{[#1]}}
\newcommand{\kum}[1]{^{[[#1]]}}
\renewcommand{\S}{\mathfrak{S}}

\def\longleadsto{\mathrel{-}\joinrel\leadsto}
\DeclareMathOperator{\Sym}{Sym}
\DeclareMathOperator{\discr}{discr}
\newcommand{\op}{\mathrm{op}}

\title{Integral cohomology of the \\generalized Kummer fourfold}
\author[Simon Kapfer,\ \ Universit\"at Augsburg]{Simon Kapfer\vspace{-0.5cm}}
\institute{Universit\"at Augsburg}
\date{Varieties with trivial canonical bundles\\B^^a6dlewo, 17 June 2016\vspace{0.6cm}\\
\includegraphics[scale=0.07]{LogoInstitut}}

%\usetheme{Warsaw}  %Warsaw, Berkeley?
\usetheme{Darmstadt}
\useoutertheme{split}
\usecolortheme{whale}
%\usecolortheme[named=Peach]{structure}
\usefonttheme{serif}
%\usepackage{palatino}
\useinnertheme{rectangles}
%\usepackage{bookman}
%\setbeamercovered{transparent}

\setbeamertemplate{navigation symbols}{}
%\setbeamertemplate{footline}{}
%\setbeamertemplate{headline}{}

%\beamertemplateboldcenterframetitle
%\setbeamerfont{frametitle}{size={\Large}}

\newcommand*\oldmacro{}%
\let\oldmacro\insertshorttitle%


\newenvironment{changemargin}[2]{%
  \begin{list}{}{%
    \setlength{\topsep}{0pt}%
    \setlength{\leftmargin}{#1}%
    \setlength{\rightmargin}{#2}%
    \setlength{\listparindent}{\parindent}%
    \setlength{\itemindent}{\parindent}%
    \setlength{\parsep}{\parskip}%
  }%
  \item[]}{\end{list}}

\newcommand{\slogan}[1]{%
  \begin{center}%
    \setlength{\fboxrule}{2pt}%
    \setlength{\fboxsep}{-3pt}%
    {\usebeamercolor[fg]{item}\fbox{\usebeamercolor[fg]{normal
    text}\parbox{0.9\textwidth}{\begin{center}#1\end{center}}}}%
  \end{center}%
}

\makeatletter
    \newenvironment{withoutheadline}{
        \setbeamertemplate{headline}[default]
        \def\beamer@entrycode{\vspace*{-\headheight}}
    }{}
\makeatother

\newcommand{\hil}[1]{{\usebeamercolor[fg]{item}{#1}}}

\begin{document}

\setbeameroption{show notes}
\setbeamertemplate{note page}[plain]
\begin{withoutheadline}
\frame{\vspace{4mm}
\titlepage
}
\end{withoutheadline}
\addtobeamertemplate{frametitle}{}{%
\begin{textblock*}{100mm}(0.88\textwidth,-1.52cm)
\includegraphics[scale=0.18]{UniLogoNeg}
\end{textblock*}}
\addtocounter{framenumber}{-1}
\renewcommand*\insertshorttitle{%
  \oldmacro\hfill\insertframenumber\,/\,\inserttotalframenumber\hfill}

\frame{
\vspace{20pt}
\begin{center}
This is joint work with Gr\'egoire Menet.
\end{center}
}

\frame[t]{\frametitle{Structure}
%\begin{itemize}
%\item Generalities on integral cohomology of IHS varieties
%\begin{enumerate}
%\item Motivation
%\item Fujiki relation
%\end{enumerate}
%\pause
%\item Cohomology of the generalized Kummer 
%\begin{enumerate}
%\item Definition
%\item Link to Hilbert scheme cohomology
%\item Torsion points
%\item Deformations
%\end{enumerate}
%\end{itemize}
\tableofcontents
}

\section{IHSM}
\subsection{Motivation}
\frame[t]{\frametitle{Motivation}
Let $X$ be a IHS manifold. Why should we be interested in $ H^*(X,\Z) $?
\pause

Most important: Second cohomology group $H^2(X,\Z)$.\\
(Beauville--Bogomolov form, Torelli type theorems,  \ldots)
BNS obtain results on automorphisms of finite prime order of $X=K3\hilb{2}$ by looking at the embedding
$$
\Sym^2 H(X,\Z) \hookrightarrow H^4(X,\Z),
$$
so it is useful to know the higher cohomology groups.
\pause
$H^*(X,\Z)$ is a deformation invariant, and all material in this talk will be, too.
Comparing $ H^*(X,\Z) $ with $ H^*(X,\C) $ gives us information about $X$, e.g.\ on projectivity.
This talk does not discuss the embedding  $ H^*(X,\Z) \subset H^*(X,\C) $. It is not a deformation invariant.
\begin{question}
Which constructions in $H^*(X,\R/\C)$ carry over to $H^*(X,\Z)$?
\end{question}
}




\frame[t]{\frametitle{Beauville--Bogomolov form}
As an example, consider the quadratic Beauville--Bogomolov form $q_X : H^*(X,\R) \rightarrow \R$
\pause
\begin{theorem}[Fujiki]
$q_X(\alpha)^m = c \int_X \alpha^{2m}$ for some $c\in\R$.
\end{theorem}
\pause
\begin{corollary}
$q_X$ can be renormalized to yield a primitive integral quadratic form: $H^*(X,\Z) \rightarrow \Z $.
\end{corollary}
}
\frame[t]{\frametitle{Discriminant formula}

\begin{theorem}
Let $d+1$ be the rank of $H^2(X,\Z)$ and denote $c_X$ the Fujiki constant.
The discriminant of $\Sym^n\!H^2(X,\Z)$ is given by
\begin{gather*}
\left(\discr \left(H^2(X,\Z)\right)\right)^{\binom{d+n}{d+1}}\cdot c_X^{\binom{d+n}{d}} \cdot \prod_{i=1}^n i^{\binom{n-i+d}{d}d} 
\cdot C, \\
\qquad \text{with } \ 
C=
\left\{
 \begin{array}{*2{l}p{5cm}}
 \displaystyle\prod_{\substack{i=1 \\ i\ \text{odd}\\\ }}^{2n+d-1}i^{\binom{n-i+d}{d}} &\text{if }d\!+\! 1\text{ is odd}, \\
 \displaystyle\prod_{i=1}^{n+\frac{d-1}{2}} i^{\binom{n-i+d}{d} - \binom{n-2i+d}{d}} &\text{if }d\! +\! 1\text{ is even}.
\end{array}
\right.
\end{gather*}
\end{theorem}

}

\section{Generalized Kummer varieties}
\subsection{Definition}
\frame[t]{\frametitle{The generalized Kummer variety}
\begin{definition}
Let $A$ be a complex abelian surface and $A\hilb{n}$ the Hilbert scheme of points. Denote $\Sigma : A\hilb{n}\rightarrow A$ the summation morphism. 
Then the generalized Kummer $A\kum n$ is defined as the fiber of a point in $A$.
\begin{center}
\begin{tikzcd}[ampersand replacement=\&]
A\kum n
\arrow[hook]{r}{\theta}
\arrow{d}
\& A\hilb n \arrow{d}{\Sigma}\\
0 \arrow[hook]{r} \& A  
\end{tikzcd}
\end{center}
\end{definition}
\pause
First objective: Collect information about
$$
\theta^* : H^*(A\hilb n ,\Z) \rightarrow H^*(A\kum n, \Z). 
$$
}

\frame[t]{\frametitle{The standard diagram}
The morphism $\theta$ fits into a pullback diagram,
\begin{center}
\begin{tikzcd}[ampersand replacement=\&]
A\kum n
\arrow[hook]{r}{0\times \id}
\arrow[hook,bend left]{rr}{\theta}
\arrow{d}\&
 A\times A\kum n \arrow{d}{pr_1} 
\arrow{r}{\Theta}
\& A\hilb n \arrow{d}{\Sigma}\\
0 \arrow[hook]{r}
\& A  \arrow{r}{n\cdot}  \& A  
\end{tikzcd}
\end{center}
so that the morphism $\Theta$ is a $n^4$-fold covering.
}

\subsection{Pullback from the Hilbert scheme}
\frame[t]{\frametitle{First properties of the pullback $\theta^*$}
\begin{proposition}
The kernel of $\theta^* : H^*(A\hilb n ,\Z) \rightarrow H^*(A\kum n, \Z)$
is given by the (two-sided) ideal $\mathcal I$ in $H^*(A\hilb n,\Z) $ generated by $H^1$.
\end{proposition}
\pause
\begin{proof} It suffices to work with $\Q$-coefficients.
It is clear that $H^1$ maps to zero, so $\mathcal I  \subset \ker \theta^*$. 
Since the morphism 
$$
\Theta :  A\times A\kum n \rightarrow A\hilb n
$$
is a covering, $\ker \theta^*$ is the annihilator of the class of $A\kum n$ in $H^*(A\hilb n,\Q)$. 
Then one shows, that this corresponds to the ideal generated by $H^1$.
\end{proof}
}

\frame[t]{\frametitle{Surjectivity}
\begin{proposition}[Beauville '83]
For, $n\geq 3$, the morphism $\theta^*: H^2(A\hilb n,\Q) \rightarrow H^2(A\kum n ,\Q)$ is surjective.
\end{proposition}
\pause
\begin{proposition}[Britze?]
The above propostion holds with integral coefficients, too.
\end{proposition}
\pause
\begin{proposition}
For $n=3$, the morphism $\theta^*: H^*(A\hilb 3,\Z) \rightarrow H^*(A\kum 3 ,\Z)$ is surjective in every degree except four.
\end{proposition} 
}




\appendix
  \addtocounter{framenumber}{-1}
%\backupstart
\renewcommand*\insertshorttitle{%
  \oldmacro\hfill}

\frame[t,shrink=8]{\frametitle{References}
%\cite{qinintegral2004}\cite{lehnsorger}\cite{gross2003calabi}\cite{boissiereniepersarti}
\bibliographystyle{plain}
\bibliography{bibl}
}
\end{document}
