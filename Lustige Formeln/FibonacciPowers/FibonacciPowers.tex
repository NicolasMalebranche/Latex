\documentclass[]{article}
\usepackage[T1]{fontenc}
\usepackage[latin1]{inputenc}
\usepackage[german]{babel}
\usepackage{amsmath,amsthm,amssymb,amscd,color,graphicx}
  
\theoremstyle{remark}
\newtheorem{satz}{Satz}
\newtheorem{aufg}{Aufgabe}
\newtheorem{definition}{Definition}

%opening
\title{Potenzen von Fibonaccizahlen}
\author{Simon Kapfer}

\begin{document}

\maketitle
Die Fibonaccizahlen $f_n$ definiert durch $f_{n+2}=f_{n+1}+f_n,\ f_1=1,\ f_0=0$. Die Lucaszahlen $l_n$ sind so definiert, da� $l_{n+2}=l_{n+1}+l_n,\ l_1=1,\ l_0=2$.
Definiere die �berspringenden Fibonaccireihen ohne konstanten Term:
\begin{equation}
F_{(k)}(x) := \frac{f_k}{1-l_kx-x^2}  = \sum_{n\geq 0} f_{k(n+1)}x^n \\
\end{equation}
und die modifizierten �berspringenden Lucasreihen:
\begin{equation}
\tilde{L}_{(k)}(x) := \left\{ \begin{array}{ll} \frac{1}{1-x},& k=0 \\ \\\frac{l_k-2x}{1-l_kx+x^2}, &k>0 \end{array}\right.
\end{equation}
Bemerkung: Es gilt f�r $k\geq 1$:
\[\tilde{L}_{(2k)}(x) = \sum_{n\geq 0 } l_{k(n+1)}x^n \]
Definiere die erzeugenden Funktionen der Potenzen der Fibonaccizahlen:
\[F^p(x) := \sum_{n\geq 0} f^p_{n+1}x^n\]

\begin{satz}
Es gilt f�r gerade Fibonacci--Potenzen:
\begin{equation}
F^{2p}(x) = \frac{1}{5^p}\sum_{k=0}^p \binom{2p}{p-k}\tilde{L}_{(2k)}((-1)^{p-k}x)
\end{equation}
und f�r ungerade Fibonacci--Potenzen:
\begin{equation}
F^{2p+1}(x) =  \frac{1}{5^p}\sum_{k=0}^p \binom{2p+1}{p-k}F_{(2k+1)}((-1)^{p-k}x)
\end{equation}
\end{satz}
\end{document}
