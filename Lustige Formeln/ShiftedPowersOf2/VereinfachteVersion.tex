\documentclass[11pt]{article}
\usepackage[T1]{fontenc}
\usepackage[latin1]{inputenc}
\usepackage[german]{babel}
\usepackage{fourier}  % Use the Adobe Utopia font for the document
\usepackage{amsmath,amsthm,amssymb,amscd,color,graphicx}

% Geschwungene Kleinbuchstaben im Mathemodus (benutze \mathpzc)
\DeclareFontFamily{OT1}{pzc}{}
\DeclareFontShape{OT1}{pzc}{m}{it}{<-> s * [1.10] pzcmi7t}{}
\DeclareMathAlphabet{\mathpzc}{OT1}{pzc}{m}{it}


%Struktur
\newcommand{\point}{\vspace{3mm}\par \noindent \refstepcounter{subsection}{\bf \thesubsection.} }
	\numberwithin{equation}{subsection}
\newcommand{\tpoint}[1]{\vspace{3mm}\par \noindent \refstepcounter{subsection}{\bf \thesubsection.} 
  \numberwithin{equation}{subsection} {\em #1. ---} }
\newcommand{\epoint}[1]{\vspace{3mm}\par \noindent \refstepcounter{subsection}{\bf \thesubsection.} 
  \numberwithin{equation}{subsection} {\em #1.} }
\newcommand{\bpoint}[1]{\vspace{3mm}\par \noindent \refstepcounter{subsection}{\bf \thesubsection.} 
  \numberwithin{equation}{subsection} {\bf\em #1.} }
  
%Abk�rzungen
\newcommand{\N}{\mathbb{N}}
\newcommand{\Z}{\mathbb{Z}}
\newcommand{\Q}{\mathbb{Q}}
\newcommand{\C}{\mathbb{C}}
\newcommand{\R}{\mathbb{R}}
\newcommand{\Cinf}{C^\infty}
\newcommand{\id}{\text{id}}


%\binoppenalty=7000
%\relpenalty=5000 

% Title Page
\title{Ein Satz �ber Teilbarkeit}
\author{Simon Kapfer}
\date{\today}
\begin{document}
\maketitle
\noindent 
Gegeben eine beliebige nat�rliche Zahl, deren letzte Ziffer gleich 1 ist, etwa 5821. Definiere die Folge $(c_n)_n$ �ber die erzeugende Funktion
$$ \sum_{n\geq 0} c_n x^n \ = \ \frac{1}{1-x}\;\left(\;\frac{5821}{1+20x+800x^2+5000x^3} - 1 \;\right)
$$
Dann gilt: $c_n$ ist durch $10^{n+1}$ teilbar. \\\\Wie die Aussage f�r andere Zahlen als 5821 lauten mu�, sollte klar sein.\\ Statt dem Dezimalsystem kann man auch ein beliebiges anderes nehmen, wenn man $10^{n+1}$ durch das Entsprechende ersetzt.
\\\\
\emph{Beweis: (Peter Uebele sei Dank)} \begin{align*}
\sum c_nx^n &= \frac{1}{1-x}\left(\frac{\sum_{i\geq 1}^Na_i d^i +1}{\sum_{i\geq 1}^Na_id^ix^i + 1 } -1  \right) \\
&= \frac{1}{1-x}\frac{\sum_{i\geq 1}^Na_i d^i -\sum_{i\geq 1}^Na_i d^ix^i}{\sum_{i\geq 1}^Na_id^ix^i + 1 }  \\
&= \frac{1}{1-x}\frac{\sum_{i\geq 1}^Na_i d^i \left(1-x^i\right)}{\sum_{i\geq 1}^Na_id^ix^i + 1 }  \\
&=\frac{\sum_{i\geq 1}^Na_i d^i \left(x^{i-1}+x^{i-2}+\ldots +1 \right)}{\sum_{i\geq 1}^Na_id^ix^i + 1 } \\
&= \sum_{i\geq 1}^Na_i d^i \left(x^{i-1}+x^{i-2}+\ldots +1 \right)\cdot\sum_{m\geq 0} \left(-\sum_{i=1}^N a_id^ix^i\right)^m
\end{align*}
\end{document}          
