\documentclass[11pt]{article}
\usepackage[T1]{fontenc}
\usepackage[latin1]{inputenc}
\usepackage[german]{babel}
\usepackage{fourier}  % Use the Adobe Utopia font for the document
\usepackage{amsmath,amsthm,amssymb,amscd,color,graphicx}

% Geschwungene Kleinbuchstaben im Mathemodus (benutze \mathpzc)
\DeclareFontFamily{OT1}{pzc}{}
\DeclareFontShape{OT1}{pzc}{m}{it}{<-> s * [1.10] pzcmi7t}{}
\DeclareMathAlphabet{\mathpzc}{OT1}{pzc}{m}{it}


%Struktur
\newcommand{\point}{\vspace{3mm}\par \noindent \refstepcounter{subsection}{\bf \thesubsection.} }
	\numberwithin{equation}{subsection}
\newcommand{\tpoint}[1]{\vspace{3mm}\par \noindent \refstepcounter{subsection}{\bf \thesubsection.} 
  \numberwithin{equation}{subsection} {\em #1. ---} }
\newcommand{\epoint}[1]{\vspace{3mm}\par \noindent \refstepcounter{subsection}{\bf \thesubsection.} 
  \numberwithin{equation}{subsection} {\em #1.} }
\newcommand{\bpoint}[1]{\vspace{3mm}\par \noindent \refstepcounter{subsection}{\bf \thesubsection.} 
  \numberwithin{equation}{subsection} {\bf\em #1.} }
  
%Abk�rzungen
\newcommand{\N}{\mathbb{N}}
\newcommand{\Z}{\mathbb{Z}}
\newcommand{\Q}{\mathbb{Q}}
\newcommand{\C}{\mathbb{C}}
\newcommand{\R}{\mathbb{R}}
\newcommand{\Cinf}{C^\infty}
\newcommand{\id}{\text{id}}


%\binoppenalty=7000
%\relpenalty=5000 

% Title Page
\title{Divisibilit�t durch $2^n$ unter affinen Abbildungen}
\author{Simon Kapfer}
\date{\today}
\begin{document}
\maketitle
\noindent Betrachte die Mengen $B_0,\;B_1,\;B_2,\;\ldots$, gegeben durch 
$$ B_n := \left\{ z\in \Z\; | \;z \equiv 2^{n} \ \text{mod}\ 2^{n+1} \right\}, \quad n\geq 0
$$
Da $B_n$ nichts anderes ist als die Teilmenge der ganzen Zahlen, deren Bin�rdarstellung auf genau $n$ Nullen endet, zerf�llt
die Menge $\Z \backslash \{0\}$ offensichtlich in die disjunkte Vereinigung dieser $B_n$. Da $2^n\in B_n$, ist $\frac{1}{1-2x} = \sum 2^n x^n$ eine erzeugende Funktion einer Folge von Elementen dieser Mengen $B_n$.
\\
Betrachte die affine, injektive Abbildung 
$$f: \Z\rightarrow \Z,\  \ x\mapsto ax + 1, \quad \text{f�r ein ungerades, positives } a.$$ 
Nun suchen wir eine Folge $(r_n)$ mit $f\left(r_n\right) \in B_n $. \\
\textbf{Vermutung:} Sei $a = 1 + 2 a_1 + 4a_2 +8a_3 +\ldots $ die Bin�rdarstellung von $a$. Dann ist eine m�gliche Folge $(r_n)$ durch folgende erzeugende Funktion gegeben:
$$
\sum r_n x^n \ = \ \frac{1}{1-2x} - \frac{1}{(1-x)(1+2a_1x+4a_2x^2+8a_3x^3 + \ldots)}
$$
\end{document}          
