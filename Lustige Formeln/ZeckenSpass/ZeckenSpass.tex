\documentclass{article}
\usepackage[T1]{fontenc}
\usepackage[latin1]{inputenc}
\usepackage[german]{babel}
\usepackage{amsmath,amsthm,amssymb,amscd,color,graphicx}
  
\theoremstyle{definition}
\newtheorem{satz}{Satz}
\newtheorem{aufg}{Aufgabe}
\newtheorem{definition}{Definition}

%opening
\title{Spa� mit der Zeckendorf--Darstellung}
\author{Simon Kapfer}

\begin{document}

\maketitle
\section{Extensionen von Ringen}
Wir betrachten exakte Sequenzen von abelschen Gruppen der Form
\begin{equation}
0 \longrightarrow A \stackrel{f}{\longrightarrow} B\stackrel{g}{\longrightarrow} A \longrightarrow 0,
\end{equation}
wobei $A$ und $B$ beides Ringe sein sollen und $f$ ein Ringhomomorphismus. Au�erdem gelte folgende Projektionsformel f�r alle $a\in A$, $b\in B$:
\begin{equation}
g\left(f(a)\cdot b\right) = a \cdot g(b).
\end{equation}
\begin{satz}
Es gilt: $g(1) = 0$.
\end{satz}
\begin{proof}
$g(1) = g\left(f(1)\cdot 1\right) = g\left(f(1)\right) =0$.
\end{proof}
\begin{satz}
$g$ bildet Ideale auf Ideale ab.
\end{satz}
\begin{proof}
Sei $J\subset B$ ein Ideal. Dann ist $g(J)$ eine additive Untergruppe von $A$. 
Seien nun $a\in A$ und $b\in J$ beliebig. Dann gilt:
$a\cdot g(b) = g\left(f(a)\cdot b \right) \in g(J)$. Also ist $g(J)\subset A$ ein Ideal.
\end{proof}
\begin{satz}
Sei $J\subset B$ ein Ideal. Dann haben wir eine Inklusion von Idealen 
$$f^{-1}(J)\subset g(J).$$
\end{satz}
\begin{proof}
W�hle zun�chst ein $x\in B$ mit $g(x) = 1$. Sei nun $a\in f^{-1}(J)$ beliebig. Dann gilt:
$a= a\cdot 1 = g\left(f(a)\cdot x\right) \in g(J)$, da $f(a)\in J$.
\end{proof}

\end{document}
