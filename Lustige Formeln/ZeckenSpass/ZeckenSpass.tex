\documentclass{article}
\usepackage[T1]{fontenc}
\usepackage[latin1]{inputenc}
\usepackage[german]{babel}
\usepackage{amsmath,amsthm,amssymb,amscd,mathtools}
  
\theoremstyle{definition}
\newtheorem{satz}{Satz}
\newtheorem{aufg}{Aufgabe}
\newtheorem{definition}{Definition}

\newcommand{\Z}{\mathbb{Z}}
\DeclareMathOperator{\ev}{ev}

%opening
\title{Spa� mit der Zeckendorf--Darstellung}
\author{Simon Kapfer}

\begin{document}

\maketitle
\section{Extensionen von Ringen}
Wir betrachten exakte Sequenzen von abelschen Gruppen der Form
\begin{equation} \label{exakt}
0 \longrightarrow A \stackrel{f}{\longrightarrow} B\stackrel{g}{\longrightarrow} A \longrightarrow 0,
\end{equation}
wobei $A$ und $B$ beides Ringe sein sollen und $f$ ein Ringhomomorphismus. Au�erdem gelte folgende Projektionsformel f�r alle $a\in A$, $b\in B$:
\begin{equation} \label{projektion}
g\left(f(a)\cdot b\right) = a \cdot g(b).
\end{equation}
\begin{satz}
Es gilt: $g(1) = 0$.
\end{satz}
\begin{proof}
$g(1) = g\left(f(1)\cdot 1\right) = g\left(f(1)\right) =0$.
\end{proof}
\begin{satz}
$g$ bildet Ideale auf Ideale ab.
\end{satz}
\begin{proof}
Sei $J\subset B$ ein Ideal. Dann ist $g(J)$ eine additive Untergruppe von $A$. 
Seien nun $a\in A$ und $b\in J$ beliebig. Dann gilt:
$a\cdot g(b) = g\left(f(a)\cdot b \right) \in g(J)$. Also ist $g(J)\subset A$ ein Ideal.
\end{proof}
\begin{satz}
Sei $J\subset B$ ein Ideal. Dann haben wir eine Inklusion von Idealen 
$$f^{-1}(J)\subset g(J).$$
\end{satz}
\begin{proof}
W�hle zun�chst ein $x\in B$ mit $g(x) = 1$. Sei nun $a\in f^{-1}(J)$ beliebig. Dann gilt:
$a= a\cdot 1 = g\left(f(a)\cdot x\right) \in g(J)$, da $f(a)\in J$.
\end{proof}

\section{Formale Zeckendorf--Summen}
Sei $Z$ die Menge der $\Z$-Linearkombinationen in den formalen Variablen $\{f_i\}_{i\in \Z}$ modulo den Relationen $f_{i+2} = f_{i+1} + f_i$. 
Man kann sich �berlegen, da� ein Element $z\in Z$ immer dargestellt werden kann als
\begin{equation}
z = \sum_{i} \alpha_if_i, \quad\text{wobei }\alpha_i\in\{-1,0,1\},\ \sum_{i}|\alpha_i|<\infty  
\end{equation}
und da� diese Darstellung eindeutig wird, wenn man noch ein paar Bedingungen mehr an die $\alpha_i$ stellt.

Auf $Z$ ist eine Ringstruktur erkl�rt, indem man f�r die Multiplikation (notiert mit dem Symbol $*$) setzt:
$$
\left(\sum_{i} \alpha_if_i\right)\left(\sum_{j} \beta_jf_j\right) \coloneqq \sum_{i,j}\alpha_i\beta_j \fallingdotseq_{i+j}.
$$ 
Das neutrale Element bez�glich der Multiplikation ist $f_0$.
\begin{satz}
Sei $\phi$ der goldene Schnitt. Wir haben einen Ringisomorphismus 
\begin{align*}
\varphi : Z &\longrightarrow \Z[\phi],\\
f_i &\longmapsto \phi^i.
\end{align*}
\end{satz}
Wir haben au�erdem die (additive) Evaluationsabbildung $\ev : Z \longrightarrow \Z$, die $f_i$ auf die $i$-te Fibonaccizahl schickt. 
\begin{satz}Diese Abbildung pa�t in eine kurze exakte Sequenz der Form (\ref{exakt}):
\begin{equation}
0 \longrightarrow \Z \stackrel{m}{\longrightarrow} Z\stackrel{\ev}{\longrightarrow} \Z \longrightarrow 0
\end{equation}
und erf�llt au�erdem (\ref{projektion}).
\end{satz}
\begin{proof}
Den Isomorphismus $\varphi$ verwendend, schreibt sich die Sequenz folgenderma�en:
\begin{align*}
0 \longrightarrow \Z \longrightarrow \Z[\phi] &\longrightarrow \Z \longrightarrow 0 ,\\
1 \ & \longmapsto 0, \\
\phi\ &\longmapsto 1.
\end{align*}
Man rechnet leicht nach, da� in dieser Darstellung die Projektionsformel (\ref{projektion}) erf�llt ist.
Andererseits ist $\ev$ schon durch $\ev(f_0)=0 $ und $\ev(f_1) =1$ eindeutig bestimmt. Also stimmt $\ev$ mit der gegebenen Abbildung �berein.
\end{proof}

\end{document}
