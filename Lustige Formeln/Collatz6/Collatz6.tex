\documentclass[]{article}
\usepackage[T1]{fontenc}
\usepackage[latin1]{inputenc}
\usepackage[german]{babel}
\usepackage{amsmath,amsthm,amssymb,amscd,color,graphicx}
  
\theoremstyle{remark}
\newtheorem{satz}{Satz}
\newtheorem{aufg}{Aufgabe}
\newtheorem{definition}{Definition}

%opening
\title{Die Collatzsche Funktion zur Basis $6$}
\author{Simon Kapfer}

\begin{document}

\maketitle
\begin{definition}
Bezeichne mit $B_i$, $i=0,1$ den von rechts wirkenden Operator $x\mapsto 2x+i$, mit $T_j$, $j=0,1,2$ den Operator $x\mapsto 3x+j$. Angewandt auf $0$, entspricht ein Produkt von $B_i$ der Bin�rdarstellung einer Zahl, ein Produkt von $T_j$ der Darstellung einer Zahl zur Basis $3$. Eine Darstellung mit abwechselnden $B_i$ und $T_j$ kann als zur Basis $6$ geh�rig interpretiert werden.
\end{definition}
Es gelten folgende Vertauschungsregeln:
\begin{equation*}
\begin{array}{r|c|c|c|c|c|c}
 &B_0T_0&B_0T_1&B_0T_2&B_1T_0&B_1T_1&B_1T_2\\ 
=&T_0B_0&T_0B_1&T_1B_0&T_1B_1&T_2B_0&T_2B_1
\end{array}
\end{equation*}
\begin{definition}
Ein Collatz-Feld ist eine Funktion $f : \mathbb{Z}^2 \supset U \rightarrow \{B_0,B_1,T_0,T_1,T_2\}$ mit folgenden Eigenschaften:
\begin{enumerate}
\item Wenn $(m,n)\in U$, dann auch $(m,n+1)\in U $ und $(m+1,n+1)\in U$.
\item $f(m,n) \in \{B_0,B_1\}$, falls $m+n$ ungerade und $f(m,n) \in \{T_0,T_1,T_2\}$, falls $m+n$ gerade.
\item $f(m,n+1)f(m,n) = f(m+1,n+1)f(m+1,n)$, falls $m+n$ gerade. Insbesondere $f(m+1,n)\in U$.
\item $f(m,n) \in \{B_0,T_0\} $ f�r $n\gg 0$.
\item Aus $f(m,n) =B_0$ und $(m,n-1)\notin U$ folgt $(m+1,n)\notin U$.
\item Aus $f(m,n) =B_1$ und $(m,n-1)\notin U$ folgt $(m+1,n)\in U$ und $f(m+1,n)=T_2$.
\end{enumerate}
\end{definition}
\begin{satz}Umrechnung der Basen. Sei $m+n$ gerade und $f$ ein Collatz-Feld.
\begin{enumerate}
\item Angewandt auf $0$, ist \\
$\ldots f(m-2,n+2)f(m-1,n+1)f(m,n) = \ldots f(m,n+2)f(m,n+1)f(m,n)$.
\item Definiere $g(0)=0,\ g(1) = 1,\ g(3)=0,\ g(4)=1$. Definiere $\tilde{g}(T_j,T_k) = B_{g(j+k)}$ und $\tilde f (m,n) = \tilde g(f(m+1,n+1),f(m,n))$. Dann gilt, angewandt auf $0$:\\
$\ldots \tilde f(m+2,n+2)\tilde f(m+1,n+1)\tilde f(m,n) = \ldots f(m,n+2)f(m,n+1)f(m,n)$.
\end{enumerate}
Man kann eine Zahl so in den Basen $2$, $3$, und $6$ lesen.
\end{satz}
\begin{proof}
Nach Induktion und weil $f(m,n) \in \{B_0,T_0\} $ f�r $n\gg 0$ gen�gt es zu zeigen, dass 
\begin{enumerate}
\item $\ldots f(m-1,n+2)f(m-1,n+1)f(m,n) = \ldots f(m,n+2)f(m,n+1)f(m,n)$.
\item $\ldots f(m+1,n+2)f(m+1,n+1)\tilde f(m,n) = \ldots f(m,n+2)f(m,n+1)f(m,n)$.
\end{enumerate}
Das folgt aber aus der Definition.
\end{proof}

\end{document}
