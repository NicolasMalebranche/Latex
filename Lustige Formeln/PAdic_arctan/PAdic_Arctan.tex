\documentclass[11pt]{article}
\usepackage[T1]{fontenc}
\usepackage[latin1]{inputenc}
\usepackage[english]{babel}
\usepackage{fourier}  % Use the Adobe Utopia font for the document
\usepackage{amsmath,amsthm,amssymb,amscd,color,graphicx}

% Geschwungene Kleinbuchstaben im Mathemodus (benutze \mathpzc)
\DeclareFontFamily{OT1}{pzc}{}
\DeclareFontShape{OT1}{pzc}{m}{it}{<-> s * [1.10] pzcmi7t}{}
\DeclareMathAlphabet{\mathpzc}{OT1}{pzc}{m}{it}


%Struktur
\newcommand{\eintrag}{\subparagraph*}
  
%Abk�rzungen
\newcommand{\N}{\mathbb{N}}
\newcommand{\Z}{\mathbb{Z}}
\newcommand{\Q}{\mathbb{Q}}
\newcommand{\C}{\mathbb{C}}
\newcommand{\R}{\mathbb{R}}
\newcommand{\Cinf}{C^\infty}
\newcommand{\id}{\text{id}}
\renewcommand{\O}{\mathcal{O}\!}
\renewcommand{\S}{\mathfrak{S}}
\newcommand{\T}{\mathcal{T}}
\newcommand{\E}{\mathcal{E}}
\newcommand{\A}{\mathcal{A}}
\renewcommand{\a}{\mathfrak{a}}
\newcommand{\m}{\mathfrak{m}}
\newcommand{\Hom}{\mathcal{H}\!\!\!\mathpzc{om}}
\newcommand{\End}{\mathcal{E}\!\!\mathpzc{nd}}
\renewcommand{\d}{d\!}
\newcommand{\del}{\partial}
\newcommand{\delbar}{\overline{\partial}}
\newcommand{\dzbar}{\d\overline{z}}
\newcommand{\diff}[1]{\frac{\partial}{\partial #1}}
\newcommand{\vac}{\left|0\right>}

%\binoppenalty=7000
%\relpenalty=5000 

% Title Page
\title{P-adic $\arctan$}
\author{Simon Kapfer}
\date{\today}

\begin{document}
\maketitle
\begin{abstract}
We define the function $\arctan$ in the p-adic fields $\Q_p$.
\end{abstract}
For $p$ a prime, $x \in \Q_p$, the series
$$\arctan(x) := x - \frac{x^3}{3} +\frac{x^5}{5} -\frac{x^7}{7} +- \ldots $$
converges, if $ \|x\|_p<1$. We would like to extend this definition to all $x \in  \Q_p$. Therefore, we must use some functional equation for $\arctan$, namely
\begin{align*}
\arctan(x) &= \operatorname{Im}\log(1+ ix) \\
&= \frac{1}{m}\operatorname{Im}\log((1+ ix)^m) \\
&= \frac{1}{m}\operatorname{Im}\log \left(\sum_k (-1)^k \binom{m}{2k}x^{2k} + i\sum_k (-1)^k \binom{m}{2k+1}x^{2k+1}\right) \\
&=  \frac{1}{m}\operatorname{Im}\log \left(1+ i \frac{\sum_k (-1)^k \binom{m}{2k+1}x^{2k+1}}{\sum_k (-1)^k \binom{m}{2k}x^{2k}}\right) \\
&=\frac{1}{m}\arctan\left(  \frac{\sum_k (-1)^k \binom{m}{2k+1}x^{2k+1}}{\sum_k (-1)^k \binom{m}{2k}x^{2k}}\right)
\end{align*}
which holds for all $m\geq 1$ and all $x\in \R$. We can use this identity to decrease the p-adic norm of the argument of $\arctan$ for an appropriate choice of $m$: 
\begin{itemize}
\item If $p=2$, take $m=4$.
\item If $p= 4n -1$, take $m=p+1$. 
\item If $p= 4n +1$, take $ m = p-1$. In this case, $i:=\sqrt{-1}\in\Q_p$ and we could also have used the identity: $\arctan(x) = \frac{1}{2i}\log\left(\frac{x+i}{x-i} \right)$ (Iwasawa logarithm).
Now we can check some formulas for $\pi$. It surprisingly turns out, that all of them become 0, e. g. 
\begin{align*}
\pi &= 4 \arctan(1) \\
&= 4\arctan\left(\tfrac{1}{2}\right) +\arctan\left(\tfrac{1}{3}\right) \\
&= 16\arctan\left(\tfrac{1}{5}\right) -  4\arctan\left(\tfrac{1}{239}\right) \\
&= 2\arctan(x) + 2\arctan\left(x^{-1}\right) \quad \forall x \in \Q_p \\
&=0.
\end{align*}

\end{itemize}  

\end{document}