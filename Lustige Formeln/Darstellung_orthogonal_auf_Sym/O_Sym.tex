\documentclass{article}
\usepackage[T1]{fontenc}
\usepackage[latin1]{inputenc}
\usepackage[german]{babel}
\usepackage{amsmath,amsthm,amssymb,amscd,mathtools}
  
\theoremstyle{definition}
\newtheorem{satz}{Satz}
\newtheorem{lemma}{Lemma}
\newtheorem{aufg}{Aufgabe}
\newtheorem{definition}{Definition}

\newcommand{\Z}{\mathbb{Z}}
\newcommand{\R}{\mathbb{R}}
\renewcommand{\H}{\mathcal{H}}
\newcommand{\bra}{\left<\!\!\!\:\left<}
\newcommand{\ket}{\right>\!\!\!\:\right>}
\DeclareMathOperator{\Sym}{Sym}

%opening
\title{Darstellungen der orthogonalen Gruppe auf homogenen Polynomen}
\author{Simon Kapfer}

\begin{document}

\maketitle
\section{Darstellungen �ber $\R$}
\begin{definition}
Wir betrachten einen endlich-dimensionalen Vektorraum $V$ �ber $\R$, versehen mit einem Skalarprodukt $(\ ,\ )$ und definieren so die orthogonale Gruppe $O(V)$ als Menge derjenigen linearen Endomorphismen von $V$, die das Skalarprodukt erhalten. Sei $x_1,\ldots, x_d \in V$ eine Orthonormalbasis.
\end{definition}
\begin{definition}
Auf $V^{\otimes n}$ wirkt $O(V)$ diagonal: F�r $U\in O(V)$ haben wir 
$$
U(v_1\otimes\cdots\otimes v_n) = U(v_1)\otimes\cdots \otimes U(v_n). 
$$
Diese Wirkung kommutiert mit Vertauschungen der Tensorfaktoren und ergibt also eine Wirkung auf $\Sym^n(V)$. Diesen Raum k�nnen wir mit den Polynomen in $x_i$ identifizieren, welche homogen vom Grad $n$ sind: $\Sym^n(V) = \R[x_1,\ldots,x_d]_n$.
\end{definition}
\begin{aufg}\label{eins}
Was sind die m�glichen symmetrischen Bilinearformen auf $\Sym^n(V)$, bez�glich derer die Wirkung von $O(V)$ orthogonal bleibt?
\end{aufg}
\begin{lemma}
Sei $B$ eine $O(V)$-invariante symmetrische Bilinearform auf $S$. Dann ist $\ker B = \{s\in S\,|\, B(s, \_ ) =0\}$ ein $O(V)$-invarianter Untervektorraum von $S$.
\end{lemma}
\begin{satz}\label{DarstellungLink}
Solche symmetrischen Bilinearformen bilden einen Vektorraum, dessen Dimension gleich der Anzahl irreduzibler $O(V)$-Darstellungen in $\Sym^n(V)$ ist.
\end{satz}
\begin{proof}
Sei $S= S_1\oplus S_2$ eine Zerlegung von Darstellungen. Sind nun $B_1$ auf $S_1$ und $B_2$ auf $S_2$ $O(V)$-invariante Bilinearformen, so ergibt jede Linearkombination 
\[
 B = \alpha_1 B_1 \oplus \alpha_2 B_2
\]
eine $O(V)$-invariante Form auf $S$.

Seien umgekehrt $B_1$, $B_2$ $O(V)$-invariante Formen auf $S$. Dann existiert eine Affinkombination $B = \alpha B_1 + (1-\alpha) B_2$, welche entartet ist. Dies kann man sich klar machen, indem man die zugeh�rigen Gramschen Matrizen betrachtet, denn jede Gerade in $\R^{m\times m}$ schneidet das Nullstellengebilde der Determinante. Nach dem obigen Lemma ist $\ker B$ eine Teildarstellung von $S$, welche aufgrund der Entartung mehr als nur die Null umfa�t. Falls $S$ jetzt irreduzibel ist, mu� $\ker B =S$ gelten. Das bedeutet aber, da� $B_1$ und $B_2$ linear abh�ngig sind.
\end{proof}
\begin{satz} \cite[Sect.~7]{DaiXu}
 Sei $\H_n^d \subset  \R[x_1,\ldots,x_d]_n$ die Menge der harmonischen Polynome und sei $r^2= x_1^2+\ldots +x_d^2$ der Radius zum Quadrat. Dann ist 
 \begin{equation}\label{HZerlegung}
 \R[x_1,\ldots,x_d]_n = \H_n^d \oplus r^2 \H_{n-2}^d \oplus r^4  \H_{n-4}^d \oplus \ldots
 \end{equation}
eine Zerlegung in irreduzible $O(V)$-Darstellungen.
\end{satz}

Damit d�rfte Aufgabe \ref{eins} wohl weitgehend gel�st sein.

\section{�ber $\Z$}
Die Frage wird etwas komplizierter, wenn wir mit ganzzahligen Koeffizienten arbeiten. Ersetzen wir also $V$ durch einen freien $\Z$-Modul, k�nnen wir die gleiche Frage nach den kompatiblen symmetrischen Bilinearformen auf $\Z[x_1,\ldots,x_d]_n$ stellen. Die Antwort wird wahrscheinlich nicht viel anders als in Satz \ref{DarstellungLink} sein, nur haben wir das Problem, da� Gleichung (\ref{HZerlegung}) nicht mehr gilt. Vielmehr haben wir
\[
  \Z[x_1,\ldots,x_d]_n \supset \H_n^d \oplus r^2 \H_{n-2}^d \oplus r^4  \H_{n-4}^d \oplus \ldots,
\]
wobei hier nat�rlich nur harmonische Polynome mit ganzzahligen Koeffizienten betrachtet werden.


\bibliographystyle{plain}
\begin{thebibliography}{}
\bibitem{DaiXu}
{Dai}, F. and {Xu}, Y., \emph{Spherical Harmonics},
arXiv 1304.2585.
\end{thebibliography}



\end{document}
