\documentclass[]{article}
\usepackage[T1]{fontenc}
\usepackage[latin1]{inputenc}
\usepackage[german]{babel}
\usepackage{amsmath,amsthm,amssymb,amscd,color,graphicx}
\usepackage{enumitem}
\theoremstyle{remark}
\newtheorem{satz}{Satz}
\newtheorem{aufg}{Aufgabe}
\newtheorem{definition}{Definition}

%opening
\title{Behandelte Themen des Mathesch�lerzirkels}
\author{Simon Kapfer}
\date{Schuljahr 2013/14}

\begin{document}

\maketitle
Meine Gruppe hat an den jeweiligen Terminen sich mit untenstehendem besch�ftigt. Klassenstufen 7 und 8.
\begin{enumerate}
\item Schriftliches Wurzelziehen
\item Diagonale des Quadrats und Satz des Pythagoras
\item Andere geometrische Aufgaben mit Wurzeln; Rechnen mit Wurzeln; Vedische Methode zum Dividieren 
\end{enumerate}
Dann waren Weihnachtsferien. Die Inhalte wurden an der Tafel vorgef�hrt. Es schien mir, als w�rde diese Art des Vorgehens den Sch�lern nicht soviel Spa� bringen. M�glicherweise waren die Aufgaben zu schwer. Frontalunterricht h�lt sie aber ruhig.
\begin{enumerate}[resume]
\item Nim--Spiele mit M�nzen. Nimm 2-3-4. 1 bis 3 Stapel und modulo-4-Strategie. 
\item Hackenbush. Normales Nim mit Bin�rsystem-Strategie. Resonanz gut. Weitere Bin�rsystem--Aufgaben gew�nscht. \\
3 von 8 Kindern 1 Stunde versp�tet. Teilweise recht lebhaft, Ermahnungen n�tig. N�chstes Mal nicht gegeneinander spielen lassen! Aggressionspotential! Kinder erwarten von einem Spiel, da� es fair ist.
\item Vertretung durch Christian Nolde. Siehe extra Anlage.
\item Arimaa-Vortrag von Peter Uebele. Resonanz gut. P�nktlichkeit der Kinder l��t zu w�nschen �brig.
\item Sierpinski-Dreieck. Muster in Pascal-Dreieck, Chaosspiel und Baryzentrische Koordinaten. Thema ist gut. Mehrere Kinder fehlten (Skilager).
\item K�stenl�nge Englands und Kochsche Kurve. Mandelbrot-Menge und erste Hinweise zu komplexen Zahlen.
\item 7. April. Wieder alle da. Rechnen mit komplexen Zahlen. Definition und Animation der Mandelbrotmenge. Thema wohl zu schwierig: Verst�ndnis bei 2 oder 3 Kindern (die fandens spannend), beim Rest nur L�rm und Quatschen. Beim Computerkino wieder alle dabei. N�chstes Mal was leichteres!
\item 5. Mai. Platonische und archimedische K�rper gebaut. Eulersche Polyederformel. Kinder hatten Spa�, man mu� aber aufpassen, da� sie nicht irgendeinen Bl�disinn machen. Insgesamt sehr unruhig. 
\item 3 Kinder fehlten, eins davon entschuldigt. Sch�nes Wetter, auf dem Balkon. Thema: Rund um Fibonaccizahlen anhand des Aufgabenblattes. Funktionierte ziemlich gut.
\item Diesmal im Computerraum, um Haskell kennenzulernen. Klappte ganz gut. Wir haben Folgen (Fibonacci, Dreieckszahlen,...) programmiert. Talent / Interesse zeigte sich bei anderen Kindern als sonst. Erst gegen Ende zunehmend unruhig, evtl. aufgrund der steilen Lernkurve.
\item 30. Juni. Formeln f�r Dreiecks- Quadrat- F�nfeckszahlen usw. Anschlie�end Symmetrien von Tapetenmustern, kristallographische Gruppe und Penrosemuster. Resonanz sehr gut. Pa�t auch gut zum Niveau der Kinder. Eines (Kay) war entschuldigt.
\item Dreiecke z�hlen. Nim-Spiel mit Schokolade. Slitherlink. F�r jedes Thema waren andere Kinder talentiert. Gegen Ende l��t die Disziplin nach. 2 Kinder fehlten. Zum Schlu� noch Themen des Schuljahrs wiederholt. Im Ged�chtnis geblieben sind (sortiert nach Reihenfolge des Erinnerns): Wurzelziehen, Vedische Methode, Bin�rzahlen, Dreieckszahlen, Fibonaccizahlen, Satz des Pythagoras, Mandelbrotmenge, Nim-Spiel mit M�nzen, Progammieren, Penrosemuster, Sierpinskidreieck.
\end{enumerate}

\end{document}
